\documentclass[12pt,a4paper,titlepage]{report}

%% font encoding.
\usepackage[utf8]{inputenc}

%% Fonts
\usepackage[T1]{fontenc}
\usepackage{lmodern}
\usepackage{couriers}

%% Bibliography.
\usepackage[bahasa]{babel}
\usepackage{csquotes}
\usepackage[
	style=apa
]{biblatex}
\DeclareLanguageMapping{bahasa}{bahasa-apa}

%% Customize chapters.
\usepackage{titlesec}

%% \includegraphics{name}
\usepackage{graphicx}

%% \url{something}
\usepackage{url}

%% \multirow{package}{width}{text}
\usepackage{multirow}

%% \cellcolor
\usepackage[table]{xcolor}

%% \caption{title}
\usepackage{caption}
	\captionsetup{format=hang}

\usepackage{subcaption}

%% \forloop
\usepackage{forloop}

%% fix underfull on footnote with URL.
\usepackage{ragged2e}

%% source code listing
\usepackage{listings}

\lstdefinelanguage{go}
{
	morekeywords={package,import,const,func,for,type,var,struct}
,	sensitive=true
,	morecomment=[l]{//}
,	morecomment=[s]{/*}{*/}
}
\lstdefinestyle{go}{%
	language=go
,	keywordstyle=\color{black}\bfseries
,	commentstyle=\color{gray}
,	breakatwhitespace=true
,	lineskip={-2.5pt}
}

\lstdefinestyle{data}{%
	breakatwhitespace=false
,	breakautoindent=false
,	literate={\,}{}{0\discretionary{,}{}{,}},
}

%% \printindex
\usepackage{makeidx}

%% table of content
\usepackage{tocloft}

%% text emphasis, including strikeout.
\usepackage[normalem]{ulem}

%% mathematics
\usepackage{mathtools}

%% arithmetic
\usepackage{calc}

%% algorithm
\usepackage{algorithm}
\usepackage{algpseudocode}

\makeatletter
	\renewcommand{\ALG@name}{Algoritma}
\makeatother

%% multiple columns
\usepackage{multicol}

%% Package for changin page margin
\usepackage[a4paper]{geometry}

%%
\usepackage{parskip}

%% Package for reading CSV to database.
\usepackage{datatool}

%% Package for scatter and line plot.
\usepackage{dataplot}

%% Long table
\usepackage{longtable}

%% Tikz
\usepackage{tikz}
\usetikzlibrary{backgrounds, shapes.geometric, positioning, patterns, external}
	\tikzexternalize

\usepackage{pgfplots}

%% Font: MnSymbol
\usepackage{MnSymbol}

%% Compact list.
\usepackage{paralist}

%% Line spacing
\usepackage{setspace}

%% For changing internal packages.
\usepackage{xpatch}

%% Create clickable TOC
\usepackage[hidelinks]{hyperref}
\hypersetup{
	colorlinks
,	allcolors=black
}

\hyphenation{
	Be-ri-kut
	Ja-nu-a-ri
	SIGKDD
	Wiki-pedia
	a-kan
	a-ku-ra-si
	ang-ka
	ba-gai-ma-na
	bayes-ian
	ber-gu-na
	ber-kas
	ber-ma-sa-lah
	ber-mak-na
	bi-a-ya
	da-lam
	data-set
	de-ngan
	di-ha-sil-kan
	di-pi-lih
	di-sing-kat
	di-tam-bah-kan
	dis-krit
	fung-si
	ga-bung-an
	ke-las
	ke-le-mah-an
	ke-mung-ki-nan
	ke-tak-se-imbang-an
	lan-guage
	ma-yo-ri-tas
	me-laku-kan
	me-me-rik-sa
	me-mi-lih
	me-ne-rap-kan
	meng-a-pli-ka-si-kan
	meng-ge-ne-ra-li-sa-si
	me-ning-kat-kan
	me-nye-dia-kan
	me-nye-im-bang-kan
	me-sin
	me-thod
	me-to-de
	mem-vi-sua-li-sa-si
	meng-gu-na-kan
	meng-hi-lang-kan
	meng-hu-bung-kan
	meng-i-kut-kan
	meng-i-kuti
	meng-im-ple-men-ta-si-kan
	meng-in-di-ka-si-kan
	mi-sal-nya
	mung-kin
	o-ver-sam-pling
	pa-ra-lel
	pe-la-ti-han
	pe-mi-sah
	pe-nan-da
	pe-ne-li-ti-an
	pe-nu-li-san
	pe-nyun-ting
	pem-ban-ding
	pen-de-kat-an
	peng-kla-si-fi-ka-si
	peng-a-pli-ka-si-an
	per-for-man-si-nya
	po-ten-si-al
	pro-ba-bi-li-tas
	pro-ses
	sam-pel
	se-im-bang
	se-jum-lah
	sun-ting-an
	ting-kat
	un-der-sam-pling
	wa-lau-pun
}

\newcommand{\mytitle}{Deteksi Vandalisme pada Wikipedia Bahasa Inggris menggunakan klasifikasi Cascaded Random Forest}
\newcommand{\myname}{Muhamad Sulhan}
\newcommand{\mysid}{23513014}
\newcommand{\myadvisorname}{Dwi Hendratmo Widyantoro}
\newcommand{\myadvisorid}{196812071994021001}
\newcommand{\mydept}{Program Studi Magister Informatika}
\newcommand{\itb}{Institut Teknologi Bandung}

\newcommand{\daftarisi}{DAFTAR ISI}
\newcommand{\daftargambar}{DAFTAR GAMBAR DAN ILUSTRASI}
\newcommand{\daftartabel}{DAFTAR TABEL}
\newcommand{\tUpAbstrak}{ABSTRAK}
\newcommand{\tDaftarPustaka}{DAFTAR PUSTAKA}
\newcommand{\tLampiran}{Lampiran}
\newcommand{\tUpLampiran}{\uppercase{\tLampiran}}

%%% My images directory
\graphicspath{{../images/}}
\newcommand{\myitbcover}{ITB-logo-hitam}
\newcommand{\myitbcoverblue}{ITB-logo-ganesha}

%%% two column signature.
\def\myadvisorsig#1{%
	\vbox{\hsize=6cm
		\textbf{#1}\\
		\addvspace{2cm}%
		\hbox to \hsize{%
			\strut\hfil%
			\myadvisorname%
			\hfil%
		}
		\hrule\kern1ex
		\hbox to \hsize{%
			\strut\hfil%
			NIP\hspace{1ex}\myadvisorid%
			\hfil%
		}
	}
}

%%% one column signature.
\def\mysignature#1#2#3{%
	\vbox{
		\textbf{#1}\\
		\addvspace{2cm}%
		\hbox to \hsize{%
			\strut\hfil%
			{#2}%
			\hfil%
		}
		\makebox[6cm][c]{
			\hrulefill
		}
		\hbox to \hsize{%
			\strut\hfil%
			NIP\hspace{1ex}{#3}%
			\hfil%
		}
	}
}

%%% source code listing
\newcommand{\includecodego}[2][c]{
	\lstinputlisting[caption=#2,escapechar=,style=go]
		{/home/ms/go/work/src/github.com/shuLhan/#2}
}

%%% data listing
\newcommand{\includedata}[2][c]{
	\lstinputlisting[caption=#2,style=data,linerange={1-10}]
		{/home/ms/go/work/src/github.com/shuLhan/#2}
}

%% Caption for algorithm
\DeclareCaptionFormat{algor}{%
	\hrulefill
	\par
	\offinterlineskip
	\vskip1pt
	\textbf{#1#2}#3
	\offinterlineskip
	\hrulefill
}

\DeclareCaptionStyle{algori}{singlelinecheck=off,format=algor,labelsep=space}
\captionsetup[algorithm]{style=algori}

%%
%% This contain formatting for ITB Thesis.
%%

%%
%% Set page margin to 4cm left, 3cm on top, right, and bottom.
%%
\geometry{
	a4paper,
	top=3cm,
	right=3cm,
	bottom=3cm,
	left=4cm
}

%%
%% Set space between paragraphs to 1.5.
%%
\setlength{\parskip}{1.5em}

%%
%% Add dot to TOC.
%%
\renewcommand{\cftchapleader}{\cftdotfill{\cftdotsep}}
\renewcommand{\cftsecleader}{\cftdotfill{\cftdotsep}}
\renewcommand{\contentsname}{}

%%
%% Reduce vspace between title in list and body
%%
\renewcommand{\contentsname}{
	\chapter*{\vspace{2cm}\daftarisi}\label{chapter:daftarisi}
	\vspace{-1.5em}
	\addcontentsline{toc}{chapter}{\daftarisi}
}

\newcommand{\listappendixname}{
	\chapter*{\vspace{2cm} \tupdaftarlampiran}
	\vspace{-1.5em}
	\addcontentsline{toc}{chapter}{\tupdaftarlampiran}
}

\renewcommand{\listfigurename}{
	\chapter*{\vspace{2cm} \daftargambar}
	\vspace{-1.5em}
	\addcontentsline{toc}{chapter}{\daftargambar}
}

\renewcommand{\listtablename}{
	\chapter*{\vspace{2cm} \daftartabel}
	\vspace{-1.5em}
	\addcontentsline{toc}{chapter}{\daftartabel}
}

%%
%% Source code listing style.
%%
\lstset{
	basicstyle=\scriptsize\ttfamily
,	breaklines=true
,	stringstyle=\scriptsize\ttfamily
,	numbers=left
,	numberstyle=\tiny\ttfamily
,	numbersep=5pt
,	tabsize=4
,	frame=single
}

\makeatletter
\def\lst@lettertrue{\let\lst@ifletter\iffalse}
\makeatother

%%
%% Format chapter and section.
%%

\setlength{\cftchapnumwidth}{5em}
\setlength{\cftsecnumwidth}{2.5em}

\setlength{\cftsecindent}{5em}
\setlength{\cftsubsecindent}{7.5em}

\renewcommand{\cftchappresnum}{Bab\space}

%% Set chapter name to Bab.
\titleformat{\chapter}[hang]
{\bfseries\large\centering}
{Bab \thechapter}
{1em}
{}

%% Set section font to normal.
\titleformat{\section}
{\normalfont\bfseries}
{\thesection}
{1em}{}

%% Set subsection font to normal.
\titleformat{\subsection}
{\normalfont\bfseries}
{\thesubsection}
{1em}{}

%% Set spacing for chapter, section, and subsection.
\titlespacing*{\chapter}{0pt}{-2.5em}{1.5em}
\titlespacing{\section}{0pt}{0pt}{0em}
\titlespacing{\subsection}{0pt}{0pt}{0em}

%% Set roman on chapter number.
\def\thechapter{\Roman{chapter}}

%% Alter latex default title on table and figure.
\captionsetup[table]{name=Tabel}
\captionsetup[figure]{name=Gambar}

%%
\renewcommand{\arraystretch}{1.5}
\setlength{\tabcolsep}{3pt}

%% Change bibliography title.
\defbibheading{bibliography}{\centerline{
	\textbf{DAFTAR PUSTAKA}}
}

%% Algorithmicx
\makeatletter
\renewcommand{\ALG@beginalgorithmic}{\footnotesize}
\makeatother

%% My bibligraphy file
\addbibresource{bibliography.bib}

%% multicolumn setting
\setlength{\columnsep}{1cm}

%% uncomment this to show overrule in black box
\overfullrule=2cm

%% The Glory Appendices

\newlistof{appendix}{app}{\listappendixname}
\setcounter{appdepth}{2}
\renewcommand{\theappendix}{\Alph{appendix}}
\renewcommand{\cftappendixpresnum}{Lampiran\space}
\setlength{\cftbeforeappendixskip}{0em}
\setlength{\cftappendixnumwidth}{6em}
\newlistentry[appendix]{subappendix}{app}{1}
\renewcommand{\thesubappendix}{\theappendix.\arabic{subappendix}}
\setlength{\cftsubappendixindent}{6em}

\newcommand{\myappendix}[1]{%
	\refstepcounter{appendix}%
	\chapter*{Lampiran\space\theappendix\space #1}%
	\addcontentsline{app}{appendix}{\protect\numberline{\theappendix}#1}%
	\par
}

\newcommand{\subappendix}[1]{%
	\refstepcounter{subappendix}%
	\section*{\thesubappendix\space #1}%
	\addcontentsline{app}{subappendix}{\protect\numberline{\thesubappendix}#1}%
}

%% Format TOC
\makeatletter
	\newskip\old@cftbeforechapskip
	\old@cftbeforechapskip\cftbeforechapskip
	\let\old@l@chapter\l@chapter
	\let\old@l@section\l@section
	\def\l@chapter#1#2{\old@l@chapter{#1}{#2}\cftbeforechapskip2pt}% set your value here
	\def\l@section#1#2{\old@l@section{#1}{#2}\cftbeforechapskip\old@cftbeforechapskip}
\makeatother


\author{\myname}
\title{\judul}


\begin{document}

\title{\mytitle}

\author{
	\IEEEauthorblockN{
		\myname\IEEEauthorrefmark{1}
		and
		\myadvisorshortname\IEEEauthorrefmark{2}
	}
	\IEEEauthorblockA{
		\myschool\\
		\itb\\
		\itbaddress\\
		Email:
		\IEEEauthorrefmark{1}ms@students.itb.ac.id,
		\IEEEauthorrefmark{2}dwi@stei.itb.ac.id
	}
}

\maketitle


\begin{abstract}
Wikipedia.org is an online encyclopedia which can be edited by anyone.
This feature makes the article in Wikipedia rapidly
increased in size and can be fixed subsequently, but also makes it prone
to vandalism in the forms of invalid information, deletion, ads, or meaningless
content.
This paper propose a framework for detecting vandalism on English Wikipedia
using machine learning technique by training Cascaded Random Forest (CRF)
classifier on PAN Wikipedia Vandalism Corpus 2010 (PAN-WVC-10) English dataset
that has been resampled using Local Neighbourhood Synthetic Minority
Oversampling Technique (LNSMOTE).
These two techniques then compared with Random Forest (RF) for classifier and
Synthetic Minority Oversampling Technique (SMOTE) for resampling.
The result of classifiers that has been tested on PAN Wikipedia Vandalism Corpus
2011 (PAN-WVC-11) English dataset
showed that dataset resampled using LNSMOTE increase the true-positive rate
(TPR) better than SMOTE in both classifiers.
CRF on SMOTE with 200 stages and 1 tree gave the better result among others
with TPR value 0.9904.
From training computation time, CRF 1.6 times faster than RF in resampled
dataset.
\end{abstract}


\section{Introduction}
\label{section:introduction}
	Vandalism, according to Merriam-Webster's dictionary is willful or malicious
destruction or defacement of public or private property.
In the context of Wikipedia, vandalism can be in the form of malicious edit
which intends to give wrong information, hiding information by deleting
the content, abusive content, ads, and/or meaningless text.
Finding and fixing the article that has been vandalized can disrupt the editor
from writing or expanding new article or other important tasks, and for the
reader they will get the wrong information or no information at all due to
deletion.

Corpus that commonly used for learning vandalism on Wikipedia is PAN Wikipedia
Vandalism Corpus 2010 (PAN-WVC-10)
\cite{potthast:2010b}
or PAN Wikipedia Vandalism Corpus 2011 (PAN-WVC-11)
\cite{potthast:2010b}.
Both of the corpus have class imbalance problem.
PAN-WVC-10 for an English articles have 32,439 sample with only $2,394$ (7.38\%)
of them is vandalism, whereas PAN-WVC-11 for an English articles have $9,985$
sample with only $1,144$ (11.45\%) of them are vandalism.

Training a classifier on imbalance dataset could lead to low performance.
This can be caused either by the minority class has low contribution to error
rate, which makes the classifier bias to majority class, or some classifier
assume that class distribution is balanced, while in real world cases this
rarely happened.

Random Forest (RF) has the disadvantages in their computation time
especially when training the classification model.  For a large dataset with
more than 10,000 samples (like the PAN-WVC-10 cases) this could lead to hours
of training time.  One of the solution is by using Cascaded Random Forest (CRF)
framework proposed by Bauman et al. \cite{baumann2013cascaded}.
Their paper state that CRF give a fast training model time and increased
performance compared to RF.

This paper attempts to overcome the dataset imbalance problem on PAN-WVC-10 by
applying resample and classifier technique that has never been used before on
the dataset.
The PAN-WVC-10 dataset is resampled using Local Neighborhood SMOTE (LNSMOTE)
technique,
which proposed by Maciejewski and Stefanowski
\cite{maciejewski2011local}.
The result from resampling then trained using CRF classifier and compared with
RF classifier to see their performance.

This paper is organized as follow.
Section \ref{section:related_works} briefly review past researches on vandalism
detection on Wikipedia.
Section \ref{section:literature_study} review the LNSMOTE and CRF techniques.
Section \ref{section:research_methodology} describes the process to generate
features from raw dataset and resampling process before it was trained and
tested on classifiers.
Section \ref{section:result_and_analysis} shows the result from each classifier
on each dataset and their analysis.
Section \ref{section:conclusion} concludes the experiments and
give some possible future works that can be extended from this paper.

\IEEEpubidadjcol



\section{Related Works}
	Vandalism detection in Wikipedia based on machine learning approach
became an interesting research topic since 2008.
Potthast et al. \cite{potthast2008automatic} contribution is the first vandalism
detection approach using machine learning with textual and basic
metadata features using Logistic Regression classifier.
Smets et al. \cite{smets08automaticvandalism} use Naive Bayes classifier on
selected words that representing vandalism edit and the first who use
compression model to detect vandalism in Wikipedia.
Itakura and Clarke \cite{itakura2009using} use Dynamic Markov Compression to
detect vandalism edit in Wikipedia.
Mola Velasco \cite{mola2012wikipedia} extend the Potthast research by adding
more textual features and word-list features.
West and Lee \cite{west2011multilingual} use spatial and temporal metadata
without required to check the text in the article and revision, and the first
to introduce \textit{ex post facto} data as feature, where prediction
take the next revision into consideration.
Adler et al. \cite{adler2011wikipedia} build a vandalism detection system using
reputation called WikiTrust and then combine it with
natural language, spatial and temporal features.
Harpalani et al. \cite{harpalani2011language} propose that vandalism has
a unique and equal lingustic property.
They build a system for detecting vandalism based on \textit{stylometric}
analysis from vandalism edit with \textit{context-free grammar} probabilistic
model.
Following the trend of classifying on cross-language vandalism, Tran and
Christen \cite{tran2013cross} evaluated several classifier based on
language feature collected from number of article viewed every hours and the
history of their edit in Wikipedia.

Gotze \cite{gotze2014advanced} combine features from
\cite{potthast2008automatic},
\cite{mola2012wikipedia},
\cite{west2011multilingual},
\cite{adler2011wikipedia},
\cite{javanmardi2011vandalism},
and
\cite{wang2010got},
with four additional and modified features.
Gotze use the random oversampling technique, to overcome imbalance problem,
called
\textit{Synthetic Minority Over-sampling TEchnique} (SMOTE)
proposed by Chawla et al.
\cite{chawla2002smote}.
Original and resampled dataset of PAN-WVC-10 then tested with two-class
classifier:
\textit{Logistic Regression},
\textit{RealAdaBoost},
\textit{Random Forest} (RF), and
\textit{Bayesian Network}.
His evaluation on original dataset showed that RF give better result than
others classifiers.
The result from resampling dataset showed increasing in performance on all
classifiers except RF.

From the previous research, seven of them
(
\cite{mola2012wikipedia}
\cite{west2011multilingual}
\cite{adler2011wikipedia}
\cite{harpalani2011language}
\cite{gotze2014advanced}
\cite{wang2010got}
\cite{adler2010detecting}
)
use PAN-WVC-10,
with the best precision value is $0.86$ and recall value is $0.57$, which
obtained by Velasco using RF without resampling on dataset.
Only two research
(
\cite{west2011multilingual}
\cite{gotze2014advanced}
)
that use PAN-WVC-11
with the best result obtained by Gotze, $0.92$ on precision and $0.39$
on recall.



\section{Literature Study}
	\label{section:literature_study}

This section review the previous work and techniques used for implementation in
this paper.

\subsection{LNSMOTE}
	\label{subsection:lnsmote}
	SMOTE method has several weakness.
First, all sample from minority class is used, this could be a problem because
not all the samples have equal benefit for learning.
Minority sample in the boundary region between minority and majority class
usually result in misclassified rather than sample located in the center of
region, while sample that located in the center may give a little contribution
to classifier.
One of the method to overcome this problem is by using sample in boundary of
minority class instead in the center which proposed by Han et al.
\cite{han2005borderline}, called Borderline-SMOTE.

Another weakness of SMOTE method is overgeneralization, where their method does
not take into consideration the distribution of minority sample in majority
class, or the outliers.
Maciejewski and Stefanowski \cite{maciejewski2011local}
introduced an extension of SMOTE called Local Neighbourhood
SMOTE (LNSMOTE) \cite{maciejewski2011local} by combining Borderline-SMOTE
with modified version of Safe-Level SMOTE (SL-SMOTE)
\cite{bunkhumpornpat2009safe}.

In SL-SMOTE, majority samples is taken into consideration before creating
synthetic sample by calculating a coefficient called safe-level.
For each minority sample, count the number of their $k$ nearest neighbours (KNN).
If KNN value is close to 0, then the sample will be considered as noise.
If KNN value is close to $k$, then the sample can be said in safe region in
minority class.
The main idea was to create synthetic sample that close to safe region.

The SL-SMOTE strategy has a problem especially when class distribution is bias
in which the minority class spread into small sub-region with low number
cardinality.
In this situation, creating synthetic sample with SLMOTE will cause an overlap
between class.
This problem is due to SL-SMOTE find only KNN for minority class.
If the sample candidate does not located in region with densed minority class,
then some of their neighbours could be far from sample candidate or surrounded
by majority class samples.
LNSMOTE overcome this overlap problem by taking into consideration the local
neighbourhood of minority sample candidate that can provide the number of
majority class around each of them.


\subsection{Cascaded Random Forest}
	\label{subsection:crf}
	Most of ensemble learning algorithm is incapable of handling imbalance training
data.
Inequality between positive and negative class usually result in low
detection accuracy.
A simulation run by Strobel et al. \cite{strobl2007bias} showed that RF skewed
in favor of majority class.
Another drawback of RF is after learning several trees, RF gradually reach its
peak, such that the classifier can not increase their detection sensitivity or
decrease their false-positive rate.

Viola and Jones proposed a detection algorithm based on AdaBoost with
cascade structure \cite{viola2004robust}.
Cascade structure motivated by assumption that it is more easy to reject a
negative sample than finding a positive one.
Viola and Jones combines several strong classifiers in several independent
stages with condition that each stage can reject a sample, so to classify a
sample as positive then all stages must be passed.
Due to rejection on early stages, computation time will be decreased.
In addition, to get better training result, Viola and Jones propose a bootstrap
strategy by deleting samples classified as true negative.
The reduced training set then refilled with sample that mis-classified, or
false-positive samples
\cite{viola2004robust}.

A cascade classifier consists of several number of stages with increasing
complexity.
Each stage have minimum one independent classifier.
Classifier added into stages until the value of true-positive and true-negative
threshold is reached.
The advantage of cascade structure is a vast number of samples can be
distributed between stages, decreasing false-positive value and shortening
computation time when training and classifying.

Baumann uses this method with RF and propose Cascaded Random Forest (CRF) which
is a combination of RF classifier with cascade structure, where in each stage
several decision tree is build with bootstrap strategy, this leads increased
learning on positive sample and the drawback of imbalanced dataset can be
avoided.
\cite{baumann2013cascaded}

CRF has six parameters, three of them shared with RF which are number of tree
($T$), percetage of bootstrap ($b$), and number of random features ($m$).
Another three parameters are number of stages ($S$), threshold
for true-positive ($maxtp$) and threshold for true-negative ($maxtn$).

The bootstrap strategy proceeds as follows: after training in each stage, the
negative test set which contains only negative samples than tested on all
previous stages in order to delete the true-negative samples from
negative test set.
Samples that classified as false-positive then moved to negative test set to be
learned later in the next stage.

Some of stage have low accuracy value than other stages.
To decrease the influence of stage with low performance, calculate the weith
factor $\alpha$ for each stage by exploiting the harmonic means of $precision$
and $recall$ on training set or also known as $F_1$ (F-Measure).a
The $\alpha$ value for each stage linearly denormalized in range of 0 to 1, so
that the weight of low performance stage reduced to make their contribution to
majority voting also decreased.

The formula to get the classification result from CRF given in picture
\ref{form:crf}.

\begin{figure}[h]
\[
	y(x) = argmax \left(
			\frac{1}{T \cdot \sum^{S}_{s=1} \alpha_{s} }
			\sum\limits_{s=1}^{S} \alpha_{s}
			\sum\limits^{T}_{t=1} I_{h_{t} (x) = c}
		\right)
\]
\caption{CRF classifier with weight.}
\label{form:crf}
\end{figure}

$x$ is a sampel to be classified,
$S$ is a number of stage in cascade structure,
$\alpha_{s}$ is the weight value for each stage,
$T$ is number of tree in each stage, and
$h_{t}$ is classification function from the tree which give class value $c$
from an indicator $I$ (e.g. a value 1 for positive or 0 for negative).



\section{Vandalism Detection Process}
	\label{section:research_methodology}

This section describe the process to generate the feature dataset and
implementation of resampling and classifiers, started from data preparation,
generating features, resampling features dataset, and implementation so it can
be used on training, testing, and analysis.
Each step on implementation process is illustrated as in figure
\ref{fig:proses}.

\begin{figure}[tp!]
\centering
\resizebox{0.3\textwidth}{!} {
\mytikzinput{diagramprocess}
\begin{tikzpicture}[
	framed,
	nodes = {
		align = center
	},
	lines/.style={
		line width=2pt,
		>=latex
	},
	data/.style={
		circle,
		draw=black,
		text centered
	},
	proc/.style={
		rectangle,
		draw=black,
		text centered
	}
]
	\node[data] (training_raw) {
		Raw training set\\
		(\textit{PAN-WVC-10})
	};

	\node[proc] (wvcgen) [below=of training_raw] {
		Features\\
		extraction
	};

	\node[proc] (trainingset) [below=of wvcgen] {
		Resampling
	};

	\node[proc] (c)  [below=of trainingset] {
		Training
	};

	\node[proc] (m)  [below=of c] {
		Vandalism detection\\
		model
	};

	\node[data] (o)  [below=of m] {
		Classification\\
		result
	};
	%%
	\node[data] (testset_raw) [right=of training_raw] {
		Raw test set\\
		(\textit{PAN-WVC-11})
	};

	\node[proc] (testset_wvcgen) [below=of testset_raw] {
		Features\\
		extraction
	};

	\node[data] (testset) [below=of testset_wvcgen] {
		Test set
	};

	\draw[lines,->] (training_raw) -- (wvcgen);
	\draw[lines,->] (wvcgen) -- (trainingset);
	\draw[lines,->] (trainingset) -- (c);
	\draw[lines,->] (c) -- (m);
	\draw[lines,->] (m) -- (o);

	\draw[lines,->] (testset_raw) -- (testset_wvcgen);
	\draw[lines,->] (testset_wvcgen) -- (testset);

	\draw[lines,->] (testset) |- (m);
\end{tikzpicture}
}
\caption{
	Workflow process on detecting vandalism
}
\label{fig:proses}
\end{figure}


\subsection{Data Preparation}
	\label{subsection:data_preparation}
	The original dataset can not be used for training and testing, they need to
combined, cleaned by removing unneeded attributes, and cleaned on their revision
to generate features.

Dataset that is used for training is PAN-WVC-10 \cite{potthast2008automatic}.
which contain two separate set, the edit set and annotation set.
The two set then combined to get only their edit ID, class, old revision ID,
new revision ID, edit time, editor, article title, and edit comment.

Dataset for testing is an English dataset of PAN-WVC-11 \cite{potthast:2010b}.
The original attribute from the set is similar with PAN-WVC-10 except they were
already combined into single set.

PAN-WVC-10 and PAN-WVC-11 contain revision files.
Revision is history of edit that contain the current text in article based on
edit ID, where each ID in dataset reference to one revision file.

In both dataset, we then add two new attributes, deletions
(text that has been deleted in previous revision) and additions (text that
has been added in new revision).
Also, the class attribute value is replaced with numeric, where
"vandalism" become "1" and "regular" become "0".

The next step is to create revision text that is clean from wiki syntax, with
an aim to help in generating feature.
Every revision files cleaned up by removing URI, wiki markups, and wiki tokens.


\subsection{Extracting Features}
	Previous papers group the features into three categories which are metadata,
text, and language.
This paper use four metadata features, 11 text features, and 10
language features which has been used and analyzed in Mola-Velasco paper
\cite{mola2012wikipedia}.

All previous feature then implemented in a program.
The program then executed on PAN-WVC-10 and PAN-WVC-11 set that has been
prepared in section \ref{subsection:data_preparation}, which generate features
dataset contain continous values.
The implementation for combining, cleaning, and generating the features is
published as open source software as \texttt{wvcgen}
\footnote{\url{https://github.com/shuLhan/wvcgen}}.


\subsubsection{Metadata Features}
	Metadata feature references to the properties of revision which can be directly
taken, for example, identity of editor, comment, or the size of changes.
Below is list of metadata features,
\begin{itemize}
\item \textbf{Anonymous}. If the editor of revision is registered user then in
the edit set it contain their username.
\item \textbf{Comment length}. Counting number of character that left by editor
in comment field.
\item \textbf{Size increment}. An absolute size of new revision. Higher size
value can indicated as deletion on whole article.
\item \textbf{Size ratio}. The size of new revision divided by the size of old
revision.
\end{itemize}


\subsubsection{Text Features}
	Below is the list of text features that are used.

\begin{itemize}
\item \textbf{Ratio of lowercase and uppercase character}. The ratio is
computed in new revision by dividing number of uppercase characters with
lowercase characters.
\item \textbf{Ratio of uppercase to all characters}. Computed by dividing
number of uppercase characters with number of all characters in new revision.
\item \textbf{Digit ratio}. Computed by dividing number of digit with number of
all character in new revision.
\item \textbf{Ratio of non-alphanumeric characters}. Computed by dividing
number of all non-alphanumeric characters with number of all characters in new
revision.
\item \textbf{Character diversity}. Computed by counting number of unique
character divided by length of text in new revision.
\item \textbf{Character distribution}. Computed using Kullback-Leibler
divergence on old revision compared with text addition in new revision.
\item \textbf{Compression rate}. Computed by applying LZW compression algorithm
on inserted text.
\item \textbf{Good token}. Computed by counting number of token that vandal
rarely used on text, for example wiki syntax.
\item \textbf{Term frequencies}. Computed by counting number of unique words
added divided by number of unique words in new revision.
\item \textbf{Longest word}. Computed by counting number of character on the
longest word inserted.
\item \textbf{Length of similar character}. Computed by counting number of
similar characters used in sequence in single word, for example
\textit{aaarrrgghhhh}, \textit{sooo huge}.
\end{itemize}


\subsubsection{Language Features}
	Language features based on number of particular words that are inserted in new
revision.
For each word categories, there are two feature to be computed: frequency and
impact.
Frequency feature computed by counting number of words in category divided by
total number of words in new revision.
Impact feature computed by counting percentage of word usage in new revision
divided by total words in old and new revision.
Below is list of language features that are used.

\begin{itemize}
\item \textbf{Vulgarism}. Counting number of vulgar, harsh, and offensive
words.
\item \textbf{Subject}. Counting number of first or second person
words used in inserted text, including colloquial words, for example
\textit{I}, \textit{you}.
\item \textbf{Bias}. Counting number of bias words, for example "coolest",
"huge".
\item \textbf{Pornography}. Counting number of pornography related words
inserted in new revision.
\item \textbf{Bad words}. Counting number of bad, non-vulgar words, usually
indicated by bad writing. For example "wanna", "gotcha".
\item \textbf{All word categories}. Combination of all word categories.
\end{itemize}


\subsection{Resampling Dataset}
	PAN-WVC-10 without resample contain 2,394 positive or vandalism samples and
30,045 negative or regular samples.
To get a balanced class, the dataset then resampled using SMOTE and LNSMOTE for
positive class.
For SMOTE, the parameter used for resampling is 1,200\% and parameter for
K-Nearest-Neighbour (KNN) is 5, which output 28,728 synthetic samples, total of
positive sample combined with original sample result in 31,122 positive
samples.
Parameter for resampling using LNSMOTE is similar with SMOTE, which generate
28,588 positive synthetic samples, in total of 30,892 positive samples.
The implementation for SMOTE and LNSMOTE published as open source software
\footnote{\url{https://github.com/shuLhan/go-mining/tree/master/resampling}}.



\subsection{Classifier Implementations}

Implementation of classifiers carried out gradually.
Started by implementing CART which is used in Random Forest, which in turn is
used in Cascaded Random Forest.
The implementation of CART based on Jiawei Han et al. book, chapter 8
\cite{han2011data}.
The implementation of Random Forest is based on original paper of Breiman
\cite{breiman2001random}, plus additional resource from internet.
The implementation of Cascaded Random Forest is based on original paper of
Baumann et al.
\cite{baumann2013cascaded}.
The result of all implementation is published as open source software to help
others in future research or for real-world usage.
\footnote{\url{https://github.com/shuLhan/go-mining/tree/master/classifiers}}.


\subsection{Training and Testing}

There are three common parameter between RF and CRF which are 200 for number of
tree, 5 for number of random features, and 64\% for percentage of
bootstrapping.
For consistency, their value are constant between training.
For CRF classifier, three separated testing will be conducted using different
parameter for number of stage and number of tree which are 200 stages with 1
tree, 100 stages with 2 trees, and 50 stages with 4 trees; all of them have
equal total number of trees.
This is an experiment to see the effect of number of trees to stage and their
performance.
Another parameter for training with CRF are thresholds for true-positive rate
(TPR) and true-negative rate (TNR), which set to constant value 0.95 and 0.95
for all training.

\begin{table}[tp]
\caption{Dataset for training and testing}
\label{table:dataset}
\centering
\begin{tabular}{c | l | c | c | c}
\hline
\multirow{2}{*}{Type} & \multirow{2}{*}{Resampling mode}
	& \multicolumn{3}{c}{Number of samples} \\
\cline{3-5}
    & & Vandalism & Regular & Total \\
\hline
\hline
\multirow{3}{*}{Training} & -       &  2.394 & 30.045 & 32.439 \\
                              & SMOTE   & 28.728 & 30.045 & 58.773 \\
                              & LNSMOTE & 28.588 & 30.045 & 58.633 \\
\hline
Testing & - & 1.143 & 8.842 & 9.985 \\
\hline
\end{tabular}
\end{table}


The dataset used for training is PAN-WVC-10 which consist of three different
set, dataset without resampled, dataset resampled with SMOTE, and dataset
resampled with LNSMOTE.
The dataset used for testing is PAN-WVC-11 which contain 1,143 positive
samples and 8,842 negative samples, in total of 9985 samples.

Training is conducted by running each classifier program, RF and CRF, on
those three different PAN-WVC-10 feature dataset.
Testing is conducted after the model has been built by giving the model the
PAN-WVC-11 feature dataset as an input.

The environment used for training and testing is Intel\textregistered
Core\texttrademark i7-4750HQ CPU 2,00 GHz, with total 8 GB of RAM.
Each training is done separatedly to avoid cache miss which affect computation
time.


\section{Evaluation}
	\label{section:result_and_analysis}
	When detecting vandalism, it is better to receive false positive than missing
vandalism edit.
Wrong classification, when regular edit detected as vandalism, will give no
effect on reader, but missing on vandalism edit can lead to information loss,
fault information, or disturb the reader.
For this reason, our evaluation based on true-positive rate of classifier
performance.
True-positive rate (TPR) or recall is number of actual positive samples divided
by sum of sample positive classified as positive and sample positive classified
as negative (false-positive).
TPR has a value between 0 and 1, with value approaching to 0 indicate poor
performance and value approaching 1 indicate good performance.
Result of testing are given in terms of performance of each classifier on table
\ref{tab:stats} and training computation time on figure \ref{graph:runtimes}.

Result from CRF classifer on LNSMOTE with 200 stages and 1 tree have the
highest TPR value $0.9904$.
RF classifier without resampling gave the lowest TPR $0,1654$.

From the computation time, CRF classifier is faster than RF on all training.
Using RF and CRF with 50 stages and 4 trees as comparison, CRF without
resampling is 11 times faster than RF, and for dataset that has been resampled
with SMOTE and LNSMOTE, CRF is 1.6 times faster than RF.

	\DTLsetseparator{;}
\DTLloaddb{stats}{./stats.csv}
\DTLmaxforcolumn{stats}{TPR}{\maxtpr}
\DTLminforcolumn{stats}{FPR}{\minfpr}
\DTLmaxforcolumn{stats}{TNR}{\maxtnr}
\DTLmaxforcolumn{stats}{Presisi}{\maxprec}
\DTLmaxforcolumn{stats}{F-Measure}{\maxfm}
\DTLmaxforcolumn{stats}{Akurasi}{\maxacc}
\DTLmaxforcolumn{stats}{AUC}{\maxauc}

\begin{table}[bp]
\caption{Performance of Random Forest and Cascaded Random Forest}
\centering
\begin{tabular}{llrrrrrrr}
\hline
\textbf{Classifier} &
\textbf{Dataset} &
\textbf{TPR}
\DTLforeach*{stats}{%
	\cl=Klasifikasi,%
	\ds=Dataset,%
	\tpr=TPR%
}{%
	\DTLifnullorempty{\cl}
		{\\ \cline{2-3}}
		{\\ \hline \hline}
	\DTLifnullorempty{\cl}
		{}
		{
			\multirow{3}{2cm}{\cl}
		}
	& \ds
	& \DTLifnumeq{\tpr}{\maxtpr}{\textbf{\tpr}}{\tpr}
}
\\
\hline
\end{tabular}
\label{tab:stats}
\end{table}


	\begin{figure}[htbp]
\centering
\begin{tikzpicture}[font=\scriptsize]
	\begin{axis}[
		width=7cm,
		ymax=250,
		ybar,
		ylabel=Running time (minutes),
		symbolic x coords={Without resampling, SMOTE, LNSMOTE},
		xtick=data,
		nodes near coords,
		every node near coord/.append style={font=\tiny},
		enlarge x limits=0.24,
		enlarge y limits=0,
		legend style={
			at={(0.5,-0.15)},
			anchor=north,
			legend columns=-1
		},
	]
		%% RF
		\addplot[pattern = dots] coordinates {
			(Without resampling, 52.5)
			(SMOTE, 217.7)
			(LNSMOTE, 199.5)
		};

		%% CRF-200-1
		\addplot[pattern=north east lines] coordinates {
			(Without resampling, 3.5)
			(SMOTE, 74.4)
			(LNSMOTE, 61.9)
		};

		%% CRF-100-2
		\addplot[pattern=horizontal lines] coordinates {
			(Without resampling, 3.6)
			(SMOTE, 74.6)
			(LNSMOTE, 67.9)
		};

		%% CRF-50-4
		\addplot coordinates {
			(Without resampling, 4.2)
			(SMOTE, 82.5)
			(LNSMOTE, 76.6)
		};
	\legend{RF, CRF-200-1, CRF-100-2, CRF-50-4}
	\end{axis}
\end{tikzpicture}
\caption{Running time for each classifer on different dataset.}
\label{graph:runtimes}
\end{figure}



\section{Conclusion}
\label{section:conclusion}

On average SMOTE increase TPR value by $0.19$ times while LNSMOTE, on average
increase TPR value $0.33$ times.
Another interisting effect of CRF classifier, when using less number of tree on
each stage on dataset without resampling, their performance almost similar with
CRF on resampling with more number of tree, for example performance of CRF with
100 stages and 2 tress on dataset without resampling is adjacent with CRF with
50 stages and 4 tress on dataset resampled with SMOTE.

The best classifier model for vandalism without resampling returned by CRF with
200 stages and 1 tree with TPR value $0.9668$.
The best classifier model for dataset that has been resampled with SMOTE is CRF
with 200 stage and 1 trees with TPR value $0.979$
The best classifier model for dataset that has been resampled with LNSMOTE is
CRF with 200 stages and 1 trees with TPR value $0.9904$.
Overall, the best model is CRF with 200 stages and 1 trees on dataset resampled
with LNSMOTE.
Beside their good performance result, CRF on average $1.6$ times faster at
training than RF on resampled dataset.

\section{Contribution}

This paper contribute on finding the best classifier on detecting vandalism on
Wikipedia and evaluating the effect of LNSMOTE resampling on imbalance dataset
and performance of CRF agains RF.
Apart from that, this paper also provides a framework to create and develop
vandalism features from raw PAN WVC dataset without having to build it again
from scratch.
Another contribution is a library for data processing and machine learning,
especially on resampling using LNSMOTE and CRF classifier which has no
open implementation on renowned program like Weka, Scikit-Learn, or R.
This framework can be used in the next research or in real-world application.

\section{Future Works}

All of training model on this paper were using RF and CRF algorithm in serial,
in which each tree is build one by one sequentially or when classifying sample
each of them was given as input to each tree sequentially to get their classes.
Using parallel algorithm, for building trees or classifying samples, can
speeding up training, testing and getting classifier result.
In the domain of machine learning, an interesting new algorithm is eXtreme
Gradient Boostring (XGBoost)
\cite{chen2016xgboost}.
Using XGBoost on Wikipedia vandalism dataset maybe can increase the accuracy
of detection.

\bibliographystyle{IEEEtran}
\bibliography{IEEEabrv,bibliography}

\end{document}
