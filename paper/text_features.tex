Below is the list of text features that are used.

\begin{itemize}
\item \textbf{Ratio of lowercase and uppercase character}. The ratio is
computed in new revision by dividing number of uppercase characters with
lowercase characters.
\item \textbf{Ratio of uppercase to all characters}. Computed by dividing
number of uppercase characters with number of all characters in new revision.
\item \textbf{Digit ratio}. Computed by dividing number of digit with number of
all character in new revision.
\item \textbf{Ratio of non-alphanumeric characters}. Computed by dividing
number of all non-alphanumeric characters with number of all characters in new
revision.
\item \textbf{Character diversity}. Computed by counting number of unique
character divided by length of text in new revision.
\item \textbf{Character distribution}. Computed using Kullback-Leibler
divergence on old revision compared with text addition in new revision.
\item \textbf{Compression rate}. Computed by applying LZW compression algorithm
on inserted text.
\item \textbf{Good token}. Computed by counting number of token that vandal
rarely used on text, for example wiki syntax.
\item \textbf{Term frequencies}. Computed by counting number of unique words
added divided by number of unique words in new revision.
\item \textbf{Longest word}. Computed by counting number of character on the
longest word inserted.
\item \textbf{Length of similar character}. Computed by counting number of
similar characters used in sequence in single word, for example
\textit{aaarrrgghhhh}, \textit{sooo huge}.
\end{itemize}
