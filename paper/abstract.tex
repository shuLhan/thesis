\begin{abstract}
Wikipedia.org is an online encyclopedia which can be edited by anyone.
These feature has benefit, which make the article in Wikipedia rapidly
increased in size and can be fixed subsequently, and their drawbacks was prone
to vandalism in the forms of invalid information, deletion, ads, or meaningless
content.
This paper propose a framework for detecting vandalism on English Wikipedia
using machine learning technique by training Cascaded Random Forest (CRF)
classifier on English Wikipedia dataset (PAN-WVC-10) that has been resampled
using Local Neighbourhood SMOTE (LNSMOTE).
These two methods then compared with Random Forest (RF) for classifier and
SMOTE for resampling.
The result of classifiers that has been tested on PAN-WVC-11 English dataset
showed that dataset resampled using LNSMOTE increase the true-positive rate
better than SMOTE in both classifiers.
CRF on SMOTE with 200 stages and 1 tree gave the better result among others
with TPR value 0.9904.
From training computation time, CRF 1.6 times faster than RF in resampled
dataset.
\end{abstract}
