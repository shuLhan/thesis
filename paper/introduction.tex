Vandalism, according to Merriam-Webster's dictionary is willful or malicious
destruction or defacement of public or private property.
In the context of Wikipedia, vandalism can be in the form of malicious edit
which intends to give wrong information, hiding information by deleting
the content, abusive content, ads, and/or meaningless text.
Finding and fixing the article that has been vandalized can disrupt the editor
from writing or expanding new article or other important tasks, and for the
reader they will get the wrong information or no information at all due to
deletion.

Corpus that commonly used for learning vandalism on Wikipedia is PAN Wikipedia
Vandalism Corpus 2010 (PAN-WVC-10)
\cite{potthast:2010b}
or PAN Wikipedia Vandalism Corpus 2011 (PAN-WVC-11)
\cite{potthast:2010b}.
Both of the corpus have class imbalance problem.
PAN-WVC-10 for an English articles have 32,439 sample with only $2,394$ (7.38\%)
of them is vandalism, whereas PAN-WVC-11 for an English articles have $9,985$
sample with only $1,144$ (11.45\%) of them are vandalism.

Training a classifier on imbalance dataset could lead to low performance.
This can be caused either by the minority class has low contribution to error
rate, which makes the classifier bias to majority class, or some classifier
assume that class distribution is balanced, while in real world cases this
rarely happened.

Random Forest (RF) has the disadvantages in their computation time
especially when training the classification model.  For a large dataset with
more than 10,000 samples (like the PAN-WVC-10 cases) this could lead to hours
of training time.  One of the solution is by using Cascaded Random Forest (CRF)
framework proposed by Bauman et al. \cite{baumann2013cascaded}.
Their paper state that CRF give a fast training model time and increased
performance compared to RF.

This paper attempts to overcome the dataset imbalance problem on PAN-WVC-10 by
applying resample and classifier technique that has never been used before on
the dataset.
The PAN-WVC-10 dataset is resampled using Local Neighborhood SMOTE (LNSMOTE)
technique,
which proposed by Maciejewski and Stefanowski
\cite{maciejewski2011local}.
The result from resampling then trained using CRF classifier and compared with
RF classifier to see their performance.

This paper is organized as follow.
Section \ref{section:related_works} briefly review past researches on vandalism
detection on Wikipedia.
Section \ref{section:literature_study} review the LNSMOTE and CRF techniques.
Section \ref{section:research_methodology} describes the process to generate
features from raw dataset and resampling process before it was trained and
tested on classifiers.
Section \ref{section:result_and_analysis} shows the result from each classifier
on each dataset and their analysis.
Section \ref{section:conclusion} concludes the experiments and
give some possible future works that can be extended from this paper.

\IEEEpubidadjcol
