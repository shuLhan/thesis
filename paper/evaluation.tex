When detecting vandalism, it is better to receive false positive than missing
vandalism edit.
False-negative classification, when regular edit detected as vandal, will
give no effect on reader, but false-positive classification or miss on vandal
edit can lead to information loss, wrong information, or disturb the reader.
For this reason, our evaluation is based on true-positive rate (TPR) of
classifier performance.
TPR or recall is the number of actual positive samples
divided by the sum of positive sample classified as positive and positive
sample classified as negative (false-positive).
TPR has a value between 0 and 1, with value approaching to 0 indicating poor
performance and value approaching 1 indicating good performance.
Result of testing are given in terms of performance of each classifier on table
\ref{tab:stats} and training computation time in
\figurename~\ref{graph:runtimes}.

Result from CRF classifer on LNSMOTE with 200 stages and 1 tree have the
highest TPR value $0.9904$, while
RF classifier without resampling gave the lowest TPR $0,1654$.

	\DTLsetseparator{;}
\DTLloaddb{stats}{./stats.csv}
\DTLmaxforcolumn{stats}{TPR}{\maxtpr}
\DTLminforcolumn{stats}{FPR}{\minfpr}
\DTLmaxforcolumn{stats}{TNR}{\maxtnr}
\DTLmaxforcolumn{stats}{Presisi}{\maxprec}
\DTLmaxforcolumn{stats}{F-Measure}{\maxfm}
\DTLmaxforcolumn{stats}{Akurasi}{\maxacc}
\DTLmaxforcolumn{stats}{AUC}{\maxauc}

\begin{table}[bp]
\caption{Performance of Random Forest and Cascaded Random Forest}
\centering
\begin{tabular}{llrrrrrrr}
\hline
\textbf{Classifier} &
\textbf{Dataset} &
\textbf{TPR}
\DTLforeach*{stats}{%
	\cl=Klasifikasi,%
	\ds=Dataset,%
	\tpr=TPR%
}{%
	\DTLifnullorempty{\cl}
		{\\ \cline{2-3}}
		{\\ \hline \hline}
	\DTLifnullorempty{\cl}
		{}
		{
			\multirow{3}{2cm}{\cl}
		}
	& \ds
	& \DTLifnumeq{\tpr}{\maxtpr}{\textbf{\tpr}}{\tpr}
}
\\
\hline
\end{tabular}
\label{tab:stats}
\end{table}


	\begin{figure}[htbp]
\centering
\begin{tikzpicture}[font=\scriptsize]
	\begin{axis}[
		width=7cm,
		ymax=250,
		ybar,
		ylabel=Running time (minutes),
		symbolic x coords={Without resampling, SMOTE, LNSMOTE},
		xtick=data,
		nodes near coords,
		every node near coord/.append style={font=\tiny},
		enlarge x limits=0.24,
		enlarge y limits=0,
		legend style={
			at={(0.5,-0.15)},
			anchor=north,
			legend columns=-1
		},
	]
		%% RF
		\addplot[pattern = dots] coordinates {
			(Without resampling, 52.5)
			(SMOTE, 217.7)
			(LNSMOTE, 199.5)
		};

		%% CRF-200-1
		\addplot[pattern=north east lines] coordinates {
			(Without resampling, 3.5)
			(SMOTE, 74.4)
			(LNSMOTE, 61.9)
		};

		%% CRF-100-2
		\addplot[pattern=horizontal lines] coordinates {
			(Without resampling, 3.6)
			(SMOTE, 74.6)
			(LNSMOTE, 67.9)
		};

		%% CRF-50-4
		\addplot coordinates {
			(Without resampling, 4.2)
			(SMOTE, 82.5)
			(LNSMOTE, 76.6)
		};
	\legend{RF, CRF-200-1, CRF-100-2, CRF-50-4}
	\end{axis}
\end{tikzpicture}
\caption{Running time for each classifer on different dataset.}
\label{graph:runtimes}
\end{figure}

