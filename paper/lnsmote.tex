SMOTE method has several weakness.
First, all sample from minority class is used, this could be a problem because
not all the samples have equal benefit for learning.
Minority sample in the boundary region between minority and majority class
usually result in misclassified rather than sample located in the center of
region, while sample that located in the center may give a little contribution
to classifier.
One of the method to overcome this problem is by using sample in boundary of
minority class instead in the center which proposed by Han et al.
\cite{han2005borderline}, called Borderline-SMOTE.

Another weakness of SMOTE method is overgeneralization, where their method does
not take into consideration the distribution of minority sample in majority
class, or the outliers.
Maciejewski and Stefanowski \cite{maciejewski2011local}
introduced an extension of SMOTE called Local Neighbourhood
SMOTE (LNSMOTE) \cite{maciejewski2011local} by combining Borderline-SMOTE
with modified version of Safe-Level SMOTE (SL-SMOTE)
\cite{bunkhumpornpat2009safe}.
In SL-SMOTE, majority samples is taken into consideration before creating
synthetic sample by calculating a coefficient called safe-level.
For each minority sample, count the number of their $k$ nearest neighbours (KNN).
If KNN value is close to 0, then the sample will be considered as noise.
If KNN value is close to $k$, then the sample can be said in safe region in
minority class.
The main idea was to create synthetic sample that close to safe region.

The SL-SMOTE strategy has a problem especially when class distribution is bias
in which the minority class spread into small sub-region with low number
cardinality.
In this situation, creating synthetic sample with SLMOTE will cause an overlap
between class.
This problem is due to SL-SMOTE find only KNN for minority class.
If the sample candidate does not located in region with densed minority class,
then some of their neighbours could be far from sample candidate or surrounded
by majority class samples.
LNSMOTE overcome this overlap problem by taking into consideration the local
neighbourhood of minority sample candidate that can provide the number of
majority class around each of them.
