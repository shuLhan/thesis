%%
%% SECTION: Latar Belakang
%%
\section{Latar Belakang}\label{sec:latar-belakang}

Wikipedia.org adalah ensiklopedia daring, yang mana artikel di Wikipedia merupakan hasil kolaborasi oleh para penyunting dari seluruh dunia. Situs Wikipedia.org merupakan situs ensiklopedia terbuka, artinya siapa pun dapat menyunting artikel tanpa perlu melakukan registrasi terlebih dahulu. Ensiklopedia daring ini memiliki artikel dari berbagai bahasa, dari bahasa umum dunia yaitu Bahasa Inggris, sampai bahasa daerah seperti Bahasa Jawa. Situs untuk artikel Wikipedia Bahasa Indonesia berada di id.wikipedia.org.

Vandalisme menurut Kamus Besar Bahasa Indonesia daring adalah, 1) perbuatan merusak dan menghancurkan hasil karya seni dan barang berharga lainnya; 2) perusakan dan penghancuran secara kasar dan ganas. Dalam konteks Wikipedia.org, vandalisme dapat berbentuk suntingan yang mengubah konten dari artikel sehingga memberikan isi yang salah, penghapusan massal, penghapusan sebagian, isi yang menghina, iklan, dan/atau teks yang tidak ada maknanya.

Jumlah artikel Bahasa Indonesia pada situs id.wikipedia.org pada bulan April 2015 yaitu sebanyak 355.264 artikel, dengan pengguna aktif, atau disebut juga editor, sebanyak 2.301 orang. Berarti, diasumsikan jika semua editor benar aktif, maka setiap pengguna aktif harus mengawasi sebanyak 154 artikel. Menemukan dan memperbaiki vandalisme tersebut dapat mengganggu editor dari menulis artikel dan pekerjaan penting lainnya, dan membuat pembaca bisa mendapatkan informasi yang salah, terhina, atau tidak mendapatkan informasi sama sekali.