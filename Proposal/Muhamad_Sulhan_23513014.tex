\documentclass[12pt,a4paper,titlepage]{article}
%%
%% document's packages
%%
\usepackage{graphicx}		% \includegraphics{name}
\usepackage{url}			% \url{something}
\usepackage{multirow}		% \multirow{package}{width}{text}
\usepackage[table]{xcolor}	% \cellcolor
\usepackage{caption}		% \caption{title}
%%% Bibliography
\usepackage[backend=bibtex,style=ieee]{biblatex}
\addbibresource{Muhamad_Sulhan_23513014.bib}
\defbibheading{bibliography}{\centerline{\textbf{DAFTAR REFERENSI}}}
%%% add dot to TOC
\usepackage{tocloft}
\renewcommand{\cftsecleader}{\cftdotfill{\cftdotsep}}
\renewcommand{\contentsname}{}
%%% uncomment this to show overrule in black box
%\overfullrule=2cm
%%% define your hyphenation here
\hyphenation{Wiki-pedia}
%%
%% document's functions
%%
%%% signature
\def\mysignature#1{%
	\vbox{\hsize=5cm
		\textbf{#1}\\
		\addvspace{2cm}%
		\hbox to \hsize{%
			\strut\hfil%
			\ldots\ldots\ldots\ldots\ldots\ldots\ldots\ldots%
			\hfil%
		}
		\hrule\kern1ex
		\hbox to \hsize{%
			\strut\hfil%
			NIP\hspace{1ex}\ldots\ldots\ldots\ldots\ldots\ldots\ldots%
			\hfil%
		}
	}
}
%%
%% document's variables
%%
\newcommand{\mytitle}{Deteksi Vandalisme di situs Wikipedia Bahasa Indonesia}
\newcommand{\myname}{Muhamad Sulhan}
\newcommand{\mysid}{23513014}
\newcommand{\myitbcover}{../images/ITB-logo-ganesha}
%%
%% document's meta-data
%%
\author{\myname}
\title{\mytitle}

%%
%% DOCUMENT
%%
\begin{document}
%%
%% PAGE: Cover
%%
\thispagestyle{empty}
\begin{center}
	\textbf{%
		\mytitle
		\vfill
		PROPOSAL TESIS
		\vfill
		Disusun sebagai syarat kelulusan matakuliah \\
		IF5099 Metodologi Penelitian/Tesis I \\
		\vfill
		Oleh \\
		\myname \\
		\mysid \\
		\vfill
		\includegraphics[width=2cm]{\myitbcover}
		\vfill
		\uppercase{%
			Program Studi Magister Informatika \\
			Sekolah Teknik Elektro dan Informatika \\
			Institut Teknologi Bandung \\
			2015
		}
	}
\end{center}
\newpage
\begin{center}
	\textbf{\large{\MakeUppercase{\judul}}}\\

	\addvspace{3cm}

	Oleh\\
	\textbf{
		\myname\\
		NIM: \mysid\\
		(\mydept)\\
	}

	\bigskip
	\itb\\

	\addvspace{3cm}

	Menyetujui\\
	Pembimbing\\
	\bigskip
	\makebox[6cm][c]{
		Tanggal \dotfill
	}

	\vfill

	\makebox[6cm][c]{
		\hrulefill
	}
	\hbox to \hsize{%
		\strut\hfil%
		(\myadvisorname)%
		\hfil%
	}
\end{center}


%%
%% PAGE: Daftar Isi
%%
\section*{\centerline{\Large DAFTAR ISI}}\label{sec:daftar-isi}
\addcontentsline{toc}{section}{DAFTAR ISI}
\tableofcontents
\newpage

%%
%% SECTIONS
%%
%%
%% SECTION: Ringkasan
%%
\section{Ringkasan Proposal}\label{sec:ringkasan}

Tesis ini mengkaji tentang metode dalam mendeteksi vandalisme yang terjadi di situs \textit{id.wikipedia.org}, yaitu situs ensiklopedia daring berbahasa Indonesia, untuk membantu editor menemukan dan memperbaiki vandalisme yang terjadi di situs tersebut dengan tujuan untuk menjaga konten dari artikel Wikipedia tetap dengan kualitas yang bagus.
%%
%% SECTION: Latar Belakang
%%
\section{Latar Belakang}\label{sec:latar-belakang}

Wikipedia.org adalah ensiklopedia daring, yang mana artikel di Wikipedia merupakan hasil kolaborasi oleh para penyunting dari seluruh dunia. Situs Wikipedia.org merupakan situs ensiklopedia terbuka, artinya siapa pun dapat menyunting artikel tanpa perlu melakukan registrasi terlebih dahulu. Ensiklopedia daring ini memiliki artikel dari berbagai bahasa, dari bahasa umum dunia yaitu Bahasa Inggris, sampai bahasa daerah seperti Bahasa Jawa. Situs untuk artikel Wikipedia Bahasa Indonesia berada di id.wikipedia.org.

Vandalisme menurut Kamus Besar Bahasa Indonesia daring adalah, 1) perbuatan merusak dan menghancurkan hasil karya seni dan barang berharga lainnya; 2) perusakan dan penghancuran secara kasar dan ganas. Dalam konteks Wikipedia.org, vandalisme dapat berbentuk suntingan yang mengubah konten dari artikel sehingga memberikan isi yang salah, penghapusan massal, penghapusan sebagian, isi yang menghina, iklan, dan/atau teks yang tidak ada maknanya.

Jumlah artikel Bahasa Indonesia pada situs id.wikipedia.org pada bulan April 2015 yaitu sebanyak 355.264 artikel, dengan pengguna aktif, atau disebut juga editor, sebanyak 2.301 orang. Berarti, diasumsikan jika semua editor benar aktif, maka setiap pengguna aktif harus mengawasi sebanyak 154 artikel. Menemukan dan memperbaiki vandalisme tersebut dapat mengganggu editor dari menulis artikel dan pekerjaan penting lainnya, dan membuat pembaca bisa mendapatkan informasi yang salah, terhina, atau tidak mendapatkan informasi sama sekali.
%%
%% SECTION: Rumusan Masalah
%%
\section{Rumusan Masalah}\label{sec:rumusan-masalah}

Tesis ini berhipotesis bahwa,
\begin{itemize}
	\item vandalisme dapat dilihat dengan pola jumlah kata tertentu yang digunakan dalam artikel dan kata yang digunakan dalam suntingan.
	\item perilaku vandal mirip di antara domain bahasa, sehingga sebuah model deteksi yang sudah ada dan digunakan pada bahasa lain dapat dibentuk dan digunakan untuk mendeteksi vandalisme pada artikel berbahasa Indonesia.
\end{itemize}
%%
%% SECTION: Tujuan
%%
\section{Tujuan}\label{sec:tujuan}

Motivasi dari tesis ini adalah untuk mendeteksi vandalisme pada penyuntingan
artikel di situs Wikipedia untuk membantu editor Wikipedia dalam mempermudah
menentukan hasil suntingan artikel yang berpotensi berisi vandal sehingga dapat
dengan cepat mengembalikan ke isi sebelumnya.
%%
%% SECTION: Batasan Masalah
%%
\section{Batasan Masalah}\label{sec:batasan-masalah}

Tesis ini hanya melakukan analisis untuk artikel Wikipedia Bahasa Indonesia yang terdapat pada situs \textit{id.wikipedia.org}. Data yang digunakan dalam melakukan analisis, implementasi, dan pengujian yaitu data \textit{dump} dari Wikipedia Bahasa Indonesia sampai pada tanggal 9 April 2015\footnote{\url{http://dumps.wikimedia.org/idwiki/20150409/}}.
%%
%% SECTION: Studi Literatur
%%
\section{Studi Literatur}\label{sec:studi-literatur}

Beberapa alat dan cara untuk menghadapi vandalisme telah dilakukan, antara lain,

\begin{itemize}
	\item Anti- vandalisme \textit{bot}, teknik pembelajaran mesin menggunakan aturan dan daftar kata-kata. Kelemahannya yaitu dapat dengan mudah ditipu bila diketahui kata yang tidak terdapat dalam aturan sehingga tidak terdeteksi oleh bot sebagai vandalisme \cite{Smets08automaticvandalism}.
	\item Visualisasi dari riwayat suntingan menggunakan diagram alur \cite{viegas2004studying}.
	\item Menganalisis kata yang digunakan dalam artikel. Saat membandingkan revisi dari sebuah artikel, ciri tingkatan kata bisa menentukan apakah penggunaan kata tertentu akan ditolak atau diterima di revisi berikutnya \cite{rzeszotarski2012learning}.
	\item Menggunakan sistem reputasi dengan menggabungkan analisis konten  dengan informasi mengenai penyunting dan pengukuran kualitas suntingan \cite{adler2010detecting}.
\end{itemize}
%%
%% SECTION: Metodologi
%%
\section{Metodologi}\label{sec:metodologi}

Penelitian ini dikembangkan dengan menggunakan metodologi kuantitatif dengan tahapan sebagai berikut,
\begin{itemize}
	\item \textbf{Studi Literatur}. Peneliti membaca beberapa penelitian yang sebelumnya telah dilakukan dalam deteksi vandalisme pada Wikipedia. Dari hasil penelitian tersebut penulis dapat melihat kelemahan dan potensi ke depan yang dapat dikembangkan, sehingga menjadi rumusan masalah dalam penulisan tesis ini. Tahapan selanjutnya dari studi literatur yaitu mengkaji sumber yang berkaitan dengan metode yang digunakan dalam pendeteksian vandalisme dalam makalah ini.
	\item \textbf{Persiapan Data dan Lingkungan Penelitian}. Dalam tahapan ini peneliti mempersiapkan data dan lingkungan pengembangan, seperti persiapan aplikasi basis data, pengaturan bahasa pemrograman, dan lainnya; yang diperlukan nantinya dalam melakukan analisis, implementasi, dan pengujian.
	\item \textbf{Analisis}. Pada tahap ini peneliti melihat data dan menentukan fungsi-fungsi yang akan diterapkan dalam implementasi untuk mendapatkan hasil yang ditujukan. Tahap ini bisa terjadi berulang kembali setelah implementasi.
	\item \textbf{Implementasi}. Tahap implementasi dalam makalah ini berupa proses pemuatan data, penerapan fungsi dalam bahasa pemrograman, dan pembuatan lingkungan pengujian.
	\item \textbf{Pengujian}. Setelah tahap implementasi selesai, fungsi dari pendeteksian vandalisme akan dilakukan secara \textit{offline}. Data dibagi dalam dua bagian waktu, sebelum $ t_{b} $ dan sesudah $ t_{s} $, kemudian fungsi deteksi vandalisme dijalankan untuk data $t_{b}$.
	\item \textbf{Evaluasi}. Tahap ini membandingkan hasil dari pengujian pada data $t_{b}$ dengan data sesudah $t_{s}$.
	
\end{itemize}
%%
%% SECTION: Implikasi
%%
\section{Implikasi}\label{sec:implikasi}

Implikasi yang didapat dari hasil penelitian ini adalah memberikan sebuah alat yang membantu editor Wikipedia dalam mengambil keputusan dari daftar suntingan yang berpotensi hasil vandal sehingga mempercepat mereka dalam mengembalikan artikel ke versi sebelumnya.
%%
%% SECTION: Sistematika Penulisan
%%
\section{Sistematika Penulisan}\label{sec:sistematika-penulisan}

Laporan tesis ini dibagi menjadi beberapa bab berikut,
\begin{enumerate}
	\item Bab I, Pendahuluan, berisi Latar Belakang, Rumusan Masalah, Tujuan, Batasan Masalah, Metodologi, dan Sistematika Penulisan.
	\item Bab II, Landasan Teori, berisi ilmu dan konsep yang mendukung pembahasan tesis ini.
	\item Bab III, Metodologi Penelitian, berisi deskripsi tentang analisis, tahap implementasi, dan tahap pengujian yang dilakukan selama penelitian.
	\item Bab IV, Hasil dan Analisis, berisi penjelasan dari hasil penelitian.
	\item Bab V, Penutup, berisi kesimpulan yang dapat diambil dari hasil\linebreak penelitian ini beserta saran untuk pengembangan selanjutnya. 
\end{enumerate}
%%
%% SECTION: Penjadwalan
%%
\section{Penjadwalan}\label{sec:penjadwalan}

Tabel \ref{tab:jadwal} menampilkan jadwal yang direncanakan dalam pengembangan tesis dari bulan ke I, September 2015, sampai dengan bulan ke VI, Januari 2016.

\renewcommand{\arraystretch}{1.5}
\setlength{\tabcolsep}{3pt}

\begin{table}[h!]
	\centering
	{\footnotesize
	\begin{tabular}{|c|p{3cm}|c|c|c|c |c|c|c|c |c|c|c|c |c|c|c|c| |c|c|c|c |c|c|c|c|}
		\hline
		\multirow{2}{*}{No.}
			& \multirow{2}{*}{Kegiatan}
			& \multicolumn{4}{c|}{Bulan I}
			& \multicolumn{4}{c|}{Bulan II}
			& \multicolumn{4}{c|}{Bulan III}
			& \multicolumn{4}{c|}{Bulan IV}
			& \multicolumn{4}{c|}{Bulan V}
			& \multicolumn{4}{c|}{Bulan VI}\\
		\cline{3-26}
		& &
			1 & 2 & 3 & 4 &
			1 & 2 & 3 & 4 &
			1 & 2 & 3 & 4 &
			1 & 2 & 3 & 4 &
			1 & 2 & 3 & 4 &
			1 & 2 & 3 & 4\\
		\hline
		1 & Studi Literatur &
			{\cellcolor[gray]{0.7}} &
			{\cellcolor[gray]{0.7}} &
			{\cellcolor[gray]{0.7}} &
			{\cellcolor[gray]{0.7}} &
			& & & &
			& & & &
			& & & &
			& & & &
			& & & \\
		\hline
		2 & Persiapan Data dan Lingkungan Penelitian &
			& &
			{\cellcolor[gray]{0.7}} &
			{\cellcolor[gray]{0.7}} &
			{\cellcolor[gray]{0.7}} &
			{\cellcolor[gray]{0.7}} &
			& &
			& & & &
			& & & &
			& & & &
			& & & \\
		\hline
		3 & Analisis &
			& & & &
			{\cellcolor[gray]{0.7}} &
			{\cellcolor[gray]{0.7}} &
			{\cellcolor[gray]{0.7}} &
			{\cellcolor[gray]{0.7}} &
			{\cellcolor[gray]{0.7}} &
			{\cellcolor[gray]{0.7}} &
			& &
			& & & &
			& & & &
			& & & \\
		\hline
		4 & Implementasi dan Pengujian &
			& & & &
			& & & &
			{\cellcolor[gray]{0.7}} &
			{\cellcolor[gray]{0.7}} &
			{\cellcolor[gray]{0.7}} &
			{\cellcolor[gray]{0.7}} &
			{\cellcolor[gray]{0.7}} &
			{\cellcolor[gray]{0.7}} &
			{\cellcolor[gray]{0.7}} &
			{\cellcolor[gray]{0.7}} &
			{\cellcolor[gray]{0.7}} &
			{\cellcolor[gray]{0.7}} &
			{\cellcolor[gray]{0.7}} &
			{\cellcolor[gray]{0.7}} &
			{\cellcolor[gray]{0.7}} &
			{\cellcolor[gray]{0.7}} &
			{\cellcolor[gray]{0.7}} \\
		\hline
		5 & Evaluasi &
			& & & &
			& & & &
			& & & &
			& & & &
			& & & &
			& &
			{\cellcolor[gray]{0.7}} &
			{\cellcolor[gray]{0.7}} \\
		\hline
	\end{tabular}
	}
	\caption{Jadwal penelitian tesis}
	\label{tab:jadwal}
\end{table}

\newpage
\printbibliography

\end{document}