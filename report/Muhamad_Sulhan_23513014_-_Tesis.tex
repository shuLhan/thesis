\documentclass[12pt,a4paper,titlepage]{report}

%% font encoding.
\usepackage[utf8]{inputenc}

%% Fonts
\usepackage[T1]{fontenc}
\usepackage{lmodern}
\usepackage{couriers}

%% Bibliography.
\usepackage[bahasa]{babel}
\usepackage{csquotes}
\usepackage[
	style=apa
]{biblatex}
\DeclareLanguageMapping{bahasa}{bahasa-apa}

%% Customize chapters.
\usepackage{titlesec}

%% \includegraphics{name}
\usepackage{graphicx}

%% \url{something}
\usepackage{url}

%% \multirow{package}{width}{text}
\usepackage{multirow}

%% \cellcolor
\usepackage[table]{xcolor}

%% \caption{title}
\usepackage{caption}
	\captionsetup{format=hang}

\usepackage{subcaption}

%% \forloop
\usepackage{forloop}

%% fix underfull on footnote with URL.
\usepackage{ragged2e}

%% source code listing
\usepackage{listings}

\lstdefinelanguage{go}
{
	morekeywords={package,import,const,func,for,type,var,struct}
,	sensitive=true
,	morecomment=[l]{//}
,	morecomment=[s]{/*}{*/}
}
\lstdefinestyle{go}{%
	language=go
,	keywordstyle=\color{black}\bfseries
,	commentstyle=\color{gray}
,	breakatwhitespace=true
,	lineskip={-2.5pt}
}

\lstdefinestyle{data}{%
	breakatwhitespace=false
,	breakautoindent=false
,	literate={\,}{}{0\discretionary{,}{}{,}},
}

%% \printindex
\usepackage{makeidx}

%% table of content
\usepackage{tocloft}

%% text emphasis, including strikeout.
\usepackage[normalem]{ulem}

%% mathematics
\usepackage{mathtools}

%% arithmetic
\usepackage{calc}

%% algorithm
\usepackage{algorithm}
\usepackage{algpseudocode}

\makeatletter
	\renewcommand{\ALG@name}{Algoritma}
\makeatother

%% multiple columns
\usepackage{multicol}

%% Package for changin page margin
\usepackage[a4paper]{geometry}

%%
\usepackage{parskip}

%% Package for reading CSV to database.
\usepackage{datatool}

%% Package for scatter and line plot.
\usepackage{dataplot}

%% Long table
\usepackage{longtable}

%% Tikz
\usepackage{tikz}
\usetikzlibrary{backgrounds, shapes.geometric, positioning, patterns, external}
	\tikzexternalize

\usepackage{pgfplots}

%% Font: MnSymbol
\usepackage{MnSymbol}

%% Compact list.
\usepackage{paralist}

%% Line spacing
\usepackage{setspace}

%% For changing internal packages.
\usepackage{xpatch}

%% Create clickable TOC
\usepackage[hidelinks]{hyperref}
\hypersetup{
	colorlinks
,	allcolors=black
}

\hyphenation{
	Be-ri-kut
	Ja-nu-a-ri
	SIGKDD
	Wiki-pedia
	a-kan
	a-ku-ra-si
	ang-ka
	ba-gai-ma-na
	bayes-ian
	ber-gu-na
	ber-kas
	ber-ma-sa-lah
	ber-mak-na
	bi-a-ya
	da-lam
	data-set
	de-ngan
	di-ha-sil-kan
	di-pi-lih
	di-sing-kat
	di-tam-bah-kan
	dis-krit
	fung-si
	ga-bung-an
	ke-las
	ke-le-mah-an
	ke-mung-ki-nan
	ke-tak-se-imbang-an
	lan-guage
	ma-yo-ri-tas
	me-laku-kan
	me-me-rik-sa
	me-mi-lih
	me-ne-rap-kan
	meng-a-pli-ka-si-kan
	meng-ge-ne-ra-li-sa-si
	me-ning-kat-kan
	me-nye-dia-kan
	me-nye-im-bang-kan
	me-sin
	me-thod
	me-to-de
	mem-vi-sua-li-sa-si
	meng-gu-na-kan
	meng-hi-lang-kan
	meng-hu-bung-kan
	meng-i-kut-kan
	meng-i-kuti
	meng-im-ple-men-ta-si-kan
	meng-in-di-ka-si-kan
	mi-sal-nya
	mung-kin
	o-ver-sam-pling
	pa-ra-lel
	pe-la-ti-han
	pe-mi-sah
	pe-nan-da
	pe-ne-li-ti-an
	pe-nu-li-san
	pe-nyun-ting
	pem-ban-ding
	pen-de-kat-an
	peng-kla-si-fi-ka-si
	peng-a-pli-ka-si-an
	per-for-man-si-nya
	po-ten-si-al
	pro-ba-bi-li-tas
	pro-ses
	sam-pel
	se-im-bang
	se-jum-lah
	sun-ting-an
	ting-kat
	un-der-sam-pling
	wa-lau-pun
}

\newcommand{\mytitle}{Deteksi Vandalisme pada Wikipedia Bahasa Inggris menggunakan klasifikasi Cascaded Random Forest}
\newcommand{\myname}{Muhamad Sulhan}
\newcommand{\mysid}{23513014}
\newcommand{\myadvisorname}{Dwi Hendratmo Widyantoro}
\newcommand{\myadvisorid}{196812071994021001}
\newcommand{\mydept}{Program Studi Magister Informatika}
\newcommand{\itb}{Institut Teknologi Bandung}

\newcommand{\daftarisi}{DAFTAR ISI}
\newcommand{\daftargambar}{DAFTAR GAMBAR DAN ILUSTRASI}
\newcommand{\daftartabel}{DAFTAR TABEL}
\newcommand{\tUpAbstrak}{ABSTRAK}
\newcommand{\tDaftarPustaka}{DAFTAR PUSTAKA}
\newcommand{\tLampiran}{Lampiran}
\newcommand{\tUpLampiran}{\uppercase{\tLampiran}}

%%% My images directory
\graphicspath{{../images/}}
\newcommand{\myitbcover}{ITB-logo-hitam}
\newcommand{\myitbcoverblue}{ITB-logo-ganesha}

%%% two column signature.
\def\myadvisorsig#1{%
	\vbox{\hsize=6cm
		\textbf{#1}\\
		\addvspace{2cm}%
		\hbox to \hsize{%
			\strut\hfil%
			\myadvisorname%
			\hfil%
		}
		\hrule\kern1ex
		\hbox to \hsize{%
			\strut\hfil%
			NIP\hspace{1ex}\myadvisorid%
			\hfil%
		}
	}
}

%%% one column signature.
\def\mysignature#1#2#3{%
	\vbox{
		\textbf{#1}\\
		\addvspace{2cm}%
		\hbox to \hsize{%
			\strut\hfil%
			{#2}%
			\hfil%
		}
		\makebox[6cm][c]{
			\hrulefill
		}
		\hbox to \hsize{%
			\strut\hfil%
			NIP\hspace{1ex}{#3}%
			\hfil%
		}
	}
}

%%% source code listing
\newcommand{\includecodego}[2][c]{
	\lstinputlisting[caption=#2,escapechar=,style=go]
		{/home/ms/go/work/src/github.com/shuLhan/#2}
}

%%% data listing
\newcommand{\includedata}[2][c]{
	\lstinputlisting[caption=#2,style=data,linerange={1-10}]
		{/home/ms/go/work/src/github.com/shuLhan/#2}
}

%% Caption for algorithm
\DeclareCaptionFormat{algor}{%
	\hrulefill
	\par
	\offinterlineskip
	\vskip1pt
	\textbf{#1#2}#3
	\offinterlineskip
	\hrulefill
}

\DeclareCaptionStyle{algori}{singlelinecheck=off,format=algor,labelsep=space}
\captionsetup[algorithm]{style=algori}

%%
%% This contain formatting for ITB Thesis.
%%

%%
%% Set page margin to 4cm left, 3cm on top, right, and bottom.
%%
\geometry{
	a4paper,
	top=3cm,
	right=3cm,
	bottom=3cm,
	left=4cm
}

%%
%% Set space between paragraphs to 1.5.
%%
\setlength{\parskip}{1.5em}

%%
%% Add dot to TOC.
%%
\renewcommand{\cftchapleader}{\cftdotfill{\cftdotsep}}
\renewcommand{\cftsecleader}{\cftdotfill{\cftdotsep}}
\renewcommand{\contentsname}{}

%%
%% Reduce vspace between title in list and body
%%
\renewcommand{\contentsname}{
	\chapter*{\vspace{2cm}\daftarisi}\label{chapter:daftarisi}
	\vspace{-1.5em}
	\addcontentsline{toc}{chapter}{\daftarisi}
}

\newcommand{\listappendixname}{
	\chapter*{\vspace{2cm} \tupdaftarlampiran}
	\vspace{-1.5em}
	\addcontentsline{toc}{chapter}{\tupdaftarlampiran}
}

\renewcommand{\listfigurename}{
	\chapter*{\vspace{2cm} \daftargambar}
	\vspace{-1.5em}
	\addcontentsline{toc}{chapter}{\daftargambar}
}

\renewcommand{\listtablename}{
	\chapter*{\vspace{2cm} \daftartabel}
	\vspace{-1.5em}
	\addcontentsline{toc}{chapter}{\daftartabel}
}

%%
%% Source code listing style.
%%
\lstset{
	basicstyle=\scriptsize\ttfamily
,	breaklines=true
,	stringstyle=\scriptsize\ttfamily
,	numbers=left
,	numberstyle=\tiny\ttfamily
,	numbersep=5pt
,	tabsize=4
,	frame=single
}

\makeatletter
\def\lst@lettertrue{\let\lst@ifletter\iffalse}
\makeatother

%%
%% Format chapter and section.
%%

\setlength{\cftchapnumwidth}{5em}
\setlength{\cftsecnumwidth}{2.5em}

\setlength{\cftsecindent}{5em}
\setlength{\cftsubsecindent}{7.5em}

\renewcommand{\cftchappresnum}{Bab\space}

%% Set chapter name to Bab.
\titleformat{\chapter}[hang]
{\bfseries\large\centering}
{Bab \thechapter}
{1em}
{}

%% Set section font to normal.
\titleformat{\section}
{\normalfont\bfseries}
{\thesection}
{1em}{}

%% Set subsection font to normal.
\titleformat{\subsection}
{\normalfont\bfseries}
{\thesubsection}
{1em}{}

%% Set spacing for chapter, section, and subsection.
\titlespacing*{\chapter}{0pt}{-2.5em}{1.5em}
\titlespacing{\section}{0pt}{0pt}{0em}
\titlespacing{\subsection}{0pt}{0pt}{0em}

%% Set roman on chapter number.
\def\thechapter{\Roman{chapter}}

%% Alter latex default title on table and figure.
\captionsetup[table]{name=Tabel}
\captionsetup[figure]{name=Gambar}

%%
\renewcommand{\arraystretch}{1.5}
\setlength{\tabcolsep}{3pt}

%% Change bibliography title.
\defbibheading{bibliography}{\centerline{
	\textbf{DAFTAR PUSTAKA}}
}

%% Algorithmicx
\makeatletter
\renewcommand{\ALG@beginalgorithmic}{\footnotesize}
\makeatother

%% My bibligraphy file
\addbibresource{bibliography.bib}

%% multicolumn setting
\setlength{\columnsep}{1cm}

%% uncomment this to show overrule in black box
\overfullrule=2cm

%% The Glory Appendices

\newlistof{appendix}{app}{\listappendixname}
\setcounter{appdepth}{2}
\renewcommand{\theappendix}{\Alph{appendix}}
\renewcommand{\cftappendixpresnum}{Lampiran\space}
\setlength{\cftbeforeappendixskip}{0em}
\setlength{\cftappendixnumwidth}{6em}
\newlistentry[appendix]{subappendix}{app}{1}
\renewcommand{\thesubappendix}{\theappendix.\arabic{subappendix}}
\setlength{\cftsubappendixindent}{6em}

\newcommand{\myappendix}[1]{%
	\refstepcounter{appendix}%
	\chapter*{Lampiran\space\theappendix\space #1}%
	\addcontentsline{app}{appendix}{\protect\numberline{\theappendix}#1}%
	\par
}

\newcommand{\subappendix}[1]{%
	\refstepcounter{subappendix}%
	\section*{\thesubappendix\space #1}%
	\addcontentsline{app}{subappendix}{\protect\numberline{\thesubappendix}#1}%
}

%% Format TOC
\makeatletter
	\newskip\old@cftbeforechapskip
	\old@cftbeforechapskip\cftbeforechapskip
	\let\old@l@chapter\l@chapter
	\let\old@l@section\l@section
	\def\l@chapter#1#2{\old@l@chapter{#1}{#2}\cftbeforechapskip2pt}% set your value here
	\def\l@section#1#2{\old@l@section{#1}{#2}\cftbeforechapskip\old@cftbeforechapskip}
\makeatother


\author{\myname}
\title{\judul}


\begin{document}

\begin{singlespace}
\thispagestyle{empty}
\begin{singlespacing}
\begin{center}
\textbf{\large
	\MakeUppercase{\judul} \\
	\vfill
	TESIS \\
	\bigskip
	{\normalsize
		Karya tulis sebagai salah satu syarat \\
		untuk memperoleh gelar Magister dari \\
		\itb{} \\
	}
	\vfill
	{\normalsize Oleh} \\
	\MakeUppercase{\myname{}} \\
	NIM: \mysid{} \\
	\mydept{} \\
	\vfill
	\includegraphics[width=2.35cm,height=3.5cm]{\myitbcover} \\
	\vfill
	\normalsize
	\MakeUppercase{\tfakultas} \\
	\MakeUppercase{\itb{}} \\
	Juni 2016 \\
}
\end{center}
\end{singlespacing}


\pagenumbering{roman}

\clearpage
\chapter*{\vspace{0em}\tUpAbstrak}
	\begin{center}
\textbf{\large
	\MakeUppercase{\mytitle{}} \\
	\bigskip
	\textnormal{Oleh} \\
	\myname{} \\
	NIM: \mysid{} \\
	(\mydept{}) \\
}
\end{center}

\bigskip
\bigskip
\bigskip

Wikipedia.org adalah ensiklopedia daring yang dapat disunting oleh siapa saja.
Sifat wiki ini dapat mempercepat perbaikan dan pertumbuhan artikel,
kekurangannya yaitu dapat menimbulkan vandalisme dalam bentuk suntingan dengan
informasi yang salah, penghapusan, iklan, atau teks yang tidak bermakna.
Tesis ini membahas deteksi vandalisme menggunakan pembelajaran mesin
dengan menerapkan metode klasifikasi
\textit{Cascaded Random Forest} (CRF)
yang dilatih pada dataset
\textit{Wikipedia Vandalism Corpus 2010}
yang telah
disampel ulang dengan
\textit{Local Neighbourhood Synthetic Minority Over-sampling Technique}
(LNSMOTE).
Kedua metode tersebut dibandingkan dengan klasifikasi
\textit{Random Forest} (RF)
dan teknik sampel ulang SMOTE.
Hasil sampel ulang dan pemodelan diuji dengan dataset
\textit{Wikipedia Vandalism Corpus 2011}
menunjukan hasil sampel ulang dengan LNSMOTE meningkatkan laju
\textit{true-positive} lebih tinggi daripada dataset yang tidak disampel ulang
dan yang disampel ulang dengan SMOTE pada kedua pengklasifikasi.
Dari segi performansi, CRF dengan sampel ulang LNSMOTE dengan 200 tingkat dan 1
pohon memberikan hasil yang paling baik dengan nilai TPR yaitu $0,9904$.
Dari segi kecepatan pemodelan pengklasifikasi CRF lebih cepat 1,6 kali daripada
RF pada data yang telah disampel ulang.

Kata kunci: wikipedia, vandalisme, dataset timpang, dataset sampel ulang,
\textit{cascaded random forest}

	\label{abstrak}
	\addcontentsline{toc}{chapter}{\tUpAbstrak}

\clearpage
\chapter*{\vspace{0em}\tupabstract}
	\begin{abstract}
Wikipedia.org is an online encyclopedia which can be edited by anyone.
This feature makes the article in Wikipedia rapidly
increased in size and can be fixed subsequently, but also makes it prone
to vandalism in the forms of invalid information, deletion, ads, or meaningless
content.
This paper propose a framework for detecting vandalism on English Wikipedia
using machine learning technique by training Cascaded Random Forest (CRF)
classifier on PAN Wikipedia Vandalism Corpus 2010 (PAN-WVC-10) English dataset
that has been resampled using Local Neighbourhood Synthetic Minority
Oversampling Technique (LNSMOTE).
These two techniques then compared with Random Forest (RF) for classifier and
Synthetic Minority Oversampling Technique (SMOTE) for resampling.
The result of classifiers that has been tested on PAN Wikipedia Vandalism Corpus
2011 (PAN-WVC-11) English dataset
showed that dataset resampled using LNSMOTE increase the true-positive rate
(TPR) better than SMOTE in both classifiers.
CRF on SMOTE with 200 stages and 1 tree gave the better result among others
with TPR value 0.9904.
From training computation time, CRF 1.6 times faster than RF in resampled
dataset.
\end{abstract}

	\label{abstract}
	\addcontentsline{toc}{chapter}{\tupabstract}

\clearpage
	\addcontentsline{toc}{chapter}{\tuppengesahan}
	\begin{center}
	\textbf{\large{\MakeUppercase{\judul}}}\\

	\addvspace{3cm}

	Oleh\\
	\textbf{
		\myname\\
		NIM: \mysid\\
		(\mydept)\\
	}

	\bigskip
	\itb\\

	\addvspace{3cm}

	Menyetujui\\
	Pembimbing\\
	\bigskip
	\makebox[6cm][c]{
		Tanggal \dotfill
	}

	\vfill

	\makebox[6cm][c]{
		\hrulefill
	}
	\hbox to \hsize{%
		\strut\hfil%
		(\myadvisorname)%
		\hfil%
	}
\end{center}


\tableofcontents

\listofappendix

\listoffigures

\listoftables
\addtocontents{toc}{\hbox to \linewidth{~}}
\end{singlespace}

\clearpage
\pagenumbering{arabic}

%%
%% BAB I: PENDAHULUAN
%%
\chapter{Pendahuluan}
\label{bab:01}

	\section{Latar Belakang}
	\label{bab:01:latar_belakang}
	Wikipedia.org adalah ensiklopedia daring dan terbuka, yang mana artikel di
Wikipedia merupakan hasil kolaborasi para penyunting dari seluruh dunia.
Terbuka artinya siapa pun dapat menyunting artikel tanpa perlu melakukan
registrasi terlebih dahulu.
Ensiklopedia daring ini memiliki artikel dari berbagai bahasa, dari bahasa umum
dunia seperti Bahasa Inggris, sampai bahasa daerah seperti Bahasa Jawa.

Vandalisme menurut Kamus Besar Bahasa Indonesia daring adalah,
1) perbuatan merusak dan menghancurkan hasil karya seni dan barang berharga
lainnya;
2) perusakan dan penghancuran secara kasar dan ganas.
Dalam konteks Wikipedia.org, vandalisme dapat berbentuk suntingan yang mengubah
konten dari artikel sehingga memberikan isi yang salah, penghapusan secara
menyeluruh, penghapusan sebagian, isi yang menghina, iklan, dan/atau teks yang
tidak ada maknanya.

Jumlah artikel Bahasa Inggris pada situs en.wikipedia.org pada bulan Juli 2015
yaitu sebanyak 4,932,627 artikel, dengan pengguna aktif, atau disebut juga
penyunting, sebanyak 31,369 orang.
Berarti, jika diasumsikan semua penyunting benar aktif, maka setiap pengguna
aktif harus mengawasi kurang lebih 157 artikel.
Menemukan dan memperbaiki vandalisme tersebut dapat mengganggu penyunting dari
menulis artikel dan pekerjaan penting lainnya, dan membuat pembaca bisa
mendapatkan informasi yang salah atau tidak mendapatkan informasi sama sekali.


	\section{Rumusan Masalah}
	\label{bab:01:rumusan_masalah}
	Tesis ini mencoba menjawab permasalahan dataset yang tidak seimbang pada
PAN-WVC-10 yang menyebabkan performansi deteksi yang rendah dan condong pada
kelas mayoritas dengan mengkaji teknik sampel ulang dan pengklasifikasi yang
belum pernah digunakan sebelumnya pada korpus tersebut.
Teknik sampel ulang yang digunakan yaitu LNSMOTE yang diajukan oleh
\textcite{maciejewski2011local}
dan teknik pengklasifikasi yang digunakan yaitu \textit{Cascaded Random Forest}
(CRF) yang diajukan oleh \textcite{baumann2013cascaded}.


	\section{Tujuan}
	\label{bab:01:tujuan}
	%%
%% SECTION: Tujuan
%%
\section{Tujuan}\label{sec:tujuan}

Motivasi dari tesis ini adalah untuk mendeteksi vandalisme pada penyuntingan
artikel di situs Wikipedia untuk membantu editor Wikipedia dalam mempermudah
menentukan hasil suntingan artikel yang berpotensi berisi vandal sehingga dapat
dengan cepat mengembalikan ke isi sebelumnya.

	\section{Batasan Masalah}
	\label{bab:01:batasan_masalah}
	Tesis ini hanya melakukan analisis untuk artikel Wikipedia Bahasa Inggris yang
terdapat pada korpus PAN-WVC-10 dan PAN-WVC-11.
Dataset yang digunakan untuk sampel ulang dan pelatihan model yaitu PAN-WVC-10.
Dataset yang digunakan dalam melakukan pengujian yaitu PAN-WVC-11.
Teknik sampel ulang yang dilakukan yaitu LNSMOTE yang dibandingkan dengan
SMOTE.
Teknik pengklasifikasi yang dijadikan dalam pelatihan yaitu CRF yang
dibandingkan dengan pengklasifikasi RF.


	\section{Metodologi}
	\label{bab:01:metodologi}
	%%
%% SECTION: Metodologi
%%
\section{Metodologi}\label{sec:metodologi}

Penelitian ini dikembangkan dengan menggunakan metodologi kuantitatif dengan tahapan sebagai berikut,
\begin{itemize}
	\item \textbf{Studi Literatur}. Peneliti membaca beberapa penelitian yang sebelumnya telah dilakukan dalam deteksi vandalisme pada Wikipedia. Dari hasil penelitian tersebut penulis dapat melihat kelemahan dan potensi ke depan yang dapat dikembangkan, sehingga menjadi rumusan masalah dalam penulisan tesis ini. Tahapan selanjutnya dari studi literatur yaitu mengkaji sumber yang berkaitan dengan metode yang digunakan dalam pendeteksian vandalisme dalam makalah ini.
	\item \textbf{Persiapan Data dan Lingkungan Penelitian}. Dalam tahapan ini peneliti mempersiapkan data dan lingkungan pengembangan, seperti persiapan aplikasi basis data, pengaturan bahasa pemrograman, dan lainnya; yang diperlukan nantinya dalam melakukan analisis, implementasi, dan pengujian.
	\item \textbf{Analisis}. Pada tahap ini peneliti melihat data dan menentukan fungsi-fungsi yang akan diterapkan dalam implementasi untuk mendapatkan hasil yang ditujukan. Tahap ini bisa terjadi berulang kembali setelah implementasi.
	\item \textbf{Implementasi}. Tahap implementasi dalam makalah ini berupa proses pemuatan data, penerapan fungsi dalam bahasa pemrograman, dan pembuatan lingkungan pengujian.
	\item \textbf{Pengujian}. Setelah tahap implementasi selesai, fungsi dari pendeteksian vandalisme akan dilakukan secara \textit{offline}. Data dibagi dalam dua bagian waktu, sebelum $ t_{b} $ dan sesudah $ t_{s} $, kemudian fungsi deteksi vandalisme dijalankan untuk data $t_{b}$.
	\item \textbf{Evaluasi}. Tahap ini membandingkan hasil dari pengujian pada data $t_{b}$ dengan data sesudah $t_{s}$.
	
\end{itemize}

	\section{Sistematika Penulisan}
	\label{bab:01:sistematika_penulisan}
	Laporan tesis ini dibagi menjadi beberapa bab berikut,
\begin{enumerate}
	\item Bab I Pendahuluan, berisi Latar Belakang, Rumusan Masalah,
	Tujuan, Batasan Masalah, Metodologi, dan Sistematika Penulisan.
	\item Bab II Tinjauan Pustaka, berisi ilmu dan konsep yang mendukung
	pembahasan tesis ini beserta makalah mengenai pekerjaan sebelumnya
	dalam deteksi vandalisme di Wikipedia.
	\item Bab III Proses Deteksi Vandalisme, berisi tahap dalam persiapan
	data, fitur, sampel ulang, implemetentasi, pelatihan model dan
	pengujian.
	\item Bab IV Evaluasi, berisi penjelasan dari hasil penelitian.
	\item Bab V Kesimpulan, berisi rangkuman yang dapat diambil dari hasil
	penelitian ini beserta saran untuk pengembangan selanjutnya.
\end{enumerate}



%%
%% BAB II: TINJAUAN PUSTAKA
%%
Bab ini berisi ilmu dan konsep yang mendukung pem bahasan tesis ini beserta
makalah mengenai pekerjaan sebelumnya dalam deteksi vandalisme di Wikipedia.
Untuk konsep sampel ulang yang dibahas adalah metode \textit{Synthetic Minority
Oversampling Technique} (SMOTE) dan ekstensi dari SMOTE yaitu
\textit{Local Neighbourhood SMOTE} (LNSMOTE).
Konsep pengklasifikasi yang dibahas dan digunakan yaitu \textit{Random Forest}
(RF) dan ekstensinya yaitu \textit{Cascaded Random Forest} (CRF).


%%
%% BAB III: PROSES
%%
Pada bab ini dijelaskan proses dari pengolahan dataset dan implementasi
algoritma sampel ulang dan klasifikasi, dimulai dari persiapan
data, pembuatan dataset fitur, sampel ulang pada dataset fitur, dan
implementasi pengklasifikasi sehingga nantinya dapat digunakan untuk pelatihan,
pengujian dan analisis.
Alur dari proses secara umum dapat dilihat pada gambar \ref{fig:proses}.


%%
%% BAB IV: EVALUASI
%%
\chapter{Evaluasi}
\label{bab:04}
Dalam deteksi vandalisme, menemukan sebuah suntingan yang vandal dengan tingkat
positif yang tinggi lebih baik daripada salah klasifikasi atau terlewatnya
suntingan vandal tersebut dari pendeteksian.
Kesalahan klasifikasi, yang bukan vandalisme terdeteksi sebagai vandalisme,
tidak akan berpengaruh pada pembaca, tetapi terlewatnya suntingan yang vandal
bisa menyebabkan hilangnya informasi, kesalahan informasi, atau mengganggu
pembaca Wikipedia.
Oleh karena itu, hasil pelatihan dan pengujian dilihat dari laju
\textit{true-positive} pada performansi.
Laju \textit{true-positive}, atau dikenal juga dengan
\textit{true-positive rate} (TPR),
yaitu total jumlah sampel yang benar positif dibagi dengan
jumlah sampel yang terklasifikasi benar positif (\textit{true-positive})
ditambah dengan jumlah sampel positif terklasifikasi dengan negatif
(\textit{false-negative}).
TPR bernilai dengan rentang antara 0 sampai 1, dengan nilai yang mendekati 0
berarti performansi yang buruk dan nilai yang mendekati 1 berarti performansi
yang baik.

Pembuatan model klasifikasi dilakukan dengan cara menjalankan program
klasifikasi RF dan CRF pada masing-masing dataset fitur PAN-WVC-10 yang belum
disampel ulang, yang telah disampel ulang dengan SMOTE, dan yang telah disampel
ulang dengan LNSMOTE.
Setelah modelnya terbangun, model diuji dengan diberikan input dataset fitur
PAN-WVC-11. Hasil dari pengujian digunakan untuk analisis.

Lingkungan pelatihan dan pengujian dilakukan pada mesin Intel\textregistered\
 Core\texttrademark \ i7-4750HQ CPU 2,00 GHz, dengan jumlah \textit{RAM} 8
GB. Setiap pelatihan model dilakukan satu per satu untuk menghindari adanya
\textit{cache miss} yang berpengaruh pada kecepatan dan waktu pemrosesan.


	\section{Dataset Pelatihan dan Pengujian}
	\label{bab:04:dataset}
	Dataset yang digunakan untuk pelatihan model yaitu PAN-WVC-10 yang terdiri dari
tiga jenis yaitu dataset tanpa sampel ulang, dataset yang telah disampel ulang
dengan SMOTE, dan dataset yang telah disampel ulang dengan LNSMOTE.
Jumlah sampel pada dataset yang tidak disampel yaitu 2.394 positif dan 30.045
negatif dengan total 32.439 sampel.
Jumlah sampel positif pada dataset hasil sampel ulang dengan SMOTE yaitu 28.728
sampel dengan total 58.773 sampel.
Jumlah sampel positif pada dataset hasil sampel ulang dengan LNSMOTE yaitu
28.588 sampel dengan total 58.633 sampel.

\begin{table}[!ht]
\centering
\caption{Dataset untuk pelatihan dan pengujian.}
\begin{tabular}{|| c | l | c | c | c ||}
\hline
\multirow{2}{*}{Tipe} & \multirow{2}{*}{Mode sampel ulang} & \multicolumn{3}{c||}{Jumlah sampel} \\
\cline{3-5}
                         &                                       & Vandalisme & Biasa & Total \\
\hline
\hline
\multirow{3}{*}{Pelatihan} & -       &  2.394 & 30.045 & 32.439 \\
                              & SMOTE   & 28.728 & 30.045 & 58.773 \\
                              & LNSMOTE & 28.588 & 30.045 & 58.633 \\
\hline
Pengujian & - & 1.143 & 8.842 & 9.985 \\
\hline
\end{tabular}
\label{table:dataset}
\end{table}

Dataset yang digunakan untuk pengujian model yaitu PAN-WVC-11 yang terdiri dari
1.143 sampel positif dan 8.842 sampel negatif dengan total 9.985 sampel.
Jumlah fitur pada PAN-WVC-11 sama dengan PAN-WVC-10 yaitu 28 fitur (berikut
dengan fitur kelas).


	\section{Parameter Pengklasifikasi}
	\label{bab:04:parameter}
	Supaya konsisten antara pengklasifikasi, digunakan parameter umum yang sama
yaitu 200 pohon, 5 (dari $\sqrt{26}$) fitur acak, dan $ 64\% $ untuk
\textit{bootstrapping}.
Untuk klasifikasi CRF dilakukan tiga pemodelan dan pengujian dengan parameter
yang berbeda yaitu 200 tingkat dengan 1 pohon, 100 tingkat dengan 2 pohon, dan
50 tingkat dengan 4 pohon; dengan jumlah pohon yang tetap sama untuk ketiganya
yaitu 200.
Hal ini dilakukan untuk melihat pengaruh dari jumlah pohon terhadap tingkat dan
hasil klasifikasi.
Parameter lain pada pemodelan CRF yaitu nilai ambang batas TPR dan TNR diset
pada nilai $0,95$ dan $0,95$ untuk mendapatkan hasil klasifikasi yang bagus dan
jumlah pohon yang konsisten.


	\section{Hasil Deteksi}
	\label{bab:04:hasil_deteksi}
	\DTLsetseparator{;}
\DTLloaddb{stats}{../result/stats.csv}

\DTLmaxforcolumn{stats}{TPR}{\maxtpr}
\DTLminforcolumn{stats}{FPR}{\minfpr}
\DTLmaxforcolumn{stats}{TNR}{\maxtnr}
\DTLmaxforcolumn{stats}{Presisi}{\maxprec}
\DTLmaxforcolumn{stats}{F-Measure}{\maxfm}
\DTLmaxforcolumn{stats}{Akurasi}{\maxacc}
\DTLmaxforcolumn{stats}{AUC}{\maxauc}

\begin{table}[htbp]
\caption{Performansi Klasifikasi RF dan CRF}
\centering
\footnotesize
\begin{tabular}{l l r}
\hline
\textbf{Klasifikasi} &
\textbf{Dataset} &
\textbf{TPR}
\DTLforeach*{stats}{%
	\cl=Klasifikasi,%
	\ds=Dataset,%
	\tpr=TPR%
}{%
	\DTLifnullorempty{\cl}
		{\\ \cline{2-3}}
		{\\ \hline \hline}
	\DTLifnullorempty{\cl}
		{}
		{
			\multirow{3}{*}{\cl}
		}
	& \ds
	& \DTLifnumeq{\tpr}{\maxtpr}{\textbf{\tpr}}{\tpr}
}
\\
\hline
\end{tabular}
\label{tab:stats}
\end{table}


SMOTE rata-rata meningkatkan nilai TPR $0,19$ kali.
Sementara pada sampel ulang dengan LNSMOTE, rata-rata meningkatkan nilai TPR
$0,33$ kali.
Pengklasifikasi CRF dengan 200 tingkat 1 pohon pada dataset yang telah disampel
ulang dengan LNSMOTE memberikan nilai TPR paling tinggi.
Pengklasifikasi RF tanpa sampel ulang memberikan nilai TPR paling rendah.
Hasil pengujian secara keseluruhan dapat dilihat pada tabel \ref{tab:stats}

Dari segi kecepatan pelatihan model, pengklasifikasi CRF lebih cepat dari RF
baik pada semua dataset pelatihan.
Sebagai pembanding, dapat dilihat pada klasifikasi RF dan CRF 50 tingkat 4
pohon.
CRF tanpa sampel ulang lebih cepat 11 kali daripada RF, dan pada sampel ulang
SMOTE dan LNSMOTE, klasifikasi CRF 1,6 kali lebih cepat daripada RF.
Kecepatan pelatihan model untuk setiap pengklasifikasi dan dataset dapat
dilihat pada gambar \ref{graph:runtimes}.

\begin{figure}[htbp]
\centering
\begin{tikzpicture}[font=\scriptsize]
	\begin{axis}[
		width=7cm,
		ymax=250,
		ybar,
		ylabel=Running time (minutes),
		symbolic x coords={Without resampling, SMOTE, LNSMOTE},
		xtick=data,
		nodes near coords,
		every node near coord/.append style={font=\tiny},
		enlarge x limits=0.24,
		enlarge y limits=0,
		legend style={
			at={(0.5,-0.15)},
			anchor=north,
			legend columns=-1
		},
	]
		%% RF
		\addplot[pattern = dots] coordinates {
			(Without resampling, 52.5)
			(SMOTE, 217.7)
			(LNSMOTE, 199.5)
		};

		%% CRF-200-1
		\addplot[pattern=north east lines] coordinates {
			(Without resampling, 3.5)
			(SMOTE, 74.4)
			(LNSMOTE, 61.9)
		};

		%% CRF-100-2
		\addplot[pattern=horizontal lines] coordinates {
			(Without resampling, 3.6)
			(SMOTE, 74.6)
			(LNSMOTE, 67.9)
		};

		%% CRF-50-4
		\addplot coordinates {
			(Without resampling, 4.2)
			(SMOTE, 82.5)
			(LNSMOTE, 76.6)
		};
	\legend{RF, CRF-200-1, CRF-100-2, CRF-50-4}
	\end{axis}
\end{tikzpicture}
\caption{Running time for each classifer on different dataset.}
\label{graph:runtimes}
\end{figure}



	\section{Analisis}
	Pengaruh sampel ulang SMOTE dan LNSMOTE sama pada pengklasifikasi RF dan
CRF yaitu meningkatnya nilai TPR tapi dengan nilai peningkatan yang berbeda,
dengan LNSMOTE lebih tinggi dari daripada SMOTE.
Pada pengklasifikasi RF, sampel ulang dengan LNSMOTE meningkatkan nilai TPR
$0.7\%$ sedangkan SMOTE hanya $0.4\%$
Pada pengklasifikasi CRF, nilai peningkatan beragam bergantung kepada jumlah
pohon keputusan disetiap tingkatan.
Pada CRF 200 tingkat 1 pohon, peningkatan TPR dari yang tanpa sampel ulang
yaitu sekitar $0,02\%$, sedangkan pada CRF 100 tingkat 2 pohon peningkatan TPR
yaitu $0,14\%$, dan pada CRF 50 tingkat 4 pohon peningkatan TPR yaitu $0,28\%$.
Semakin banyak jumlah pohon pada setiap tingkatan maka semakin nilai TPR akan
semakin tinggi daripada yang tanpa sampel ulang.

%% Apakah ada outlier?

%% Kenapa LNSMOTE lebih baik?

%% Kenapa CRF menghasilkan TPR lebih tinggi?

Fokus dari pengklasifikasi CRF yaitu pada pembelajaran sampel negatif,
khususnya \textit{false-positive} yaitu sampel yang bukan vandal tapi
terdeteksi sebagai vandal.
Hal ini dapat diteliti lebih lanjut dengan melihat strategi \textit{bootstrap}
dari CRF.
Asumsikan $D$ adalah dataset pengujian yang berisi sampel positif dan negatif,
dan dataset $T$ yang berisi hanya sampel yang \textit{true-negative} (TP).
Pada setiap tingkatan CRF, setelah semua pohon dibangun, semua tingkatan pada
model diuji dengan dataset $D$ dan $T$.
Hasil pengujian pada dataset $D$ akan menghasilkan sampel yang tergolong ke
dalam benar positif (TP), benar terklasifikasi negatif (TN), dan yang salah
terklasifikasi (FP dan FN).
Sampel yang tergolong TN dihapus dari dataset pelatihan $D$ dan dimasukan ke
dataset $T$ yang hanya berisi sampel TN, sehingga dataset $D$ meninggalkan
hanya sampel yang tergolong TP, FP, dan FN.
Hasil pengujian pada dataset $T$ menghasilkan sampel yang tergolong benar
negatif (TN) dan salah positif (FP).
Sampel yang tergolong FP pada dataset $T$ kemudian dimasukan kembali ke dataest
$D$ untuk dipelajari kembali pada tingkatan selajutnya.
Berbeda dengan RF, yang mana tidak ada sampel yang dihapus saat
pelatihan, CRF membuat dataset yang tadinya condong pada kelas mayoritas
sedikit demi sedikit pada setiap tingkatan menjadi sama dan meningkatkan
performansi klasifikasi yang benar positif.



%%
%% BAB V: KESIMPULAN
%%
\chapter{Kesimpulan}

Rata-rata SMOTE menaikan nilai TPR $0,19$ kali dan nilai FPR $1,6$ kali.
Sementara pada sampel ulang dengan LNSMOTE, rata-rata menaikan nilai TPR $0,33$
dan juga menaikan nilai FPR $2,75$ kali.
Kedua sampel ulang tersebut cukup bagus dalam meningkatkan performansi laju
positif dengan konsekuensi juga meningkatnya laju \textit{false-positive},
pengaruh ini lebih terasa pada sampel ulang LNSMOTE.
Secara keseluruhan performansi dari sampel ulang SMOTE lebih baik dari
LNSMOTE untuk pengklasifikasi CRF.
Hal ini disebabkan karena sifat algoritma dari CRF yang berfokus pembelajaran
sampel yang negatif, bukan pada sampel yang positif, sehingga menambahkan
sampel sintetis menyebabkan pembelajar menjadi \textit{over-fitting} pada
sampel positif.
Pengaruh menarik lainnya yaitu pada pengklasifikasi CRF, dengan menggunakan
jumlah pohon lebih sedikit pada setiap tingkatan menghasilkan performansi yang
hampir sama dengan melakukan sampel ulang pada dataset asli, seperti yang
terlihat pada performansi CRF-100-2 tanpa sampel ulang hampir sama dengan CRF
50 tingkat 4 pohon dengan sampel ulang SMOTE.

Untuk model klasifikasi vandalisme yang terbaik tanpa sampel ulang dihasilkan
dari pengklasifikasi CRF dengan 200 tingkat dan 1 pohon, untuk model terbaik
pada dataset yang telah disampel ulang dengan SMOTE yaitu CRF dengan 100
tingkat dan 1 pohon, dan untuk model terbaik dari sampel ulang LNSMOTE yaitu
CRF dengan 200 tingkat 1 pohon. Secara keseluruhan model terbaik yaitu dari CRF
100 tingkat dan 2 pohon yang telah disampel ulang dengan SMOTE.
Selain performansi yang lebih baik, pengklasifikasi CRF juga lebih cepat $1,6$
kali dalam pembentukan model daripada RF pada dataset yang telah disampel
ulang.

\section{Kontribusi}

Kontribusi dari karya tulis ini selain membantu menemukan pengklasifikasi yang
lebih baik dalam mendeteksi vandalisme juga memberikan kerangka kerja untuk
menciptakan dan mengembangkan fitur vandalisme dari dataset mentah tanpa harus
mulai dari awal untuk dapat digunakan dalam penelitian selanjutnya atau
digunakan langsung pada kasus asli. Selain itu juga menyediakan pustaka untuk
pengolahan data dan fungsi pembelajaran mesin terutama pengklasifikasi
\textit{Cascaded Random Forest} yang belum ada implementasi secara terbuka pada
program terkenal seperti \textit{Weka}, \textit{Scikit-Learn}, atau \textit{R}.

\section{Pekerjaan Selanjutnya}

Semua pelatihan model dalam karya tulis ini masih menggunakan algoritma
pengklasifikasi RF dan CRF dalam bentuk serial, yang mana satu pohon dibangun
satu per satu bergantian atau pada saat klasifikasi setiap sampel dimasukan ke
dalam pohon secara bergantian untuk mendapatkan kelasnya.
Penggunaan algoritma paralel, seperti dalam pembentukan dua atau empat pohon
bersamaan dan klasifikasi sampel (pengumpulan \textit{vote} di setiap pohon)
bersamaan, bisa membantu mempercepat pelatihan, pengujian dan hasil
klasifikasi.
Pada domain pembelajaran mesin, hal menarik yaitu adanya algoritma
\textit{eXtreme Gradient Boostring} (XGBoost) \cite{chen2016xgboost} yang
mungkin bisa diuji dan diterapkan untuk meningkatkan deteksi vandalisme.


%%
%% DAFTAR PUSTAKA
%%
\clearpage
\addcontentsline{toc}{chapter}{\tDaftarPustaka}
\printbibliography

%%
%% LAMPIRAN
%%
\vspace*{\fill}
\begin{center}
	\begin{minipage}{\textwidth}
		\centering
		\textbf{\Large\tUpLampiran}
		\addcontentsline{toc}{chapter}{\tUpLampiran}
	\end{minipage}
\end{center}
\vfill

\myappendix{Daftar Kategori Kata dan Token}
	\label{lampiran:daftar_token_dan_kata}

Lampiran ini berisi daftar kategori kata dan token yang digunakan untuk
penghitungan fitur vandalisme.

%%{{{ TOKEN WIKI
\subappendix{Token Wiki}
	\label{lampiran:words_wiki_token}

Kategori sintaks wiki digunakan dalam penghitungan fitur \textit{token umum}.
Berikut daftar token yang digunakan,

	\lstinputlisting[style=data,basicstyle=\scriptsize\ttfamily]
		{lampiran/token_wiki}
%%}}}

%%{{{ KATA VULGAR
\subappendix{Kategori Kata Vulgar}
\label{lampiran:words_vulgar}

Kategori kata vulgar yaitu kata yang kasar dan menghina.
Berikut daftar kata vulgar yang digunakan,

	\lstinputlisting[style=data,basicstyle=\scriptsize\ttfamily]
		{lampiran/kata_vulgar}
%%}}}

%%{{{ KATA SUBJEK
\subappendix{Kategori Kata Subjek}
\label{lampiran:words_pronoun}

Kategori kata subjek yaitu kata yang merujuk pada pihak pertama dan kedua yang
digunakan dalam percakapan sehari-hari.
Berikut daftar kata subjek yang digunakan,

	\lstinputlisting[style=data,basicstyle=\scriptsize\ttfamily]
		{lampiran/kata_subjek}
%%}}}

%%{{{ KATA BIAS
\subappendix{Kategori Kata Bias}
\label{lampiran:words_bias}

Kategori kata bias berisi kata yang mengesankan penekanan yang berlebihan
sehingga cenderung membuat bias.
Berikut daftar kata bias yang digunakan,

	\lstinputlisting[style=data,basicstyle=\scriptsize\ttfamily]
		{lampiran/kata_bias}
%%}}}

%%{{{ KATA PORNOGRAFI
\subappendix{Kategori Kata Pornografi}
\label{lampiran:words_sex}

Berikut daftar kata pornografi yang digunakan,

	\lstinputlisting[style=data,basicstyle=\scriptsize\ttfamily]
		{lampiran/kata_pornografi}
%%}}}

%%{{{ KATA BURUK
\subappendix{Kategori Kata Buruk}
\label{lampiran:words_bad}

Kategori kata buruk mengikutkan kata-kata yang bersifat negatif, bukan vulgar
dan bukan pornografi, atau kata slang yang digunakan dalam sehari-hari.
Berikut daftar kata buruk yang digunakan,

	\lstinputlisting[style=data,basicstyle=\scriptsize\ttfamily]
		{lampiran/kata_buruk}

%%}}}


\myappendix{Contoh Dataset}
	\label{lampiran:dataset}

Lampiran ini berisi contoh dataset yang digunakan pada pelatihan dan pengujian
model klasifikasi.

%%{{{ WVC-10 SUNTINGAN
\subappendix{Dataset Suntingan PAN-WVC-10}
	\label{lampiran:dataset_wvc10_suntingan}

Berikut contoh dataset suntingan pada PAN-WVC-10,

\lstinputlisting[style=data,basicstyle=\scriptsize\ttfamily]
	{lampiran/dataset_wvc10_suntingan}
%%}}}

%%{{{ WVC-10 ANOTASI
\subappendix{Dataset Anotasi PAN-WVC-10}
	\label{lampiran:dataset_wvc10_anotasi}

Berikut contoh dataset anotasi pada PAN-WVC-10,

\lstinputlisting[style=data,basicstyle=\scriptsize\ttfamily]
	{lampiran/dataset_wvc10_anotasi}
%%}}}

%%{{{ WVC-10 GABUNGAN
\subappendix{Dataset Gabungan PAN-WVC-10}
	\label{lampiran:dataset_wvc10_gabungan}

Berikut contoh dataset PAN-WVC-10 yang telah dibersihkan dan ditambahkan dengan
atribut \textit{additions} dan \textit{deletions},

\lstinputlisting[
	style=data
,	basicstyle=\scriptsize\ttfamily
,	texcl=true
]{lampiran/dataset_wvc10_gabungan}
%%}}}

%%{{{ WVC-11 MENTAH
\subappendix{Dataset Mentah PAN-WVC-11}
	\label{lampiran:dataset_wvc11_mentah}

Berikut contoh dataset mentah dari PAN-WVC-11 Bahasa Inggris,

\lstinputlisting[
	style=data
,	basicstyle=\scriptsize\ttfamily
,	texcl=true
]{lampiran/dataset_wvc11_mentah}
%%}}}

%%{{{ WVC-11 GABUNGAN
\subappendix{Dataset Gabungan PAN-WVC-11}
	\label{lampiran:dataset_wvc11_gabungan}

Berikut contoh dataset PAN-WVC-11 yang telah dibersihkan dan ditambahkan
atribut \textit{additions} dan \textit{deletions},

\lstinputlisting[
	style=data
,	basicstyle=\scriptsize\ttfamily
,	texcl=true
,	alsoletter={\\}
,	morekeywords={\"}
]{lampiran/dataset_wvc11_gabungan}
%%}}}


\end{document}
