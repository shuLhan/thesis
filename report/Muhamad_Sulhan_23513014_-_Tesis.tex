\documentclass[12pt,a4paper,titlepage,oneside]{report}

%%{{{ packages
%% font encoding.
\usepackage[utf8]{inputenc}
\usepackage[T1]{fontenc}
\usepackage{lmodern}
\usepackage{couriers}
%% Customize chapters.
\usepackage{titlesec}
%% bibliography.
\usepackage[
	backend=bibtex
,	style=ieee
,	sorting=nyt
]{biblatex}
%% \includegraphics{name}
\usepackage{graphicx}
%% \url{something}
\usepackage{url}
%% \multirow{package}{width}{text}
\usepackage{multirow}
%% \cellcolor
\usepackage[table]{xcolor}
%% \caption{title}
\usepackage{caption}
\usepackage{subcaption}
%% \forloop
\usepackage{forloop}
%% fix underfull on footnote with URL.
\usepackage{ragged2e}
%% source code listing
\usepackage{listings}
%% \printindex
\usepackage{makeidx}
%% table of content
\usepackage{tocloft}
%% text emphasis, including strikeout.
\usepackage[normalem]{ulem}
%% mathematics
\usepackage{mathtools}
%% arithmetic
\usepackage{calc}
%% algorithm
\usepackage{algorithm}
\usepackage{algpseudocode}
%% multiple columns
\usepackage{multicol}
%% Change the margin
\usepackage[a4paper]{geometry}
\geometry{
	a4paper,
	top=3cm,
	right=3cm,
	bottom=3cm,
	left=4cm
}
%%
\usepackage{parskip}
%% Package for reading CSV to database.
\usepackage{datatool}
%% Package for scatter and line plot.
\usepackage{dataplot}
%% Long table
\usepackage{longtable}
%% Tikz
\usepackage{tikz}
\usetikzlibrary{backgrounds}

\usepackage{pgfplots}

%% MnSymbol
\usepackage{MnSymbol}

%% Compact list.
\usepackage{paralist}
%%}}}

%%{{{ hyphenation: sorted in ascending.
%%
\hyphenation{
	Ja-nu-a-ri
	SIGKDD
	Wiki-pedia
	a-kan
	a-ku-ra-si
	ang-ka
	ba-gai-ma-na
	bayes-ian
	ber-kas
	ber-ma-sa-lah
	ber-mak-na
	bi-a-ya
	da-lam
	data-set
	de-ngan
	di-ha-sil-kan
	di-pi-lih
	di-sing-kat
	di-tam-bah-kan
	dis-krit
	fung-si
	ga-bung-an
	ke-las
	ke-le-mah-an
	ke-mung-ki-nan
	ke-tak-se-imbang-an
	lan-guage
	ma-yo-ri-tas
	me-laku-kan
	me-me-rik-sa
	me-mi-lih
	me-ne-rap-kan
	meng-a-pli-ka-si-kan
	me-ning-kat-kan
	me-nye-dia-kan
	me-nye-im-bang-kan
	me-sin
	me-thod
	me-to-de
	mem-vi-sua-li-sa-si
	meng-gu-na-kan
	meng-hi-lang-kan
	meng-hu-bung-kan
	meng-i-kut-kan
	meng-i-kuti
	meng-im-ple-men-ta-si-kan
	meng-in-di-ka-si-kan
	mi-sal-nya
	mung-kin
	o-ver-sam-pling
	pa-ra-lel
	pe-la-ti-han
	pe-mi-sah
	pe-nan-da
	pe-ne-li-ti-an
	pe-nu-li-san
	pe-nyun-ting
	pem-ban-ding
	pen-de-kat-an
	peng-kla-si-fi-ka-si
	peng-a-pli-ka-si-an
	per-for-man-si-nya
	po-ten-si-al
	pro-ba-bi-li-tas
	pro-ses
	sam-pel
	se-im-bang
	se-jum-lah
	sun-ting-an
	ting-kat
	un-der-sam-pling
	wa-lau-pun
}
%%}}}

%%{{{ override default latex setting.
%%

%% Space between paragraphs.
\setlength{\parskip}{1.5em}

%% Add dot to TOC.
\renewcommand{\cftsecleader}{\cftdotfill{\cftdotsep}}
\renewcommand{\contentsname}{}

\lstset{
	basicstyle=\scriptsize\ttfamily
,	breaklines=true
,	stringstyle=\scriptsize\ttfamily
,	numbers=left
,	numberstyle=\tiny\ttfamily
,	numbersep=5pt
,	tabsize=4
,	frame=single
}

\makeatletter
\def\lst@lettertrue{\let\lst@ifletter\iffalse}
\makeatother

%% Format chapter and section.
%%
%%% Set chapter name to Bab.
\titleformat{\chapter}[hang]
{\bfseries\large\centering}
{Bab \thechapter}
{1em}
{}

\titleformat{\section}[hang]
{\bfseries\large}
{\thesection}
{1em}{}

%% Set spacing for sections.
\titlespacing{\chapter}{0ex}{0ex}{1.5em}
\titlespacing{\section}{0ex}{0ex}{0em}

%% Set roman on chapter number.
\def\thechapter{\Roman{chapter}}

%% Alter latex default title on table.
\captionsetup[table]{name=Tabel}
\captionsetup[figure]{name=Gambar}

\renewcommand{\arraystretch}{1.5}
\setlength{\tabcolsep}{3pt}

%% Change bibliography title.
\defbibheading{bibliography}{\centerline{
	\textbf{DAFTAR PUSTAKA}}
}

%%% uncomment this to show overrule in black box
\overfullrule=2cm

%% Algorithmicx
\makeatletter
\renewcommand{\ALG@beginalgorithmic}{\footnotesize}
\makeatother

%% multicolumn setting
\setlength{\columnsep}{1cm}

%% pgfplots setting.
\pgfplotsset{
	/pgf/number format/read comma as period,
	/pgf/text mark as node=false,
	table/col sep=semicolon,
	xmax=1,
	xmin=0,
	ymax=1,
	ymin=0,
	xtick distance=0.2,
	ytick distance=0.2,
	grid=major,
	cycle list name=linestyles,
}
%%}}}

%%{{{ variables
%%
\newcommand{\mytitle}{Deteksi Vandalisme pada Wikipedia Bahasa Inggris menggunakan klasifikasi Cascaded Random Forest}
\newcommand{\myname}{Muhamad Sulhan}
\newcommand{\mysid}{23513014}
\newcommand{\myadvisorname}{Dwi Hendratmo Widyantoro}
\newcommand{\myadvisorid}{196812071994021001}
\newcommand{\mydept}{Program Studi Magister Informatika}
\newcommand{\itb}{Institut Teknologi Bandung}

%%% My images directory
\graphicspath{{../images/}}
\newcommand{\myitbcover}{ITB-logo-hitam}

%%% My bibligraphy file
\addbibresource{bibliography.bib}
%%}}}

%%{{{ document's meta-data
%%
\author{\myname}
\title{\mytitle}
%%}}}

%%{{{ document's macros
%%
%%% two column signature.
\def\myadvisorsig#1{%
	\vbox{\hsize=6cm
		\textbf{#1}\\
		\addvspace{2cm}%
		\hbox to \hsize{%
			\strut\hfil%
			\myadvisorname%
			\hfil%
		}
		\hrule\kern1ex
		\hbox to \hsize{%
			\strut\hfil%
			NIP\hspace{1ex}\myadvisorid%
			\hfil%
		}
	}
}

%%% one column signature.
\def\mysignature#1#2#3{%
	\vbox{
		\textbf{#1}\\
		\addvspace{2cm}%
		\hbox to \hsize{%
			\strut\hfil%
			{#2}%
			\hfil%
		}
		\makebox[6cm][c]{
			\hrulefill
		}
		\hbox to \hsize{%
			\strut\hfil%
			NIP\hspace{1ex}{#3}%
			\hfil%
		}
	}
}

%%% source code listing
\lstdefinelanguage{go}
{
	morekeywords={package,import,const,func,for,type,var,struct}
,	sensitive=true
,	morecomment=[l]{//}
,	morecomment=[s]{/*}{*/}
}
\lstdefinestyle{go}{%
	language=go
,	keywordstyle=\color{black}\bfseries
,	commentstyle=\color{gray}
,	breakatwhitespace=true
,	lineskip={-2.5pt}
}
\newcommand{\includecodego}[2][c]{
	\lstinputlisting[caption=#2,escapechar=,style=go]
		{/home/ms/go/work/src/github.com/shuLhan/#2}
}

%%% data listing
\lstdefinestyle{data}{%
	breakatwhitespace=false
,	breakautoindent=false
,	literate={\,}{}{0\discretionary{,}{}{,}},
}
\newcommand{\includedata}[2][c]{
	\lstinputlisting[caption=#2,style=data,linerange={1-10}]
		{/home/ms/go/work/src/github.com/shuLhan/#2}
}
%%}}}


\newcommand{\daftarisi}{DAFTAR ISI}
\renewcommand{\contentsname}{
	\chapter*{\daftarisi}\label{chapter:daftarisi}
	\vspace{-1.5em}
	\addcontentsline{toc}{chapter}{\daftarisi}
}

\newcommand{\daftargambar}{DAFTAR GAMBAR DAN ILUSTRASI}
\renewcommand{\listfigurename}{
	\chapter*{\daftargambar}
	\vspace{-1.5em}
	\addcontentsline{toc}{chapter}{\daftargambar}
}

\newcommand{\daftartabel}{DAFTAR TABEL}
\renewcommand{\listtablename}{
	\chapter*{\daftartabel}
	\vspace{-1.5em}
	\addcontentsline{toc}{chapter}{\daftartabel}
}

\begin{document}

\thispagestyle{empty}
\begin{singlespacing}
\begin{center}
\textbf{\large
	\MakeUppercase{\judul} \\
	\vfill
	TESIS \\
	\bigskip
	{\normalsize
		Karya tulis sebagai salah satu syarat \\
		untuk memperoleh gelar Magister dari \\
		\itb{} \\
	}
	\vfill
	{\normalsize Oleh} \\
	\MakeUppercase{\myname{}} \\
	NIM: \mysid{} \\
	\mydept{} \\
	\vfill
	\includegraphics[width=2.35cm,height=3.5cm]{\myitbcover} \\
	\vfill
	\normalsize
	\MakeUppercase{\tfakultas} \\
	\MakeUppercase{\itb{}} \\
	Juni 2016 \\
}
\end{center}
\end{singlespacing}


\newpage
\pagenumbering{roman}
\newcommand{\abstrak}{ABSTRAK}
\chapter*{\abstrak}\label{chapter:abstrak}
\addcontentsline{toc}{chapter}{\abstrak}
\begin{center}
\textbf{\large
	\MakeUppercase{\mytitle{}} \\
	\bigskip
	\textnormal{Oleh} \\
	\myname{} \\
	NIM: \mysid{} \\
	(\mydept{}) \\
}
\end{center}

\bigskip
\bigskip
\bigskip

Wikipedia.org adalah ensiklopedia daring yang dapat disunting oleh siapa saja.
Sifat wiki ini dapat mempercepat perbaikan dan pertumbuhan artikel,
kekurangannya yaitu dapat menimbulkan vandalisme dalam bentuk suntingan dengan
informasi yang salah, penghapusan, iklan, atau teks yang tidak bermakna.
Tesis ini membahas deteksi vandalisme menggunakan pembelajaran mesin
dengan menerapkan metode klasifikasi
\textit{Cascaded Random Forest} (CRF)
yang dilatih pada dataset
\textit{Wikipedia Vandalism Corpus 2010}
yang telah
disampel ulang dengan
\textit{Local Neighbourhood Synthetic Minority Over-sampling Technique}
(LNSMOTE).
Kedua metode tersebut dibandingkan dengan klasifikasi
\textit{Random Forest} (RF)
dan teknik sampel ulang SMOTE.
Hasil sampel ulang dan pemodelan diuji dengan dataset
\textit{Wikipedia Vandalism Corpus 2011}
menunjukan hasil sampel ulang dengan LNSMOTE meningkatkan laju
\textit{true-positive} lebih tinggi daripada dataset yang tidak disampel ulang
dan yang disampel ulang dengan SMOTE pada kedua pengklasifikasi.
Dari segi performansi, CRF dengan sampel ulang LNSMOTE dengan 200 tingkat dan 1
pohon memberikan hasil yang paling baik dengan nilai TPR yaitu $0,9904$.
Dari segi kecepatan pemodelan pengklasifikasi CRF lebih cepat 1,6 kali daripada
RF pada data yang telah disampel ulang.

Kata kunci: wikipedia, vandalisme, dataset timpang, dataset sampel ulang,
\textit{cascaded random forest}


\tableofcontents

\listoffigures

\listoftables

\newpage
\pagenumbering{arabic}

%%
%% BAB I: PENDAHULUAN
%%
\chapter{Pendahuluan}
\label{chapter:pendahuluan}

\section{Latar Belakang}
\label{sec:latar_belakang}
Wikipedia.org adalah ensiklopedia daring dan terbuka, yang mana artikel di
Wikipedia merupakan hasil kolaborasi para penyunting dari seluruh dunia.
Terbuka artinya siapa pun dapat menyunting artikel tanpa perlu melakukan
registrasi terlebih dahulu.
Ensiklopedia daring ini memiliki artikel dari berbagai bahasa, dari bahasa umum
dunia seperti Bahasa Inggris, sampai bahasa daerah seperti Bahasa Jawa.

Vandalisme menurut Kamus Besar Bahasa Indonesia daring adalah,
1) perbuatan merusak dan menghancurkan hasil karya seni dan barang berharga
lainnya;
2) perusakan dan penghancuran secara kasar dan ganas.
Dalam konteks Wikipedia.org, vandalisme dapat berbentuk suntingan yang mengubah
konten dari artikel sehingga memberikan isi yang salah, penghapusan secara
menyeluruh, penghapusan sebagian, isi yang menghina, iklan, dan/atau teks yang
tidak ada maknanya.

Jumlah artikel Bahasa Inggris pada situs en.wikipedia.org pada bulan Juli 2015
yaitu sebanyak 4,932,627 artikel, dengan pengguna aktif, atau disebut juga
penyunting, sebanyak 31,369 orang.
Berarti, jika diasumsikan semua penyunting benar aktif, maka setiap pengguna
aktif harus mengawasi kurang lebih 157 artikel.
Menemukan dan memperbaiki vandalisme tersebut dapat mengganggu penyunting dari
menulis artikel dan pekerjaan penting lainnya, dan membuat pembaca bisa
mendapatkan informasi yang salah atau tidak mendapatkan informasi sama sekali.


\section{Rumusan Masalah}
\label{sec:rumusan_masalah}
Tesis ini mencoba menjawab permasalahan dataset yang tidak seimbang pada
PAN-WVC-10 yang menyebabkan performansi deteksi yang rendah dan condong pada
kelas mayoritas dengan mengkaji teknik sampel ulang dan pengklasifikasi yang
belum pernah digunakan sebelumnya pada korpus tersebut.
Teknik sampel ulang yang digunakan yaitu LNSMOTE yang diajukan oleh
\textcite{maciejewski2011local}
dan teknik pengklasifikasi yang digunakan yaitu \textit{Cascaded Random Forest}
(CRF) yang diajukan oleh \textcite{baumann2013cascaded}.


\section{Tujuan}
\label{sec:tujuan}
%%
%% SECTION: Tujuan
%%
\section{Tujuan}\label{sec:tujuan}

Motivasi dari tesis ini adalah untuk mendeteksi vandalisme pada penyuntingan
artikel di situs Wikipedia untuk membantu editor Wikipedia dalam mempermudah
menentukan hasil suntingan artikel yang berpotensi berisi vandal sehingga dapat
dengan cepat mengembalikan ke isi sebelumnya.

\section{Batasan Masalah}
\label{sec:batasan_masalah}
Tesis ini hanya melakukan analisis untuk artikel Wikipedia Bahasa Inggris yang
terdapat pada korpus PAN-WVC-10 dan PAN-WVC-11.
Dataset yang digunakan untuk sampel ulang dan pelatihan model yaitu PAN-WVC-10.
Dataset yang digunakan dalam melakukan pengujian yaitu PAN-WVC-11.
Teknik sampel ulang yang dilakukan yaitu LNSMOTE yang dibandingkan dengan
SMOTE.
Teknik pengklasifikasi yang dijadikan dalam pelatihan yaitu CRF yang
dibandingkan dengan pengklasifikasi RF.


\section{Metodologi}
\label{sec:metodologi}
%%
%% SECTION: Metodologi
%%
\section{Metodologi}\label{sec:metodologi}

Penelitian ini dikembangkan dengan menggunakan metodologi kuantitatif dengan tahapan sebagai berikut,
\begin{itemize}
	\item \textbf{Studi Literatur}. Peneliti membaca beberapa penelitian yang sebelumnya telah dilakukan dalam deteksi vandalisme pada Wikipedia. Dari hasil penelitian tersebut penulis dapat melihat kelemahan dan potensi ke depan yang dapat dikembangkan, sehingga menjadi rumusan masalah dalam penulisan tesis ini. Tahapan selanjutnya dari studi literatur yaitu mengkaji sumber yang berkaitan dengan metode yang digunakan dalam pendeteksian vandalisme dalam makalah ini.
	\item \textbf{Persiapan Data dan Lingkungan Penelitian}. Dalam tahapan ini peneliti mempersiapkan data dan lingkungan pengembangan, seperti persiapan aplikasi basis data, pengaturan bahasa pemrograman, dan lainnya; yang diperlukan nantinya dalam melakukan analisis, implementasi, dan pengujian.
	\item \textbf{Analisis}. Pada tahap ini peneliti melihat data dan menentukan fungsi-fungsi yang akan diterapkan dalam implementasi untuk mendapatkan hasil yang ditujukan. Tahap ini bisa terjadi berulang kembali setelah implementasi.
	\item \textbf{Implementasi}. Tahap implementasi dalam makalah ini berupa proses pemuatan data, penerapan fungsi dalam bahasa pemrograman, dan pembuatan lingkungan pengujian.
	\item \textbf{Pengujian}. Setelah tahap implementasi selesai, fungsi dari pendeteksian vandalisme akan dilakukan secara \textit{offline}. Data dibagi dalam dua bagian waktu, sebelum $ t_{b} $ dan sesudah $ t_{s} $, kemudian fungsi deteksi vandalisme dijalankan untuk data $t_{b}$.
	\item \textbf{Evaluasi}. Tahap ini membandingkan hasil dari pengujian pada data $t_{b}$ dengan data sesudah $t_{s}$.
	
\end{itemize}

\section{Sistematika Penulisan}
\label{sec:sistematika_penulisan}
Laporan tesis ini dibagi menjadi beberapa bab berikut,
\begin{enumerate}
	\item Bab I Pendahuluan, berisi Latar Belakang, Rumusan Masalah,
	Tujuan, Batasan Masalah, Metodologi, dan Sistematika Penulisan.
	\item Bab II Tinjauan Pustaka, berisi ilmu dan konsep yang mendukung
	pembahasan tesis ini beserta makalah mengenai pekerjaan sebelumnya
	dalam deteksi vandalisme di Wikipedia.
	\item Bab III Proses Deteksi Vandalisme, berisi tahap dalam persiapan
	data, fitur, sampel ulang, implemetentasi, pelatihan model dan
	pengujian.
	\item Bab IV Evaluasi, berisi penjelasan dari hasil penelitian.
	\item Bab V Kesimpulan, berisi rangkuman yang dapat diambil dari hasil
	penelitian ini beserta saran untuk pengembangan selanjutnya.
\end{enumerate}


%%
%% BAB II: TINJAUAN PUSTAKA
%%
\chapter{Tinjauan Pustaka}

\section{SMOTE}
Metode \textit{Synthetic Minority Over-sampling Technique} (SMOTE)
\cite{chawla2002smote} menggunakan pendekatan \textit{over-sampling} yang mana
kelas minoritas ditambah dengan membuat sampel "sintetis" bukan dengan
mengganti sampel dari kelas mayoritas menjadi kelas minoritas.
Sampel sintetis dibuat lebih kurang lewat aplikasi, dengan beroperasi pada
"ruang fitur" bukan pada "ruang data".
Kelas minoritas ditambah dengan mengambil setiap sampel-sampel dari kelas
minoritas dan membuat sampel sintetis diantara segmen garis yang menggabungkan
setiap/semua \textit{k-nearest-neighbors} dari kelas minoritas.
Bergantung kepada jumlah \textit{over-sampling} yang dibutuhkan, instan dari
\textit{k-nearest-neighbors} dipilih secara acak.

\begin{figure}[b]
	\centering
	\includegraphics[keepaspectratio=true,scale=0.6]{SMOTE-example}
	\caption{Ilustrasi pembuatan sampel sintetis pada SMOTE.
\textit{p} adalah sampel minoritas, \textit{n} adalah salah satu
\textit{k-nearest-neighbors} dari \textit{p}.
Sampel sintetis yang baru akan berada digaris antara \textit{p} dan \textit{n}.
	}
	\label{fig:smote}
\end{figure}

Sampel sintetis dibuat dengan cara berikut,
\begin{itemize}
	\item Hitung selisih antara vektor fitur (sampel) dengan tetangga
	terdekatnya.
	\item Kalikan selisih tersebut dengan angka ril acak antara 0 sampai 1,
	dan
	\item tambahkan hasilnya ke vektor fitur.
\end{itemize}

Cara ini membuat sampel secara acak pada segmen garis antara dua fitur yang
terpilih, seperti yang terlihat pada gambar \ref{fig:smote}.
Pendekatan ini secara efektif mendorong wilayah pembelajaran dari kelas
minoritas menjadi lebih besar tanpa menyebabkan \textit{overfitting}.

Ambil contoh sebuah sampel (6,4) dan (4,3) sebagai tetangga terdekatnya.
(6,4) adalah sampel yang akan dicari \textit{k-nearest-neighbors}-nya.
(4,3) adalah salah satu dari \textit{k-nearest-neighbors}-nya.
Misalkan,
\[
\begin{matrix}
f1\_1 = 6 & f1\_2 = 4 \\
f2\_1 = 4 & f2\_2 = 3
\end{matrix}
\]
\[
\begin{matrix}
f2\_1 - f1\_1 = 4 - 6 = -2 \\
f2\_2 - f1\_2 = 3 - 4 = -1
\end{matrix}
\]

Sampel baru dihasilkan dengan,
\[
(f1', f2') = (6,4) + random(0-1) * (-2,-1)
\]

Fungsi \texttt{random(0-1)} menghasilkan bilangan ril acak dari 0 sampai 1.


\section{LNSMOTE}
SMOTE method has several weakness.
First, all sample from minority class is used, this could be a problem because
not all the samples have equal benefit for learning.
Minority sample in the boundary region between minority and majority class
usually result in misclassified rather than sample located in the center of
region, while sample that located in the center may give a little contribution
to classifier.
One of the method to overcome this problem is by using sample in boundary of
minority class instead in the center which proposed by Han et al.
\cite{han2005borderline}, called Borderline-SMOTE.

Another weakness of SMOTE method is overgeneralization, where their method does
not take into consideration the distribution of minority sample in majority
class, or the outliers.
Maciejewski and Stefanowski \cite{maciejewski2011local}
introduced an extension of SMOTE called Local Neighbourhood
SMOTE (LNSMOTE) \cite{maciejewski2011local} by combining Borderline-SMOTE
with modified version of Safe-Level SMOTE (SL-SMOTE)
\cite{bunkhumpornpat2009safe}.

In SL-SMOTE, majority samples is taken into consideration before creating
synthetic sample by calculating a coefficient called safe-level.
For each minority sample, count the number of their $k$ nearest neighbours (KNN).
If KNN value is close to 0, then the sample will be considered as noise.
If KNN value is close to $k$, then the sample can be said in safe region in
minority class.
The main idea was to create synthetic sample that close to safe region.

The SL-SMOTE strategy has a problem especially when class distribution is bias
in which the minority class spread into small sub-region with low number
cardinality.
In this situation, creating synthetic sample with SLMOTE will cause an overlap
between class.
This problem is due to SL-SMOTE find only KNN for minority class.
If the sample candidate does not located in region with densed minority class,
then some of their neighbours could be far from sample candidate or surrounded
by majority class samples.
LNSMOTE overcome this overlap problem by taking into consideration the local
neighbourhood of minority sample candidate that can provide the number of
majority class around each of them.


\section{\textit{Random Forest}}
\textit{Random Forest} (RF)
is a combination of several decision tree such that each tree depends on the
value of random vector sampled independently and with equal distribution for
all trees in the forest
\cite{breiman2001random}.

There are three common paremeter in building RF.
The first parameter is the number of trees in forest ($n$),
the other two parameters are percentage of samples ($b$) and number of random
features ($m$) for building the tree.
All of the parameters are set before building each tree and their value is
constant.

Percentage of sample for training that was selected randomly usually two third
from overall samples, which left one third of them as out of bag (OOB) samples.
For the number of random features $m$, the common value is the square root or
log of all features \cite{breiman2001random}.

Procedure to build RF is as follows.
Let $S$ be a training set.
After their parameter has been set, when building each tree, take $b$ samples
randomly from $S$ without replacement (sample that get selected can be picked
again in the next iteration).
This process also know as bootstrapping.
Samples that does not included in $b$ is called out-of-bag (OOB), which can be
used to calculate the misclassification rate.
From $b$ samples, take $m$ random features, and then build the tree using $b$
samples with $m$ number of features without pruning.
Repeat the process until the $n$-tree has been built.

Classification process on RF proceed as follows.
Given test set $T$, with the same number of features with $S$.
For each sample $t$ in $T$, insert the sample $t$ into each tree and collect
their classification result.
After $n$ trees or $n$ number of classes, compute the majority class from all
tree classification.


\section{\textit{Cascaded Random Forest}}
Most of ensemble learning algorithm is incapable of handling imbalance training
data.
Inequality between positive and negative class usually result in low
detection accuracy.
A simulation run by Strobel et al. \cite{strobl2007bias} showed that RF skewed
in favor of majority class.
Another drawback of RF is after learning several trees, RF gradually reach its
peak, such that the classifier can not increase their detection sensitivity or
decrease their false-positive rate.

Viola and Jones proposed a detection algorithm based on AdaBoost with
cascade structure \cite{viola2004robust}.
Cascade structure motivated by assumption that it is more easy to reject a
negative sample than finding a positive one.
Viola and Jones combines several strong classifiers in several independent
stages with condition that each stage can reject a sample, so to classify a
sample as positive then all stages must be passed.
Due to rejection on early stages, computation time will be decreased.
In addition, to get better training result, Viola and Jones propose a bootstrap
strategy by deleting samples classified as true negative.
The reduced training set then refilled with sample that mis-classified, or
false-positive samples
\cite{viola2004robust}.

A cascade classifier consists of several number of stages with increasing
complexity.
Each stage have minimum one independent classifier.
Classifier added into stages until the value of true-positive and true-negative
threshold is reached.
The advantage of cascade structure is a vast number of samples can be
distributed between stages, decreasing false-positive value and shortening
computation time when training and classifying.

Baumann uses this method with RF and propose Cascaded Random Forest (CRF) which
is a combination of RF classifier with cascade structure, where in each stage
several decision tree is build with bootstrap strategy, this leads increased
learning on positive sample and the drawback of imbalanced dataset can be
avoided.
\cite{baumann2013cascaded}

CRF has six parameters, three of them shared with RF which are number of tree
($T$), percetage of bootstrap ($b$), and number of random features ($m$).
Another three parameters are number of stages ($S$), threshold
for true-positive ($maxtp$) and threshold for true-negative ($maxtn$).

The bootstrap strategy proceeds as follows: after training in each stage, the
negative test set which contains only negative samples than tested on all
previous stages in order to delete the true-negative samples from
negative test set.
Samples that classified as false-positive then moved to negative test set to be
learned later in the next stage.

Some of stage have low accuracy value than other stages.
To decrease the influence of stage with low performance, calculate the weith
factor $\alpha$ for each stage by exploiting the harmonic means of $precision$
and $recall$ on training set or also known as $F_1$ (F-Measure).a
The $\alpha$ value for each stage linearly denormalized in range of 0 to 1, so
that the weight of low performance stage reduced to make their contribution to
majority voting also decreased.

The formula to get the classification result from CRF given in picture
\ref{form:crf}.

\begin{figure}[h]
\[
	y(x) = argmax \left(
			\frac{1}{T \cdot \sum^{S}_{s=1} \alpha_{s} }
			\sum\limits_{s=1}^{S} \alpha_{s}
			\sum\limits^{T}_{t=1} I_{h_{t} (x) = c}
		\right)
\]
\caption{CRF classifier with weight.}
\label{form:crf}
\end{figure}

$x$ is a sampel to be classified,
$S$ is a number of stage in cascade structure,
$\alpha_{s}$ is the weight value for each stage,
$T$ is number of tree in each stage, and
$h_{t}$ is classification function from the tree which give class value $c$
from an indicator $I$ (e.g. a value 1 for positive or 0 for negative).


\section{Pekerjaan Sebelumnya}
\label{subsec:bot-wikipedia}
\subsection{Bot Wikipedia}

Permasalahan vandalisme di Wikipedia telah terjadi sejak adanya Wikipedia itu
sendiri.
Komunitas Wikipedia menangani permasalahan tersebut dengan membuat fungsi
pengaman pada artikel secara manual supaya artikel tidak bisa disunting, bila
sebuah artikel terlalu sering divandal.
Sejak tahun 2006, bot pendeteksi vandalisme digunakan, yang secara otomatis
memantau suntingan vandalisme dan terkadang mengembalikannya.
Umumnya bot ini menggunakan aturan heuristik sederhana, daftar hitam kata, dan
daftar alamat \textit{Internet Protocol} (IP) pengguna yang
diblokir yang terdeteksi melakukan vandalisme (contohnya, VoABot II
\footnote{\url{https://en.wikipedia.org/wiki/User:VoABot_II}}
dan ClueBot
\footnote{\url{https://en.wikipedia.org/wiki/User:ClueBot}}).
ClueBot-NG
\footnote{\url{https://en.wikipedia.org/wiki/User:ClueBot_NG}}
menggantikan ClueBot, menggunakan pendekatan pembelajaran mesin.
Bot ini mencoba memperbaiki teknik berbasis heuristik yang susah untuk dirawat
dan mudah dilewat.
Bot ini menggunakan dataset suntingan pra-klasifikasi yang dianotasikan oleh
pengguna Wikipedia untuk melatih Jaringan Saraf Tiruan.
Pengklasifikasinya bekerja pada beberapa fitur suntingan, seperti probabilitas
tingkat-kata vandalisme untuk mengklasifikasi suntingan baru.

\label{subsec:pendekatan-pembelajaran-mesin}
\subsection{Pendekatan Pembelajaran Mesin}

Sejak tahun 2008 deteksi vandalisme di Wikipedia berbasis pendekatan pembelajaran mesin telah menjadi topik penelitian yang menarik.
Potthast \cite{potthast2008automatic} berkontribusi untuk pendekatan deteksi
vandalisme dengan pembelajaran mesin yang pertama menggunakan fitur tekstual
berikut fitur meta data dasar dengan menggunakan pengklasifikasi
\textit{logistic regression}.
Smets \cite{smets08automaticvandalism} menggunakan pengklasifikasi Naive Bayes
pada sekumpulan kata yang merepresentasikan suntingan dan yang pertama
menggunakan model kompresi untuk mendeteksi vandalisme di Wikipedia.
Itakura dan Clarke \cite{itakura2009using} menggunakan Kompresi Markov Dinamis
untuk mendeteksi suntingan vandalisme di Wikipedia.
Mola Velasco \cite{mola2012wikipedia} mengembangkan pendekatan yang dilakukan
oleh Potthast \cite{potthast2008automatic} dengan menambahkan beberapa fitur
tekstual dan berbagai fitur berbasis daftar-kata.
Velasco memenangi \textit{1st International Competition on Wikipedia Vandalism
Detection}.  West dkk. \cite{west2011multilingual} adalah yang pertama
mengajukan sebuah pendekatan deteksi vandalisme hanya berdasarkan meta data
spasial dan temporal, tanpa perlu memeriksa teks pada artikel dan revisi.
Adler dkk. \cite{adler2010detecting} membangun sebuah sistem deteksi vandalisme
menggunakan sistem reputasi WikiTrust.
Adler dkk. \cite{adler2011wikipedia} kemudian menggabungkan bahasa alami,
spasial, temporal, dan fitur reputasi yang digunakan pada karya sebelumnya.
West dan Lee \cite{west2011multilingual} adalah yang pertama memperkenalkan
data \textit{ex post facto} sebagai fitur, yang mana perhitungannya
mempertimbangkan revisi selanjutnya.
Sistem deteksi vandalisme West dan Lee memenangkan \textit{2nd International
Competition on Wikipedia Vandalism Detection}.
Harpalani dkk. \cite{harpalani2011language} menyatakan suntingan vandalisme
memiliki properti lingustik yang unik dan sama.
Harpalani dkk. membangun sistem deteksi vandalisme berdasarkan analisis
\textit{stylometric} dari suntingan vandalisme dengan model probabilitas
\textit{context-free grammar}.
Pendekatan Harpalani dkk. mengalahkan sistem berbasis fitur dengan pola
dangkal, yang menyamakan struktur sintaksis dan token teks.
Mengikuti tren dari klasifikasi vandalisme antar bahasa, Tran dan Christen
\cite{tran2013cross} mengevaluasi berbagai pengklasifikasi berbasiskan pada
sekumpulan fitur independen bahasa yang dikumpulkan dari jumlah artikel dilihat
setiap jam dan riwayat suntingan Wikipedia.

Gotze \cite{gotze2014advanced} menggabungkan fitur dari Adler dkk.
\cite{adler2011wikipedia}, Javanmardi dkk. \cite{javanmardi2011vandalism}, Mola
Velasco \cite{mola2012wikipedia}, Potthast dkk. \cite{potthast2008automatic},
Wang dan McKeown \cite{wang2010got}, dan West dan Lee
\cite{west2011multilingual} dengan empat fitur tambahan dan perubahan.
Gotze mengaplikasikan SMOTE untuk mensampel ulang dataset PAN-WVC-10 dan
PAN-WVC-11.
Untuk mengevaluasi hasil sampel ulang tersebut Gotze berfokus pada
pengaplikasian teknik \textit{Logistic Regression} dan \textit{Random Forest};
dan sebagai tambahan mengikutkan juga pengklasifikasi \textit{RealAdaBoost} dan
\textit{Bayesian Network}.

Dari penelitian di atas, tujuh diantaranya menggunakan PAN-WVC-10
\cite{adler2010detecting}
\cite{adler2011wikipedia}
\cite{gotze2014advanced}
\cite{harpalani2011language}
\cite{mola2012wikipedia}
\cite{wang2010got}
\cite{west2011multilingual},
dengan nilai presisi terbaik yaitu $0,86$, nilai \textit{recall} $0,57$, dan
PR-AUC $0,66$ didapat oleh Velasco menggunakan \textit{Random Forest} tanpa
penyeimbangan dataset.
Hanya dua yang menggunakan PAN-WVC-11 \cite{gotze2014advanced}
\cite{west2011multilingual} dengan hasil terbaik dipegang oleh Gotze yaitu
dengan nilai presisi $0,92$, \textit{recall} $0,39$, dan PR-AUC $0,74$.


%%
%% BAB III
%%
\chapter{Proses Deteksi Vandalisme}
\chapter{Metodologi Penelitian}
\section{Pembuatan Fitur}
\section{Pembuatan Sampel-ulang}


\section{Persiapan Data}
\label{persiapan_data}
\input{metodologi_penelitian/persiapan_data}

\subsection{Persiapan Dataset Pelatihan}
Dataset yang digunakan untuk pelatihan yaitu PAN-WVC-10
\cite{potthast:2010b}.
Dataset PAN-WVC-10 terbagi menjadi dua yaitu dataset suntingan dari artikel
Wikipedia dan dataset anotasi yang berisi hasil klasifikasi vandalisme pada
dataset suntingan tersebut.
Dataset suntingan memiliki atribut sebagai berikut,

\begin{itemize}
	\item \textbf{editid}, format angka, berisi identifikasi (ID) unik dari setiap suntingan.
	\item \textbf{editor}, format string, berisi nama penyunting.
	\item \textbf{oldrevisionid}, format angka, berisi ID untuk suntingan lama.
	\item \textbf{newrevisionid}, format angka, berisi ID untuk suntingan baru.
	\item \textbf{diffurl}, format string, berisi URL yang mengacu pada perbedaan suntingan baru dengan lama.
	\item \textbf{edittime}, format string, berisi tanggal dan pukul
	suntingan.
	\item \textbf{editcomment}, format string, berisi komentar yang ditambahkan oleh penyunting saat menyimpan hasil suntingan.
	\item \textbf{articleid}, format angka, berisi ID unik dari artikel.
	\item \textbf{articletitle}, format string, berisi judul dari artikel yang disunting.
\end{itemize}

Dataset anotasi memiliki atribut sebagai berikut,

\begin{itemize}
	\item \textbf{editid}, format angka, mengacu pada \textit{editid} di
	dataset suntingan.
	\item \textbf{class}, format string, berisi tipe suntingan yang
	bernilai "regular" yang menyatakan bahwa suntingan tersebut bukan
	vandalisme, dan "vandalism" yang menyatakan bahwa suntingan tersebut
	adalah vandalisme.
	\item \textbf{annotators}, format angka, berisi jumlah orang yang
	menandai (penanda) bahwa suntingan dengan ID tersebut termasuk ke dalam
	kelas "regular" atau "vandalism".
	\item \textbf{totalannotators}, format angka, berisi jumlah total penanda yang memeriksa suntingan.
\end{itemize}

\newpage
Kedua dataset kemudian digabung untuk menghasilkan atribut \textit{editid},
\textit{class}, \textit{oldrevisionid}, \textit{newrevisionid},
\textit{edittime}, \textit{editor}, \textit{articletitle},
\textit{editcomment}, \textit{deletions}, dan \textit{additions}.
Nilai dari atribut \textit{class} diganti dari teks menjadi angka, yaitu 1
untuk "vandalism" dan 0 untuk "regular".
Atribut tambahan \textit{deletions} berisi teks yang dihapus dalam revisi yang
lama.
Atribut tambahan \textit{additions} berisi teks yang ditambahkan dalam revisi
yang baru.
Kedua atribut tersebut didapat dengan membandingkan isi dari revisi yang lama
dengan yang baru.

Pemrosesan selanjutnya yaitu membuat berkas revisi yang bersih dari sintaks
wiki.
Tujuan dari revisi ini yaitu supaya tidak ada \textit{noise} pada saat
melakukan penghitungan fitur dan mempercepat proses pembuatan fitur.
Setiap berkas revisi dibaca kemudian dilakukan pembersihan berikut,

\begin{itemize}
\item penghapusan URI yang berawalan dengan
\textit{http://}, \textit{https://}, \textit{ftp://}, dan \textit{ftps}.
\item Menghapus \textit{mark-up} wiki yaitu konten yang dilingkupi oleh marka
berikut:
\texttt{[[Category:]]}, \texttt{[[:Category:]]}, \texttt{[[File:]]}, \\
\texttt{[[Help:]]}, \texttt{[[Image:]]}, \texttt{[[Special:]]},
\texttt{[[Wikipedia:]]}, \\
\texttt{\{\{DEFAULTSORT:\}\}}, \texttt{\{\{Template:\}\}}, dan \texttt{<ref/>}.
\item Mengganti karakter dan token berikut dengan karakter kosong (spasi):
\texttt{[}, \texttt{]}, \texttt{\{}, \texttt{\}}, \texttt{|}, \texttt{=},
\texttt{\#}, \texttt{'s}, \texttt{'}, \texttt{<ref>}, \texttt{</ref>},
\texttt{<br />}, \texttt{<br/>}, \texttt{<br>}, \texttt{<nowiki>},
\texttt{</nowiki>}, \texttt{\&nbsp;}.
\item Menghapus karakter kosong yang berlebihan.
\end{itemize}


\subsection{Persiapan Dataset Pengujian}
Dataset yang digunakan untuk pengujian yaitu PAN-WVC-11
\cite{potthast:2010b}.
Korpus PAN-WVC-11 terdiri dari tiga bahasa yaitu Inggris, Jerman, dan
Spanyol, yang digunakan untuk pelatihan hanya yang bahasa Inggris.
Dataset asli memiliki atribut yang sama dengan PAN-WVC-10, yaitu
\textit{editid},
\textit{editor},
\textit{oldrevisionid},
\textit{newrevisionid},
\textit{diffurl},
\textit{class},
\textit{annotators},
\textit{totalannotators},
\textit{edittime},
\textit{editcomment},
\textit{articleid}, dan
\textit{articletitle}.

Atribut \textit{annotators} dan \textit{totalannotators} dihilangkan dan
ditambah dengan dua atribut baru yaitu \textit{deletions} yang berisi teks yang
dihapus pada revisi lama, dan \textit{additions} yang berisi teks yang
ditambahkan pada revisi yang baru.
Nilai dari atribut \textit{class} diganti dari string menjadi integer, yaitu
dari "vandalism" menjadi bilangan \texttt{1}
dan "regular" menjadi bilangan \texttt{0}.

Untuk proses pembersihan data dilakukan prosedur yang sama seperti pada
PAN-WVC-10.


\section{Fitur Vandalisme}
Beberapa makalah sebelumnya mengelompokan fitur ke dalam tiga kelompok yaitu
\textit{metadata}, teks, dan bahasa.
Dalam tesis ini digunakan 4 fitur metadata, 11 fitur teks, dan 10 fitur
bahasa yang diambil dari hasil analisis makalah
\textcite{mola2012wikipedia}.

\subsubsection{Fitur Kelompok Metadata}

Kelompok metadata mengacu pada properti dari sebuah revisi yang secara langsung
dapat diambil, seperti identitas penyunting, komentar, atau ukuran perubahan.

Berikut daftar fitur metadata yang digunakan,

\begin{itemize}

\item \textbf{Anonim}.
Penyunting anonim yaitu yang tidak menggunakan akun Wikipedia saat melakukan
penyuntingan, sehingga yang tercatat hanya alamat IP bukan nama pengguna.
Fitur ini melihat apakah suntingan anonim atau bukan pada atribut
\textit{editor}, jika benar maka ditandai dengan nilai 1, atau 0 sebaliknya.
Vandal lebih condong berlaku anonim karena jika menggunakan akun asli akan
membuat akun mereka mudah diblokir dan membuat akun baru membutuhkan waktu dan
identitas berupa alamat surel.

\item \textbf{Panjang komentar}.
Melihat dari jumlah karakter yang diinputkan di kolom rangkuman suntingan, yang
tersimpan dalam atribut \textit{editcomment} pada dataset, saat menyimpan hasil
suntingan, tanpa mengikutkan bagian \textit{header} yaitu penanda di awal
komentar yang menunjuk ke bagian yang di sunting, biasanya dalam format
"\texttt{/* Nama header */}".
Komentar yang panjang mungkin mengindikasikan suntingan normal dan yang pendek
atau kosong mungkin menyarankan suatu vandalisme.

\pagebreak

\item \textbf{Peningkatan ukuran}.
Peningkatan absolut dari ukuran konten artikel.
Berkurangnya ukuran dalam jumlah besar bisa mengindikasikan
pengosongan artikel. Fitur ini dihitung dengan,
\begin{equation}
|\text{Ukuran suntingan baru} - \text{Ukuran suntingan lama}|
\end{equation}

\item \textbf{Rasio ukuran}.
Ukuran revisi baru relatif terhadap revisi lama.
Pada suntingan biasa, penghapusan biasanya diikuti dengan sejumlah perbaikan.
Fitur ini mendeteksi dengan menghitung penambahan atau penghapusan yang
berlebihan, dengan cara

\begin{equation}
\frac{1 + |\text{Ukuran suntingan baru}|}{1 + |\text{Ukuran suntingan lama}|}
\end{equation}

\end{itemize}


\subsubsection{Fitur Kelompok Teks}

Berikut daftar fitur berbasiskan teks yang digunakan,

\begin{itemize}

\item \textbf{Rasio huruf besar dan kecil}.
Pelaku vandal biasanya tidak mengikuti aturan huruf kapital, menulis semuanya
dengan huruf kecil atau huruf besar.
Rasio ini dihitung pada teks yang ditambahkan di revisi baru dengan menggunakan
rumus

\begin{equation}
\frac{1 + |\text{Jumlah huruf besar}|}{1 + |\text{Jumlah huruf kecil}|}
\end{equation}

\item \textbf{Rasio huruf besar terhadap semua karakter}
Rasio ini dihitung dengan menggunakan rumus

\begin{equation}
\frac{1 + |\text{Jumlah huruf besar}|}%
	{1 + |\text{Jumlah huruf besar}| + |\text{Jumlah huruf kecil}|}
\end{equation}

\item \textbf{Rasio angka}.
Rasio semua karakter terhadap angka, yaitu

\begin{equation}
\frac{1 + |\text{Jumlah karakter angka}|}{1 + |\text{Jumlah semua karakter}|}
\end{equation}

Fitur ini membantu menemukan perubahan kecil yang hanya mengubah angka.
Contoh kasusnya perubahan sebuah tanggal atau perhitungan yang disengaja untuk
memberikan informasi yang salah.

\item \textbf{Rasio non-alfanumerik}.
Rasio semua karakter terhadap karakter selain huruf dan angka, yaitu

\begin{equation}
\frac{1 + |\text{Jumlah karakter non alfanumerik}|}%
	{1 + |\text{Jumlah semua karakter}|}
\end{equation}

Penggunaan karakter selain angka-huruf yang berlebihan bisa mengindikasikan
penggunaan \textit{emoticon}, tanda baca, atau kata tak bermakna.

\item \textbf{Diversitas karakter}.
Menghitung jumlah karakter berbeda yang digunakan pada penambahan, dibandingkan dengan panjang teks yang dimasukan,
dihitung dengan rumus,

\begin{equation}
\text{panjang teks}^{\frac{1}{1+\text{karakter unik}}}
\end{equation}

Fitur ini membantu menemukan penggunaan karakter secara acak dan kata tak
bermakna.

\item \textbf{Distribusi karakter}.
Menggunakan Divergensi Kullback-Leibler dari distribusi karakter yang dimasukan
terhadap ekspektasi.
Fitur ini berguna untuk mendeteksi kata tak bermakna.

\item \textbf{Laju kompresi}.
Melihat tingkat kompresi dari penambahan teks menggunakan algoritma kompresi
LZW.
Fitur ini berguna untuk mendeteksi kata tak bermakna, pengulangan kata atau
karakter, dll.
Vandalisme biasanya memiliki ukuran kompresi yang rendah.

\item \textbf{Token umum}.
Menghitung token yang biasanya jarang digunakan oleh vandal yaitu sintaks
wiki, seperti \textit{\_\_TOC\_\_}. Daftar token dapat dilihat pada lampiran
\ref{lampiran:words_wiki_token}.

\item \textbf{Frekuensi rerata kata}.
Frekuensi relatif rerata dari kata yang dimasukan pada revisi baru.
Pada artikel yang panjang, semakin banyak kata yang dimasukan yang tidak ada
pada artikel mengindikasikan bahwa suntingan tersebut bisa tak bermakna atau
tidak berhubungan dengan isinya.

\item \textbf{Kata terpanjang}.
Panjang dari kata yang dimasukan.
Nilainya akan 0 jika tidak ada kata yang dimasukan.
Fitur ini berguna untuk mendeteksi suntingan tak-bermakna.

\item \textbf{Urutan karakter terpanjang}.
Urutan terpanjang dari karakter yang sama pada teks yang dimasukan sering
digunakan pada vandalisme, contohnya \textit{aaarrrrggghhh! sooo huge}.

\end{itemize}

\subsubsection{Fitur Kelompok Bahasa}

Fitur kelompok bahasa didasarkan pada jumlah kata tertentu yang ditambahkan
pada suntingan atau revisi yang baru.
Untuk setiap kategori kata dihitung dua jenis fiturnya: frekuensi dan impak.

Fitur frekuensi yaitu menghitung frekuensi dari kategori kata terhadap
total seluruh kata pada suntingan yang baru, yaitu

\begin{equation}
	\frac{\text{Jumlah kategori kata}}
		{\text{Jumlah seluruh kata}}
\end{equation}

Fitur impak yaitu menghitung persentase peningkatan kategori kata di revisi
yang baru, dihitung dengan,

\begin{equation}
	\frac{\text{Jumlah kata di revisi lama}}%
		{
			\text{Jumlah kata di revisi lama} +
			\text{Jumlah kata di revisi baru}
		}
\end{equation}

Berikut daftar kategori kata yang digunakan,

\begin{itemize}
\item \textbf{Vulgarisme}.
Menghitung kata-kata vulgar, kasar dan menghina.
Daftar kategori kata ini dapat dilihat pada lampiran
\ref{lampiran:words_vulgar}.

\item \textbf{Subjek}.
Menghitung kata subjek pertama atau kedua, termasuk pengejaan tidak baku,
misalnya \textit{I}, \textit{you}, \textit{ya}.
Daftar kategori kata ini dapat dilihat pada lampiran
\ref{lampiran:words_pronoun}.

\item \textbf{Bias}.
Menghitung penggunaan kata sehari-hari yang mengandung bias, misalnya
\textit{coolest}, \textit{huge}.
Daftar kategori kata ini dapat dilihat pada lampiran
\ref{lampiran:words_bias}.

\item \textbf{Pornografi}.
Menghitung penggunaan kata berhubungan dengan pornografi.
Daftar kategori kata ini dapat dilihat pada lampiran
\ref{lampiran:words_sex}.

\item \textbf{Kata buruk}.
Menghitung penggunaan kata sehari-hari dan beberapa penulisan yang buruk
(misalnya, \textit{wanna}, \textit{gotcha}).
Daftar kategori kata ini dapat dilihat pada lampiran
\ref{lampiran:words_bad}.

\item \textbf{Seluruh kategori kata}.
Gabungan dari kategori kata vulgarisme, subjek, bias, pornografi, dan kata
buruk.

\end{itemize}



\section{Pembuatan Fitur}
Dari analisis fitur vandalisme, kemudian dilanjutkan
dengan mengimplementasikan semua fitur tersebut ke dalam sebuah program.
Program tersebut kemudian dijalankan dengan input yaitu dataset PAN-WVC-10
dan PAN-WVC-11 yang telah dibersihkan pada subbab \ref{persiapan_data},
sehingga menghasilkan dataset fitur yang bernilai kontinu.
Implementasi program pembersihan data dan generator fitur ini dapat dilihat dan
digunakan secara terbuka pada repositori \textit{Wikipedia Vandalism Corpus
Generator} (\texttt{wvcgen})
\footnote{\url{https://github.com/shuLhan/wvcgen}}.


\section{Pembuatan Sampel-ulang}
Dataset PAN-WVC-10 tanpa sampel ulang memiliki jumlah sampel vandal atau
positif yaitu 2.394 sampel dan 30.045 sampel bukan vandal atau negatif.
Untuk mendapatkan jumlah sampel yang seimbang, setiap dataset disampel ulang
dengan teknik SMOTE atau LNSMOTE untuk kelas yang positif.


\section{Implementasi Pengklasifikasi}
Implementasi pengklasifikasi dilakukan bertahap karena keterkaitan antara
modul.
Dimulai dari implementasi untuk pohon keputusan CART yang digunakan untuk
implementasi \textit{Random Forest} (RF) yang kemudian digunakan dalam
implementasi \textit{Cascaded Random Forest} (CRF).


\section{Pelatihan dan Pengujian}
Dataset yang digunakan untuk pelatihan model yaitu PAN-WVC-10 yang terdiri dari
tiga jenis yaitu dataset tanpa sampel ulang, dataset yang telah disampel ulang
dengan SMOTE, dan dataset yang telah disampel ulang dengan LNSMOTE.
Jumlah sampel pada dataset yang tidak disampel yaitu 2.394 positif dan 30.045
negatif dengan total 32.439 sampel.
Jumlah sampel positif pada dataset hasil sampel ulang dengan SMOTE yaitu 28.728
sampel dengan total 58.773 sampel.
Jumlah sampel positif pada dataset hasil sampel ulang dengan LNSMOTE yaitu
28.588 sampel dengan total 58.633 sampel.

Dataset yang digunakan untuk pengujian model yaitu PAN-WVC-11 yang terdiri dari
1.143 sampel positif dan 8.842 sampel negatif dengan total 9.985 sampel.
Jumlah fitur pada PAN-WVC-11 sama dengan PAN-WVC-10 yaitu 26 fitur.

Supaya konsisten antara pengklasifikasi, digunakan parameter umum yang sama
yaitu 200 pohon, 5 (dari $\sqrt{26}$) fitur acak, dan $ 64\% $ untuk
\textit{bootstrapping}.
Untuk klasifikasi CRF dilakukan tiga pemodelan dan pengujian dengan parameter
yang berbeda yaitu 200 tingkat dengan 1 pohon, 100 tingkat dengan 2 pohon, dan
50 tingkat dengan 4 pohon; dengan jumlah pohon yang tetap sama untuk ketiganya
yaitu 200.
Hal ini dilakukan untuk melihat pengaruh dari jumlah pohon terhadap tingkat dan
hasil klasifikasi.
Parameter lain pada pemodelan CRF yaitu nilai ambang batas TPR dan TNR diset
pada nilai $0,95$ dan $0,95$ untuk mendapatkan hasil klasifikasi yang bagus dan
jumlah pohon yang konsisten.

Pembuatan model klasifikasi dilakukan dengan cara menjalankan program
klasifikasi RF dan CRF pada masing-masing dataset fitur PAN-WVC-10 yang belum
disampel ulang, yang telah disampel ulang dengan SMOTE, dan yang telah disampel
ulang dengan LNSMOTE.
Setelah modelnya terbangun, model diuji dengan diberikan input dataset fitur
PAN-WVC-11. Hasil dari pengujian digunakan untuk analisis.

Lingkungan pelatihan dan pengujian dilakukan pada mesin Intel\textregistered\
 Core\texttrademark \ i7-4750HQ CPU 2,00 GHz, dengan jumlah \textit{RAM} 8
GB. Setiap pelatihan model dilakukan satu per satu untuk menghindari adanya
\textit{cache miss} yang berpengaruh pada kecepatan dan waktu pemrosesan.


%%
%% BAB IV: EVALUASI
%%
\chapter{Evaluasi}

Klasifikasi CRF LNSMOTE dengan 200 tingkat 1 pohon memberikan nilai TPR paling
tinggi yaitu $0,9904$ tapi dengan nilai FPR yang paling tinggi yaitu $0,8558$
dan TNR yang paling rendah yaitu $0,1442$ di antara model yang lainnya.
Kebalikannya, pengklasifikasi RF tanpa sampel ulang memberikan nilai TNR paling
tinggi yaitu $0,9988$ dan nilai FPR paling rendah yaitu $0,0012$.
Untuk presisi, RF tanpa sampel ulang memberikan nilai tertinggi yaitu $0,9450$,
dan nilai terendah diberikan oleh klasifikasi CRF LNSMOTE dengan 200 tingkat 1
pohon.
Untuk nilai \textit{F-Measure}, nilai tertinggi diberikan oleh klasifikasi CRF
tanpa sampel ulang dengan 50 tingkat dan 4 pohon yaitu $0,5353$, dengan nilai
terendah diberikan oleh klasifikasi CRF LNSMOTE 200 tingkat 1 pohon.
Untuk nilai akurasi tertinggi didapat dengan klasifikasi RF LNSMOTE dengan
yang terendah diberikan oleh klasifikasi CRF LNSMOTE 200 tingkat 1 pohon.
Klasifikasi dengan nilai AUC tertinggi yaitu CRF SMOTE 100 tingkat 2 pohon
dengan yang terendah diberikan oleh klasifikasi CRF tanpa sampel ulang dengan
100 tingkat 2 pohon.

Dari segi kecepatan pelatihan model, pengklasifikasi CRF lebih cepat dari RF
baik pada semua dataset pelatihan.
Sebagai pembanding, dapat dilihat pada klasifikasi RF dan CRF 50 tingkat 4
pohon.
CRF tanpa sampel ulang lebih cepat 11 kali daripada RF, dan pada sampel ulang
SMOTE dan LNSMOTE, klasifikasi CRF 1,6 kali lebih cepat daripada RF.

Pengaruh sampel ulang SMOTE dan LNSMOTE sama pada pengklasifikasi RF dan
CRF yaitu meningkatnya nilai TPR tapi dengan nilai peningkatan yang berbeda,
dengan LNSMOTE lebih tinggi dari daripada SMOTE.
Pada pengklasifikasi RF, sampel ulang dengan LNSMOTE meningkatkan nilai TPR
$0.7\%$ sedangkan SMOTE hanya $0.4\%$
Pada pengklasifikasi CRF, nilai peningkatan beragam bergantung kepada jumlah
pohon keputusan disetiap tingkatan.
Pada CRF 200 tingkat 1 pohon, peningkatan TPR dari yang tanpa sampel ulang
yaitu sekitar $0,02\%$, sedangkan pada CRF 100 tingkat 2 pohon peningkatan TPR
yaitu $0,14\%$, dan pada CRF 50 tingkat 4 pohon peningkatan TPR yaitu $0,28\%$.
Semakin banyak jumlah pohon pada setiap tingkatan maka semakin nilai TPR akan
semakin tinggi daripada yang tanpa sampel ulang.

%% Apakah ada outlier?

%% Kenapa LNSMOTE lebih baik?

%% Kenapa CRF menghasilkan TPR lebih tinggi?

Fokus dari pengklasifikasi CRF yaitu pada pembelajaran sampel negatif,
khususnya \textit{false-positive} yaitu sampel yang bukan vandal tapi
terdeteksi sebagai vandal.
Hal ini dapat diteliti lebih lanjut dengan melihat strategi \textit{bootstrap}
dari CRF.
Asumsikan $D$ adalah dataset pengujian yang berisi sampel positif dan negatif,
dan dataset $T$ yang berisi hanya sampel yang \textit{true-negative} (TP).
Pada setiap tingkatan CRF, setelah semua pohon dibangun, semua tingkatan pada
model diuji dengan dataset $D$ dan $T$.
Hasil pengujian pada dataset $D$ akan menghasilkan sampel yang tergolong ke
dalam benar positif (TP), benar terklasifikasi negatif (TN), dan yang salah
terklasifikasi (FP dan FN).
Sampel yang tergolong TN dihapus dari dataset pelatihan $D$ dan dimasukan ke
dataset $T$ yang hanya berisi sampel TN, sehingga dataset $D$ meninggalkan
hanya sampel yang tergolong TP, FP, dan FN.
Hasil pengujian pada dataset $T$ menghasilkan sampel yang tergolong benar
negatif (TN) dan salah positif (FP).
Sampel yang tergolong FP pada dataset $T$ kemudian dimasukan kembali ke dataest
$D$ untuk dipelajari kembali pada tingkatan selajutnya.
Berbeda dengan RF, yang mana tidak ada sampel yang dihapus saat
pelatihan, CRF membuat dataset yang tadinya condong pada kelas mayoritas
sedikit demi sedikit pada setiap tingkatan menjadi sama dan meningkatkan
performansi klasifikasi yang benar positif.


\chapter{Kesimpulan}

Rata-rata SMOTE menaikan nilai TPR $0,19$ kali dan nilai FPR $1,6$ kali.
Sementara pada sampel ulang dengan LNSMOTE, rata-rata menaikan nilai TPR $0,33$
dan juga menaikan nilai FPR $2,75$ kali.
Kedua sampel ulang tersebut cukup bagus dalam meningkatkan performansi laju
positif dengan konsekuensi juga meningkatnya laju \textit{false-positive},
pengaruh ini lebih terasa pada sampel ulang LNSMOTE.
Secara keseluruhan performansi dari sampel ulang SMOTE lebih baik dari
LNSMOTE untuk pengklasifikasi CRF.
Hal ini disebabkan karena sifat algoritma dari CRF yang berfokus pembelajaran
sampel yang negatif, bukan pada sampel yang positif, sehingga menambahkan
sampel sintetis menyebabkan pembelajar menjadi \textit{over-fitting} pada
sampel positif.
Pengaruh menarik lainnya yaitu pada pengklasifikasi CRF, dengan menggunakan
jumlah pohon lebih sedikit pada setiap tingkatan menghasilkan performansi yang
hampir sama dengan melakukan sampel ulang pada dataset asli, seperti yang
terlihat pada performansi CRF-100-2 tanpa sampel ulang hampir sama dengan CRF
50 tingkat 4 pohon dengan sampel ulang SMOTE.

Untuk model klasifikasi vandalisme yang terbaik tanpa sampel ulang dihasilkan
dari pengklasifikasi CRF dengan 200 tingkat dan 1 pohon, untuk model terbaik
pada dataset yang telah disampel ulang dengan SMOTE yaitu CRF dengan 100
tingkat dan 1 pohon, dan untuk model terbaik dari sampel ulang LNSMOTE yaitu
CRF dengan 200 tingkat 1 pohon. Secara keseluruhan model terbaik yaitu dari CRF
100 tingkat dan 2 pohon yang telah disampel ulang dengan SMOTE.
Selain performansi yang lebih baik, pengklasifikasi CRF juga lebih cepat $1,6$
kali dalam pembentukan model daripada RF pada dataset yang telah disampel
ulang.

\section{Kontribusi}

Kontribusi dari karya tulis ini selain membantu menemukan pengklasifikasi yang
lebih baik dalam mendeteksi vandalisme juga memberikan kerangka kerja untuk
menciptakan dan mengembangkan fitur vandalisme dari dataset mentah tanpa harus
mulai dari awal untuk dapat digunakan dalam penelitian selanjutnya atau
digunakan langsung pada kasus asli. Selain itu juga menyediakan pustaka untuk
pengolahan data dan fungsi pembelajaran mesin terutama pengklasifikasi
\textit{Cascaded Random Forest} yang belum ada implementasi secara terbuka pada
program terkenal seperti \textit{Weka}, \textit{Scikit-Learn}, atau \textit{R}.

\section{Pekerjaan Selanjutnya}

Semua pelatihan model dalam karya tulis ini masih menggunakan algoritma
pengklasifikasi RF dan CRF dalam bentuk serial, yang mana satu pohon dibangun
satu per satu bergantian atau pada saat klasifikasi setiap sampel dimasukan ke
dalam pohon secara bergantian untuk mendapatkan kelasnya.
Penggunaan algoritma paralel, seperti dalam pembentukan dua atau empat pohon
bersamaan dan klasifikasi sampel (pengumpulan \textit{vote} di setiap pohon)
bersamaan, bisa membantu mempercepat pelatihan, pengujian dan hasil
klasifikasi.
Pada domain pembelajaran mesin, hal menarik yaitu adanya algoritma
\textit{eXtreme Gradient Boostring} (XGBoost) \cite{chen2016xgboost} yang
mungkin bisa diuji dan diterapkan untuk meningkatkan deteksi vandalisme.


\clearpage
\addcontentsline{toc}{chapter}{DAFTAR PUSTAKA}
\printbibliography

%%
%% LAMPIRAN
%%
\titleformat{\chapter}[hang]
{\bfseries\large\centering}
{Lampiran \thechapter}
{1em}
{}

\appendix
\chapter*{LAMPIRAN}
\addcontentsline{toc}{chapter}{LAMPIRAN}

\chapter{Daftar Kategori Kata dan Token}
\label{lampiran:daftar_token_dan_kata}
\input{lampiran/daftar_token_dan_kata}

\end{document}
