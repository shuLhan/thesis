\begin{center}
\textbf{\large
	\MakeUppercase{\mytitle{}} \\
	\bigskip
	\textnormal{By} \\
	\myname{} \\
	NIM: \mysid{} \\
	(\mydept{}) \\
}
\end{center}

\bigskip
\bigskip
\bigskip

Wikipedia.org is an online encyclopedia which can edited by anyone.
Those feature has benefit, which make the article in Wikipedia rapidly
increased in size and can be fixed subsequently, and their drawbacks was prone
to vandalism in the forms of invalid information, deletion, ads, or meaningless
content.
This paper propose a framework for detecting vandalism on English Wikipedia
using machine learning technique by training Cascaded Random Forest (CRF)
classifier on English Wikipedia dataset (PAN-WVC-10) that has been resampled
using Local Neighbourhood SMOTE (LNSMOTE).
Those two methods then compared with Random Forest (RF) for classifier and
SMOTE for resampling.
The result of training both classifiers that has been tested on Wikipedia
Vandalism Corpus 2011 (PAN-WVC-11) English only dataset showed that the dataset
resampled using LNSMOTE have true-positive rate better than SMOTE.
CRF on LNSMOTE with 200 stages and 1 tree gave the better result among others
with true-positive rate value 0.9904.
From training computation time, CRF 1.6 times faster than RF in resampled
dataset.

Keywords: wikipedia, vandalism, imbalanced dataset, dataset resampling,
cascaded random forest
