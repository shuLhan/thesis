Klasifikasi CRF LNSMOTE dengan 200 tingkat 1 pohon memberikan nilai TPR paling
tinggi yaitu $0,9904$ tapi dengan nilai FPR yang paling tinggi yaitu $0,8558$
dan TNR yang paling rendah yaitu $0,1442$ di antara model yang lainnya.
Kebalikannya, pengklasifikasi RF tanpa sampel ulang memberikan nilai TNR paling
tinggi yaitu $0,9988$ dan nilai FPR paling rendah yaitu $0,0012$.
Untuk presisi, RF tanpa sampel ulang memberikan nilai tertinggi yaitu $0,9450$,
dan nilai terendah diberikan oleh klasifikasi CRF LNSMOTE dengan 200 tingkat 1
pohon.
Untuk nilai \textit{F-Measure}, nilai tertinggi diberikan oleh klasifikasi CRF
tanpa sampel ulang dengan 50 tingkat dan 4 pohon yaitu $0,5353$, dengan nilai
terendah diberikan oleh klasifikasi CRF LNSMOTE 200 tingkat 1 pohon.
Untuk nilai akurasi tertinggi didapat dengan klasifikasi RF LNSMOTE dengan
yang terendah diberikan oleh klasifikasi CRF LNSMOTE 200 tingkat 1 pohon.
Klasifikasi dengan nilai AUC tertinggi yaitu CRF SMOTE 100 tingkat 2 pohon
dengan yang terendah diberikan oleh klasifikasi CRF tanpa sampel ulang dengan
100 tingkat 2 pohon.

Dari segi kecepatan pelatihan model, pengklasifikasi CRF lebih cepat dari RF
baik pada semua dataset pelatihan.
Sebagai pembanding, dapat dilihat pada klasifikasi RF dan CRF 50 tingkat 4
pohon.
CRF tanpa sampel ulang lebih cepat 11 kali daripada RF, dan pada sampel ulang
SMOTE dan LNSMOTE, klasifikasi CRF 1,6 kali lebih cepat daripada RF.
