Dalam deteksi vandalisme, menemukan sebuah suntingan yang vandal dengan tingkat
positif yang tinggi lebih baik daripada salah klasifikasi atau terlewatnya
suntingan vandal tersebut dari pendeteksian.
Kesalahan klasifikasi, yang bukan vandalisme terdeteksi sebagai vandalisme,
tidak akan berpengaruh pada pembaca, tetapi terlewatnya suntingan yang vandal
bisa menyebabkan hilangnya informasi, kesalahan informasi, atau mengganggu
pembaca Wikipedia.
Oleh karena itu, hasil pelatihan dan pengujian dilihat dari laju
\textit{true-positive} pada performansi.
Laju \textit{true-positive}, atau dikenal juga dengan
\textit{true-positive rate} (TPR),
yaitu total jumlah sampel yang benar positif dibagi dengan
jumlah sampel yang terklasifikasi benar positif (\textit{true-positive})
ditambah dengan jumlah sampel positif terklasifikasi dengan negatif
(\textit{false-negative}).
TPR bernilai dengan rentang antara 0 sampai 1, dengan nilai yang mendekati 0
berarti performansi yang buruk dan nilai yang mendekati 1 berarti performansi
yang baik.
Hasil pengujian diberikan dalam bentuk performansi pengklasifikasi pada tabel
\ref{tab:stats} dan kecepatan pelatihan model pada gambar \ref{graph:runtimes}.

\DTLsetseparator{;}
\DTLloaddb{stats}{../result/stats.csv}

\DTLmaxforcolumn{stats}{TPR}{\maxtpr}
\DTLminforcolumn{stats}{FPR}{\minfpr}
\DTLmaxforcolumn{stats}{TNR}{\maxtnr}
\DTLmaxforcolumn{stats}{Presisi}{\maxprec}
\DTLmaxforcolumn{stats}{F-Measure}{\maxfm}
\DTLmaxforcolumn{stats}{Akurasi}{\maxacc}
\DTLmaxforcolumn{stats}{AUC}{\maxauc}

\begin{table}[htbp]
\caption{Performansi Klasifikasi RF dan CRF}
\centering
\footnotesize
\begin{tabular}{l l r}
\hline
\textbf{Klasifikasi} &
\textbf{Dataset} &
\textbf{TPR}
\DTLforeach*{stats}{%
	\cl=Klasifikasi,%
	\ds=Dataset,%
	\tpr=TPR%
}{%
	\DTLifnullorempty{\cl}
		{\\ \cline{2-3}}
		{\\ \hline \hline}
	\DTLifnullorempty{\cl}
		{}
		{
			\multirow{3}{*}{\cl}
		}
	& \ds
	& \DTLifnumeq{\tpr}{\maxtpr}{\textbf{\tpr}}{\tpr}
}
\\
\hline
\end{tabular}
\label{tab:stats}
\end{table}


Pengklasifikasi CRF dengan 200 tingkat 1 pohon pada dataset yang telah disampel
ulang dengan LNSMOTE memberikan nilai TPR paling tinggi yaitu $0,9904$.
Pengklasifikasi RF tanpa sampel ulang memberikan nilai TPR paling rendah yaitu
$0,1654$.

\begin{figure}[htb]
\centering
\mytikzinput{graph_runtimes}
\begin{tikzpicture}
	\begin{axis}[
		width=13cm,
		ymax=250,
		ybar=8pt,
		ylabel=Waktu (menit),
		symbolic x coords={Tanpa sampel ulang, SMOTE, LNSMOTE},
		xtick=data,
		nodes near coords,
		every node near coord/.append style={font=\footnotesize},
		enlarge x limits=0.24,
		enlarge y limits=0,
		bar width = 13pt,
		legend style={
			at={(0.5,-0.15)},
			anchor=north,
			legend columns=-1,
			/tikz/every even column/.append style={column sep=0.5cm}
		},
	]
		%% RF
		\addplot[fill=white] coordinates {
			(Tanpa sampel ulang, 52.5)
			(SMOTE, 217.7)
			(LNSMOTE, 199.5)
		};

		%% CRF-200-1
		\addplot[pattern=north east lines] coordinates {
			(Tanpa sampel ulang, 3.5)
			(SMOTE, 74.4)
			(LNSMOTE, 61.9)
		};

		%% CRF-100-2
		\addplot[pattern=horizontal lines] coordinates {
			(Tanpa sampel ulang, 3.6)
			(SMOTE, 74.6)
			(LNSMOTE, 67.9)
		};

		%% CRF-50-4
		\addplot coordinates {
			(Tanpa sampel ulang, 4.2)
			(SMOTE, 82.5)
			(LNSMOTE, 76.6)
		};
	\legend{RF, CRF-200-1, CRF-100-2, CRF-50-4}
	\end{axis}
\end{tikzpicture}
\caption{Waktu pelatihan untuk setiap klasifikasi berdasarkan dataset.}
\label{graph:runtimes}
\end{figure}


Dari segi kecepatan pelatihan model, pengklasifikasi CRF lebih cepat dari RF
baik pada semua dataset pelatihan.
Sebagai pembanding, dapat dilihat pada klasifikasi RF dan CRF 50 tingkat 4
pohon.
CRF tanpa sampel ulang lebih cepat 11 kali daripada RF, dan pada sampel ulang
SMOTE dan LNSMOTE, klasifikasi CRF 1,6 kali lebih cepat daripada RF.
