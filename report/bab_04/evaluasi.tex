Dalam deteksi vandalisme, menemukan sebuah suntingan yang vandal dengan tingkat
positif yang tinggi lebih baik daripada salah klasifikasi atau terlewatnya
suntingan vandal tersebut dari pendeteksian.
Kesalahan klasifikasi, yang bukan vandalisme terdeteksi sebagai vandalisme,
tidak akan berpengaruh pada pembaca, tetapi terlewatnya suntingan yang vandal
bisa menyebabkan hilangnya informasi, kesalahan informasi, atau mengganggu
pembaca Wikipedia.
Oleh karena itu, hasil pelatihan dan pengujian dilihat dari laju
\textit{true-positive} pada performansi.
Laju \textit{true-positive}, atau dikenal juga dengan
\textit{true-positive rate} (TPR),
yaitu total jumlah sampel yang benar positif dibagi dengan
jumlah sampel yang terklasifikasi benar positif (\textit{true-positive})
ditambah dengan jumlah sampel positif terklasifikasi dengan negatif
(\textit{false-negative}).
TPR bernilai dengan rentang antara 0 sampai 1, dengan nilai yang mendekati 0
berarti performansi yang buruk dan nilai yang mendekati 1 berarti performansi
yang baik.

Pembuatan model klasifikasi dilakukan dengan cara menjalankan program
klasifikasi RF dan CRF pada masing-masing dataset fitur PAN-WVC-10 yang belum
disampel ulang, yang telah disampel ulang dengan SMOTE, dan yang telah disampel
ulang dengan LNSMOTE.
Setelah modelnya terbangun, model diuji dengan diberikan input dataset fitur
PAN-WVC-11. Hasil dari pengujian digunakan untuk analisis.

Lingkungan pelatihan dan pengujian dilakukan pada mesin Intel\textregistered\
 Core\texttrademark \ i7-4750HQ CPU 2,00 GHz, dengan jumlah \textit{RAM} 8
GB. Setiap pelatihan model dilakukan satu per satu untuk menghindari adanya
\textit{cache miss} yang berpengaruh pada kecepatan dan waktu pemrosesan.
