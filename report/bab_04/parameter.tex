Supaya konsisten antara pengklasifikasi, digunakan parameter umum yang sama
yaitu 200 pohon, 5 fitur acak (yang didapat dari
$\sqrt{\text{jumlah fitur} = 26}$),
dan $ 64\% $ untuk \textit{bootstrapping}.
Untuk klasifikasi CRF dilakukan tiga pemodelan dan pengujian dengan parameter
yang berbeda yaitu 200 tingkat dengan 1 pohon, 100 tingkat dengan 2 pohon, dan
50 tingkat dengan 4 pohon; dengan jumlah pohon keputusan yang tetap sama untuk
ketiganya yaitu 200.
Hal ini dilakukan untuk melihat pengaruh dari jumlah pohon terhadap tingkat dan
hasil klasifikasi.
Parameter lain pada pemodelan CRF yaitu nilai ambang batas TPR dan TNR diset
pada nilai $0,95$ dan $0,95$ untuk mendapatkan hasil klasifikasi yang bagus dan
jumlah pohon yang konsisten.
