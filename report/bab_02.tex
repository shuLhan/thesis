\chapter{Tinjauan Pustaka}
\label{bab:02}

Bab ini berisi ilmu dan konsep yang mendukung pem bahasan tesis ini beserta
makalah mengenai pekerjaan sebelumnya dalam deteksi vandalisme di Wikipedia.
Untuk konsep sampel ulang yang dibahas adalah metode \textit{Synthetic Minority
Oversampling Technique} (SMOTE) dan ekstensi dari SMOTE yaitu
\textit{Local Neighbourhood SMOTE} (LNSMOTE).
Konsep pengklasifikasi yang dibahas dan digunakan yaitu \textit{Random Forest}
(RF) dan ekstensinya yaitu \textit{Cascaded Random Forest} (CRF).

	\section{SMOTE}
	\label{bab:02:smote}
	Metode \textit{Synthetic Minority Over-sampling Technique} (SMOTE)
\cite{chawla2002smote} menggunakan pendekatan \textit{over-sampling} yang mana
kelas minoritas ditambah dengan membuat sampel "sintetis" bukan dengan
mengganti sampel dari kelas mayoritas menjadi kelas minoritas.
Sampel sintetis dibuat lebih kurang lewat aplikasi, dengan beroperasi pada
"ruang fitur" bukan pada "ruang data".
Kelas minoritas ditambah dengan mengambil setiap sampel-sampel dari kelas
minoritas dan membuat sampel sintetis diantara segmen garis yang menggabungkan
setiap/semua \textit{k-nearest-neighbors} dari kelas minoritas.
Bergantung kepada jumlah \textit{over-sampling} yang dibutuhkan, instan dari
\textit{k-nearest-neighbors} dipilih secara acak.

\begin{figure}[b]
	\centering
	\includegraphics[keepaspectratio=true,scale=0.6]{SMOTE-example}
	\caption{Ilustrasi pembuatan sampel sintetis pada SMOTE.
\textit{p} adalah sampel minoritas, \textit{n} adalah salah satu
\textit{k-nearest-neighbors} dari \textit{p}.
Sampel sintetis yang baru akan berada digaris antara \textit{p} dan \textit{n}.
	}
	\label{fig:smote}
\end{figure}

Sampel sintetis dibuat dengan cara berikut,
\begin{itemize}
	\item Hitung selisih antara vektor fitur (sampel) dengan tetangga
	terdekatnya.
	\item Kalikan selisih tersebut dengan angka ril acak antara 0 sampai 1,
	dan
	\item tambahkan hasilnya ke vektor fitur.
\end{itemize}

Cara ini membuat sampel secara acak pada segmen garis antara dua fitur yang
terpilih, seperti yang terlihat pada gambar \ref{fig:smote}.
Pendekatan ini secara efektif mendorong wilayah pembelajaran dari kelas
minoritas menjadi lebih besar tanpa menyebabkan \textit{overfitting}.

Ambil contoh sebuah sampel (6,4) dan (4,3) sebagai tetangga terdekatnya.
(6,4) adalah sampel yang akan dicari \textit{k-nearest-neighbors}-nya.
(4,3) adalah salah satu dari \textit{k-nearest-neighbors}-nya.
Misalkan,
\[
\begin{matrix}
f1\_1 = 6 & f1\_2 = 4 \\
f2\_1 = 4 & f2\_2 = 3
\end{matrix}
\]
\[
\begin{matrix}
f2\_1 - f1\_1 = 4 - 6 = -2 \\
f2\_2 - f1\_2 = 3 - 4 = -1
\end{matrix}
\]

Sampel baru dihasilkan dengan,
\[
(f1', f2') = (6,4) + random(0-1) * (-2,-1)
\]

Fungsi \texttt{random(0-1)} menghasilkan bilangan ril acak dari 0 sampai 1.


	\section{LNSMOTE}
	\label{bab:02:lnsmote}
	SMOTE method has several weakness.
First, all sample from minority class is used, this could be a problem because
not all the samples have equal benefit for learning.
Minority sample in the boundary region between minority and majority class
usually result in misclassified rather than sample located in the center of
region, while sample that located in the center may give a little contribution
to classifier.
One of the method to overcome this problem is by using sample in boundary of
minority class instead in the center which proposed by Han et al.
\cite{han2005borderline}, called Borderline-SMOTE.

Another weakness of SMOTE method is overgeneralization, where their method does
not take into consideration the distribution of minority sample in majority
class, or the outliers.
Maciejewski and Stefanowski \cite{maciejewski2011local}
introduced an extension of SMOTE called Local Neighbourhood
SMOTE (LNSMOTE) \cite{maciejewski2011local} by combining Borderline-SMOTE
with modified version of Safe-Level SMOTE (SL-SMOTE)
\cite{bunkhumpornpat2009safe}.

In SL-SMOTE, majority samples is taken into consideration before creating
synthetic sample by calculating a coefficient called safe-level.
For each minority sample, count the number of their $k$ nearest neighbours (KNN).
If KNN value is close to 0, then the sample will be considered as noise.
If KNN value is close to $k$, then the sample can be said in safe region in
minority class.
The main idea was to create synthetic sample that close to safe region.

The SL-SMOTE strategy has a problem especially when class distribution is bias
in which the minority class spread into small sub-region with low number
cardinality.
In this situation, creating synthetic sample with SLMOTE will cause an overlap
between class.
This problem is due to SL-SMOTE find only KNN for minority class.
If the sample candidate does not located in region with densed minority class,
then some of their neighbours could be far from sample candidate or surrounded
by majority class samples.
LNSMOTE overcome this overlap problem by taking into consideration the local
neighbourhood of minority sample candidate that can provide the number of
majority class around each of them.


	\section{\textit{Random Forest}}
	\label{bab:02:rf}
	\textit{Random Forest} (RF)
is a combination of several decision tree such that each tree depends on the
value of random vector sampled independently and with equal distribution for
all trees in the forest
\cite{breiman2001random}.

There are three common paremeter in building RF.
The first parameter is the number of trees in forest ($n$),
the other two parameters are percentage of samples ($b$) and number of random
features ($m$) for building the tree.
All of the parameters are set before building each tree and their value is
constant.

Percentage of sample for training that was selected randomly usually two third
from overall samples, which left one third of them as out of bag (OOB) samples.
For the number of random features $m$, the common value is the square root or
log of all features \cite{breiman2001random}.

Procedure to build RF is as follows.
Let $S$ be a training set.
After their parameter has been set, when building each tree, take $b$ samples
randomly from $S$ without replacement (sample that get selected can be picked
again in the next iteration).
This process also know as bootstrapping.
Samples that does not included in $b$ is called out-of-bag (OOB), which can be
used to calculate the misclassification rate.
From $b$ samples, take $m$ random features, and then build the tree using $b$
samples with $m$ number of features without pruning.
Repeat the process until the $n$-tree has been built.

Classification process on RF proceed as follows.
Given test set $T$, with the same number of features with $S$.
For each sample $t$ in $T$, insert the sample $t$ into each tree and collect
their classification result.
After $n$ trees or $n$ number of classes, compute the majority class from all
tree classification.


	\section{\textit{Cascaded Random Forest}}
	\label{bab:02:crf}
	Most of ensemble learning algorithm is incapable of handling imbalance training
data.
Inequality between positive and negative class usually result in low
detection accuracy.
A simulation run by Strobel et al. \cite{strobl2007bias} showed that RF skewed
in favor of majority class.
Another drawback of RF is after learning several trees, RF gradually reach its
peak, such that the classifier can not increase their detection sensitivity or
decrease their false-positive rate.

Viola and Jones proposed a detection algorithm based on AdaBoost with
cascade structure \cite{viola2004robust}.
Cascade structure motivated by assumption that it is more easy to reject a
negative sample than finding a positive one.
Viola and Jones combines several strong classifiers in several independent
stages with condition that each stage can reject a sample, so to classify a
sample as positive then all stages must be passed.
Due to rejection on early stages, computation time will be decreased.
In addition, to get better training result, Viola and Jones propose a bootstrap
strategy by deleting samples classified as true negative.
The reduced training set then refilled with sample that mis-classified, or
false-positive samples
\cite{viola2004robust}.

A cascade classifier consists of several number of stages with increasing
complexity.
Each stage have minimum one independent classifier.
Classifier added into stages until the value of true-positive and true-negative
threshold is reached.
The advantage of cascade structure is a vast number of samples can be
distributed between stages, decreasing false-positive value and shortening
computation time when training and classifying.

Baumann uses this method with RF and propose Cascaded Random Forest (CRF) which
is a combination of RF classifier with cascade structure, where in each stage
several decision tree is build with bootstrap strategy, this leads increased
learning on positive sample and the drawback of imbalanced dataset can be
avoided.
\cite{baumann2013cascaded}

CRF has six parameters, three of them shared with RF which are number of tree
($T$), percetage of bootstrap ($b$), and number of random features ($m$).
Another three parameters are number of stages ($S$), threshold
for true-positive ($maxtp$) and threshold for true-negative ($maxtn$).

The bootstrap strategy proceeds as follows: after training in each stage, the
negative test set which contains only negative samples than tested on all
previous stages in order to delete the true-negative samples from
negative test set.
Samples that classified as false-positive then moved to negative test set to be
learned later in the next stage.

Some of stage have low accuracy value than other stages.
To decrease the influence of stage with low performance, calculate the weith
factor $\alpha$ for each stage by exploiting the harmonic means of $precision$
and $recall$ on training set or also known as $F_1$ (F-Measure).a
The $\alpha$ value for each stage linearly denormalized in range of 0 to 1, so
that the weight of low performance stage reduced to make their contribution to
majority voting also decreased.

The formula to get the classification result from CRF given in picture
\ref{form:crf}.

\begin{figure}[h]
\[
	y(x) = argmax \left(
			\frac{1}{T \cdot \sum^{S}_{s=1} \alpha_{s} }
			\sum\limits_{s=1}^{S} \alpha_{s}
			\sum\limits^{T}_{t=1} I_{h_{t} (x) = c}
		\right)
\]
\caption{CRF classifier with weight.}
\label{form:crf}
\end{figure}

$x$ is a sampel to be classified,
$S$ is a number of stage in cascade structure,
$\alpha_{s}$ is the weight value for each stage,
$T$ is number of tree in each stage, and
$h_{t}$ is classification function from the tree which give class value $c$
from an indicator $I$ (e.g. a value 1 for positive or 0 for negative).


	\section{Pekerjaan Terkait}\label{bab:02:pekerjaan_terkait}
	Deteksi vandalisme di Wikipedia berbasis pendekatan pembelajaran mesin telah
menjadi topik penelitian yang menarik sejak tahun 2008.
\textcite{potthast2008automatic} berkontribusi untuk pendekatan deteksi
vandalisme dengan pembelajaran mesin yang pertama, dengan menggunakan fitur
tekstual berikut fitur meta data dasar dengan menggunakan pengklasifikasi
\textit{logistic regression}.
\textcite{smets08automaticvandalism} menggunakan pengklasifikasi Naive Bayes
pada sekumpulan kata yang merepresentasikan suntingan dan yang pertama
menggunakan model kompresi untuk mendeteksi vandalisme di Wikipedia.
\textcite{itakura2009using} menggunakan Kompresi Markov Dinamis
untuk mendeteksi suntingan vandalisme di Wikipedia.
\textcite{mola2012wikipedia} mengembangkan pendekatan yang dilakukan
oleh \textcite{potthast2008automatic} dengan menambahkan beberapa fitur
tekstual dan berbagai fitur berbasis daftar-kata.
Velasco memenangi \textit{1st International Competition on Wikipedia Vandalism
Detection}.
\textcite{west2011multilingual} adalah yang pertama
mengajukan pendekatan deteksi hanya berdasarkan meta data
spasial dan temporal, tanpa perlu memeriksa teks pada artikel dan revisi.
\textcite{adler2010detecting} membangun sebuah sistem deteksi vandalisme
menggunakan sistem reputasi WikiTrust.
\textcite{adler2011wikipedia} kemudian menggabungkan bahasa alami,
spasial, temporal, dan fitur reputasi yang digunakan pada karya sebelumnya.
\textcite{west2011multilingual} adalah yang pertama memperkenalkan
data \textit{ex post facto} sebagai fitur, yang mana perhitungannya
mempertimbangkan revisi selanjutnya.
Sistem deteksi vandalisme West dan Lee memenangkan \textit{2nd International
Competition on Wikipedia Vandalism Detection}.
\textcite{harpalani2011language} menyatakan suntingan vandalisme
memiliki properti lingustik yang unik dan sama.
Harpalani dkk. membangun sistem deteksi vandalisme berdasarkan analisis
\textit{stylometric} dari suntingan vandalisme dengan model probabilitas
\textit{context-free grammar}.
Pendekatan Harpalani dkk. mengalahkan sistem berbasis fitur dengan pola
dangkal, yang menyamakan struktur sintaksis dan token teks.
Mengikuti tren dari klasifikasi vandalisme antar bahasa,
\textcite{tran2013cross} mengevaluasi berbagai pengklasifikasi berbasiskan pada
sekumpulan fitur independen bahasa yang dikumpulkan dari jumlah artikel dilihat
setiap jam dan riwayat suntingan Wikipedia.

\textcite{gotze2014advanced} menggabungkan fitur dari
\textcite{adler2011wikipedia},
\textcite{javanmardi2011vandalism},
\textcite{mola2012wikipedia},
\textcite{potthast2008automatic},
\textcite{wang2010got}, dan
\textcite{west2011multilingual} dengan empat fitur tambahan dan perubahan.
Untuk mengatasi masalah ketimpangan pada korpus,
\textcite{gotze2014advanced}
mengaplikasikan teknik
\textit{random oversampling}
bernama
\textit{Synthetic Minority Over-sampling TEchnique} (SMOTE)
yang diajukan oleh
\textcite{chawla2002smote},
dan kombinasi dari SMOTE dan
\textit{random undersampling}.
Dataset latihan yang orisinal dan hasil sampel ulang diuji dengan
pengklasifikasi satu-kelas dan dua-kelas.
Pengklasifikasi satu-kelas yang diterapkan diantaranya
\textcite{hempstalk2008one}
dan SVM oleh
\textcite{scholkopf1999support}
yang diimplementasikan oleh
\textcite{chang2011libsvm}
pada korpus PAN-WVC.
Pengklasifikasi dua-kelas yang diterapkan diantaranya
\textit{Logistic Regression},
\textit{RealAdaBoost},
\textit{Random Forest} (RF), dan
\textit{Bayesian Network}.
Hasil percobaan yang didapat memperlihatkan performansi pengklasifikasi
satu-kelas tidak kompetitif dengan satu pun pengklasifikasi dua-kelas.
Hal ini bisa disebabkan karena tidak sesuainya kelompok fitur yang digunakan
untuk menjelaskan suntingan vandalisme, sebagaimana juga parameter pengaturan
yang tidak sesuai pada pendekatan yang digunakan.
Hasil dari pelatihan pada dataset orisinal memperlihatkan RF
lebih unggul dari pengklasifikasi lainnya.
Hasil dari pelatihan pada dataset hasil sampel ulang memperlihatkan adanya
peningkatan pada semua pengklasifikasi kecuali pada RF.



Dari penelitian di atas, tujuh diantaranya menggunakan PAN-WVC-10
\parencites{adler2010detecting}
{adler2011wikipedia}
{gotze2014advanced}
{harpalani2011language}
{mola2012wikipedia}
{wang2010got}
{west2011multilingual},
dengan nilai presisi terbaik yaitu $0,86$, nilai \textit{recall} $0,57$, dan
PR-AUC $0,66$ didapat oleh Velasco menggunakan \textit{Random Forest} tanpa
penyeimbangan dataset.
Hanya dua yang menggunakan PAN-WVC-11
\parencites{gotze2014advanced}
{west2011multilingual}
dengan hasil terbaik dipegang oleh Gotze yaitu
dengan nilai presisi $0,92$, \textit{recall} $0,39$, dan PR-AUC $0,74$.


	\section{Fitur Vandalisme}
	\label{bab:02:fitur_vandalisme}
	Beberapa makalah sebelumnya mengelompokan fitur ke dalam tiga kelompok yaitu
\textit{metadata}, teks, dan bahasa.
Dalam tesis ini digunakan 4 fitur metadata, 11 fitur teks, dan 10 fitur
bahasa yang diambil dari hasil analisis makalah
\textcite{mola2012wikipedia}.

\subsubsection{Fitur Kelompok Metadata}

Kelompok metadata mengacu pada properti dari sebuah revisi yang secara langsung
dapat diambil, seperti identitas penyunting, komentar, atau ukuran perubahan.

Berikut daftar fitur metadata yang digunakan,

\begin{itemize}

\item \textbf{Anonim}.
Penyunting anonim yaitu yang tidak menggunakan akun Wikipedia saat melakukan
penyuntingan, sehingga yang tercatat hanya alamat IP bukan nama pengguna.
Fitur ini melihat apakah suntingan anonim atau bukan pada atribut
\textit{editor}, jika benar maka ditandai dengan nilai 1, atau 0 sebaliknya.
Vandal lebih condong berlaku anonim karena jika menggunakan akun asli akan
membuat akun mereka mudah diblokir dan membuat akun baru membutuhkan waktu dan
identitas berupa alamat surel.

\item \textbf{Panjang komentar}.
Melihat dari jumlah karakter yang diinputkan di kolom rangkuman suntingan, yang
tersimpan dalam atribut \textit{editcomment} pada dataset, saat menyimpan hasil
suntingan, tanpa mengikutkan bagian \textit{header} yaitu penanda di awal
komentar yang menunjuk ke bagian yang di sunting, biasanya dalam format
"\texttt{/* Nama header */}".
Komentar yang panjang mungkin mengindikasikan suntingan normal dan yang pendek
atau kosong mungkin menyarankan suatu vandalisme.

\pagebreak

\item \textbf{Peningkatan ukuran}.
Peningkatan absolut dari ukuran konten artikel.
Berkurangnya ukuran dalam jumlah besar bisa mengindikasikan
pengosongan artikel. Fitur ini dihitung dengan,
\begin{equation}
|\text{Ukuran suntingan baru} - \text{Ukuran suntingan lama}|
\end{equation}

\item \textbf{Rasio ukuran}.
Ukuran revisi baru relatif terhadap revisi lama.
Pada suntingan biasa, penghapusan biasanya diikuti dengan sejumlah perbaikan.
Fitur ini mendeteksi dengan menghitung penambahan atau penghapusan yang
berlebihan, dengan cara

\begin{equation}
\frac{1 + |\text{Ukuran suntingan baru}|}{1 + |\text{Ukuran suntingan lama}|}
\end{equation}

\end{itemize}


\subsubsection{Fitur Kelompok Teks}

Berikut daftar fitur berbasiskan teks yang digunakan,

\begin{itemize}

\item \textbf{Rasio huruf besar dan kecil}.
Pelaku vandal biasanya tidak mengikuti aturan huruf kapital, menulis semuanya
dengan huruf kecil atau huruf besar.
Rasio ini dihitung pada teks yang ditambahkan di revisi baru dengan menggunakan
rumus

\begin{equation}
\frac{1 + |\text{Jumlah huruf besar}|}{1 + |\text{Jumlah huruf kecil}|}
\end{equation}

\item \textbf{Rasio huruf besar terhadap semua karakter}
Rasio ini dihitung dengan menggunakan rumus

\begin{equation}
\frac{1 + |\text{Jumlah huruf besar}|}%
	{1 + |\text{Jumlah huruf besar}| + |\text{Jumlah huruf kecil}|}
\end{equation}

\item \textbf{Rasio angka}.
Rasio semua karakter terhadap angka, yaitu

\begin{equation}
\frac{1 + |\text{Jumlah karakter angka}|}{1 + |\text{Jumlah semua karakter}|}
\end{equation}

Fitur ini membantu menemukan perubahan kecil yang hanya mengubah angka.
Contoh kasusnya perubahan sebuah tanggal atau perhitungan yang disengaja untuk
memberikan informasi yang salah.

\item \textbf{Rasio non-alfanumerik}.
Rasio semua karakter terhadap karakter selain huruf dan angka, yaitu

\begin{equation}
\frac{1 + |\text{Jumlah karakter non alfanumerik}|}%
	{1 + |\text{Jumlah semua karakter}|}
\end{equation}

Penggunaan karakter selain angka-huruf yang berlebihan bisa mengindikasikan
penggunaan \textit{emoticon}, tanda baca, atau kata tak bermakna.

\item \textbf{Diversitas karakter}.
Menghitung jumlah karakter berbeda yang digunakan pada penambahan, dibandingkan dengan panjang teks yang dimasukan,
dihitung dengan rumus,

\begin{equation}
\text{panjang teks}^{\frac{1}{1+\text{karakter unik}}}
\end{equation}

Fitur ini membantu menemukan penggunaan karakter secara acak dan kata tak
bermakna.

\item \textbf{Distribusi karakter}.
Menggunakan Divergensi Kullback-Leibler dari distribusi karakter yang dimasukan
terhadap ekspektasi.
Fitur ini berguna untuk mendeteksi kata tak bermakna.

\item \textbf{Laju kompresi}.
Melihat tingkat kompresi dari penambahan teks menggunakan algoritma kompresi
LZW.
Fitur ini berguna untuk mendeteksi kata tak bermakna, pengulangan kata atau
karakter, dll.
Vandalisme biasanya memiliki ukuran kompresi yang rendah.

\item \textbf{Token umum}.
Menghitung token yang biasanya jarang digunakan oleh vandal yaitu sintaks
wiki, seperti \textit{\_\_TOC\_\_}. Daftar token dapat dilihat pada lampiran
\ref{lampiran:words_wiki_token}.

\item \textbf{Frekuensi rerata kata}.
Frekuensi relatif rerata dari kata yang dimasukan pada revisi baru.
Pada artikel yang panjang, semakin banyak kata yang dimasukan yang tidak ada
pada artikel mengindikasikan bahwa suntingan tersebut bisa tak bermakna atau
tidak berhubungan dengan isinya.

\item \textbf{Kata terpanjang}.
Panjang dari kata yang dimasukan.
Nilainya akan 0 jika tidak ada kata yang dimasukan.
Fitur ini berguna untuk mendeteksi suntingan tak-bermakna.

\item \textbf{Urutan karakter terpanjang}.
Urutan terpanjang dari karakter yang sama pada teks yang dimasukan sering
digunakan pada vandalisme, contohnya \textit{aaarrrrggghhh! sooo huge}.

\end{itemize}

\subsubsection{Fitur Kelompok Bahasa}

Fitur kelompok bahasa didasarkan pada jumlah kata tertentu yang ditambahkan
pada suntingan atau revisi yang baru.
Untuk setiap kategori kata dihitung dua jenis fiturnya: frekuensi dan impak.

Fitur frekuensi yaitu menghitung frekuensi dari kategori kata terhadap
total seluruh kata pada suntingan yang baru, yaitu

\begin{equation}
	\frac{\text{Jumlah kategori kata}}
		{\text{Jumlah seluruh kata}}
\end{equation}

Fitur impak yaitu menghitung persentase peningkatan kategori kata di revisi
yang baru, dihitung dengan,

\begin{equation}
	\frac{\text{Jumlah kata di revisi lama}}%
		{
			\text{Jumlah kata di revisi lama} +
			\text{Jumlah kata di revisi baru}
		}
\end{equation}

Berikut daftar kategori kata yang digunakan,

\begin{itemize}
\item \textbf{Vulgarisme}.
Menghitung kata-kata vulgar, kasar dan menghina.
Daftar kategori kata ini dapat dilihat pada lampiran
\ref{lampiran:words_vulgar}.

\item \textbf{Subjek}.
Menghitung kata subjek pertama atau kedua, termasuk pengejaan tidak baku,
misalnya \textit{I}, \textit{you}, \textit{ya}.
Daftar kategori kata ini dapat dilihat pada lampiran
\ref{lampiran:words_pronoun}.

\item \textbf{Bias}.
Menghitung penggunaan kata sehari-hari yang mengandung bias, misalnya
\textit{coolest}, \textit{huge}.
Daftar kategori kata ini dapat dilihat pada lampiran
\ref{lampiran:words_bias}.

\item \textbf{Pornografi}.
Menghitung penggunaan kata berhubungan dengan pornografi.
Daftar kategori kata ini dapat dilihat pada lampiran
\ref{lampiran:words_sex}.

\item \textbf{Kata buruk}.
Menghitung penggunaan kata sehari-hari dan beberapa penulisan yang buruk
(misalnya, \textit{wanna}, \textit{gotcha}).
Daftar kategori kata ini dapat dilihat pada lampiran
\ref{lampiran:words_bad}.

\item \textbf{Seluruh kategori kata}.
Gabungan dari kategori kata vulgarisme, subjek, bias, pornografi, dan kata
buruk.

\end{itemize}



	\section{Evaluasi Performansi}
	%% Jelaskan matriks confusion

Dalam konteks vandalisme, setiap sampel dalam dataset memiliki sebuah label
atau fitur kelas yang menandakan sampel tersebut termasuk ke dalam vandalisme
atau bukan.
Secara formalnya, setiap sampel dipetakan ke salah satu elemen dalam set
$\{p, n\}$, atau label kelas positif dan negatif.
Sebuah model klasifikasi (atau pengklasifikasi) adalah pemetaan dari sampel ke
kelas prediksi.
Untuk membedakan dengan kelas sebenarnya digunakan label $\{Y, N\}$ untuk kelas
prediksi yang dihasilkan oleh pengklasifikasi
\parencite{fawcett2006introduction}.

Diberikan sebuah pengklasifikasi dan sampel, maka ada empat kemungkinan
keluaran.
Jika label dari sampel positif dan diklasifikasi positif, maka dihitung sebagai
\textit{true-positive}, jika diklasifikasi negatif, maka dihitung sebagai
\textit{false-negative}.
Jika label dari sampel negatif dan diklasifikasi negatif, maka dihitung sebagai
\textit{true-negative}, jika diklasifikasi positif maka dihitung sebagai
\textit{false-positive}.
Jika diberikan sebuah pengklasifikasi dan sekumpulkan sampel (dataset
pengujian), sebuah \textit{confusion matrix} (atau disebut juga tabel
\textit{contingency}) dapat dibuat yang merepresentasikan penempatan dari
setiap sampel.

\begin{figure}[htbp]
	\centering
	\tikzsetnextfilename{confusionmatrix}
	\begin{tikzpicture}[
		framed,
		scale = 1,
		nodes = {
			align = center
		},
		proc/.style={
			rectangle,
			draw=black,
			text centered,
			minimum width=3cm,
			minimum height=2cm,
			outer sep=0pt,
		},
		label/.style={
			draw=none,
			text centered,
		}
	]
		\node[proc] (tp) {
			True Positive\\
			(TP)
		};

			\node[label] (p) [above=4pt of tp] {
				\textbf{p}
			};
			\node[label] (Y) [left=4pt of tp] {
				\textbf{Y}
			};

		\node[proc] (fp) [right=0pt of tp] {
			False Positive\\
			(FP)
		};
			\node[label] (n) [above=4pt of fp] {
				\textbf{n}
			};

		\node[proc] (fn) [below=0pt of tp] {
			False Negative\\
			(FN)
		};
			\node[label] (N) [left=4pt of fn] {
				\textbf{N}
			};
			\node[label] (P) [below=8pt of fn] {
				\textbf{P}
			};
			\node[label] (P) [left= of P] {
				\textbf{Total kolom:}
			};

		\node[proc] (tn) [right=0pt of fn] {
			True Negative\\
			(TN)
		};
			\node[label] (N) [below=8pt of tn] {
				\textbf{N}
			};

		\node[label] (actual) [above=of tp, xshift=1.5cm] {
			\underline{Kelas sebenarnya}
		};

		\node[label] (prediction) [left=of tp, yshift=-1cm] {
			\underline{Kelas prediksi}
		};
	\end{tikzpicture}
\caption{%
	\textit{Confusion matrix}
}
\label{confusion_matrix}
\end{figure}


Gambar \ref{confusion_matrix} memperlihatkan sebuah \textit{confusion matrix}
dari dua buah kelas.
Nilai pada kolom diagonal mayor (TP dan TN) merepresentasikan hasil klasifikasi
yang benar, sementara nilai pada diagonal minor (FN dan FP) merepresentasikan
galat -- \textit{confusion} -- antara kelas.
Nilai laju \textit{true-positive} atau \textit{true-positive rate} (TPR)
dikenal juga dengan \textit{recall} atau \textit{hit rate}) dari sebuah
pengklasifikasi dihitung dengan,

\begin{equation}
	\textit{tp rate} \approx \frac{\text{Benar terklasifikasi positif}}%
		{\text{Total positif}}
		= \frac{TP}{TP + FN}
		= \frac{TP}{P}
\end{equation}

TPR bernilai dengan rentang antara 0 sampai 1, dengan nilai yang mendekati 0
berarti performansi yang buruk dan nilai yang mendekati 1 berarti performansi
yang baik.

%% Jelaskan kenapa mengambil TPR / recall saja

Dalam deteksi vandalisme, menemukan sebuah suntingan yang vandal dengan tingkat
positif yang tinggi lebih baik daripada salah klasifikasi atau terlewatnya
suntingan vandal tersebut dari pendeteksian.
Kesalahan klasifikasi, yang bukan vandalisme terdeteksi sebagai vandalisme,
tidak akan berpengaruh pada pembaca, tetapi terlewatnya suntingan yang vandal
bisa menyebabkan hilangnya informasi, kesalahan informasi, atau mengganggu
pembaca Wikipedia.
Oleh karena itu, hasil pelatihan dan pengujian dilihat dari laju
\textit{true-positive} pada performansi pengklasifikasi.

