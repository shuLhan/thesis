Meskipun hasil SMOTE dibuktikan bagus dalam makalah Chawla dkk.
\cite{chawla2002smote}, metode ini masih memiliki beberapa kelemahan.
Pertama, cara menentukan sampel minoritas sebagai calon untuk
\textit{over-sampling} bisa bermasalah.
Pada SMOTE, semua sampel dari kelas minoritas digunakan.
Namun, bukan berarti semua sampel tersebut sama bergunanya bagi pembelajaran
klasifikasi.
Pada khususnya, sampel yang ada pada batas \textit{decision}, atau berada
dibatas kelas minoritas dengan kelas mayoritas, lebih sering salah
diklasifikasi dibandingkan dengan sampel yang berada jauh dari batas kelas,
oleh karena itu mereka lebih penting untuk klasifikasi.
Sampel yang jauh dari batas kelas, berada di tengah kelas minoritas mungkin
berkontribusi sedikit pada klasifikasi.
Salah satu metode untuk menangani permasalahan ini yaitu dengan hanya
mengambil sampel pada batas kelas minoritas yang dijadikan untuk
\textit{oversampling}, seperti yang diajukan oleh Han dkk.
\cite{han2005borderline} dengan menggunakan metode bernama \textit{Borderline
SMOTE}.

\begin{figure}[htbp]
	\centering
	\begin{subfigure}[b]{0.4\textwidth}
		\centering
		\includegraphics[width=\textwidth]{SMOTE-problem-outliers}
		\caption{}
		\label{fig:smote-outliers}
	\end{subfigure}
	\begin{subfigure}[b]{0.5\textwidth}
		\centering
		\includegraphics[width=\textwidth]{SMOTE-problem-overlapping}
		\caption{}
		\label{fig:smote-overlapping}
	\end{subfigure}
	\caption{
Kelemahan metode SMOTE:
(a) \textit{Outliers} pada kelas minoritas tidak diperhitungkan,
(b) Pembuatan sampel sintetis yang baru berada di wilayah kelas mayoritas
yang tumpang-tindih dengan sampel kelas mayoritas.
	}
	\label{fig:smote-problems}
\end{figure}

Kelemahan lain dari SMOTE yaitu permasalahan \textit{overgeneralization} yang
begitu saja menggeneralisasi wilayah dari kelas minoritas.
SMOTE tidak mempertimbangkan distribusi dari sampel pada kelas mayoritas, atau
adanya \textit{outliers}.

Untuk mengatasi permasalahan di atas, Maciejewski dkk. memperkenalkan ekstensi
dari metode SMOTE bernama \textit{Local-neighbourhood SMOTE} atau disingkat
LNSMOTE \cite{maciejewski2011local} yang menggunakan modifikasi dari
pendekatan \textit{Safe-Level SMOTE} \cite{bunkhumpornpat2009safe}.
Pada metode \textit{Safe-level SMOTE} keberadaan sampel mayoritas
diperhitungkan sebelum membuat sampel sintetis dengan menghitung sebuah
koefisien khusus yang disebut \textit{safe level}.
Untuk setiap sampel dari kelas minoritas, dihitung jumlah sampel kelas
minoritas di antara \textit{k-nearest-neighbors} (KNN).
Jika nilainya sama atau mendekati $ 0 $, sampel tersebut dianggap sebagai
\textit{noise}.
Jika nilainya mendekati $ k $, maka sampel tersebut bisa dikatakan berada
di wilayah aman dari kelas minoritas.
Gagasan utamanya adalah membuat sampel sintetis yang dekat dengan wilayah aman.

Lebih rincinya, misalkan $ p $ adalah sampel dari kelas minoritas yang akan
menjadi calon untuk \textit{over-sampling}, maka $ k $ sampel terdekat dengan
$ p $ yang termasuk pada kelas minoritas $ P $ ditentukan.
Seperti pada SMOTE, setidaknya satu dari tetangga tersebut dipilih, sebutlah
dengan $ n $.
Untuk kedua sampel, $ p $ dan $ n $, $ k $ sampel terdekatnya untuk keseluruhan
data pelatihan $ S $ dihitung \textit{safe level}-nya dengan notasi $ sl(p) $
dan $ sl(n) $.
Dari pengetahuan tersebut, koefisien rasio \textit{safe-level} ditentukan
dengan $ slratio = sl(p) / sl(n) $.
Langkah selanjutnya ditentukan berdasarkan 5 kasus berikut:
\begin{enumerate}
	\item \label{case:safe-1} Jika $ sl(p) = 0 $ dan $ sl(n) = 0 $, sampel
	$ p $ dan $ n $ dianggap sebagai \textit{noise outliers}, dan tidak ada
	sampel sintetis yang dibuat.
	\item Jika $ sl(p) > 0 $ dan $ sl(n) = 0 $, maka $ n $
	adalah \textit{noise}.
	Sampel sintetis akan dibuat jauh dari $ n $ dengan menduplikasi $ p $.
	\item Jika $ slratio = 1 $, maka $ p $ dan $ n $ memiliki tetangga
	yang sama dan sampel sintetis yang baru akan dibuat diantara garis yang
	menghubungkan mereka dengan cara yang sama seperti pada SMOTE.
	\item Jika $ slratio > 1 $, maka $ p $ berada di wilayah
	aman minoritas daripada $ n $ dan sampel sintetis yang baru akan
	dibuat dekat dengan $ p $, dengan parameter acak pada SMOTE akan
	memiliki rentang $ [0, 1 / slratio] $.
	\item Jika \textit{slratio < 1}, maka \textit{n} berada di wilayah
	aman minoritas daripada $ p $ dan sampel sintetis yang baru akan
	dibuat dekat dengan $ n $, yaitu parameter acak pada SMOTE akan
	memiliki rentang $ [1 - slratio, 1] $.
\end{enumerate}

Strategi \textit{safe-level SMOTE} bermasalah khususnya pada distribusi kelas
yang bias yang mana kelas minoritas menyebar sehingga terdiri dari beberapa
sub-wilayah dengan kardinalitas yang kecil.
Situasi ini mengacu pada permasalahan yang disebut \textit{small disjuncts}
yang diakui sebagai sumber kesulitan yang paling penting bagi pembelajaran
klasifikasi untuk data timpang \cite{jo2004class}.
Pada kasus seperti ini pembuatan sampel sintetis dengan \textit{Safe-level
SMOTE} bisa menyebabkan terjadinya tumpang-tindih antara kelas.

Sebagai contohnya, misalkan dua kelompok dari kelas minoritas terpisah
dikelilingi oleh sampel dari kelas mayoritas.
Katakanlah, jarak antara kedua kelompok minoritas tersebut cukup jauh, satu
kelompok berada di bawah dan kelompok lainnya di atas.
Misalkan calon dari sebuah sampel berada di kelompok yang dibawah.
Jika parameter $ k $ dari fungsi pencarian KNN lebih besar dari jumlah sampel
kelas minoritas di dalam kelompok tersebut, maka tetangga dari kelas minoritas
selanjutnya akan menjadi sampel dari kelompok yang lain.
Jika rasio \textit{safe-level} dari sampel kedua kelompok sama, sampel sintetis
yang baru bisa dibuat diantara garis yang menggabungkan sampel-sampel dari
kedua kelompok tersebut.
Dengan kata lain, sampel sintetis yang baru bisa berada di wilayah yang dihuni
oleh kelas mayoritas.
Makanya, strategi ini masih memungkinkan terjadinya tumpang-tindih antara
kelas.

Situasi tidak diinginkan seperti di atas disebabkan karena teknik
\textit{Safe-Level SMOTE} mencari KNN yang dimiliki oleh kelas minoritas saja.
Jika calon sampel tidak berada di wilayah yang padat dengan kelas minoritas,
maka beberapa tetangganya bisa saja cukup jauh dan juga sampel dari kelas
mayoritas mengelilingi calon sampel tersebut.

LNSMOTE mengatasi masalah tumpang tindih ini dengan lebih mempertimbangkan
tetangga sekitar
(\textit{local neighbourhood})
dari calon sampel minoritas yang mungkin bisa memberikan perkiraan dari
keberadaan sampel kelas mayoritas.
Jadi, pencarian sampel yang terlalu jauh sebaiknya dihindarkan.

Modifikasi teknik LNSMOTE terhadap \textit{Safe-Level} SMOTE yaitu,
\begin{itemize}
\item jika calon sampel $ p $ diidentifikasi berada di tetangga terdekat $ k
$ dari $ n $, sampel tersebut tidak langsung dihitung dengan $ sl(n) $ tapi
dicari tetangga dari $ k + 1 $ selanjutnya.
\item Membatasi rentang interval di mana sampel baru secara acak ditempatkan.
Jadi, pada beberapa kasus dari \textit{safe-level}, LNSMOTE tidak menentukan
batas kanan dari interval dengan 1 tapi sebagai ambang batas $\tau < 1$.
Ambang batas tersebut tidak baku tapi ditentukan secara dinamis bergantung
kepada \textit{safe-level} dari sampel mayoritas.
\item Jika $ sl(n) $ relatif rendah, artinya $ n $ dikelilingi oleh banyak
sampel dari kelas mayoritas, sampel baru seharusnya ditempatkan lebih dekat ke
$ p $.
\item Jika $ n $ dikelilingi oleh sejumlah sampel minoritas yang seimbang,
sehingga nilai $ sl(n) $ tinggi, sampel yang baru bisa berada di dekat $ n $.
Hal ini supaya bisa mengontrol tingkat ekspansi dari kelas minoritas dengan
cara dinamis, memperhitungkan distribusi lokal dari sampel.
\end{itemize}

Algoritma LNSMOTE yang digunakan pada implementasi tesis berdasarkan pada
makalah Maciejewski dkk.
\cite{maciejewski2011local}
yang dapat dilihat pada lampiran
\ref{lampiran:alg_lnsmote}.
