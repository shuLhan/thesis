\textit{Random Forest} (RF) \cite{breiman2001random}
adalah sebuah kombinasi dari beberapa pohon keputusan yang mana setiap pohon
bergantung kepada nilai dari vektor acak yang disampel secara independen dan
dengan distribusi yang sama.

Pohon keputusan yang digunakan dalam RF dapat menggunakan ID3
(\textit{Iterative Dichotomiser}), C4.5 (pengganti dari ID3), atau CART
(\textit{Classification and Regression Trees}).
Dalam tesis ini, dan umumnya pada implementasi RF, menggunakan CART sebagai
pohon keputusan.

Ada tiga parameter utama dalam pembentukan RF.
Parameter pertama yaitu jumlah pohon yang akan dibangun dalam \textit{forest}
(katakanlah $n$),
parameter kedua yaitu persentase dari sampel pelatihan yang digunakan untuk
membangun pohon (katakanlah $b$),
dan parameter ketiga yaitu jumlah fitur acak yang juga digunakan untuk
membangun pohon (katakanlah $m$).
Ketiga parameter diatur sebelum memulai pembangunan RF dan nilainya sama untuk
semua pohon.

Persentase dari sampel pelatihan yang diambil secara acak umumnya dua per tiga
dari sampel keseluruhan, sehingga meninggalkan sepertiganya sebagai
\textit{out-of-bag} (OOB).
Untuk nilai acak fitur $m$, nilai yang biasa digunakan yaitu akar dari jumlah
fitur atau $log$ dari jumlah fitur \cite{breiman2001random}.

Proses pembangunan RF sebagai berikut.
Diasumsikan diberikan sebuah dataset latihan $S$.
Setelah parameter ditentukan, pada saat pembuatan pohon, ambil sampel dari $S$
secara acak tanpa diganti (sampel yang telah diambil bisa terpilih kembali pada
pengambilan acak berikutnya) sejumlah $b$.
Proses ini dikenal juga dengan \textit{bootstraping}.
Sampel yang tidak terpilih disebut dengan \textit{out-of-bag} (OOB), yang
bisa digunakan untuk penghitungan laju galat mis-klasifikasi.
Dari sampel yang terpilih tersebut ambil $m$ buah fitur secara acak, kemudian
bangun pohon dari $m$ fitur dan $b$ sampel acak tersebut tanpa dibersihkan
(\textit{pruning}).
Ulangi proses tersebut sampai pohon ke-$n$.

Proses klasifikasi pada RF berjalan seperti berikut.
Diberikan dataset $T$ yaitu dataset yang berisi fitur yang sama dengan dataset
pelatihan $S$.
Untuk setiap sampel yang akan diklasifikasi pada dataset $T$, masukan sampel
tersebut ke setiap pohon.
Kumpulkan hasil klasifikasi dari setiap pohon, kemudian ambil mayoritas kelas
dari semua hasil klasifikasi tersebut.
Fungsi selengkapnya dari pembuatan dan klasifikasi RF dapat dilihat pada
algoritma \ref{alg:rf}.

\begin{center}
\captionof{algorithm}{Random Forest}
\label{alg:rf}
	\begin{algorithmic}[1]
\Require \\
$ TSET $: dataset pelatihan \\
$ T $: jumlah pohon untuk \textit{forest} \\
$ B $: persentase sampel yang dipilih secara acak tanpa diganti dari $TSET$ \\
$ m $: jumlah fitur acak \\

\Function{RandomForest}{$ TSET, T, B, m $}
	\State $ n \gets $ jumlah sampel pada dataset $ TSET $
	\State $ b \gets n * (B / 100) $
	\Comment Jumlah sampel untuk setiap pohon.

	\State $ forest \gets nil $
	\For{$ t \gets 1,T $}
		\State $ tp, tn, tree \gets \Call{GrowTree}{forest, TSET, b, m} $
		\State $ \Call{forest.Add}{tree} $
	\EndFor
	
	\State \Return $forest$
\EndFunction
\\
\Function{GrowTree}{$ forest, TSET, b, m $}
	\label{bagging}
	\State $ bag, oob, bagIdx, oobIdx \gets \Call{RandomPickSample}{$ TSET,
	b $} $

	\State $ tree \gets \Call{BuildTree}{bag, m} $

	\State $ \Call{forest.SaveBagIndex}{bagIdx} $
	\Comment Simpan pohon dan indeks dari \textit{bagging}

	\State $ cm \gets \Call{Classify}{forest, oob, oobIdx} $
	\State $ tp, tn \gets \Call{ComputeStat}{cm} $
	\Comment Hitung laju galat OOB dari \textit{confusion matrix}
	$cm$.

	\State \Return{$ tp, tn, tree $}
\EndFunction
\\
\Function{Classify}{$ forest, set, setIdx $}
	\State $ predictions \gets nil $
	\For{\textbf{each} $ sample, idx $ \textbf{in} $ set $}
		\State $ votes \gets \Call{ForestVotes}{forest, sample, idx} $
		\State $ class \gets \Call{Majority}{votes} $
		\Comment ambil mayoritas suara kelas pada $votes$
		\State $ predictions.push(class) $
	\EndFor

	\State \Return $predictions$
\EndFunction
\\
\Function{ForestVotes}{$ forest, sample, idx $}
	\State $ votes \gets nil $
	\For{\textbf{each} $ tree $ \textbf{in} $forest$ }
		\State $ exist \gets \Call{IsSampleInTheBag}{idx} $
		\Comment Periksa apakah sampel indeks digunakan pada pohon ini.

		\If{exist}
			\State continue
			\Comment Jika digunakan, lanjutkan ke pohon berikutnya.
		\EndIf

		\State $ class \gets \Call{tree.Classify}{sample} $

		\State $ \Call{votes.push}{class} $
	\EndFor

	\State \Return $ votes $
\EndFunction
	\end{algorithmic}
\end{center}

