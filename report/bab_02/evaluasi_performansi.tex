%% Jelaskan matriks confusion

Dalam konteks vandalisme, setiap sampel dalam dataset memiliki sebuah label
atau fitur kelas yang menandakan sampel tersebut termasuk ke dalam vandalisme
atau bukan.
Secara formalnya, setiap sampel dipetakan ke salah satu elemen dalam set
$\{p, n\}$, atau label kelas positif dan negatif.
Sebuah model klasifikasi (atau pengklasifikasi) adalah pemetaan dari sampel ke
kelas prediksi.
Untuk membedakan dengan kelas sebenarnya digunakan label $\{Y, N\}$ untuk kelas
prediksi yang dihasilkan oleh pengklasifikasi
\parencite{fawcett2006introduction}.

Diberikan sebuah pengklasifikasi dan sampel, maka ada empat kemungkinan
keluaran.
Jika label dari sampel positif dan diklasifikasi positif, maka dihitung sebagai
\textit{true-positive}, jika diklasifikasi negatif, maka dihitung sebagai
\textit{false-negative}.
Jika label dari sampel negatif dan diklasifikasi negatif, maka dihitung sebagai
\textit{true-negative}, jika diklasifikasi positif maka dihitung sebagai
\textit{false-positive}.
Jika diberikan sebuah pengklasifikasi dan sekumpulkan sampel (dataset
pengujian), sebuah \textit{confusion matrix} (atau disebut juga tabel
\textit{contingency}) dapat dibuat yang merepresentasikan penempatan dari
setiap sampel.

\begin{figure}[htbp]
	\centering
	\tikzsetnextfilename{confusionmatrix}
	\begin{tikzpicture}[
		framed,
		scale = 2,
		nodes = {
			align = center
		},
		proc/.style={
			rectangle,
			draw=black,
			text centered,
			minimum width=3cm,
			minimum height=2cm,
			outer sep=0pt,
		},
		label/.style={
			draw=none,
			text centered,
		}
	]
		\node[proc] (tp) {
			True Positive\\
			(TP)
		};

			\node[label] (p) [above=4pt of tp] {
				\textbf{p}
			};
			\node[label] (Y) [left=4pt of tp] {
				\textbf{Y}
			};

		\node[proc] (fp) [right=0pt of tp] {
			False Positive\\
			(FP)
		};
			\node[label] (n) [above=4pt of fp] {
				\textbf{n}
			};

		\node[proc] (fn) [below=0pt of tp] {
			False Negative\\
			(FN)
		};
			\node[label] (N) [left=4pt of fn] {
				\textbf{N}
			};
			\node[label] (P) [below=8pt of fn] {
				\textbf{P}
			};
			\node[label] (P) [left= of P] {
				\textbf{Total kolom:}
			};

		\node[proc] (tn) [right=0pt of fn] {
			True Negative\\
			(TN)
		};
			\node[label] (N) [below=8pt of tn] {
				\textbf{N}
			};

		\node[label] (actual) [above=of tp, xshift=1.5cm] {
			\underline{Kelas sebenarnya}
		};

		\node[label] (prediction) [left=of tp, yshift=-1cm] {
			\underline{Kelas prediksi}
		};
	\end{tikzpicture}
\caption{%
	\textit{Confusion matrix}
}
\label{confusion_matrix}
\end{figure}


Gambar \ref{confusion_matrix} memperlihatkan sebuah \textit{confusion matrix}
dari dua buah kelas.
Nilai pada kolom diagonal mayor (TP dan TN) merepresentasikan hasil klasifikasi
yang benar, sementara nilai pada diagonal minor (FN dan FP) merepresentasikan
galat -- \textit{confusion} -- antara kelas.
Nilai laju \textit{true-positive} atau \textit{true-positive rate} (TPR)
dikenal juga dengan \textit{recall} atau \textit{hit rate}) dari sebuah
pengklasifikasi dihitung dengan,

\begin{equation}
	\textit{tp rate} \approx \frac{\text{Benar terklasifikasi positif}}%
		{\text{Total positif}}
		= \frac{TP}{TP + FN}
		= \frac{TP}{P}
\end{equation}

TPR bernilai dengan rentang antara 0 sampai 1, dengan nilai yang mendekati 0
berarti performansi yang buruk dan nilai yang mendekati 1 berarti performansi
yang baik.

%% Jelaskan kenapa mengambil TPR / recall saja

Dalam deteksi vandalisme, menemukan sebuah suntingan yang vandal dengan tingkat
positif yang tinggi lebih baik daripada salah klasifikasi atau terlewatnya
suntingan vandal tersebut dari pendeteksian.
Kesalahan klasifikasi, yang bukan vandalisme terdeteksi sebagai vandalisme,
tidak akan berpengaruh pada pembaca, tetapi terlewatnya suntingan yang vandal
bisa menyebabkan hilangnya informasi, kesalahan informasi, atau mengganggu
pembaca Wikipedia.
Oleh karena itu, hasil pelatihan dan pengujian dilihat dari laju
\textit{true-positive} pada performansi pengklasifikasi.
