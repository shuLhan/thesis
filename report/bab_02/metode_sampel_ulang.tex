Sebuah dataset dikatakan timpang jika berisi lebih banyak sampel untuk satu
kelas daripada kelas lainnya.
Dataset dikatakan tidak seimbang bila paling tidak satu kelas direpresentasikan
hanya oleh sejumlah kecil contoh sampel (disebut dengan kelas minoritas)
sementara kelas lainnya menjadi mayoritas.
Dalam skenario ini, pengklasifikasi akan memiliki akurasi yang bagus pada kelas
mayoritas tapi akurasi yang buruk pada kelas minoritas yang disebabkan karena
pengaruh yang besar dari kelas mayoritas pada kriteria pelatihan sampel.
Dengan dataset yang tidak seimbang, algoritma penggalian data menghasilkan
model yang cacat yang tidak memperhitungkan kelas minoritas, secara kebanyakan
algoritma penggalian data mengasumsikan dataset yang seimbang.

Pada umumnya ada dua pendekatan dalam menghadapi permasalahan ketakseimbangan
pada kelas.
Pendekatan pertama beroperasi pada tingkat data yang terdiri dari
beberapa teknik sampel ulang data.
Pendekatan kedua bekerja pada tingkat algoritma yang mengikutkan penyesuaian
terhadap biaya misklasifikasi atau estimasi probabilitas.
Kebanyakan algoritma klasifikasi mencoba meminimalkan laju galat: persentase
prediksi yang tidak tepat dari label kelas.
Algoritma penyeimbang mengindahkan perbedaan antara tipe dari galat
misklasifikasi.
Khususnya, algoritma tersebut secara implisit mengasumsikan bahwa semua galat
misklasifikasi memiliki biaya yang sama.
Sejumlah solusi terhadap permasalahan ketimpangan kelas telah diajukan untuk
tingkat data dan algoritma \parencite{chawla2004editorial}.

%%{{{ subsec:metode-resampling
%%
\label{subsec:metode-resampling}
\subsection{Metode sampel ulang}

Metode yang paling mudah untuk menyeimbangkan kelas yaitu dengan melakukan
pengambilan ulang sampel, atau \textit{resampling}, baik dengan
\textit{over-sampling} pada kelas minoritas atau \textit{under-sampling} pada
kelas mayoritas, sampai kelas-kelas tersebut mendekati representasi yang sama.
Kedua strategi tersebut dapat digunakan pada sistem pembelajaran, secara mereka
bekerja pada fase pra-proses, membolehkan sistem pembelajaran menggunakan
instan data latihan seolah-olah berasal dari dataset yang seimbang.
Maka, bias apa pun dari sistem terhadap kelas mayoritas disebabkan karena
proporsi sampel yang berbeda bisa diharapkan berkurang.

Metode sampel ulang memiliki kelemahan.
\textit{Undersampling} bisa menghilangkan data yang berpotensi berguna,
sementara \textit{over-sampling} secara artifisial meningkatkan jumlah dataset
dan akibatnya memperburuk kerja komputasi pada algoritma pembelajaran.

Dengan kelemahan-kelemahan pada teknik penyeimbangan sampel, tetap metode
sampel ulang masih merupakan cara yang paling sering digunakan untuk menangani
masalah ketimpangan data daripada menggunakan algoritma pembelajaran
\textit{cost-sensitive}.
Hal ini karena, pertama, tidak semua implementasi algoritma
\textit{cost-sentitive} ada pada bahasa pemrograman, maka pendekatan
menggunakan sampel ulang menjadi satu-satunya pilihan.
Alasan kedua, kebanyakan dataset yang timpang sangatlah besar dan ukuran
dataset latihan harus dikurangi supaya pembelajaran dapat menghasilkan model.
Untuk alasan ini, \textit{under-sampling} menjadi strategi yang masuk akal dan
valid.
Jika ingin menghilangkan beberapa data latihan akan lebih menguntungkan bila
sampel pada kelas mayoritas yang dihilangkan dengan tujuan untuk mengurangi
jumlah data latihan, dan kemudian menggunakan algoritma \textit{cost-sensitive}
terhadapnya.
Alasan terakhir kenapa sampel ulang digunakan adalah biaya misklasifikasi
terkadang tidak diketahui.
%%}}}

%%{{{ subsubsec:oversampling
\subsubsection{\textit{Oversampling}}\label{subsubsec:oversampling}

Metode paling sederhana untuk meningkatkan jumlah kelas minoritas bisa
menggunakan \textit{oversampling} acak, yaitu metode non-heuristik yang
menyeimbangkan distribusi kelas dengan replikasi acak dari sampel positif.
Disebabkan karena metode ini mereplikasi sampel yang ada pada kelas minoritas,
maka permasalahan \textit{overfit} akan mungkin terjadi, yang mana model akan
bagus untuk data latihan tapi buruk bila digunakan pada data yang baru.
Kelemahan lain yaitu menambah jumlah waktu pada pelatihan sampel karena jumlah
data yang bertambah.
%%}}}

%%{{{undersampling
%%
\label{subsubsec:undersampling}
\subsubsection{\textit{Undersampling}}

\textit{Under-sampling} adalah metode yang efisien untuk pembelajaran kelas
yang timpang.
Metode ini menggunakan subset dari kelas mayoritas untuk melatih
pengklasifikasi.  Secara banyak sampel dari kelas mayoritas yang dihilangkan,
data latihan menjadi lebih seimbang dan proses pelatihan menjadi lebih cepat.
Teknik paling umum adalah \textit{random majority under-sampling} (RUS).  Pada
RUS, instan dari kelas mayoritas secara acak dihilangkan dari dataset.

Kelemahan utama dari \textit{under-sampling} adalah informasi yang secara
potensial berguna yang terdapat pada sampel yang dihilangkan tersebut menjadi
terindahkan.

%%{{{
\label{subsec:algcostsensitive}
\subsection{Metode Algoritma Pembelajaran \textit{Cost-Sensitive}}

Pada tingkat algoritma, beberapa solusinya adalah menyesuaikan biaya pada
beberapa kelas untuk menangani ketimpangan kelas, mengatur estimasi
probabilitas pada \textit{tree leaf} (bila menggunakan \textit{decision tree}),
mengatur ambang batas keputusan, dan menggunakan teknik pembelajaran berbasis
rekognisi (yaitu, pembelajaran dengan satu kelas) daripada berbasis
diskriminasi (dua kelas).

Pembelajaran \textit{Cost-Sensitive} adalah sebuah tipe dari pembelajaran dalam
penggalian data yang menggunakan biaya misklasifikasi (dan bisa juga tipe biaya
yang lain) menjadi pertimbangan.
Ada banyak cara untuk mengimplementasikan pembelajaran \textit{cost sensitive},
yang bisa dikategorikan menjadi tiga kelas \parencite{he2009learning}.
Kelas pertama menggunakan biaya misklasifikasi terhadap dataset sebagai sebuah
bentuk berat ruang data, kelas kedua menerapkan teknik minimalisasi biaya
terhadap skema kombinasi dari metode \textit{ensemble}, dan kelas terakhir
menggabungkan fitur \textit{cost sensitive} langsung ke paradigma klasifikasi
untuk secara esensial menyesuaikan kerangka kerja \textit{cost sensitive} ke
dalam pengklasifikasi.

Menggabungkan biaya ke dalam algoritma klasifikasi \textit{decision tree}
adalah yang paling banyak digunakan dan pengklasifikasi yang paling sederhana.
Biaya dapat diikutkan dengan berbagai cara.  Cara pertama adalah biaya dapat
diterapkan untuk menyesuaikan ambang batas keputusan, cara kedua adalah biaya
dapat digunakan dalam memisahkan pemilihan atribut saat pembentukan
\textit{decision tree}, dan terakhir yaitu skema pemotongan \textit{cost
sensitive} dapat diterapkan pada \textit{tree}.
%%}}}
