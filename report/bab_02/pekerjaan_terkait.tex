Deteksi vandalisme di Wikipedia berbasis pendekatan pembelajaran mesin telah
menjadi topik penelitian yang menarik sejak tahun 2008.
Potthast \cite{potthast2008automatic} berkontribusi untuk pendekatan deteksi
vandalisme dengan pembelajaran mesin yang pertama, dengan menggunakan fitur
tekstual berikut fitur meta data dasar dengan menggunakan pengklasifikasi
\textit{logistic regression}.
Smets \cite{smets08automaticvandalism} menggunakan pengklasifikasi Naive Bayes
pada sekumpulan kata yang merepresentasikan suntingan dan yang pertama
menggunakan model kompresi untuk mendeteksi vandalisme di Wikipedia.
Itakura dan Clarke \cite{itakura2009using} menggunakan Kompresi Markov Dinamis
untuk mendeteksi suntingan vandalisme di Wikipedia.
Mola Velasco \cite{mola2012wikipedia} mengembangkan pendekatan yang dilakukan
oleh Potthast \cite{potthast2008automatic} dengan menambahkan beberapa fitur
tekstual dan berbagai fitur berbasis daftar-kata.
Velasco memenangi \textit{1st International Competition on Wikipedia Vandalism
Detection}.  West dkk. \cite{west2011multilingual} adalah yang pertama
mengajukan sebuah pendekatan deteksi vandalisme hanya berdasarkan meta data
spasial dan temporal, tanpa perlu memeriksa teks pada artikel dan revisi.
Adler dkk. \cite{adler2010detecting} membangun sebuah sistem deteksi vandalisme
menggunakan sistem reputasi WikiTrust.
Adler dkk. \cite{adler2011wikipedia} kemudian menggabungkan bahasa alami,
spasial, temporal, dan fitur reputasi yang digunakan pada karya sebelumnya.
West dan Lee \cite{west2011multilingual} adalah yang pertama memperkenalkan
data \textit{ex post facto} sebagai fitur, yang mana perhitungannya
mempertimbangkan revisi selanjutnya.
Sistem deteksi vandalisme West dan Lee memenangkan \textit{2nd International
Competition on Wikipedia Vandalism Detection}.
Harpalani dkk. \cite{harpalani2011language} menyatakan suntingan vandalisme
memiliki properti lingustik yang unik dan sama.
Harpalani dkk. membangun sistem deteksi vandalisme berdasarkan analisis
\textit{stylometric} dari suntingan vandalisme dengan model probabilitas
\textit{context-free grammar}.
Pendekatan Harpalani dkk. mengalahkan sistem berbasis fitur dengan pola
dangkal, yang menyamakan struktur sintaksis dan token teks.
Mengikuti tren dari klasifikasi vandalisme antar bahasa, Tran dan Christen
\cite{tran2013cross} mengevaluasi berbagai pengklasifikasi berbasiskan pada
sekumpulan fitur independen bahasa yang dikumpulkan dari jumlah artikel dilihat
setiap jam dan riwayat suntingan Wikipedia.

Gotze \cite{gotze2014advanced} menggabungkan fitur dari Adler dkk.
\cite{adler2011wikipedia}, Javanmardi dkk. \cite{javanmardi2011vandalism}, Mola
Velasco \cite{mola2012wikipedia}, Potthast dkk. \cite{potthast2008automatic},
Wang dan McKeown \cite{wang2010got}, dan West dan Lee
\cite{west2011multilingual} dengan empat fitur tambahan dan perubahan.
Untuk mengatasi masalah ketimpangan pada korpus, Gotze
\cite{gotze2014advanced}
mengaplikasikan teknik
\textit{random oversampling}
bernama
\textit{Synthetic Minority Over-sampling TEchnique} (SMOTE)
yang diajukan oleh Chawla
\cite{chawla2002smote},
dan kombinasi dari SMOTE dan
\textit{random undersampling}.
Dataset latihan yang orisinal dan hasil sampel ulang diuji dengan
pengklasifikasi satu-kelas dan dua-kelas.
Pengklasifikasi satu-kelas yang diterapkan diantaranya Hempstalk dkk.
\cite{hempstalk2008one}
dan SVM oleh Schölkopf dkk.
\cite{scholkopf1999support}
yang diimplementasikan oleh Chang dan Lin
\cite{chang2011libsvm}
pada korpus PAN-WVC.
Pengklasifikasi dua-kelas yang diterapkan diantaranya
\textit{Logistic Regression},
\textit{RealAdaBoost},
\textit{Random Forest} (RF), dan
\textit{Bayesian Network}.
Hasil percobaan yang didapat memperlihatkan performansi pengklasifikasi
satu-kelas tidak kompetitif dengan satu pun pengklasifikasi dua-kelas.
Hal ini bisa disebabkan karena tidak sesuainya kelompok fitur yang digunakan
untuk menjelaskan suntingan vandalisme, sebagaimana juga parameter pengaturan
yang tidak sesuai pada pendekatan yang digunakan.
Hasil dari pelatihan pada dataset orisinal memperlihatkan RF
lebih unggul dari pengklasifikasi lainnya.
Hasil dari pelatihan pada dataset hasil sampel ulang memperlihatkan adanya
peningkatkan pada semua pengklasifikasi kecuali pada RF.



Dari penelitian di atas, tujuh diantaranya menggunakan PAN-WVC-10
\cite{adler2010detecting}
\cite{adler2011wikipedia}
\cite{gotze2014advanced}
\cite{harpalani2011language}
\cite{mola2012wikipedia}
\cite{wang2010got}
\cite{west2011multilingual},
dengan nilai presisi terbaik yaitu $0,86$, nilai \textit{recall} $0,57$, dan
PR-AUC $0,66$ didapat oleh Velasco menggunakan \textit{Random Forest} tanpa
penyeimbangan dataset.
Hanya dua yang menggunakan PAN-WVC-11 \cite{gotze2014advanced}
\cite{west2011multilingual} dengan hasil terbaik dipegang oleh Gotze yaitu
dengan nilai presisi $0,92$, \textit{recall} $0,39$, dan PR-AUC $0,74$.
