Permasalahan vandalisme di Wikipedia telah terjadi sejak adanya Wikipedia itu
sendiri.
Komunitas Wikipedia menangani permasalahan tersebut dengan membuat fungsi
pengaman pada artikel secara manual supaya artikel tidak bisa disunting, bila
sebuah artikel terlalu sering divandal.
Sejak tahun 2006, bot pendeteksi vandalisme digunakan, yang secara otomatis
memantau suntingan vandalisme dan terkadang mengembalikannya.
Umumnya bot ini menggunakan aturan heuristik sederhana, daftar hitam kata, dan
daftar alamat \textit{Internet Protocol} (IP) pengguna yang
diblokir yang terdeteksi melakukan vandalisme (contohnya, VoABot II
\footnote{\url{https://en.wikipedia.org/wiki/User:VoABot_II}}
dan ClueBot
\footnote{\url{https://en.wikipedia.org/wiki/User:ClueBot}}).
ClueBot-NG
\footnote{\url{https://en.wikipedia.org/wiki/User:ClueBot_NG}}
menggantikan ClueBot, menggunakan pendekatan pembelajaran mesin.
Bot ini mencoba memperbaiki teknik berbasis heuristik yang susah untuk dirawat
dan mudah dilewat.
Bot ini menggunakan dataset suntingan pra-klasifikasi yang dianotasikan oleh
pengguna Wikipedia untuk melatih Jaringan Saraf Tiruan.
Pengklasifikasinya bekerja pada beberapa fitur suntingan, seperti probabilitas
tingkat-kata vandalisme untuk mengklasifikasi suntingan baru.
