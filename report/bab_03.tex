\chapter{Proses Deteksi Vandalisme}
\label{bab:03}

Pada bab ini dijelaskan proses dari pengolahan dataset dan implementasi
algoritma sampel ulang dan klasifikasi, dimulai dari persiapan
data, pembuatan dataset fitur, sampel ulang pada dataset fitur, dan
implementasi pengklasifikasi sehingga nantinya dapat digunakan untuk pelatihan,
pengujian dan analisis.
Alur dari proses secara umum dapat dilihat pada gambar \ref{fig:diagramproses}.

\begin{figure}[htbp]
\centering
\resizebox{.5\textwidth} {!} {%
\mytikzinput{diagramproses}
\begin{tikzpicture}[
	framed,
	nodes = {
		align = center
	},
	lines/.style={
		line width=2pt,
		>=latex
	},
	data/.style={
		circle,
		draw=black,
		text centered
	},
	proc/.style={
		rectangle,
		draw=black,
		text centered
	}
]
	\node[data] (training_raw) {
		Dataset pelatihan\\
		mentah\\
		(\textit{PAN-WVC-10})
	};

	\node[proc] (wvcgen) [below=of training_raw] {
		Ekstraksi\\fitur
	};

	\node[proc] (trainingset) [below=of wvcgen] {
		Sampel ulang
	};

	\node[proc] (c)  [below=of trainingset] {
		Pembangkitan model\\
		deteksi
	};

	\node[proc] (m)  [below=of c] {
		Model deteksi\\
		vandalisme
	};

	\node[data] (o)  [below=of m] {
		Hasil deteksi
	};
	%%
	\node[data] (testset_raw) [right=of training_raw] {
		Dataset pengujian\\mentah\\
		(\textit{PAN-WVC-11})
	};

	\node[proc] (testset_wvcgen) [below=of testset_raw] {
		Ekstraksi\\fitur
	};

	\node[data] (testset) [below=of testset_wvcgen] {
		Dataset\\
		pengujian
	};

	\draw[lines,->] (training_raw) -- (wvcgen);
	\draw[lines,->] (wvcgen) -- (trainingset);
	\draw[lines,->] (trainingset) -- (c);
	\draw[lines,->] (c) -- (m);
	\draw[lines,->] (m) -- (o);

	\draw[lines,->] (testset_raw) -- (testset_wvcgen);
	\draw[lines,->] (testset_wvcgen) -- (testset);

	\draw[lines,->] (testset) |- (m);
\end{tikzpicture}
}
\caption{
	Proses implementasi deteksi vandalisme
}
\label{fig:diagramproses}
\end{figure}


	\section{Gambaran Umum}
	\label{bab:03:gambaran_umum}
	Dataset yang didapat tidak langsung bisa digunakan untuk pelatihan dan
pengujian, melainkan perlu ada pra-proses terlebih dahulu yang mengikutkan
penggabungan data dan pembersihan data dengan menghapus atribut yang tidak
diperlukan dan pembersihan dari isi data itu sendiri.
Setelah data mentah dibersihkan, kemudian diekstrak fitur untuk masing-masing
dataset pelatihan dan pengujian sehingga menghasilkan dataset pelatihan dan
pengujian yang bernilai kontinu.
Dataset pelatihan disampel ulang dengan metode SMOTE dan LNSMOTE,
sehingga menghasilkan tiga dataset yang akan dilatih yaitu dataset tanpa sampel
ulang, dataset yang telah disampel ulang dengan SMOTE, dan dataset yang telah
disampel ulang dengan LNSMOTE.
Ketiga dataset pelatihan tersebut kemudian dijalankan pada program
pengklasifikasi \textit{Random Forest} (RF) dan \textit{Cascaded Random Forest}
(CRF) satu per satu, yang menghasilkan sebuah model deteksi vandalisme.
Model tersebut kemudian diuji dengan menginputkan dataset fitur pengujian
sehingga menghasilkan kelas-kelas dari sampel yang nantinya dapat digunakan
untuk mengevaluasi sampel ulang dan pengklasifikasi.


	\section{Persiapan Dataset Pelatihan}
	\label{bab:03:persiapan_data_pelatihan}
	Dataset yang digunakan untuk pelatihan yaitu PAN-WVC-10
\cite{potthast:2010b}.
Dataset PAN-WVC-10 terbagi menjadi dua yaitu dataset suntingan dari artikel
Wikipedia dan dataset anotasi yang berisi hasil klasifikasi vandalisme pada
dataset suntingan tersebut.
Dataset suntingan memiliki atribut sebagai berikut,

\begin{itemize}
	\item \textbf{editid}, format angka, berisi identifikasi (ID) unik dari setiap suntingan.
	\item \textbf{editor}, format string, berisi nama penyunting.
	\item \textbf{oldrevisionid}, format angka, berisi ID untuk suntingan lama.
	\item \textbf{newrevisionid}, format angka, berisi ID untuk suntingan baru.
	\item \textbf{diffurl}, format string, berisi URL yang mengacu pada perbedaan suntingan baru dengan lama.
	\item \textbf{edittime}, format string, berisi tanggal dan pukul
	suntingan.
	\item \textbf{editcomment}, format string, berisi komentar yang ditambahkan oleh penyunting saat menyimpan hasil suntingan.
	\item \textbf{articleid}, format angka, berisi ID unik dari artikel.
	\item \textbf{articletitle}, format string, berisi judul dari artikel yang disunting.
\end{itemize}

Dataset anotasi memiliki atribut sebagai berikut,

\begin{itemize}
	\item \textbf{editid}, format angka, mengacu pada \textit{editid} di
	dataset suntingan.
	\item \textbf{class}, format string, berisi tipe suntingan yang
	bernilai "regular" yang menyatakan bahwa suntingan tersebut bukan
	vandalisme, dan "vandalism" yang menyatakan bahwa suntingan tersebut
	adalah vandalisme.
	\item \textbf{annotators}, format angka, berisi jumlah orang yang
	menandai (penanda) bahwa suntingan dengan ID tersebut termasuk ke dalam
	kelas "regular" atau "vandalism".
	\item \textbf{totalannotators}, format angka, berisi jumlah total penanda yang memeriksa suntingan.
\end{itemize}

\newpage
Kedua dataset kemudian digabung untuk menghasilkan atribut \textit{editid},
\textit{class}, \textit{oldrevisionid}, \textit{newrevisionid},
\textit{edittime}, \textit{editor}, \textit{articletitle},
\textit{editcomment}, \textit{deletions}, dan \textit{additions}.
Nilai dari atribut \textit{class} diganti dari teks menjadi angka, yaitu 1
untuk "vandalism" dan 0 untuk "regular".
Atribut tambahan \textit{deletions} berisi teks yang dihapus dalam revisi yang
lama.
Atribut tambahan \textit{additions} berisi teks yang ditambahkan dalam revisi
yang baru.
Kedua atribut tersebut didapat dengan membandingkan isi dari revisi yang lama
dengan yang baru.

Pemrosesan selanjutnya yaitu membuat berkas revisi yang bersih dari sintaks
wiki.
Tujuan dari revisi ini yaitu supaya tidak ada \textit{noise} pada saat
melakukan penghitungan fitur dan mempercepat proses pembuatan fitur.
Setiap berkas revisi dibaca kemudian dilakukan pembersihan berikut,

\begin{itemize}
\item penghapusan URI yang berawalan dengan
\textit{http://}, \textit{https://}, \textit{ftp://}, dan \textit{ftps}.
\item Menghapus \textit{mark-up} wiki yaitu konten yang dilingkupi oleh marka
berikut:
\texttt{[[Category:]]}, \texttt{[[:Category:]]}, \texttt{[[File:]]}, \\
\texttt{[[Help:]]}, \texttt{[[Image:]]}, \texttt{[[Special:]]},
\texttt{[[Wikipedia:]]}, \\
\texttt{\{\{DEFAULTSORT:\}\}}, \texttt{\{\{Template:\}\}}, dan \texttt{<ref/>}.
\item Mengganti karakter dan token berikut dengan karakter kosong (spasi):
\texttt{[}, \texttt{]}, \texttt{\{}, \texttt{\}}, \texttt{|}, \texttt{=},
\texttt{\#}, \texttt{'s}, \texttt{'}, \texttt{<ref>}, \texttt{</ref>},
\texttt{<br />}, \texttt{<br/>}, \texttt{<br>}, \texttt{<nowiki>},
\texttt{</nowiki>}, \texttt{\&nbsp;}.
\item Menghapus karakter kosong yang berlebihan.
\end{itemize}


	\section{Persiapan Dataset Pengujian}
	\label{bab:03:persiapan_data_pengujian}
	Dataset yang digunakan untuk pengujian yaitu PAN-WVC-11
\cite{potthast:2010b}.
Korpus PAN-WVC-11 terdiri dari tiga bahasa yaitu Inggris, Jerman, dan
Spanyol, yang digunakan untuk pelatihan hanya yang bahasa Inggris.
Dataset asli memiliki atribut yang sama dengan PAN-WVC-10, yaitu
\textit{editid},
\textit{editor},
\textit{oldrevisionid},
\textit{newrevisionid},
\textit{diffurl},
\textit{class},
\textit{annotators},
\textit{totalannotators},
\textit{edittime},
\textit{editcomment},
\textit{articleid}, dan
\textit{articletitle}.

Atribut \textit{annotators} dan \textit{totalannotators} dihilangkan dan
ditambah dengan dua atribut baru yaitu \textit{deletions} yang berisi teks yang
dihapus pada revisi lama, dan \textit{additions} yang berisi teks yang
ditambahkan pada revisi yang baru.
Nilai dari atribut \textit{class} diganti dari string menjadi integer, yaitu
dari "vandalism" menjadi bilangan \texttt{1}
dan "regular" menjadi bilangan \texttt{0}.

Untuk proses pembersihan data dilakukan prosedur yang sama seperti pada
PAN-WVC-10.


	\section{Ekstraksi Fitur}
	\label{bab:03:ekstraksi_fitur}
	Dari analisis fitur vandalisme pada \ref{bab:02:fitur_vandalisme} kemudian
ditransformasikan ke dalam sebuah program.
Program ekstraksi fitur kemudian dijalankan dengan input dataset PAN-WVC-10
dan PAN-WVC-11 yang telah dibersihkan pada subbab
\ref{bab:03:persiapan_data_pelatihan} dan
\ref{bab:03:persiapan_data_pengujian},
sehingga menghasilkan dataset fitur yang bernilai kontinu.
Implementasi program pembersihan data dan generator fitur ini dapat dilihat dan
digunakan secara terbuka pada repositori \texttt{wvcgen}
\footnote{\url{https://github.com/shuLhan/wvcgen}}.
Nilai dari fitur yang desimal dibulatkan menjadi lima angka dibelakang koma.

Urutan ekstrasi fitur adalah sebagai berikut: \textit{editid}, anonim,
panjang komentar, peningkatan ukuran, rasio ukuran, rasio huruf besar dan
kecil, rasio huruf besar, rasio angka, rasio non-alfanumerik, diversitas
karakter, distribusi karakter, laju kompresi, token umum, frekuensi rerata
kata, kata terpanjang, urutan karakter terpanjang, frekuensi kata vulgar, impak
kata vulgar, frekuensi kata subjek, impak kata subjek, frekuensi kata bias,
impak kata bias, frekuensi kata pornografi, impak kata pornografi, frekuensi
kata buruk, impak kata buruk, frekuensi semua kategori kata, impak semua
kategori kata, dan kelas. Berikut contoh lima baris awal dan akhir dari hasil
ekstraksi fitur pada dataset PAN-WVC-10.

\begin{lstlisting}[style=data,basicstyle=\footnotesize\ttfamily]
1,0,0,4,0.99964,1,1,0.5,1,1,0,3,0,0,0,0,0,0.5,0,0.5,0,0.5,0,0.5,0,0.5,0,0.5,0
2,0,35,978,0.96855,0,0,0,0,0,0,0,0,0,0,0,0,0.52173,0,0.50563,0,0.50156,0,0.5,0,0.5283,0,0.50534,0
3,1,0,1,0.99995,0.5,0.5,0.5,0,1,7.29251,3,0,0,1,0,0,0.5,0,0.5,0,0.5,0,0.5,0,0.5,0,0.5,0
4,0,67,94,1.00678,0.29761,0.23148,0.05956,0.25471,1.10837,1.18222,0.6415,12,0.02692,10,2,0,0.65,0.3,0.72838,0,0.60606,0,0.5,0,0.77906,0,0.72837,0
5,0,130,1,0.99986,0.27777,0.22727,0.03846,0.04,1.25849,2.16454,1,0,0.04682,8,0,0,0.5,0.5,0.5,0,0.5,0,0.5,0,0.5,0,0.5,0
...
45962,0,0,35,1.00036,0.5,0.34615,0.05555,0.11428,1.15967,2.19877,1.08571,1,0.06435,11,2,0,0.5,0.83333,0.5,0,0.5,0,0.5,0,0.5,0,0.5,0
45963,0,0,59,1.00239,0.11904,0.10869,0.01666,0.08474,1.16979,0.96304,1.03389,3,0.08203,10,2,0,0.5,0.75,0.49878,0,0.5,0,0.5,0,0.5,0,0.49891,0
45964,0,0,144,1.00293,0.08888,0.08247,0.10958,0.10344,1.11425,0.96351,1.01379,3,0,0,2,0,0.5,0,0.5,0,0.5,0,0.5,0,0.5,0,0.5,0
45965,1,0,1,0.99964,0,0,0,0,0,0,0,0,0,0,0,0,0.5,0,0.5,0,0.5,0,0.5,0,0.5,0,0.5,0
45966,0,17,385,1.045,0.10943,0.09897,0.03341,0.04639,1.11671,0.33307,0.76288,4,0.0851,11,2,0.01562,0.5,0.54687,0.5,0,0.5,0,0.5,0,0.5,0,0.5,0
\end{lstlisting}

Berikut contoh lima baris awal dan akhir dari hasil ekstraksi fitur pada
dataset PAN-WVC-11, dengan urutan fitur yang sama dengan PAN-WVC-10,

\begin{lstlisting}[style=data,basicstyle=\footnotesize\ttfamily]
416341,0,133,51,1.03148,0.19444,0.16666,0.01923,0.0196,1.18643,0.72515,1.05882,0,0.21505,8,0,0,0.5,0.18181,0.4923,0,0.5,0,0.5,0,0.5,0,0.49275,0
416342,0,0,101,1.00663,0.08196,0.07692,0.03669,0.18518,1.14764,0.55365,0.9537,3,0.12137,11,2,0.05882,0.5,0.47058,0.5,0,0.5,0,0.5,0,0.5,0,0.5,0
416343,1,0,1744,0.95762,17,1,0.03846,0,1.33994,7.50425,1.08,0,0.09672,5,0,0,0.50299,0,0.51133,0.5,0.49777,0,0.5,0,0.51029,0.5,0.51047,1
416345,0,0,41,1.00776,0.09677,0.0909,0.0238,0.14634,1.20403,1.0729,1.09756,2,0.04375,13,2,0.2,0.5,0.4,0.5,0,0.5,0,0.5,0,0.5,0,0.5,0
416346,0,31,50,0.99857,0.03643,0.03517,0.03539,0.09296,1.10003,0.0772,0.66356,49,0.43945,13,2,0.00357,0.5,0.41785,0.50261,0.01071,0.52631,0,0.5,0,0.5,0.00714,0.50279,0
...
480155,1,123,1,1.00028,0.16666,0.16666,0.16666,0,1.49534,4.13689,1.4,0,0,5,2,0,0.5,1,0.49886,0,0.5,0,0.5,0,0.5,0,0.49896,0
480156,0,152,6,0.99299,0.1,0.1,0.07692,0.25,1.23007,2.24007,1.25,1,0,10,2,0,0.5,0,0.5,0,0.5,0,0.5,0,0.5,0,0.5,0
480158,1,0,768,1.03608,0.04703,0.045,0.00129,0.0286,1.16302,0.26067,0.71521,0,0.34586,12,2,0.02083,0.46938,0.41666,0.4744,0.01388,0.47058,0,0.5,0,0.46902,0.04861,0.47368,1
480159,1,48,52381,0.00024,0,0,0,0,0,0,0,0,0,0,0,0,1,0,1,0,1,0,0.5,0,1,0,1,1
480160,0,14,1109,1.10722,0.08823,0.08119,0.05327,0.15122,1.10911,0.18691,0.64685,71,0.29261,32,3,0.03539,0.48437,0.50442,0.47991,0,0.5,0,0.5,0,0.48965,0,0.48135,0
\end{lstlisting}


	\section{Pembuatan Sampel Ulang}
	\label{bab:03:pembuatan_sampel_ulang}
	Dataset PAN-WVC-10 tanpa sampel ulang memiliki jumlah sampel vandal atau
positif yaitu 2.394 sampel dan 30.045 sampel bukan vandal atau negatif.
Untuk mendapatkan jumlah sampel yang seimbang, setiap dataset disampel ulang
dengan teknik SMOTE atau LNSMOTE untuk kelas yang positif.


	\subsection{Pembuatan Sampel Ulang SMOTE}
	\label{bab:03:sampel_ulang_smote}
	Implementasi SMOTE dibuat ke dalam sebuah program
\footnote{\url{%
https://github.com/shuLhan/go-mining/tree/master/resampling/smote
}}
berdasarkan makalah aslinya
\parencite{chawla2002smote}.
Gambaran umum dari implementasi dapat dilihat pada algoritma \ref{alg:smote}.

Program SMOTE dijalankan dengan memberikan input sampel minoritas dari dataset
fitur PAN-WVC-10.
Untuk mendapatkan jumlah sampel positif yang mendekati jumlah sampel negatif,
maka parameter persentase jumlah sampel sintetis diberi dengan nilai 1.200\%,
yang berarti 12 kali jumlah sampel minoritas.
Parameter untuk jumlah tetangga terdekat yang dicari untuk dapat membuat sebuah
sampel diberi dengan nilai 5.
Nilai ini diberikan untuk menghindari sampel minoritas yang \textit{outliers}
tapi juga masih dapat mencakup kelompok-kelompok kelas minoritas yang kecil.
Dari ketiga parameter tersebut didapat hasil sampel ulang dengan jumlah sampel
sintetis positif yaitu 28.728 sampel, digabung dengan sampel positif asli
totalnya yaitu 31.122 sampel positif.

	\begin{center}
	\captionof{algorithm}{SMOTE}
	\label{alg:smote}
	\begin{algorithmic}[1]
\Require \\
$ P $: sampel minoritas \\
$ o $: persentase jumlah sampel sintetis yang akan dibuat \\
$ k $: jumlah tetangga terdekat dari sampel yang akan dibuat
\\
\Function{SMOTE}{$ P, o $}
	\State $ S \gets [] $
	\Comment Penampung untuk semua sampel sintetis yang akan dibuat.

	\State $ nsyn \gets 0 $
	\Comment Jumlah sampel sintetis yang akan dibuat.

	\State $ lenP \gets \Call{Length}{P} $

	\If{$o < 100$}
		\State $ nsyn \gets (o / 100) \times lenP $
		\State $ P \gets \Call{RandomPickWithoutReplacement}{P, nsyn} $
		\State $ o \gets 100.0 $
	\EndIf

	\State $ nsyn \gets o / 100.0 $

	\For{\textbf{each} $sample$ \textbf{in} $P$}
		\State $ neighbors \gets \Call{FindNeighbours}{P, sample, k} $
		\Comment Cari k buah tetangga terdekat dari $sample$ di P.

		\State $ s \gets \Call{CreateSynthetic}{sample, neighbors} $

		\State $ \Call{S.Push}{s} $
	\EndFor

	\State \Return{S}
\EndFunction
\\
\Function{CreateSynthetic}{$ sample, neighbors, nsyn $}
	\State $ s \gets [] $
	\Comment Penampung untuk semua sampel sintetis yang akan dibuat.

	\State $ lenAttr \gets \Call{LengthAttribute}{sample} $
	\Comment Simpan jumlah atribut dari sampel.

	\For{$ x \gets 1,nsyn $}
		\State $ neighbor \gets \Call{RandomPick}{neighbors} $
		\State $ newsample \gets [] $
		\Comment Penampung untuk sampel sintetis yang baru.

		\For{$ y \gets 1,lenAttr $}
			\If{y \textbf{is} classAttribute}
				\State continue
			\EndIf

			\State $ nattr \gets neighbor.\Call{AttributeAt}{y} $
			\State $ sattr \gets sample.\Call{AttributeAt}{y} $
			\State $ diff \gets sattr - nattr $
			\State $ gap \gets \Call{Random}{0,1} $
			\State $ newAttr \gets nattr + (gap \times diff) $

			\State $ newSample.\Call{PushAttribute}{newAttr} $
		\EndFor

		\State $ \Call{s.Push}{newSample} $
	\EndFor

	\State \Return{s}
\EndFunction

	\end{algorithmic}
\end{center}


Sampel ulang SMOTE memiliki tiga parameter yaitu dataset minoritas yang akan
disampel ulang ($P$), persentase sampel sintetis yang akan
dibuat ($o$), dan jumlah tetangga terdekat yang dicari pada sampel
yang akan dibuat ($k$).
Dataset minoritas yaitu dataset yang berisi hanya sampel dengan kelas
minoritas.
Jika parameter persentase lebih kecil dari 100 (persen) maka hanya $o$ persen
dari dataset $P$ yang disampel ulang.
Parameter jumlah tetangga terdekat ($k$) menentukan jumlah sampel minoritas
lain yang dijadikan sebagai patokan dalam pembuatan sampel sintentis.
Semakin besar nilai $k$ maka distribusi sampel sintetis di dalam dataset juga
semakin luas, semakin kecil nilai $k$ maka antara sampel sintetis
dengan sampel asli semakin mirip/dekat.
Berikut penjelasan fungsi-fungsi tambahan yang ada dalam algoritma.

Fungsi \texttt{RandomPickWithoutReplacement} memiliki dua parameter yaitu
dataset ($P$) dan jumlah sampel yang akan diambil ($n$).  Fungsi ini
mengembalikan $n$ buah sampel dari $P$ yang diambil secara acak tanpa diganti,
artinya setelah sampel diambil sampel tersebut dihapus dari $P$.

Fungsi \texttt{FindNeighbours} memiliki tiga parameter yaitu dataset ($P$),
sebuah sampel yang akan dicari tetangganya ($s$), dan jumlah tetangga yang akan
dicari di sekitar $s$ ($k$).
Fungsi ini mengembalikan $k$ sampel tetangga dari $s$ di dalam dataset $P$.
Jumlah sampel tetangga yang dikembalikan bisa lebih kecil dari nilai $k$.

Fungsi \texttt{CreateSynthetic} memiliki tiga parameter yaitu sebuah sampel
(\textit{sample}), sampel-sampel tetangga dari $sample$ ($neighbors$), dan
jumlah sampel sintetis yang akan dibuat ($nsyn$).
Fungsi ini membuat dan mengembalikan $nsyn$ buah sampel sintetis dari $sample$
yang berdekatan dengan tetangganya ($neighbors$).

Fungsi \texttt{RandomPick} memiliki sebuah parameter yaitu sebuah dataset dan
fungsi ini mengembalikan sebuah sampel yang diambil secara acak dari dataset.

Fungsi \texttt{Random} memiliki dua parameter bilangan yaitu nilai minimum dan
maksimum, fungsi ini mengembalikan bilangan acak real dengan rentang antara
minimum dan maksimum.


	\subsection{Pembuatan Sampel Ulang LNSMOTE}
	\label{bab:03:sampel_ulang_lnsmote}
	Proses sampel ulang LNSMOTE yang digunakan pada implementasi berdasarkan
pada makalah Maciejewski dkk.
\cite{maciejewski2011local}
yang dapat dilihat pada algoritma
\ref{alg:lnsmote}.
Hasil implementasinya dalam bentuk sebuah program
\footnote{\url{%
https://github.com/shuLhan/go-mining/tree/master/resampling/lnsmote
}}.
Program LNSMOTE dijalankan dengan diberikan input dataset fitur PAN-WVC-10
dengan persentase sampel sintetis yaitu 1.100\% dan jumlah tetangga terdekat
yaitu 5, sehingga didapat 28.588 sampel sintetis positif, ditambah dengan
sampel positif asli totalnya menjadi 30.982 sampel positif.

	\newpage
	\begin{algorithm}[h]
	\caption{Local Neighbourhood SMOTE}
	\label{alg:lnsmote}
	\begin{algorithmic}[1]
\Require \\
S: dataset sampel keseluruhan \\
P: sample minoritas dari $ S $ \\
o: rasio \textit{over-sampling}, jumlah sampel sintetis yang akan dibuat \\

\Function{LNSMOTE}{$ S, P, o $}
	\State $ SEEDS \gets P $
	\State $ OUT \gets S $
	\For{\textbf{each} $ p $ \textbf{in} $ SEEDS $}
		\State $ NN \gets \Call{KNearestNeighbor}{S, p} $
		\For{$ i \gets 1 \dots o $}
			\State $ s \gets \Call{CreateSynthetic}{p, NN, S} $

			\If{$ s \not= nil $}
				\State tambahkan $ s $ ke $OUT$
			\EndIf
		\EndFor
	\EndFor
	\State \Return{OUT}
\EndFunction
\\
\Function{CreateSynthetic}{$ p, NN, S $}
	\State $ n \gets $ pilih \textit{nearest neighbourhood} secara acak
	pada $ NN $
	\If{\textbf{not} \Call{CanCreate}{$ S, p, n $}}
		\State \Return nil
	\EndIf

	\State $ x_{new} \gets \Call{Clone}{p} $
	\For{\textbf{each} $ a $ \textbf{in} \Call{Attributes}{S}}a
		\If{\Call{IsQuantitative}{$ a $}}
			\State $ \delta \gets \Call{RandomGap}{S, p, n} $
			\State $ diff \gets n(a) - p(a) $
			\State $ x_{new}(a) \gets p(a) + \delta \cdot diff $
		\Else
			\State $ x_{new} \gets \Call{MostFrequent}{p \cup NN, a} $
		\EndIf
	\EndFor

	\State \Return $ x_{new} $
\EndFunction
	\algstore{lnsmote_break}
	\end{algorithmic}
\end{algorithm}

\begin{algorithm}[h]
	\caption{Local Neighbourhood SMOTE bagian 2}
	\begin{algorithmic}[1]
	\algrestore{lnsmote_break}

\Function{CanCreate}{$ S, p, n $}
	\State $ slp \gets \Call{SafeLevel}{S, p} $
	\State $ sln \gets \Call{SafeLevel2}{S, p, n} $
	\State \Return{$ slp \not= 0 $ \textbf{or} $ sln \not= 0 $}
\EndFunction
\\
\Function{SafeLevel}{$ S, p $}
	\State $ pneighbor \gets \Call{KNearestNeighbor}{S, p} $
	\State \Return{$ pneighbor $ yang kelasnya minoritas}
\EndFunction
\\
\Function{SafeLevel2}{$ S, p, n $}
	\State $ nneighbor \gets \Call{KNearestNeighbor}{ S, n } $
	\If{$\Call{Class}{n} \not= P $ \textbf{and} $ p \in nneighbor $}
		\State Ganti $ p $ di $ nneighbor $ dengan tetanggga $ k + 1 $
		dari $ n $
	\EndIf
	\State \Return{$ nneighbor $ yang kelasnya minoritas}
\EndFunction
\\
\Function{RandomGap}{$ S, p, n $}
	\State $ slp \gets \Call{SafeLevel}{S, p} $
	\State $ sln \gets \Call{SafeLevel}{S, p, n} $
	\State $ \delta \gets 0 $

	\If{$ sln = 0 $ \textbf{and} $ slp > 0 $}
		\State \Return{$ \delta $}
	\EndIf
	\\
	\State $ slratio \gets \frac{slp}{sln} $
	\If{$ slratio = 1 $}
		\State $ \delta \gets \Call{Random}{1} $
	\ElsIf{$ slratio > 1 $}
		\State $ \delta \gets \Call{Random}{\frac{1}{slratio}} $
	\Else
		\State $ \delta = 1 - \Call{Random}{slratio} $
	\EndIf
	\\
	\If{$ \Call{Class}{n} \not= P $}
		\State $ \delta = \delta \cdot \frac{sln}{k} $
	\EndIf
	\\
	\State \Return{$ \delta $}
\EndFunction
	\end{algorithmic}
\end{algorithm}


LNSMOTE memiliki empat parameter yaitu dataset keseluruhan ($D$), kelas
minoritas yang akan disampel ulang ($minor$) pada dataset $D$, persentase
sampel sintetis yang akan dibuat ($o$), dan jumlah tetangga terdekat yang akan
dicari ($k$) di sekitar sampel minoritas.

Fungsi \texttt{CreateSynthetic} memiliki tiga parameter yaitu dataset ($D$),
sebuah sampel ($sample$), dan semua tetangga dari $sample$ ($neighbors$).
Fungsi ini mengembalikan sebuah sampel sintetis yang dibuat antara $sample$
dengan salah satu tetangga $neighbours$ yang dipilih secara acak ($n$), dengan
syarat nilai \textit{safe-level} dari $sample$ atau $n$ di dalam dataset $D$
lebih besar dari $0$.

Fungsi \texttt{CanCreate} memiliki tiga parameter yaitu dataset ($D$), sebuah
sampel ($sample$), dan sebuah tetangga dari $sample$ ($neighbour$).
Fungsi ini memeriksa \textit{safe-level} dari $sample$ dan $neigbour$ di dalam
dataset $D$, jika nilai \textit{safe-level} kedua sampel tersebut sama dengan
0, maka fungsi mengembalikan nilai $false$ yang berarti sampel sintetis tidak
bisa dibuat antara $p$ dan $n$, sebaliknya jika salah satu tidak 0
maka akan mengembalikan nilai $true$ yang berarti sampel sintetis bisa dibuat.

Fungsi \texttt{SafeLevel} memiliki tiga parameter yaitu dataset ($D$), sebuah
sampel minoritas $p$ yang ada di dalam $D$, dan jumlah tetangga $p$ yang akan dicari
dalam $D$ ($k$).
Fungsi ini mencari $k$ buah tetangga dari $p$ dan mengembalikan hanya tetangga
yang kelasnya minoritas.

Fungsi \texttt{SafeLevel2} memiliki empat parameter yaitu dataset ($D$), sebuah
sampel minoritas ($p$) yang ada di dalam $D$, sebuah tetangga dari $p$ ($n$),
dan jumlah tetangga dari $p$ yang akan dicari ($k$).
Fungsi ini mencari $k$ tetangga terdekat dari $n$, katakanlah $nneighbours$,
jika $n$ bukan sampel minoritas dan $p$ adalah tetangga dari $n$ maka sampel
$p$ yang ada di dalam $nneighbours$ diganti dengan tetangga yang selanjutnya
($k+1$).
Fungsi ini mengembalikan semua tetangga $nneighbours$ yang kelasnya minoritas.

Fungsi \texttt{RandomGap} memiliki empat parameter yaitu dataset ($D$), sebuah
sampel minoritas ($p$) yang ada di dalam $D$, sebuah tetangga dari $p$ ($n$),
dan jumlah tetangga yang akan dicari ($k$).
Fungsi ini menghitung jarak acak antara sampel minoritas $p$ dengan tetangganya
$n$ berdasarkan nilai \textit{safe-level} dari $p$ dan $n$.
Jika \textit{safe-level} $n$ sama dengan 0 dan \textit{safe-level} dari $p$
besar dari 0, artinya tidak ada tetangga dari $n$ yang minoritas yang mana $n$
berarti adalah \textit{outlier}, maka fungsi ini mengembalikan nilai jarak 0.
Jika kedua \textit{safe-level} dari $p$ dan $n$ besar dari 0 maka dihitung
rasionya ($slratio$).
Jika $slratio$ sama dengan 1, artinya sampel $p$ dan $n$ memiliki jumlah
tetangga yang sama, maka fungsi ini mengembalikan jarak yang diambil secara
acak dengan rentang antara 0 dan 1.
Jika $slratio$ besar dari 1, artinya jumlah tetangga $p$ lebih besar dari
jumlah tetangga $n$, maka fungsi mengembalikan nilai jarak acak dengan rentang
0 dan $1/slratio$.
Selain itu, fungsi mengembalikan jarak acak antara 0 sampai
$1-Random(slratio)$.


	\section{Implementasi Pengklasifikasi}
	Implementasi pengklasifikasi dilakukan bertahap karena keterkaitan antara
modul.
Dimulai dari implementasi untuk pohon keputusan CART yang digunakan untuk
implementasi \textit{Random Forest} (RF) yang kemudian digunakan dalam
implementasi \textit{Cascaded Random Forest} (CRF).


	\subsection{Implementasi \textit{Random Forest}}
	\label{subsection:implementasi_rf}
	Implementasi RF berdasarkan pada makalah Breiman \cite{breiman2001random}
dan beberapa referensi dari sejumlah sumber lainnya.
Hasil dari implementasi dapat dilihat dan digunakan secara terbuka pada
repositori \texttt{go-mining}
\footnote{
\url{%
https://github.com/shuLhan/go-mining/tree/master/classifiers/rf
}}.
Gambaran umum dari pengklasifikasi RF dapat dilihat pada algoritma
\ref{alg:rf}.

	\begin{algorithm}[h]
\caption{Random Forest}
\label{alg:rf}
	\begin{algorithmic}[1]
\Require \\
$ TSET $: dataset pelatihan \\
$ T $: jumlah pohon untuk \textit{forest} \\
$ B $: persentase sampel yang dipilih secara acak tanpa diganti dari $TSET$ \\
$ m $: jumlah fitur acak \\

\Function{RandomForest}{$ TSET, T, B, m $}
	\State $ n \gets $ jumlah sampel pada dataset $ TSET $
	\State $ b \gets n * (B / 100) $
	\State $ forest \gets nil $
	\For{$ t \gets 1,T $}
		\State \Call{GrowTree}{forest, TSET, b, m}
	\EndFor
	
	\State \Return $forest$
\EndFunction
\\
\Function{GrowTree}{$ forest, TSET, b, m $}
	\label{bagging}
	\State $ bag, oob, bagIdx, oobIdx \gets \Call{RandomPickSample}{$ TSET,
	b $} $

	\State $ tree \gets \Call{BuildTree}{bag, m} $

	\State $ forest.add(tree, bagIdx) $
	\Comment Simpan pohon dan indeks dari \textit{bagging} untuk
	klasifikasi OOB.

	\State $ cm \gets \Call{Classify}{forest, oob, oobIdx} $
	\State $ stat \gets \Call{ComputeStat}{cm} $
	\Comment Hitung laju galat OOB dari \textit{confusion matrix}
	$cm$.
\EndFunction
\\
\Function{Classify}{$ forest, set, setIdx $}
	\State $ predictions \gets nil $
	\For{\textbf{each} $ sample, idx $ \textbf{in} $ set $}
		\State $ votes \gets \Call{ForestVotes}{forest, sample, idx} $
		\State $ class \gets \Call{Majority}{votes} $
		\Comment ambil mayoritas suara kelas pada $votes$
		\State $ predictions.push(class) $
	\EndFor

	\State \Return $predictions$
\EndFunction
	\algstore{rf_break}
	\end{algorithmic}
\end{algorithm}

\begin{algorithm}[h]
	\caption{Random Forest bagian 2}
	\begin{algorithmic}[1]
	\algrestore{rf_break}


\Function{ForestVotes}{$ forest, sample, idx $}
	\State $ votes \gets nil $
	\For{\textbf{each} $ tree $ \textbf{in} $forest$ }
		\State $ exist \gets \Call{IsSampleInTheBag}{idx} $
		\Comment Periksa apakah sampel indeks digunakan pada pohon ini.

		\If{exist}
			\State continue
			\Comment Jika digunakan, lanjutkan ke pohon berikutnya.
		\EndIf

		\State $ class \gets \Call{tree.Classify}{sample} $

		\State $ \Call{votes.push}{class} $
	\EndFor

	\State \Return $ votes $
\EndFunction
	\end{algorithmic}
\end{algorithm}


Fungsi \texttt{RandomForest} memiliki empat parameter yaitu dataset pelatihan
($TSET$),
jumlah pohon dalam \textit{forest} ($T$), jumlah sampel acak yang digunakan untuk
membangun setiap pohon ($B$), dan jumlah fitur acak yang digunakan untuk membangun
setiap pohon ($m$).
Fungsi ini mengembalikan sebuah \textit{forest} yang berisi kumpulan
pohon-pohon keputusan dengan jumlah $T$.

Fungsi \texttt{GrowTree} memiliki empat parameter yaitu sebuah $forest$, sebuah
dataset $TSET$, jumlah sampel acak untuk pembangunan pohon $b$, dan jumlah
fitur acak $m$.
Fungsi ini mengambil $b$ buah sampel acak dari $TSET$ ($bag$) dan menyimpan
indeks dari sampel yang terambil ($bagIdx$) untuk proses penghitungan galat
OOB, nantinya.
Fungsi ini mengembalikan sebuah pohon keputusan $tree$.

Fungsi \texttt{Classify} memiliki tiga parameter yaitu sebuah pengklasifikasi
RandomForest ($forest$), dataset yang akan diklasifikasi ($set$), dan indeks
dari setiap sampel dalam $set$ ($setIdx$, opsional).
Fungsi ini memberi label kelas pada setiap sampel dalam $set$ dengan
mengumpulkan hasil keputusan dari setiap pohon yang ada di dalam $forest$.

Fungsi \texttt{ForestVotes} memiliki tiga parameter yaitu sebuah
pengklasifikasi RandomForest ($forest$), sebuah sampel yang akan beri label
kelasnya (\textit{sample}), dan nilai index dari sampel tersebut ($idx$).
Fungsi ini memanggil fungsi klasifikasi pada setiap pohon di dalam $forest$ dan
mengumpulkan hasil klasifikasi ($class$) dari semua pohon tersebut untuk
kemudian dihitung mayoritas hasil klasifikasinya.
Sebelum pemanggilkan fungsi klasifikasi pada pohon $t$, diperiksa terlebih
dahulu apakah $sample$ tersebut digunakan untuk membangun pohon $t$, jika benar
maka fungsi klasifikasi tidak dilewati dan dilanjutkan dengan pohon
selanjutnya.


	\subsection{Implementasi \textit{Cascaded Random Forest}}
	\label{subsection:implementasi_crf}
	Implementasi \textit{Cascaded Random Forest} (CRF) berdasarkan pada makalah
Bauman dkk.  \cite{baumann2013cascaded}.
Hasil dari implementasi dapat dilihat dan digunakan secara terbuka pada
repositori \texttt{go-mining}
\footnote{
\url{%
https://github.com/shuLhan/go-mining/tree/master/classifiers/crf
}}.
Gambaran umum dari pengklasifikasi CRF dapat dilihat pada algoritma
\ref{alg:crf}.

Pengklasifikasi CRF memiliki tujuh parameter, empat diantaranya sama dengan RF
yaitu dataset fitur yang digunakan untuk pelatihan, jumlah pohon, jumlah sampel
acak disetiap pohon, dan jumlah fitur acak di setiap pohon.
Tiga parameter lainnya yaitu jumlah tingkatan, nilai ambang batas untuk
\textit{true-positive} ($maxtp$) dan nilai ambang batas untuk
\textit{true-negative rate} ($maxtn$).

	\begin{center}
	\captionof{algorithm}{Cascaded Random Forest}
	\label{alg:crf}
	\begin{algorithmic}[1]
\Require \\
$ TSET $: dataset pelatihan \\
$ S $: jumlah tingkatan \\
$ T $: jumlah pohon untuk setiap tingkatan \\
$ B $: persentase sampel yang dipilih secara acak tanpa diganti dari $TSET$ \\
$ m $: jumlah fitur acak \\
$ maxtp $: ambang batas untuk nilai \textit{true-positive} \\
$ maxtn $: ambang batas untuk nilai \textit{true-negative}
\\
\Function{CascadedRandomForest}{$ TSET, S, T, B, m, maxtp, maxtn $}
	\State $ n \gets $ jumlah sampel pada dataset $ TSET $
	\State $ b \gets n * (B / 100) $
	\State $ forest \gets nil $
	\State $ TNSET \gets nil $
	\For{$ s \gets 1,S $}
		\For{$ t \gets 1,T $}
			\State $ tp_{s,t}, tn_{s,t} \gets \Call{GrowTree}{forest, S, b, m} $
			\If{$ tp_{s,t} > maxtp $ \textbf{and} $ tn_{s,t} > maxtn $}
				\State Tingkatan selesai
			\EndIf
		\EndFor
		\State $\alpha_{s} \gets exp(fmeasure) $
		\If{$ s = 1 $}
			\State Pindahkan sampel \textit{true-negative} dari
			$TSET$ ke $TNSET$
		\Else
			\State Uji tingkatan sebelumnya dengan $TNSET$
			\State Hapus sampel \textit{true-negative} dari TNSET
			\State Isi ulang $TSET$ dengan sampel
			\textit{false-positive} dari hasil uji $TNSET$
		\EndIf
	\EndFor
\EndFunction
	\end{algorithmic}
\end{center}


Fungsi \texttt{GrowTree} pada CRF memiliki parameter dan kembalian yang sama
dengan fungsi \textit{GrowTree} pada \textit{RandomForest}.
Setiap selesai membangun sebuah pohon, nilai kembalian $tp$ dan $tn$ diperiksa.
Jika nilai $tp$ dan $tn$ lebih besar dari nilai $maxtp$ dan $maxtn$ maka proses
pembangunan tingkatan selesai, jika belum lanjutkan membangun pohon sampai
nilai $maxtp$ dan $maxtn$ tercapai atau pohon ke-$T$ terbangun.

Saat sebuah tingkatan selesai, simpan $forest$ yang telah terbangun berikut
dengan nilai \textit{F-Measure} dari pohon yang terakhir sebagai bobot ($w$)
dari tingkatan yang nanti digunakan untuk proses klasifikasi.

Fungsi \texttt{Classify} memiliki dua parameter yaitu pengklasifikasi CRF
($crf$) dan kumpulan sampel yang akan diklasifikasi ($set$).
Untuk setiap sampel dalam $set$, fungsi ini melakukan klasifikasi
\textit{Random-Forest} dan mengumpulkan probabilitas dari kelas yang telah
dikalikan dengan bobot dari setiap tingkatan.
Fungsi ini mengembalikan prediksi dari kelas pada $set$.

