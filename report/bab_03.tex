\chapter{Proses Deteksi Vandalisme}
\label{bab:03}

Pada bab ini dijelaskan proses dari pengolahan dataset dan implementasi
algoritma sampel ulang dan klasifikasi, dimulai dari persiapan
data, pembuatan dataset fitur, sampel ulang pada dataset fitur, dan
implementasi pengklasifikasi sehingga nantinya dapat digunakan untuk pelatihan,
pengujian dan analisis.
Alur dari proses secara umum dapat dilihat pada gambar \ref{fig:diagramproses}.

\begin{figure}[htbp]
\centering
\resizebox{.5\textwidth} {!} {
\begin{tikzpicture}[
	framed,
	nodes = {
		align = center
	},
	lines/.style={
		line width=2pt,
		>=latex
	},
	data/.style={
		circle,
		draw=black,
		text centered
	},
	proc/.style={
		rectangle,
		draw=black,
		text centered
	}
]
	\node[data] (training_raw) {
		Dataset pelatihan\\
		mentah\\
		(\textit{PAN-WVC-10})
	};

	\node[proc] (wvcgen) [below=of training_raw] {
		Ekstraksi\\fitur
	};

	\node[proc] (trainingset) [below=of wvcgen] {
		Sampel ulang
	};

	\node[proc] (c)  [below=of trainingset] {
		Pembangkitan model\\
		deteksi
	};

	\node[proc] (m)  [below=of c] {
		Model deteksi\\
		vandalisme
	};

	\node[data] (o)  [below=of m] {
		Hasil deteksi
	};
	%%
	\node[data] (testset_raw) [right=of training_raw] {
		Dataset pengujian\\mentah\\
		(\textit{PAN-WVC-11})
	};

	\node[proc] (testset_wvcgen) [below=of testset_raw] {
		Ekstraksi\\fitur
	};

	\node[data] (testset) [below=of testset_wvcgen] {
		Dataset\\
		pengujian
	};

	\draw[lines,->] (training_raw) -- (wvcgen);
	\draw[lines,->] (wvcgen) -- (trainingset);
	\draw[lines,->] (trainingset) -- (c);
	\draw[lines,->] (c) -- (m);
	\draw[lines,->] (m) -- (o);

	\draw[lines,->] (testset_raw) -- (testset_wvcgen);
	\draw[lines,->] (testset_wvcgen) -- (testset);

	\draw[lines,->] (testset) |- (m);
\end{tikzpicture}
}
\caption{
	Proses implementasi deteksi vandalisme
}
\label{fig:proses}
\end{figure}


	\section{Gambaran Umum}
	\label{bab:03:gambaran_umum}
	Dataset yang didapat tidak langsung bisa digunakan untuk pelatihan dan
pengujian, melainkan perlu ada pra-proses terlebih dahulu yang mengikutkan
penggabungan data dan pembersihan data dengan menghapus atribut yang tidak
diperlukan dan pembersihan dari isi data itu sendiri.
Setelah data mentah dibersihkan, kemudian diekstrak fitur untuk masing-masing
dataset pelatihan dan pengujian sehingga menghasilkan dataset pelatihan dan
pengujian yang bernilai kontinu.
Dataset pelatihan disampel ulang dengan metode SMOTE dan LNSMOTE,
sehingga menghasilkan tiga dataset yang akan dilatih yaitu dataset tanpa sampel
ulang, dataset yang telah disampel ulang dengan SMOTE, dan dataset yang telah
disampel ulang dengan LNSMOTE.
Ketiga dataset pelatihan tersebut kemudian dijalankan pada program
pengklasifikasi \textit{Random Forest} (RF) dan \textit{Cascaded Random Forest}
(CRF) satu per satu, yang menghasilkan sebuah model deteksi vandalisme.
Model tersebut kemudian diuji dengan menginputkan dataset fitur pengujian
sehingga menghasilkan kelas-kelas dari sampel yang nantinya dapat digunakan
untuk mengevaluasi sampel ulang dan pengklasifikasi.


	\section{Persiapan Dataset Pelatihan}
	\label{bab:03:persiapan_data_pelatihan}
	Dataset yang digunakan untuk pelatihan yaitu \textit{PAN Wikipedia Vandalism
Corpus 2010} (PAN-WVC-10)
\cite{potthast:2010b}
yang dapat diambil dari situs Universitas Bauhaus Weimar
\footnote{%
	\RaggedRight\url{%
http://www.uni-weimar.de/en/media/chairs/webis/corpora/pan-wvc-10/
}}.
Korpus ini terdiri dari 32.452 suntingan artikel Wikipedia, 2.391 diantaranya
adalah suntingan vandal yang telah diidentifikasi.
Korpus ini dianotasi oleh 753 orang dengan total 150.000 suara dari semua
suntingan, sehingga setiap suntingan telah diperiksa setidaknya oleh 3 orang
penganotasi.
Dari suara-suara penganotasi tersebut setiap sampel diberi dengan label
"vandalism" atau "regular" (bukan vandalisme).
PAN-WVC-10 telah sukses digunakan pada \textit{1st International Competition on
Wikipedia Vandalism Detection}, PAN'10, yang diselenggarakan bersama dengan
konferensi CLEF 2010 (\textit{Conference on Multilingual and Multimodal
Information Access Evaluation}).

Dataset PAN-WVC-10 terbagi menjadi dua yaitu dataset suntingan dari artikel
Wikipedia dan dataset anotasi yang berisi hasil klasifikasi vandalisme pada
dataset suntingan tersebut.
Dataset suntingan memiliki atribut sebagai berikut,

\begin{itemize}
	\item \textbf{editid}, format angka, berisi identifikasi (ID) unik dari setiap suntingan.
	\item \textbf{editor}, format string, berisi nama penyunting.
	\item \textbf{oldrevisionid}, format angka, berisi ID untuk suntingan lama.
	\item \textbf{newrevisionid}, format angka, berisi ID untuk suntingan baru.
	\item \textbf{diffurl}, format string, berisi URL yang mengacu pada perbedaan suntingan baru dengan lama.
	\item \textbf{edittime}, format string, berisi tanggal dan pukul
	suntingan.
	\item \textbf{editcomment}, format string, berisi komentar yang ditambahkan oleh penyunting saat menyimpan hasil suntingan.
	\item \textbf{articleid}, format angka, berisi ID unik dari artikel.
	\item \textbf{articletitle}, format string, berisi judul dari artikel yang disunting.
\end{itemize}

Dataset anotasi memiliki atribut sebagai berikut,

\begin{itemize}
	\item \textbf{editid}, format angka, mengacu pada \textit{editid} di
	dataset suntingan.
	\item \textbf{class}, format string, berisi tipe suntingan yang
	bernilai "regular" yang menyatakan bahwa suntingan tersebut bukan
	vandalisme, dan "vandalism" yang menyatakan bahwa suntingan tersebut
	adalah vandalisme.
	\item \textbf{annotators}, format angka, berisi jumlah orang yang
	menandai (penanda) bahwa suntingan dengan ID tersebut termasuk ke dalam
	kelas "regular" atau "vandalism".
	\item \textbf{totalannotators}, format angka, berisi jumlah total penanda yang memeriksa suntingan.
\end{itemize}

Kedua dataset kemudian digabung untuk menghasilkan atribut \textit{editid},
\textit{class}, \textit{oldrevisionid}, \textit{newrevisionid},
\textit{edittime}, \textit{editor}, \textit{articletitle},
\textit{editcomment}, \textit{deletions}, dan \textit{additions}.
Nilai dari atribut \textit{class} diganti dari teks menjadi angka, yaitu 1
untuk "vandalism" dan 0 untuk "regular".
Atribut tambahan \textit{deletions} berisi teks yang dihapus dalam revisi yang
lama.
Atribut tambahan \textit{additions} berisi teks yang ditambahkan dalam revisi
yang baru.
Kedua atribut tersebut didapat dengan membandingkan isi dari revisi yang lama
dengan yang baru.

Pemrosesan selanjutnya yaitu membuat berkas revisi yang bersih dari sintaks
wiki.
Tujuan dari revisi ini yaitu supaya tidak ada \textit{noise} pada saat
melakukan penghitungan fitur dan mempercepat proses pembuatan fitur.
Setiap berkas revisi dibaca kemudian dilakukan pembersihan berikut,

\begin{itemize}
\item penghapusan URI yang berawalan dengan
\textit{http://}, \textit{https://}, \textit{ftp://}, dan \textit{ftps}.
\item Menghapus \textit{mark-up} wiki yaitu konten yang dilingkupi oleh marka
berikut:
\texttt{[[Category:]]}, \texttt{[[:Category:]]}, \texttt{[[File:]]}, \\
\texttt{[[Help:]]}, \texttt{[[Image:]]}, \texttt{[[Special:]]},
\texttt{[[Wikipedia:]]}, \\
\texttt{\{\{DEFAULTSORT:\}\}}, \texttt{\{\{Template:\}\}}, dan \texttt{<ref/>}.

\item Mengganti karakter dan token berikut dengan karakter kosong (spasi):
\texttt{[}, \texttt{]}, \texttt{\{}, \texttt{\}}, \texttt{|}, \texttt{=},
\texttt{\#}, \texttt{'s}, \texttt{'}, \texttt{<ref>}, \texttt{</ref>},
\texttt{<br />}, \texttt{<br/>}, \texttt{<br>}, \texttt{<nowiki>},
\texttt{</nowiki>}, \texttt{\&nbsp;}.
\end{itemize}


	\section{Persiapan Dataset Pengujian}
	\label{bab:03:persiapan_data_pengujian}
	Dataset yang digunakan untuk pengujian yaitu \textit{PAN Wikipedia Vandalism
Corpus 2011} (PAN-WVC-11)
\cite{potthast:2010b}
yang dapat diambil dari situs Universitas Universitas Bauhaus Weimar
\footnote{%
	\RaggedRight\url{%
http://www.uni-weimar.de/en/media/chairs/webis/corpora/corpus-pan-wvc-11/
	}
}
Korpus ini terdiri dari 9.985 suntingan bahasa Inggris, 9.990 suntingan bahasa
Jerman, dan 9.974 suntingan bahasa Spanyol.
Korpus yang digunakan untuk pelatihan hanya yang bahasa Inggris.

Dataset asli memiliki atribut yang sama dengan PAN-WVC-10, yaitu
\textit{editid},
\textit{editor},
\textit{oldrevisionid},
\textit{newrevisionid},
\textit{diffurl},
\textit{class},
\textit{annotators},
\textit{totalannotators},
\textit{edittime},
\textit{editcomment},
\textit{articleid}, dan
\textit{articletitle}.
Atribut \textit{annotators} dan \textit{totalannotators} dihilangkan dan
ditambah dengan dua atribut baru yaitu \textit{deletions} yang berisi teks yang
dihapus pada revisi lama, dan \textit{additions} yang berisi teks yang
ditambahkan pada revisi yang baru.
Nilai dari atribut \textit{class} diganti dari string menjadi integer, yaitu
dari "vandalism" menjadi bilangan \texttt{1}
dan "regular" menjadi bilangan \texttt{0}.

Untuk proses pembersihan data dilakukan prosedur yang sama seperti pada
PAN-WVC-10.


	\section{Ekstraksi Fitur}
	\label{bab:03:ekstraksi_fitur}
	Beberapa makalah sebelumnya mengelompokan fitur ke dalam tiga kelompok yaitu
\textit{metadata}, teks, dan bahasa.
Dalam tesis ini digunakan 4 fitur metadata, 11 fitur teks, dan 10 fitur
bahasa yang diambil dari hasil analisis makalah Mola-Velasco
\cite{mola2012wikipedia}.

Dari analisis fitur vandalisme tersebut kemudian ditransformasikan ke dalam
sebuah program.
Program ekstraksi fitur kemudian dijalankan dengan input dataset PAN-WVC-10
dan PAN-WVC-11 yang telah dibersihkan pada subbab
\ref{bab:03:persiapan_data_pelatihan} dan
\ref{bab:03:persiapan_data_pengujian},
sehingga menghasilkan dataset fitur yang bernilai kontinu.
Implementasi program pembersihan data dan generator fitur ini dapat dilihat dan
digunakan secara terbuka pada repositori \texttt{wvcgen}
\footnote{\url{https://github.com/shuLhan/wvcgen}}.

\subsection{Fitur Kelompok Metadata}

Kelompok metadata mengacu pada properti dari sebuah revisi yang secara langsung
dapat diambil, seperti identitas penyunting, komentar, atau ukuran perubahan.

Berikut daftar fitur metadata yang digunakan,

\begin{itemize}

\item \textbf{Anonim}.
Penyunting anonim yaitu yang tidak menggunakan akun Wikipedia saat melakukan
penyuntingan, sehingga yang tercatat hanya alamat IP bukan nama pengguna.
Fitur ini melihat apakah suntingan anonim atau bukan pada atribut
\textit{editor}, jika benar maka ditandai dengan nilai 1, atau 0 sebaliknya.
Vandal lebih condong berlaku anonim karena jika menggunakan akun asli akan
membuat akun mereka mudah diblokir dan membuat akun baru membutuhkan waktu dan
identitas berupa alamat surel.

\item \textbf{Panjang komentar}.
Melihat dari jumlah karakter yang diinputkan di kolom rangkuman suntingan, yang
tersimpan dalam atribut \textit{editcomment} pada dataset, saat menyimpan hasil
suntingan, tanpa mengikutkan bagian \textit{header} yaitu penanda di awal
komentar yang menunjuk ke bagian yang di sunting, biasanya dalam format
"\texttt{/* Nama header */}".
Komentar yang panjang mungkin mengindikasikan suntingan normal dan yang pendek
atau kosong mungkin menyarankan suatu vandalisme.

\item \textbf{Peningkatan ukuran}.
Peningkatan absolut dari ukuran konten artikel, dihitung dengan
\[
|\text{ukuran\_suntingan\_baru} - \text{ukuran\_suntingan\_lama}|
\]
Misalnya, berkurangnya ukuran dalam jumlah besar bisa mengindikasikan
pengosongan artikel.

\item \textbf{Rasio ukuran}.
Ukuran revisi baru relatif terhadap revisi lama, dihitung dengan cara
\[
\frac{1 + |baru|}{1 + |lama|}
\]

\end{itemize}


\subsection{Fitur Kelompok Teks}

Berikut daftar fitur berbasiskan teks yang digunakan,

\begin{itemize}

\item \textbf{Rasio huruf besar dan kecil}.
Pelaku vandal biasanya tidak mengikuti aturan huruf kapital, menulis semuanya
dengan huruf kecil atau huruf besar.
Rasio ini dihitung pada teks yang ditambahkan di revisi baru dengan menggunakan
rumus
\[
\frac{1 + |huruf\ besar|}{1 + |huruf\ kecil|}
\]

\item \textbf{Rasio huruf besar terhadap semua karakter}
Rasio ini dihitung dengan menggunakan rumus
\[
\frac{1 + |huruf\ besar|}{1 + |huruf\ besar| + |huruf\ kecil|}
\]

\item \textbf{Rasio angka}.
Rasio semua karakter terhadap angka, yaitu
\[
\frac{1 + |angka|}{1 + |semua karakter|}
\]
Fitur ini membantu menemukan perubahan kecil yang hanya mengubah angka.
Contoh kasusnya perubahan sebuah tanggal atau perhitungan untuk menyebabkan
kesalahan informasi.

\item \textbf{Rasio non-alfanumerik}.
Rasio semua karakter terhadap karakter selain huruf dan angka, yaitu
\[
\frac{1 + |non\ alfanumerik|}{1 + |semua\ karakter|}
\]
Penggunaan karakter selain angka-huruf yang berlebihan bisa mengindikasikan
penggunaan \textit{emoticon}, tanda baca, atau kata tak bermakna.

\item \textbf{Diversitas karakter}.
Menghitung jumlah karakter berbeda yang digunakan pada penambahan, dibandingkan dengan panjang teks yang dimasukan,
dihitung dengan rumus,
\[
length \frac{1}{karakter\ berbeda}
\]
Fitur ini membantu menemukan penggunaan karakter secara acak dan kata tak
bermakna.

\item \textbf{Distribusi karakter}.
Menggunakan Divergensi Kullback-Leibler dari distribusi karakter yang dimasukan
terhadap ekspektasi.
Fitur ini berguna untuk mendeteksi kata tak bermakna.

\item \textbf{Laju kompresi}.
Melihat tingkat kompres dari penambahan teks menggunakan algoritma kompresi
LZW.
Fitur ini berguna untuk mendeteksi kata tak bermakna, pengulangan kata atau
karakter, dll.
Vandalisme biasanya memiliki ukuran kompresi yang rendah.

\item \textbf{Token umum}.
Menghitung token yang biasanya jarang digunakan oleh vandal yaitu sintaks
wiki, seperti \textit{\_\_TOC\_\_}. Daftar token dapat dilihat pada lampiran
\ref{lampiran:words_wiki_token}.

\item \textbf{Frekuensi rerata kata}.
Frekuensi relatif rerata dari kata yang dimasukan pada revisi baru.
Pada artikel yang panjang, semakin banyak kata yang dimasukan yang tidak ada
pada artikel mengindikasikan bahwa suntingan tersebut bisa tak bermakna atau
tidak berhubungan dengan isinya.

\item \textbf{Kata terpanjang}.
Panjang dari kata yang dimasukan.
Nilainya akan 0 jika tidak ada kata yang dimasukan.
Fitur ini berguna untuk mendeteksi suntingan tak-bermakna.

\item \textbf{Urutan karakter terpanjang}.
Urutan terpanjang dari karakter yang sama pada teks yang dimasukan sering
digunakan pada vandalisme, contohnya \textit{aaarrrrggghhh! sooo huge}.

\end{itemize}

\subsection{Fitur Kelompok Bahasa}

Fitur kelompok bahasa didasarkan pada jumlah kata tertentu yang ditambahkan
pada suntingan atau revisi yang baru.
Untuk setiap kategori kata dihitung dua jenis fiturnya: frekuensi dan impak.

Fitur frekuensi yaitu menghitung frekuensi dari kategori kata terhadap
total seluruh kata pada suntingan yang baru, yaitu
\[
	\frac{jumlah\ kategori\ kata}
		{jumlah\ seluruh\ kata}
\]

Fitur impak yaitu menghitung persentase peningkatan kategori kata di revisi
yang baru, dihitung dengan,
\[
	\frac{jumlah\ kategori\ kata\ di\ revisi\ lama}
		{jumlah\ kategori\ kata\ di\ revisi\ lama +
		jumlah\ kategori\ kata\ di\ revisi\ baru}
\]

Berikut daftar kategori kata yang digunakan,

\begin{itemize}
\item \textbf{Vulgarisme}.
Menghitung kata-kata vulgar, kasar dan menghina.
Daftar kategori kata ini dapat dilihat pada lampiran
\ref{lampiran:words_vulgar}.

\item \textbf{Subjek}.
Menghitung kata subjek pertama atau kedua, termasuk pengejaan tidak baku,
misalnya \textit{I}, \textit{you}, \textit{ya}.
Daftar kategori kata ini dapat dilihat pada lampiran
\ref{lampiran:words_pronoun}.

\item \textbf{Bias}.
Menghitung penggunaan kata sehari-hari yang mengandung bias, misalnya
\textit{coolest}, \textit{huge}.
Daftar kategori kata ini dapat dilihat pada lampiran
\ref{lampiran:words_bias}.

\item \textbf{Pornografi}.
Menghitung penggunaan kata berhubungan dengan pornografi.
Daftar kategori kata ini dapat dilihat pada lampiran
\ref{lampiran:words_sex}.

\item \textbf{Kata buruk}.
Menghitung penggunaan kata sehari-hari dan beberapa penulisan yang buruk
(misalnya, \textit{wanna}, \textit{gotcha}).
Daftar kategori kata ini dapat dilihat pada lampiran
\ref{lampiran:words_bad}.

\item \textbf{Seluruh kategori kata}.
Gabungan dari kategori kata vulgarisme, subjek, bias, pornografi, dan kata
buruk.

\end{itemize}



	\section{Pembuatan Sampel Ulang}
	\label{bab:03:pembuatan_sampel_ulang}
	Dataset PAN-WVC-10 tanpa sampel ulang memiliki jumlah sampel vandal atau
positif yaitu 2.394 sampel dan 30.045 sampel bukan vandal atau negatif.
Untuk mendapatkan jumlah sampel yang seimbang, dataset disampel ulang dengan
teknik SMOTE dan LNSMOTE untuk kelas yang positif.
Untuk teknik SMOTE parameter persentase sampel ulang yang digunakan yaitu
1.100\% dan parameter \textit{K-Nearest-Neighbourhood} (KNN) yaitu 5,
sehingga didapat hasil sampel ulang dengan jumlah sampel sintetis positif yaitu
28.728 sampel, ditambah dengan sampel positif asli totalnya yaitu 31.122 sampel
positif.
Untuk teknik LNSMOTE parameter persentase sampel ulang dan KNN yang digunakan
sama dengan SMOTE yaitu 1.100\% dan 5, sehingga didapat 28.588 sampel
sintetis positif dengan total menjadi 30.982 sampel positif.
Hasil implementasi algoritma SMOTE dan LNSMOTE dapat dilihat dan digunakan
secara terbuka pada repositori \texttt{go-mining}
\footnote{\url{https://github.com/shuLhan/go-mining/tree/master/resampling}}.


	\subsection{Pembuatan Sampel Ulang SMOTE}
	\label{bab:03:sampel_ulang_smote}
	Implementasi SMOTE dibuat ke dalam sebuah program
\footnote{\url{%
https://github.com/shuLhan/go-mining/tree/master/resampling/smote
}}
berdasarkan makalah aslinya \cite{chawla2002smote}.
Gambaran umum dari implementasi dapat dilihat pada algoritma \ref{alg:smote}.

Program SMOTE dijalankan dengan memberikan input sampel minoritas dari dataset
fitur PAN-WVC-10.
Untuk mendapatkan jumlah sampel positif yang mendekati jumlah sampel negatif,
maka parameter persentase jumlah sampel sintetis diberi dengan nilai 1.200\%,
yang berarti 12 kali jumlah sampel minoritas.
Parameter untuk jumlah tetangga terdekat yang dicari untuk dapat membuat sebuah
sampel diberi dengan nilai 5.
Nilai ini diberikan untuk menghindari sampel minoritas yang \textit{outliers}
tapi juga masih dapat mencakup kelompok-kelompok kelas minoritas yang kecil.
Dari ketiga parameter tersebut didapat hasil sampel ulang dengan jumlah sampel
sintetis positif yaitu 28.728 sampel, digabung dengan sampel positif asli
totalnya yaitu 31.122 sampel positif.

	\newpage
	\begin{center}
	\captionof{algorithm}{SMOTE}
	\label{alg:smote}
	\begin{algorithmic}[1]
\Require \\
$ P $: dataset minoritas \\
$ o $: persentase jumlah sampel sintetis yang akan dibuat \\
$ k $: jumlah tetangga terdekat dari sampel yang akan dibuat
\\
\Function{SMOTE}{$ P, o, k $}
	\State $ S \gets [] $
	\Comment Penampung untuk semua sampel sintetis yang akan dibuat.

	\State $ nsyn \gets 0 $
	\Comment Jumlah sampel sintetis yang akan dibuat.

	\State $ lenP \gets \Call{Length}{P} $

	\If{$o < 100$}
		\label{o_less_100}
		\State $ nsyn \gets (o / 100) \times lenP $
		\State $ P \gets \Call{RandomPickWithoutReplacement}{P, nsyn} $
	\Else
		\State $ nsyn \gets o / 100.0 $
	\EndIf

	\For{\textbf{each} $sample$ \textbf{in} $P$}
		\State $ neighbors \gets \Call{FindNeighbours}{P, sample, k} $
		\Comment Cari k buah tetangga terdekat dari $sample$ di P.

		\State $ s \gets \Call{CreateSynthetic}{sample, neighbors, nsyn} $

		\State $ \Call{S.Push}{s} $
	\EndFor

	\State \Return{S}
\EndFunction
\\
\Function{CreateSynthetic}{$ sample, neighbors, nsyn $}
	\State $ s \gets [] $
	\Comment Penampung untuk semua sampel sintetis yang akan dibuat.

	\State $ lenAttr \gets \Call{LengthAttribute}{sample} $
	\Comment Simpan jumlah atribut dari sampel.

	\For{$ x \gets 1,nsyn $}
		\State $ neighbor \gets \Call{RandomPick}{neighbors} $
		\State $ newsample \gets [] $
		\Comment Penampung untuk sampel sintetis yang baru.

		\For{$ y \gets 1,lenAttr $}
			\If{y \textbf{is} classAttribute}
				\State continue
			\EndIf

			\State $ nattr \gets neighbor.\Call{AttributeAt}{y} $
			\State $ sattr \gets sample.\Call{AttributeAt}{y} $
			\State $ diff \gets sattr - nattr $
			\State $ gap \gets \Call{Random}{0,1} $
			\State $ newAttr \gets nattr + (gap \times diff) $

			\State $ newsample.\Call{PushAttribute}{newAttr} $
		\EndFor

		\State $ \Call{s.Push}{newsample} $
	\EndFor

	\State \Return{s}
\EndFunction

	\end{algorithmic}
\end{center}


Sampel ulang SMOTE memiliki tiga parameter yaitu dataset minoritas yang akan
disampel ulang ($P$), persentase sampel sintetis yang akan
dibuat ($o$), dan jumlah tetangga terdekat yang dicari pada sampel
yang akan dibuat ($k$).
Dataset minoritas yaitu dataset yang berisi hanya sampel dengan kelas
minoritas.
Jika parameter persentase lebih kecil dari 100 (persen) maka hanya $o$ persen
dari dataset $P$ yang disampel ulang.
Parameter jumlah tetangga terdekat ($k$) menentukan jumlah sampel minoritas
lain yang dijadikan sebagai patokan dalam pembuatan sampel sintentis.
Semakin besar nilai $k$ maka distribusi sampel sintetis di dalam dataset juga
semakin luas, semakin kecil nilai $k$ maka antara sampel sintetis
dengan sampel asli semakin mirip/dekat.
Berikut penjelasan fungsi-fungsi tambahan yang ada dalam algoritma.

Fungsi \texttt{RandomPickWithoutReplacement} memiliki dua parameter yaitu
dataset ($P$) dan jumlah sampel yang akan diambil ($n$).  Fungsi ini
mengembalikan $n$ buah sampel dari $P$ yang diambil secara acak tanpa diganti,
artinya setelah sampel diambil sampel tersebut dihapus dari $P$.

Fungsi \texttt{FindNeighbours} memiliki tiga parameter yaitu dataset ($P$),
sebuah sampel yang akan dicari tetangganya ($s$), dan jumlah tetangga yang akan
dicari di sekitar $s$ ($k$).
Fungsi ini mengembalikan $k$ sampel tetangga dari $s$ di dalam dataset $P$.
Jumlah sampel tetangga yang dikembalikan bisa lebih kecil dari nilai $k$.

Fungsi \texttt{CreateSynthetic} memiliki tiga parameter yaitu sebuah sampel
(\textit{sample}), sampel-sampel tetangga dari $sample$ ($neighbors$), dan
jumlah sampel sintetis yang akan dibuat ($nsyn$).
Fungsi ini membuat dan mengembalikan $nsyn$ buah sampel sintetis dari $sample$
yang berdekatan dengan tetangganya ($neighbors$).

Fungsi \texttt{RandomPick} memiliki sebuah parameter yaitu sebuah dataset dan
fungsi ini mengembalikan sebuah sampel yang diambil secara acak dari dataset.

Fungsi \texttt{Random} memiliki dua parameter bilangan yaitu nilai minimum dan
maksimum, fungsi ini mengembalikan bilangan acak real dengan rentang antara
minimum dan maksimum.


	\subsection{Pembuatan Sampel Ulang LNSMOTE}
	\label{bab:03:sampel_ulang_lnsmote}
	Proses sampel ulang LNSMOTE yang digunakan pada implementasi berdasarkan
pada makalah Maciejewski dkk.
\cite{maciejewski2011local}
yang dapat dilihat pada algoritma
\ref{alg:lnsmote}.
Hasil implementasinya dalam bentuk sebuah program
\footnote{\url{%
https://github.com/shuLhan/go-mining/tree/master/resampling/lnsmote
}}.
Program LNSMOTE dijalankan dengan diberikan input dataset fitur PAN-WVC-10
dengan persentase sampel sintetis yaitu 1.100\% dan jumlah tetangga terdekat
yaitu 5, sehingga didapat 28.588 sampel sintetis positif, ditambah dengan
sampel positif asli totalnya menjadi 30.982 sampel positif.

	\begin{center}
	\captionof{algorithm}{Local Neighbourhood SMOTE}
	\label{alg:lnsmote}
	\begin{algorithmic}[1]
\Require \\
$ D $: dataset sampel keseluruhan \\
$ minor $: kelas minoritas pada dataset $D$ \\
$ o $: persentase jumlah sampel sintetis yang akan dibuat \\
$ k $: jumlah tetangga terdekat dari sampel yang akan dibuat
\\
\Function{LNSMOTE}{$ D, P, o, k $}
	\State $ S \gets [] $
	\Comment Penampung untuk semua sampel sintetis yang akan dibuat.
	\State $ nsyn \gets o/100 $
	\Comment Jumlah sampel sintetis yang akan dibuat.
	\State $ P \gets $ semua sampel minoritas $minor$ pada dataset $D$.
	\\
	\For{\textbf{each} $ sample $ \textbf{in} $ P $}
		\State $ neighbours \gets \Call{FindNeighbours}{D, sample, k} $
		\For{$ i \gets 1,nsyn $}
			\State $ s \gets \Call{CreateSynthetic}{D, sample, neighbours} $

			\If{$ s \not= nil $}
				\State $ \Call{S.Push}{s} $
			\EndIf
		\EndFor
	\EndFor
	\\
	\State \Return{S}
\EndFunction
\\
\Function{CreateSynthetic}{$ D, sample, neighbors $}
	\State $ neighbor \gets \Call{RandomPick}{neighbors} $
	\If{$ \Call{CanCreate}{D, sample, neighbor} = false $}
		\State \Return nil
	\EndIf
	\\
	\State $ lenAttr \gets \Call{LengthAttribute}{sample} $
	\Comment Simpan jumlah atribut dari sampel.

	\State $ newsample \gets [] $
	\Comment Penampung untuk sampel sintetis yang baru.
	\\
	\For{$ x \gets 1,lenAttr $}
		\If{y \textbf{is} classAttribute}
			\State continue
		\EndIf

		\State $ gap \gets \Call{RandomGap}{D, sample, neighbor, k} $

		\State $ sattr \gets sample.\Call{AttributeAt}{y} $
		\State $ nattr \gets neighbor.\Call{AttributeAt}{y} $
		\State $ diff \gets sattr - nattr $
		\State $ newAttr \gets sattr + (gap \times diff) $

		\State $ newsample.\Call{PushAttribute}{newAttr} $
	\EndFor
	\\
	\State \Return{$ newsample $}
\EndFunction
\\
\Function{CanCreate}{$ D, sample, neighbour $}
	\State $ slp \gets \Call{SafeLevel}{D, sample, k} $
	\State $ sln \gets \Call{SafeLevel2}{D, sample, neighbor, k} $
	\State \Return{$ slp \not= 0 $ \textbf{or} $ sln \not= 0 $}
\EndFunction
\\
\Function{SafeLevel}{$ D, p, k $}
	\State $ pneighbor \gets \Call{FindNeighbours}{ D, p, k } $
	\State \Return{$ pneighbor $ yang kelasnya minoritas}
\EndFunction
\\
\Function{SafeLevel2}{$ D, p, n, k $}
	\State $ nneighbours \gets \Call{FindNeighbours}{ D, n, k } $
	\If{$\Call{Class}{n} \not= minor $ \textbf{and} $ p \in nneighbours $}
		\State Ganti $ p $ di $ nneighbours $ dengan tetanggga $ k + 1 $
		dari $ n $
	\EndIf
	\State \Return{$ nneighbours $ yang kelasnya minoritas}
\EndFunction
\\
\Function{RandomGap}{$ D, p, n, k $}
	\State $ slp \gets \Call{SafeLevel}{D, p, k} $
	\State $ sln \gets \Call{SafeLevel}{D, p, n, k} $
	\State $ \delta \gets 0 $

	\If{$ sln = 0 $ \textbf{and} $ slp > 0 $}
		\State \Return{$ \delta $}
	\EndIf
	\\
	\State $ slratio \gets slp / sln $
	\If{$ slratio = 1 $}
		\State $ \delta \gets \Call{Random}{1} $
	\ElsIf{$ slratio > 1 $}
		\State $ \delta \gets \Call{Random}{1 / slratio} $
	\Else
		\State $ \delta = 1 - \Call{Random}{slratio} $
	\EndIf
\pagebreak
	\If{$ \Call{Class}{n} \not= minor $}
		\State $ \delta = \delta \times (sln / k) $
	\EndIf
	\\
	\State \Return{$ \delta $}
\EndFunction
	\end{algorithmic}
\end{center}


LNSMOTE memiliki empat parameter yaitu dataset keseluruhan ($D$), kelas
minoritas yang akan disampel ulang ($minor$) pada dataset $D$, persentase
sampel sintetis yang akan dibuat ($o$), dan jumlah tetangga terdekat yang akan
dicari ($k$) di sekitar sampel minoritas.

Fungsi \texttt{CreateSynthetic} memiliki tiga parameter yaitu dataset ($D$),
sebuah sampel ($sample$), dan semua tetangga dari $sample$ ($neighbors$).
Fungsi ini mengembalikan sebuah sampel sintetis yang dibuat antara $sample$
dengan salah satu tetangga $neighbours$ yang dipilih secara acak ($n$), dengan
syarat nilai \textit{safe-level} dari $sample$ atau $n$ di dalam dataset $D$
lebih besar dari $0$.

Fungsi \texttt{CanCreate} memiliki tiga parameter yaitu dataset ($D$), sebuah
sampel ($sample$), dan sebuah tetangga dari $sample$ ($neighbour$).
Fungsi ini memeriksa \textit{safe-level} dari $sample$ dan $neigbour$ di dalam
dataset $D$, jika nilai \textit{safe-level} kedua sampel tersebut sama dengan
0, maka fungsi mengembalikan nilai $false$ yang berarti sampel sintetis tidak
bisa dibuat antara $p$ dan $n$, sebaliknya jika salah satu tidak 0
maka akan mengembalikan nilai $true$ yang berarti sampel sintetis bisa dibuat.

Fungsi \texttt{SafeLevel} memiliki tiga parameter yaitu dataset ($D$), sebuah
sampel minoritas $p$ yang ada di dalam $D$, dan jumlah tetangga $p$ yang akan dicari
dalam $D$ ($k$).
Fungsi ini mencari $k$ buah tetangga dari $p$ dan mengembalikan hanya tetangga
yang kelasnya minoritas.
\newpage
Fungsi \texttt{SafeLevel2} memiliki empat parameter yaitu dataset ($D$), sebuah
sampel minoritas ($p$) yang ada di dalam $D$, sebuah tetangga dari $p$ ($n$),
dan jumlah tetangga dari $p$ yang akan dicari ($k$).
Fungsi ini mencari $k$ tetangga terdekat dari $n$, katakanlah $nneighbours$,
jika $n$ bukan sampel minoritas dan $p$ adalah tetangga dari $n$ maka sampel
$p$ yang ada di dalam $nneighbours$ diganti dengan tetangga yang selanjutnya
($k+1$).
Fungsi ini mengembalikan semua tetangga $nneighbours$ yang kelasnya minoritas.

Fungsi \texttt{RandomGap} memiliki empat parameter yaitu dataset ($D$), sebuah
sampel minoritas ($p$) yang ada di dalam $D$, sebuah tetangga dari $p$ ($n$),
dan jumlah tetangga yang akan dicari ($k$).
Fungsi ini menghitung jarak acak antara sampel minoritas $p$ dengan tetangganya
$n$ berdasarkan nilai \textit{safe-level} dari $p$ dan $n$.
Jika \textit{safe-level} $n$ sama dengan 0 dan \textit{safe-level} dari $p$
besar dari 0, artinya tidak ada tetangga dari $n$ yang minoritas yang mana $n$
berarti adalah \textit{outlier}, maka fungsi ini mengembalikan nilai jarak 0.
Jika kedua \textit{safe-level} dari $p$ dan $n$ besar dari 0 maka dihitung
rasionya ($slratio$).
Jika $slratio$ sama dengan 1, artinya sampel $p$ dan $n$ memiliki jumlah
tetangga yang sama, maka fungsi ini mengembalikan jarak yang diambil secara
acak dengan rentang antara 0 dan 1.
Jika $slratio$ besar dari 1, artinya jumlah tetangga $p$ lebih besar dari
jumlah tetangga $n$, maka fungsi mengembalikan nilai jarak acak dengan rentang
0 dan $1/slratio$.
Selain itu, fungsi mengembalikan jarak acak antara 0 sampai
$1-Random(slratio)$.


	\section{Implementasi Pengklasifikasi}
	Implementasi pengklasifikasi dilakukan bertahap karena keterkaitan antara
modul.
Dimulai dari implementasi untuk pohon keputusan CART yang digunakan untuk
implementasi \textit{Random Forest} (RF) yang kemudian digunakan dalam
implementasi \textit{Cascaded Random Forest} (CRF).


	\subsection{Implementasi \textit{Random Forest}}
	\label{subsection:implementasi_rf}
	Implementasi RF berdasarkan pada makalah Breiman \cite{breiman2001random}
dan beberapa referensi tambahan dari sejumlah sumber lainnya
(situs, makalah, dan video).
Hasil dari implementasi dapat dilihat dan digunakan secara terbuka pada
repositori \texttt{go-mining}
\footnote{
\url{%
https://github.com/shuLhan/go-mining/tree/master/classifiers/rf
}}.
Gambaran umum dari pengklasifikasi RF dapat dilihat pada algoritma
\ref{alg:rf}.

RF memiliki empat parameter yaitu dataset pelatihan, jumlah pohon dalam
\textit{forest}, jumlah sampel acak yang digunakan untuk membangun setiap
pohon, dan jumlah fitur acak yang digunakan untuk membangun setiap pohon.
Untuk dataset pelatihan yang digunakan terdiri dari tiga jenis yaitu
dataset fitur tanpa sampel ulang, dataset yang telah disampel ulang dengan
SMOTE, dan dataset yang telah disampel ulang dengan LNSMOTE.
Untuk setiap dataset, jumlah pohon yang digunakan yaitu 200 pohon, dengan
parameter sampel acak yaitu 66\% dari total sampel pada dataset, dan 5 untuk
jumlah fitur acak.


	\subsection{Implementasi \textit{Cascaded Random Forest}}
	\label{subsection:implementasi_crf}
	Implementasi \textit{Cascaded Random Forest} (CRF) berdasarkan pada makalah
Bauman dkk.  \cite{baumann2013cascaded}.
Hasil dari implementasi dapat dilihat dan digunakan secara terbuka pada
repositori \texttt{go-mining}
\footnote{
\url{%
https://github.com/shuLhan/go-mining/tree/master/classifiers/crf
}}.
Gambaran umum dari pengklasifikasi CRF dapat dilihat pada algoritma
\ref{alg:crf}.

Pengklasifikasi CRF memiliki tujuh parameter, empat diantaranya sama dengan RF
yaitu dataset fitur yang digunakan untuk pelatihan, jumlah pohon, jumlah sampel
acak disetiap pohon, dan jumlah fitur acak di setiap pohon.
Tiga parameter lainnya yaitu jumlah tingkatan, nilai ambang batas untuk
\textit{true-positive} ($maxtp$) dan nilai ambang batas untuk
\textit{true-negative rate} ($maxtn$).
Dalam pelatihan digunakan tiga pengklasifikasi CRF yang memiliki parameter
tingkat dan pohon yang berbeda yaitu CRF dengan 200 tingkat dan 1 pohon, CRF
dengan 100 tingkat dan 2 pohon, dan CRF dengan 50 tingkat dan 4 pohon.
Ketiga pengklasifikasi CRF tersebut memiliki total jumlah pohon yang sama yaitu
200 pohon, menggunakan jumlah sampel acak dan fitur acak yang sama yaitu 64\%
dan 5 fitur acak, dan nilai ambang batas $maxtp$ dan $maxtn$ yang sama yaitu
$0,95$ dan $0,95$.

