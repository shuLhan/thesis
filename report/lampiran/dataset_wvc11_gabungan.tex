"editid","class","oldrevisionid","newrevisionid","edittime","editor","articletitle","editcomment","deletions","additions"
...
416341,"regular",308672260,326825757,"2009-11-19T23:25:19Z","YUL89YYZ","Elwin Palmer","Disambiguate [[CMG]] to [[Companion of the Order of St Michael and St George]] using [[:en:Wikipedia:Tools/Navigation_popups|popups]]","","ompanion of the Order of St Michael and St George|C"
416342,"regular",326560892,326560971,"2009-11-18T16:48:35Z","Abyssal","List of stratigraphic units with dinosaur tracks","/* Thyreophorans */","
","{| class=\"wikitable sortable\" align=\"center\" width=\"100%\"
 |-
 ! Name
 ! Age
 ! Location
 ! Description
 |-
"
416343,"vandalism",326540780,326785787,"2009-11-19T19:46:03Z","24.199.196.3","Air Jordan","/* Air Jordan XI */","The Air Jordan XI model was designed by Tinker Hatfield. While Jordan was still pursuing a career in baseball, Hatfield designed the shoe in hopes that Michael would eventually wear it if he returned to the NBA.

 The Jordan XI was meant to stand out with a fusion of performance and style. Taken from the world of high-end mountaineering backpacks, the [[condura nylon]] upper gave the Air Jordan XI model lightweight durability. Further innovation came with the use of a carbon fiber plate on the sole of the shoe, that can be seen underneath the clear outsole, which gave the shoe exceptional torsional rigidity. The most visually distinct aspect of the shoe was its shiny patent leather mid and toebox. A material long used in the fashion industry, patent leather was extremely lightweight, when compared to genuine leather, and also tended not to stretch - a very useful property to help keep the foot within the bounds of the shoebed during quick direction changes on the court. The shiny leather gave the XI what many described as a \"formal\" look - a fact that many owners of the shoe took advantage of thereafter, pairing the shoe off the court with business suits in substitution for dress shoes. [[Boyz II Men]] wore black and white Air Jordan XI shoes with white suits at one of their concerts.

 Jordan wore the Air Jordan XI model to help the Chicago Bulls claim the 1995-1996 NBA championship. The legacy of the shoe was transferred to the silver screen as Jordan wore a black/white/royal blue colorway of the Air Jordan XI model in the 1996 Warner Bros. animated movie \"Space Jam\" (these shoes were eventually released in 2001 with the tag name \"Space Jams\") which was then fined for
$5,000 for not respecting the Bulls colorway policy.


 Air Jordan XI","

  THE BEST SHOES EVER "
416345,"regular",326886207,326886384,"2009-11-20T06:53:06Z","Brillbananaman","Peter Dazeley","/* Photographic career */","","\"<ref>Association of Photographers<ref>\" "
416346,"regular",326324241,326860676,"2009-11-20T03:11:50Z","Lalax3","Human
population control","Made a few grammar corrections."," and or  Kautilya [[ ]]
[[emigration]] to colonies would be encouraged should the population become too
big.<ref name=\"Neurath 1994 6\"/> Like [[Confucius]], [[Aristotle]] concluded
that a large increase in population would bring \"certain  on the citizenry,
and poverty  \". \". , In \". \", a  \"t \". , war, disaster and famine, are
factors that Malthus considered to increase the death rate. <ref
name=\"geography.about.com\">Rosenberg, M. (2007, September 09). Thomas Malthus
on Population. Retrieved June 20, 2009, from About.com:Geography Web site:
http://geography.about.com/od/populationgeography/a/malthus.htm</ref>
\"Preventative checks\" were factors that Malthus believed to affect the birth
rate such as moral restraint, abstinence and birth control <ref
name=\"geography.about.com\"/>. He predicted that \"positive checks\" on
[[exponential growth|exponential population growth]], such as [[poverty nd
[[war]], \". \", \": \", [[ ]] [[ ]] [[ genocide]].<ref name=\"Knudsen 2006 2-3\"/> In a 2004 interview, Ehrlich reviewed the predictions in his book and found that while the specific dates within his predictions may have been wrong, his predictions about climate change and disease were valid. Ehrlich continued to advocate for population control and co-authored the book ''The Population Explosion'' released in 1990 with his wife Anne Ehrlich. Paige Whaley Eager argues that the shift in perception that occurred in the 1960s must be understood in the context of the demographic changes that took place at the time. It was only in the first decade of the 19th Century that the world's population reached one billion. The second billion was added in the 1930s, the next billion in the 1960s. 90 percent of this net increase occurred in developing countries.<ref>{{cite book |title=Global Population Policy |last= Whaley Eager |first=Paige |authorlink= |coauthors= |year=2004 |publisher= Ashgate Publishing|location= |isbn=0754641627, 9780754641629 |pages=36|url=http://books.google.com/books?id=G2WBj4BDLqYC&dq=reproductive+rights&source=gbs_summary_s&cad=0 }}</ref> Whaley Eager also argues that at the time the US recognised that these demographic changes could significantly affect global geopolitics. Large increases occurred in [[China]], [[Mexico]] and [[ and   \"o \". poverty and famine, and  ,",", he , , , . [[Emigration]] to colonies would be encouraged should the population become too large. <ref name=\"Neurath 1994 6\"/> Aristotle concluded that a large increase in population would bring, \"certain poverty on the citizenry .\" , .\" , , During , .\" , , , , ,\" , , the , \"T , , .\" , [[ war]], [[disaster]] and [[ re factors that Malthus considered to increase the death rate. <ref name=\"geography.about.com\">Rosenberg, M. (2007, September 09). Thomas Malthus on Population. Retrieved June 20, 2009, from About.com:Geography Web site: http://geography.about.com/od/populationgeography/a/malthus.htm</ref> \"Preventative checks\" were factors that Malthus believed to affect the birth rate such as moral restraint, abstinence and birth control <ref name=\"geography.about.com\"/>. He predicted that \"positive checks\" on [[exponential growth|exponential population growth]]  .\" ,\" ,\" ,\" ,
and genocide.<ref name=\"Knudsen 2006 2-3\"/> In a 2004 interview, Ehrlich reviewed the predictions in his book, and found that while the specific dates within his predictions may have been wrong, his predictions about climate change and disease were valid. Ehrlich continued to advocate for population control and co-authored the book ''The Population Explosion'', released in 1990 with his wife Anne Ehrlich. Paige Whaley Eager argues that the shift in perception that occurred in the 1960s must be understood in the context of the demographic changes that took place at the time. It was only in the first decade of the 19th century that the world's population reached one billion. The second billion was added in the 1930s, and the next billion in the 1960s. 90 percent of this net increase occurred in developing countries.<ref>{{cite book |title=Global Population Policy |last= Whaley Eager |first=Paige |authorlink= |coauthors= |year=2004 |publisher= Ashgate Publishing|location= |isbn=0754641627, 9780754641629 |pages=36|url=http://books.google.com/books?id=G2WBj4BDLqYC&dq=reproductive+rights&source=gbs_summary_s&cad=0 }}</ref> Eager also argues that, at the time, the [[United States]] recognised that these demographic changes could significantly affect global geopolitics. Large increases occurred in [[China]], [[Mexico , , , , \"O .\"  and poverty and famine"
...
