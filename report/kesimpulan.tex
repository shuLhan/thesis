\chapter{Kesimpulan}

Secara keseluruhan, performansi dari sampel ulang LNSMOTE lebih baik daripada
SMOTE untuk kedua pengklasifikasi.
Pengaruh menarik lainnya yaitu pada pengklasifikasi CRF, dengan menggunakan
jumlah pohon lebih sedikit pada setiap tingkatan menghasilkan performansi yang
hampir sama dengan melakukan sampel ulang pada dataset asli, seperti yang
terlihat pada performansi CRF 100 tingkat 2 pohon pada dataset tanpa sampel
ulang, hampir sama dengan CRF 50 tingkat 4 pohon dengan sampel ulang SMOTE.

Untuk model klasifikasi vandalisme yang terbaik tanpa sampel ulang dihasilkan
dari pengklasifikasi CRF dengan 200 tingkat dan 1 pohon dengan nilai TPR
$0,9668$,
untuk model terbaik pada dataset yang telah disampel ulang dengan SMOTE yaitu
CRF dengan 200 tingkat dan 1 pohon dengan nilai TPR $0,9790$,
dan untuk model terbaik dari sampel ulang LNSMOTE yaitu CRF dengan 200 tingkat
1 pohon dengan nilai TPR $0.9904$.
Secara keseluruhan model terbaik yaitu dari CRF 200 tingkat dan 1 pohon yang
telah disampel ulang dengan LNSMOTE.
Selain performansi yang lebih baik, pengklasifikasi CRF juga lebih cepat $1,6$
kali dalam pembentukan model daripada RF pada dataset yang telah disampel
ulang.

\section{Kontribusi}

Kontribusi dari karya tulis ini selain membantu menemukan pengklasifikasi yang
lebih baik dalam mendeteksi vandalisme juga memberikan kerangka kerja untuk
menciptakan dan mengembangkan fitur vandalisme dari dataset mentah tanpa harus
mulai dari awal untuk dapat digunakan dalam penelitian selanjutnya atau
digunakan langsung pada kasus asli. Selain itu juga menyediakan pustaka program
untuk pengolahan data dan fungsi pembelajaran mesin terutama untuk sampel ulang
LNSMOTE dan pengklasifikasi
\textit{Cascaded Random Forest} yang belum ada implementasi secara terbuka pada
program terkenal seperti \textit{Weka}, \textit{Scikit-Learn}, atau \textit{R}.

\section{Pekerjaan Selanjutnya}

Jumlah artikel Bahasa Indonesia di Wikipedia setiap tahun semakin meningkat
seiring dengan meningkatnya jumlah pengguna internet di Indonesia.
Semakin banyak pengguna, kemungkinan vandalisme juga semakin meningkat.
Untuk mengatasi masalah vandalisme tersebut lebih awal, akan lebih baik bila
disiapkan sebuah sistem yang dapat mendeteksi vandalisme pada Wikipedia
Bahasa Indonesia.
Langkah awalnya mungkin dengan mengumpulkan data artikel yang telah dianotasi
dengan vandalisme, supaya dapat dianalisis untuk pembuatan fitur dan pembuatan
model deteksi vandalisme untuk Wikipedia Bahasa Indonesia.

Semua pelatihan model dalam karya tulis ini masih menggunakan algoritma
pengklasifikasi RF dan CRF dalam bentuk serial, yang mana satu pohon dibangun
satu per satu bergantian atau pada saat klasifikasi setiap sampel dimasukan ke
dalam pohon secara bergantian untuk mendapatkan kelasnya.
Penggunaan algoritma paralel, seperti dalam pembentukan dua atau empat pohon
bersamaan dan klasifikasi sampel (pengumpulan kelas di setiap pohon)
bersamaan, bisa membantu mempercepat pelatihan, pengujian dan hasil
klasifikasi.
Pada domain pembelajaran mesin, hal menarik yaitu adanya algoritma
\textit{eXtreme Gradient Boostring} (XGBoost) \cite{chen2016xgboost} yang
mungkin bisa diuji dan diterapkan untuk meningkatkan deteksi vandalisme.
