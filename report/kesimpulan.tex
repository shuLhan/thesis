\chapter{Kesimpulan}

Rata-rata SMOTE menaikan nilai TPR $0,19$ kali dan nilai FPR $1,6$ kali.
Sementara pada sampel ulang dengan LNSMOTE, rata-rata menaikan nilai TPR $0,33$
dan juga menaikan nilai FPR $2,75$ kali.
Kedua sampel ulang tersebut cukup bagus dalam meningkatkan performansi laju
positif dengan konsekuensi juga meningkatnya laju \textit{false-positive},
pengaruh ini lebih terasa pada sampel ulang LNSMOTE.
Secara keseluruhan performansi dari sampel ulang SMOTE lebih baik dari
LNSMOTE untuk pengklasifikasi CRF.
Hal ini disebabkan karena sifat algoritma dari CRF yang berfokus pembelajaran
sampel yang negatif, bukan pada sampel yang positif, sehingga menambahkan
sampel sintetis menyebabkan pembelajar menjadi \textit{over-fitting} pada
sampel positif.
Pengaruh menarik lainnya yaitu pada pengklasifikasi CRF, dengan menggunakan
jumlah pohon lebih sedikit pada setiap tingkatan menghasilkan performansi yang
hampir sama dengan melakukan sampel ulang pada dataset asli, seperti yang
terlihat pada performansi CRF-100-2 tanpa sampel ulang hampir sama dengan CRF
50 tingkat 4 pohon dengan sampel ulang SMOTE.

Untuk model klasifikasi vandalisme yang terbaik tanpa sampel ulang dihasilkan
dari pengklasifikasi CRF dengan 200 tingkat dan 1 pohon, untuk model terbaik
pada dataset yang telah disampel ulang dengan SMOTE yaitu CRF dengan 100
tingkat dan 1 pohon, dan untuk model terbaik dari sampel ulang LNSMOTE yaitu
CRF dengan 200 tingkat 1 pohon. Secara keseluruhan model terbaik yaitu dari CRF
100 tingkat dan 2 pohon yang telah disampel ulang dengan SMOTE.
Selain performansi yang lebih baik, pengklasifikasi CRF juga lebih cepat $1,6$
kali dalam pembentukan model daripada RF pada dataset yang telah disampel
ulang.

\section{Kontribusi}

Kontribusi dari karya tulis ini selain membantu menemukan pengklasifikasi yang
lebih baik dalam mendeteksi vandalisme juga memberikan kerangka kerja untuk
menciptakan dan mengembangkan fitur vandalisme dari dataset mentah tanpa harus
mulai dari awal untuk dapat digunakan dalam penelitian selanjutnya atau
digunakan langsung pada kasus asli. Selain itu juga menyediakan pustaka untuk
pengolahan data dan fungsi pembelajaran mesin terutama pengklasifikasi
\textit{Cascaded Random Forest} yang belum ada implementasi secara terbuka pada
program terkenal seperti \textit{Weka}, \textit{Scikit-Learn}, atau \textit{R}.

\section{Pekerjaan Selanjutnya}

Semua pelatihan model dalam karya tulis ini masih menggunakan algoritma
pengklasifikasi RF dan CRF dalam bentuk serial, yang mana satu pohon dibangun
satu per satu bergantian atau pada saat klasifikasi setiap sampel dimasukan ke
dalam pohon secara bergantian untuk mendapatkan kelasnya.
Penggunaan algoritma paralel, seperti dalam pembentukan dua atau empat pohon
bersamaan dan klasifikasi sampel (pengumpulan \textit{vote} di setiap pohon)
bersamaan, bisa membantu mempercepat pelatihan, pengujian dan hasil
klasifikasi.
Pada domain pembelajaran mesin, hal menarik yaitu adanya algoritma
\textit{eXtreme Gradient Boostring} (XGBoost) yang mungkin bisa diuji dan
diterapkan untuk meningkatkan deteksi vandalisme.
