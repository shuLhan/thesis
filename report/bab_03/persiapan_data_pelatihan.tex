Dataset yang digunakan untuk pelatihan yaitu PAN-WVC-10
\cite{potthast:2010b}.
Dataset PAN-WVC-10 terbagi menjadi dua yaitu dataset suntingan dari artikel
Wikipedia dan dataset anotasi yang berisi hasil klasifikasi vandalisme pada
dataset suntingan tersebut.
Dataset suntingan memiliki atribut sebagai berikut,

\begin{itemize}
	\item \textbf{editid}, format angka, berisi identifikasi (ID) unik dari setiap suntingan.
	\item \textbf{editor}, format string, berisi nama penyunting.
	\item \textbf{oldrevisionid}, format angka, berisi ID untuk suntingan lama.
	\item \textbf{newrevisionid}, format angka, berisi ID untuk suntingan baru.
	\item \textbf{diffurl}, format string, berisi URL yang mengacu pada perbedaan suntingan baru dengan lama.
	\item \textbf{edittime}, format string, berisi tanggal dan pukul
	suntingan.
	\item \textbf{editcomment}, format string, berisi komentar yang ditambahkan oleh penyunting saat menyimpan hasil suntingan.
	\item \textbf{articleid}, format angka, berisi ID unik dari artikel.
	\item \textbf{articletitle}, format string, berisi judul dari artikel yang disunting.
\end{itemize}

Dataset anotasi memiliki atribut sebagai berikut,

\begin{itemize}
	\item \textbf{editid}, format angka, mengacu pada \textit{editid} di
	dataset suntingan.
	\item \textbf{class}, format string, berisi tipe suntingan yang
	bernilai "regular" yang menyatakan bahwa suntingan tersebut bukan
	vandalisme, dan "vandalism" yang menyatakan bahwa suntingan tersebut
	adalah vandalisme.
	\item \textbf{annotators}, format angka, berisi jumlah orang yang
	menandai (penanda) bahwa suntingan dengan ID tersebut termasuk ke dalam
	kelas "regular" atau "vandalism".
	\item \textbf{totalannotators}, format angka, berisi jumlah total penanda yang memeriksa suntingan.
\end{itemize}

Kedua dataset kemudian digabung untuk menghasilkan atribut \textit{editid},
\textit{class}, \textit{oldrevisionid}, \textit{newrevisionid},
\textit{edittime}, \textit{editor}, \textit{articletitle},
\textit{editcomment}, \textit{deletions}, dan \textit{additions}.
Nilai dari atribut \textit{class} diganti dari teks menjadi angka, yaitu 1
untuk "vandalism" dan 0 untuk "regular".
Atribut tambahan \textit{deletions} berisi teks yang dihapus dalam revisi yang
lama.
Atribut tambahan \textit{additions} berisi teks yang ditambahkan dalam revisi
yang baru.
Kedua atribut tersebut didapat dengan membandingkan isi dari revisi yang lama
dengan yang baru.

Pemrosesan selanjutnya yaitu membuat berkas revisi yang bersih dari sintaks
wiki.
Tujuan dari revisi ini yaitu supaya tidak ada \textit{noise} pada saat
melakukan penghitungan fitur dan mempercepat proses pembuatan fitur.
Setiap berkas revisi dibaca kemudian dilakukan pembersihan berikut,

\begin{itemize}
\item penghapusan URI yang berawalan dengan
\textit{http://}, \textit{https://}, \textit{ftp://}, dan \textit{ftps}.
\item Menghapus \textit{mark-up} wiki yaitu konten yang dilingkupi oleh marka
berikut:
\texttt{[[Category:]]}, \texttt{[[:Category:]]}, \texttt{[[File:]]}, \\
\texttt{[[Help:]]}, \texttt{[[Image:]]}, \texttt{[[Special:]]},
\texttt{[[Wikipedia:]]}, \\
\texttt{\{\{DEFAULTSORT:\}\}}, \texttt{\{\{Template:\}\}}, dan \texttt{<ref/>}.
\item Mengganti karakter dan token berikut dengan karakter kosong (spasi):
\texttt{[}, \texttt{]}, \texttt{\{}, \texttt{\}}, \texttt{|}, \texttt{=},
\texttt{\#}, \texttt{'s}, \texttt{'}, \texttt{<ref>}, \texttt{</ref>},
\texttt{<br />}, \texttt{<br/>}, \texttt{<br>}, \texttt{<nowiki>},
\texttt{</nowiki>}, \texttt{\&nbsp;}.
\end{itemize}
