Implementasi \textit{Cascaded Random Forest} (CRF) berdasarkan pada makalah
Bauman dkk.  \cite{baumann2013cascaded}.
Hasil dari implementasi dapat dilihat dan digunakan secara terbuka pada
repositori \texttt{go-mining}
\footnote{
\url{%
https://github.com/shuLhan/go-mining/tree/master/classifiers/crf
}}.
Gambaran umum dari pengklasifikasi CRF dapat dilihat pada algoritma
\ref{alg:crf}.

Pengklasifikasi CRF memiliki tujuh parameter, empat diantaranya sama dengan RF
yaitu dataset fitur yang digunakan untuk pelatihan, jumlah pohon, jumlah sampel
acak disetiap pohon, dan jumlah fitur acak di setiap pohon.
Tiga parameter lainnya yaitu jumlah tingkatan, nilai ambang batas untuk
\textit{true-positive} ($maxtp$) dan nilai ambang batas untuk
\textit{true-negative rate} ($maxtn$).

	\newpage
	\begin{center}
	\captionof{algorithm}{Cascaded Random Forest}
	\label{alg:crf}
	\begin{algorithmic}[1]
\Require \\
$ TSET $: dataset pelatihan \\
$ S $: jumlah tingkatan \\
$ T $: jumlah pohon untuk setiap tingkatan \\
$ B $: persentase sampel yang dipilih secara acak tanpa diganti dari $TSET$ \\
$ m $: jumlah fitur acak \\
$ maxtp $: ambang batas untuk nilai \textit{true-positive} \\
$ maxtn $: ambang batas untuk nilai \textit{true-negative}
\\
\Function{CascadedRandomForest}{$ TSET, S, T, B, m, maxtp, maxtn $}
	\State $ n \gets $ jumlah sampel pada dataset $ TSET $
	\State $ b \gets n * (B / 100) $
	\Comment Jumlah sampel acak untuk setiap pohon.

	\State $ TNSET \gets nil $
	\Comment Penampung untuk sampel \textit{true-negative}.

	\State $ crf \gets nil $
	\Comment Pengklasifikasi yang berisi tingkatan.

	\For{$ s \gets 1,S $}
		\State $ forest \gets nil $

		\For{$ t \gets 1,T $}
			\State $ tp, tn, tree \gets \Call{GrowTree}{forest, S, b, m} $
			\If{$ tp > maxtp $ \textbf{and} $ tn > maxtn $}
				\State Tingkatan selesai
			\Else
				\State $ forest.\Call{Add}{tree} $
			\EndIf
		\EndFor
		\\
		\State $w \gets exp(\Call{FMeasure}{tree}) $
		\Comment Simpan nilai \textit{F-Measure} dari pohon terakhir sebagai
		bobot dari tingkatan.
		\State $ crf.\Call{Add}{forest, w} $
		\Comment Simpan forest dan bobot dari tingkatan.
		\If{$ s = 1 $}
			\State Pindahkan sampel \textit{true-negative} dari
			$TSET$ ke $TNSET$
		\Else
			\State Uji tingkatan sebelumnya dengan $TNSET$
			\State Hapus sampel \textit{true-negative} dari TNSET
			\State Isi ulang $TSET$ dengan sampel
			\textit{false-positive} dari hasil uji $TNSET$
		\EndIf
	\EndFor

	\State \Return{crf}
\EndFunction
\\
\Function{Classify}{$ crf, set $}
	\State $ predictions \gets nil $
	\Comment Penampung untuk prediksi klasifikasi dari $set$.
	\State $ sumw \gets $ Jumlah semua bobot tingkatan dalam $crf$.

	\For{\textbf{each} $sample$ \textbf{in} $set$}
		\State $ stageProbs \gets nil $
		\Comment Penampung untuk probabilitas kelas dari setiap $forest$.

		\For{\textbf{each} $forest$, $w$ \textbf{in} $crf$}
			\State $ votes \gets \Call{Classify}{forest, set, nil} $
			\State $ cwprobs \gets $
				(probabilitas setiap kelas dalam $votes \times w$) /
				($ sumw ~ \times $ jumlah pohon dalam $forest$)
			\State $ stageProbs.\Call{Push}{cwprobs} $
		\EndFor

		\State $ class \gets $ Pilih kelas dengan probabilitas maksimum pada $stageProbs$
		\State $ predictions.\Call{Push}{class} $
	\EndFor
	\\
	\State \Return{predictions}
\EndFunction
	\end{algorithmic}
\end{center}


Fungsi \texttt{GrowTree} pada CRF memiliki parameter dan kembalian yang sama
dengan fungsi \textit{GrowTree} pada \textit{RandomForest}.
Setiap selesai membangun sebuah pohon, nilai kembalian $tp$ dan $tn$ diperiksa.
Jika nilai $tp$ dan $tn$ lebih besar dari nilai $maxtp$ dan $maxtn$ maka proses
pembangunan tingkatan selesai, jika belum lanjutkan membangun pohon sampai
nilai $maxtp$ dan $maxtn$ tercapai atau pohon ke-$T$ terbangun.

Saat sebuah tingkatan selesai, simpan $forest$ yang telah terbangun berikut
dengan nilai \textit{F-Measure} dari pohon yang terakhir sebagai bobot ($w$)
dari tingkatan yang nanti digunakan untuk proses klasifikasi.

Fungsi \texttt{Classify} memiliki dua parameter yaitu pengklasifikasi CRF
($crf$) dan kumpulan sampel yang akan diklasifikasi ($set$).
Untuk setiap sampel dalam $set$, fungsi ini melakukan klasifikasi
\textit{Random-Forest} dan mengumpulkan probabilitas dari kelas yang telah
dikalikan dengan bobot dari setiap tingkatan.
Fungsi ini mengembalikan prediksi dari kelas pada $set$.
