Implementasi \textit{Cascaded Random Forest} (CRF) berdasarkan pada makalah
Bauman dkk.  \cite{baumann2013cascaded}.
Hasil dari implementasi dapat dilihat dan digunakan secara terbuka pada
repositori \texttt{go-mining}
\footnote{
\url{%
https://github.com/shuLhan/go-mining/tree/master/classifiers/crf
}}.
Gambaran umum dari pengklasifikasi CRF dapat dilihat pada algoritma
\ref{alg:crf}.

Pengklasifikasi CRF memiliki tujuh parameter, empat diantaranya sama dengan RF
yaitu dataset fitur yang digunakan untuk pelatihan, jumlah pohon, jumlah sampel
acak disetiap pohon, dan jumlah fitur acak di setiap pohon.
Tiga parameter lainnya yaitu jumlah tingkatan, nilai ambang batas untuk
\textit{true-positive} ($maxtp$) dan nilai ambang batas untuk
\textit{true-negative rate} ($maxtn$).
Dalam pelatihan digunakan tiga pengklasifikasi CRF yang memiliki parameter
tingkat dan pohon yang berbeda yaitu CRF dengan 200 tingkat dan 1 pohon, CRF
dengan 100 tingkat dan 2 pohon, dan CRF dengan 50 tingkat dan 4 pohon.
Ketiga pengklasifikasi CRF tersebut memiliki total jumlah pohon yang sama yaitu
200 pohon, menggunakan jumlah sampel acak dan fitur acak yang sama yaitu 64\%
dan 5 fitur acak, dan nilai ambang batas $maxtp$ dan $maxtn$ yang sama yaitu
$0,95$ dan $0,95$.
