Dataset yang didapat tidak langsung bisa digunakan untuk pelatihan dan
pengujian, melainkan perlu ada pra-proses terlebih dahulu yang mengikutkan
penggabungan data dan pembersihan data dengan menghapus atribut yang tidak
diperlukan dan pembersihan dari isi data itu sendiri.
Setelah data mentah dibersihkan, kemudian diekstrak fitur untuk masing-masing
dataset pelatihan dan pengujian sehingga menghasilkan dataset pelatihan dan
pengujian yang bernilai kontinu.
Dataset pelatihan disampel ulang dengan metode SMOTE dan LNSMOTE,
sehingga menghasilkan tiga dataset yang akan dilatih yaitu dataset tanpa sampel
ulang, dataset yang telah disampel ulang dengan SMOTE, dan dataset yang telah
disampel ulang dengan LNSMOTE.
Ketiga dataset pelatihan tersebut kemudian dijalankan pada program
pengklasifikasi \textit{Random Forest} (RF) dan \textit{Cascaded Random Forest}
(CRF) satu per satu, yang menghasilkan sebuah model deteksi vandalisme.
Model tersebut kemudian diuji dengan menginputkan dataset fitur pengujian
sehingga menghasilkan kelas-kelas dari sampel yang nantinya dapat digunakan
untuk mengevaluasi sampel ulang dan pengklasifikasi.
