Proses sampel ulang LNSMOTE yang digunakan pada implementasi berdasarkan
pada makalah Maciejewski dkk.
\cite{maciejewski2011local}
yang dapat dilihat pada algoritma
\ref{alg:lnsmote}.
Hasil implementasinya dalam bentuk sebuah program
\footnote{\url{%
https://github.com/shuLhan/go-mining/tree/master/resampling/lnsmote
}}.
Program LNSMOTE dijalankan dengan diberikan input dataset fitur PAN-WVC-10
dengan persentase sampel sintetis yaitu 1.100\% dan jumlah tetangga terdekat
yaitu 5, sehingga didapat 28.588 sampel sintetis positif, ditambah dengan
sampel positif asli totalnya menjadi 30.982 sampel positif.

	\newpage
	\begin{algorithm}[h]
	\caption{Local Neighbourhood SMOTE}
	\label{alg:lnsmote}
	\begin{algorithmic}[1]
\Require \\
S: dataset sampel keseluruhan \\
P: sample minoritas dari $ S $ \\
o: rasio \textit{over-sampling}, jumlah sampel sintetis yang akan dibuat \\

\Function{LNSMOTE}{$ S, P, o $}
	\State $ SEEDS \gets P $
	\State $ OUT \gets S $
	\For{\textbf{each} $ p $ \textbf{in} $ SEEDS $}
		\State $ NN \gets \Call{KNearestNeighbor}{S, p} $
		\For{$ i \gets 1 \dots o $}
			\State $ s \gets \Call{CreateSynthetic}{p, NN, S} $

			\If{$ s \not= nil $}
				\State tambahkan $ s $ ke $OUT$
			\EndIf
		\EndFor
	\EndFor
	\State \Return{OUT}
\EndFunction
\\
\Function{CreateSynthetic}{$ p, NN, S $}
	\State $ n \gets $ pilih \textit{nearest neighbourhood} secara acak
	pada $ NN $
	\If{\textbf{not} \Call{CanCreate}{$ S, p, n $}}
		\State \Return nil
	\EndIf

	\State $ x_{new} \gets \Call{Clone}{p} $
	\For{\textbf{each} $ a $ \textbf{in} \Call{Attributes}{S}}a
		\If{\Call{IsQuantitative}{$ a $}}
			\State $ \delta \gets \Call{RandomGap}{S, p, n} $
			\State $ diff \gets n(a) - p(a) $
			\State $ x_{new}(a) \gets p(a) + \delta \cdot diff $
		\Else
			\State $ x_{new} \gets \Call{MostFrequent}{p \cup NN, a} $
		\EndIf
	\EndFor

	\State \Return $ x_{new} $
\EndFunction
	\algstore{lnsmote_break}
	\end{algorithmic}
\end{algorithm}

\begin{algorithm}[h]
	\caption{Local Neighbourhood SMOTE bagian 2}
	\begin{algorithmic}[1]
	\algrestore{lnsmote_break}

\Function{CanCreate}{$ S, p, n $}
	\State $ slp \gets \Call{SafeLevel}{S, p} $
	\State $ sln \gets \Call{SafeLevel2}{S, p, n} $
	\State \Return{$ slp \not= 0 $ \textbf{or} $ sln \not= 0 $}
\EndFunction
\\
\Function{SafeLevel}{$ S, p $}
	\State $ pneighbor \gets \Call{KNearestNeighbor}{S, p} $
	\State \Return{$ pneighbor $ yang kelasnya minoritas}
\EndFunction
\\
\Function{SafeLevel2}{$ S, p, n $}
	\State $ nneighbor \gets \Call{KNearestNeighbor}{ S, n } $
	\If{$\Call{Class}{n} \not= P $ \textbf{and} $ p \in nneighbor $}
		\State Ganti $ p $ di $ nneighbor $ dengan tetanggga $ k + 1 $
		dari $ n $
	\EndIf
	\State \Return{$ nneighbor $ yang kelasnya minoritas}
\EndFunction
\\
\Function{RandomGap}{$ S, p, n $}
	\State $ slp \gets \Call{SafeLevel}{S, p} $
	\State $ sln \gets \Call{SafeLevel}{S, p, n} $
	\State $ \delta \gets 0 $

	\If{$ sln = 0 $ \textbf{and} $ slp > 0 $}
		\State \Return{$ \delta $}
	\EndIf
	\\
	\State $ slratio \gets \frac{slp}{sln} $
	\If{$ slratio = 1 $}
		\State $ \delta \gets \Call{Random}{1} $
	\ElsIf{$ slratio > 1 $}
		\State $ \delta \gets \Call{Random}{\frac{1}{slratio}} $
	\Else
		\State $ \delta = 1 - \Call{Random}{slratio} $
	\EndIf
	\\
	\If{$ \Call{Class}{n} \not= P $}
		\State $ \delta = \delta \cdot \frac{sln}{k} $
	\EndIf
	\\
	\State \Return{$ \delta $}
\EndFunction
	\end{algorithmic}
\end{algorithm}


LNSMOTE memiliki empat parameter yaitu dataset keseluruhan ($D$), kelas
minoritas yang akan disampel ulang ($minor$) pada dataset $D$, persentase
sampel sintetis yang akan dibuat ($o$), dan jumlah tetangga terdekat yang akan
dicari ($k$) di sekitar sampel minoritas.

Fungsi \texttt{CreateSynthetic} memiliki tiga parameter yaitu dataset ($D$),
sebuah sampel ($sample$), dan semua tetangga dari $sample$ ($neighbors$).
Fungsi ini mengembalikan sebuah sampel sintetis yang dibuat antara $sample$
dengan salah satu tetangga $neighbours$ yang dipilih secara acak ($n$), dengan
syarat nilai \textit{safe-level} dari $sample$ atau $n$ di dalam dataset $D$
lebih besar dari $0$.

Fungsi \texttt{CanCreate} memiliki tiga parameter yaitu dataset ($D$), sebuah
sampel ($sample$), dan sebuah tetangga dari $sample$ ($neighbour$).
Fungsi ini memeriksa \textit{safe-level} dari $sample$ dan $neigbour$ di dalam
dataset $D$, jika nilai \textit{safe-level} kedua sampel tersebut sama dengan
0, maka fungsi mengembalikan nilai $false$ yang berarti sampel sintetis tidak
bisa dibuat antara $p$ dan $n$, sebaliknya jika salah satu tidak 0
maka akan mengembalikan nilai $true$ yang berarti sampel sintetis bisa dibuat.

Fungsi \texttt{SafeLevel} memiliki tiga parameter yaitu dataset ($D$), sebuah
sampel minoritas $p$ yang ada di dalam $D$, dan jumlah tetangga $p$ yang akan dicari
dalam $D$ ($k$).
Fungsi ini mencari $k$ buah tetangga dari $p$ dan mengembalikan hanya tetangga
yang kelasnya minoritas.

Fungsi \texttt{SafeLevel2} memiliki empat parameter yaitu dataset ($D$), sebuah
sampel minoritas ($p$) yang ada di dalam $D$, sebuah tetangga dari $p$ ($n$),
dan jumlah tetangga dari $p$ yang akan dicari ($k$).
Fungsi ini mencari $k$ tetangga terdekat dari $n$, katakanlah $nneighbours$,
jika $n$ bukan sampel minoritas dan $p$ adalah tetangga dari $n$ maka sampel
$p$ yang ada di dalam $nneighbours$ diganti dengan tetangga yang selanjutnya
($k+1$).
Fungsi ini mengembalikan semua tetangga $nneighbours$ yang kelasnya minoritas.

Fungsi \texttt{RandomGap} memiliki empat parameter yaitu dataset ($D$), sebuah
sampel minoritas ($p$) yang ada di dalam $D$, sebuah tetangga dari $p$ ($n$),
dan jumlah tetangga yang akan dicari ($k$).
Fungsi ini menghitung jarak acak antara sampel minoritas $p$ dengan tetangganya
$n$ berdasarkan nilai \textit{safe-level} dari $p$ dan $n$.
Jika \textit{safe-level} $n$ sama dengan 0 dan \textit{safe-level} dari $p$
besar dari 0, artinya tidak ada tetangga dari $n$ yang minoritas yang mana $n$
berarti adalah \textit{outlier}, maka fungsi ini mengembalikan nilai jarak 0.
Jika kedua \textit{safe-level} dari $p$ dan $n$ besar dari 0 maka dihitung
rasionya ($slratio$).
Jika $slratio$ sama dengan 1, artinya sampel $p$ dan $n$ memiliki jumlah
tetangga yang sama, maka fungsi ini mengembalikan jarak yang diambil secara
acak dengan rentang antara 0 dan 1.
Jika $slratio$ besar dari 1, artinya jumlah tetangga $p$ lebih besar dari
jumlah tetangga $n$, maka fungsi mengembalikan nilai jarak acak dengan rentang
0 dan $1/slratio$.
Selain itu, fungsi mengembalikan jarak acak antara 0 sampai
$1-Random(slratio)$.
