Proses sampel ulang LNSMOTE yang digunakan pada implementasi berdasarkan
pada makalah Maciejewski dkk.
\cite{maciejewski2011local}
yang dapat dilihat pada algoritma
\ref{alg:lnsmote}.
Hasil implementasinya dalam bentuk sebuah program
\footnote{\url{%
https://github.com/shuLhan/go-mining/tree/master/resampling/lnsmote
}}.

LNSMOTE memiliki empat parameter yaitu dataset keseluruhan yang akan disampel
ulang, dataset minoritas, persentase sampel sintetis yang akan dibuat, dan
jumlah tetangga terdekat.
Program LNSMOTE dijalankan dengan diberikan input dataset fitur PAN-WVC-10
dengan persentase sampel sintetis yaitu 1.100\% dan jumlah tetangga terdekat
yaitu 5, sehingga didapat 28.588 sampel sintetis positif, ditambah dengan
sampel positif asli totalnya menjadi 30.982 sampel positif.
