Proses sampel ulang LNSMOTE yang digunakan pada implementasi berdasarkan
pada makalah
\textcite{maciejewski2011local}
yang dapat dilihat pada algoritma
\ref{alg:lnsmote}.
Hasil implementasinya dalam bentuk sebuah program
\footnote{\url{%
https://github.com/shuLhan/go-mining/tree/master/resampling/lnsmote
}}.
Program LNSMOTE dijalankan dengan diberikan input dataset fitur PAN-WVC-10
dengan persentase sampel sintetis yaitu 1.100\% dan jumlah tetangga terdekat
yaitu 5, sehingga didapat 28.588 sampel sintetis positif, ditambah dengan
sampel positif asli totalnya menjadi 30.982 sampel positif.

	\begin{center}
	\captionof{algorithm}{Local Neighbourhood SMOTE}
	\label{alg:lnsmote}
	\begin{algorithmic}[1]
\Require \\
$ D $: dataset sampel keseluruhan \\
$ minor $: kelas minoritas pada dataset $D$ \\
$ o $: persentase jumlah sampel sintetis yang akan dibuat \\
$ k $: jumlah tetangga terdekat dari sampel yang akan dibuat
\\
\Function{LNSMOTE}{$ D, P, o, k $}
	\State $ S \gets [] $
	\Comment Penampung untuk semua sampel sintetis yang akan dibuat.
	\State $ nsyn \gets o/100 $
	\Comment Jumlah sampel sintetis yang akan dibuat.
	\State $ P \gets $ semua sampel minoritas $minor$ pada dataset $D$.
	\\
	\For{\textbf{each} $ sample $ \textbf{in} $ P $}
		\State $ neighbours \gets \Call{FindNeighbours}{D, sample, k} $
		\For{$ i \gets 1,nsyn $}
			\State $ s \gets \Call{CreateSynthetic}{D, sample, neighbours} $

			\If{$ s \not= nil $}
				\State $ \Call{S.Push}{s} $
			\EndIf
		\EndFor
	\EndFor
	\\
	\State \Return{S}
\EndFunction
\\
\Function{CreateSynthetic}{$ D, sample, neighbors $}
	\State $ neighbor \gets \Call{RandomPick}{neighbors} $
	\If{$ \Call{CanCreate}{D, sample, neighbor} = false $}
		\State \Return nil
	\EndIf
	\\
	\State $ lenAttr \gets \Call{LengthAttribute}{sample} $
	\Comment Simpan jumlah atribut dari sampel.

	\State $ newsample \gets [] $
	\Comment Penampung untuk sampel sintetis yang baru.
	\\
	\For{$ x \gets 1,lenAttr $}
		\If{y \textbf{is} classAttribute}
			\State continue
		\EndIf

		\State $ gap \gets \Call{RandomGap}{D, sample, neighbor, k} $

		\State $ sattr \gets sample.\Call{AttributeAt}{y} $
		\State $ nattr \gets neighbor.\Call{AttributeAt}{y} $
		\State $ diff \gets sattr - nattr $
		\State $ newAttr \gets sattr + (gap \times diff) $

		\State $ newsample.\Call{PushAttribute}{newAttr} $
	\EndFor
	\\
	\State \Return{$ newsample $}
\EndFunction
\\
\Function{CanCreate}{$ D, sample, neighbour $}
	\State $ slp \gets \Call{SafeLevel}{D, sample, k} $
	\State $ sln \gets \Call{SafeLevel2}{D, sample, neighbor, k} $
	\State \Return{$ slp \not= 0 $ \textbf{or} $ sln \not= 0 $}
\EndFunction
\\
\Function{SafeLevel}{$ D, p, k $}
	\State $ pneighbor \gets \Call{FindNeighbours}{ D, p, k } $
	\State \Return{$ pneighbor $ yang kelasnya minoritas}
\EndFunction
\\
\Function{SafeLevel2}{$ D, p, n, k $}
	\State $ nneighbours \gets \Call{FindNeighbours}{ D, n, k } $
	\If{$\Call{Class}{n} \not= minor $ \textbf{and} $ p \in nneighbours $}
		\State Ganti $ p $ di $ nneighbours $ dengan tetanggga $ k + 1 $
		dari $ n $
	\EndIf
	\State \Return{$ nneighbours $ yang kelasnya minoritas}
\EndFunction
\\
\Function{RandomGap}{$ D, p, n, k $}
	\State $ slp \gets \Call{SafeLevel}{D, p, k} $
	\State $ sln \gets \Call{SafeLevel}{D, p, n, k} $
	\State $ \delta \gets 0 $

	\If{$ sln = 0 $ \textbf{and} $ slp > 0 $}
		\State \Return{$ \delta $}
	\EndIf
	\\
	\State $ slratio \gets slp / sln $
	\If{$ slratio = 1 $}
		\State $ \delta \gets \Call{Random}{1} $
	\ElsIf{$ slratio > 1 $}
		\State $ \delta \gets \Call{Random}{1 / slratio} $
	\Else
		\State $ \delta = 1 - \Call{Random}{slratio} $
	\EndIf
\pagebreak
	\If{$ \Call{Class}{n} \not= minor $}
		\State $ \delta = \delta \times (sln / k) $
	\EndIf
	\\
	\State \Return{$ \delta $}
\EndFunction
	\end{algorithmic}
\end{center}


LNSMOTE memiliki empat parameter yaitu dataset keseluruhan ($D$), kelas
minoritas yang akan disampel ulang ($minor$) pada dataset $D$, persentase
sampel sintetis yang akan dibuat ($o$), dan jumlah tetangga terdekat yang akan
dicari ($k$) di sekitar sampel minoritas.

Fungsi \texttt{CreateSynthetic} memiliki tiga parameter yaitu dataset ($D$),
sebuah sampel ($sample$), dan semua tetangga dari $sample$ ($neighbors$).
Fungsi ini mengembalikan sebuah sampel sintetis yang dibuat antara $sample$
dengan salah satu tetangga $neighbours$ yang dipilih secara acak ($n$), dengan
syarat nilai \textit{safe-level} dari $sample$ atau $n$ di dalam dataset $D$
lebih besar dari $0$.

Fungsi \texttt{CanCreate} memiliki tiga parameter yaitu dataset ($D$), sebuah
sampel ($sample$), dan sebuah tetangga dari $sample$ ($neighbour$).
Fungsi ini memeriksa \textit{safe-level} dari $sample$ dan $neigbour$ di dalam
dataset $D$, jika nilai \textit{safe-level} kedua sampel tersebut sama dengan
0, maka fungsi mengembalikan nilai $false$ yang berarti sampel sintetis tidak
bisa dibuat antara $p$ dan $n$, sebaliknya jika salah satu tidak 0
maka akan mengembalikan nilai $true$ yang berarti sampel sintetis bisa dibuat.

Fungsi \texttt{SafeLevel} memiliki tiga parameter yaitu dataset ($D$), sebuah
sampel minoritas $p$ yang ada di dalam $D$, dan jumlah tetangga $p$ yang akan dicari
dalam $D$ ($k$).
Fungsi ini mencari $k$ buah tetangga dari $p$ dan mengembalikan hanya tetangga
yang kelasnya minoritas.

Fungsi \texttt{SafeLevel2} memiliki empat parameter yaitu dataset ($D$), sebuah
sampel minoritas ($p$) yang ada di dalam $D$, sebuah tetangga dari $p$ ($n$),
dan jumlah tetangga dari $p$ yang akan dicari ($k$).
Fungsi ini mencari $k$ tetangga terdekat dari $n$, katakanlah $nneighbours$,
jika $n$ bukan sampel minoritas dan $p$ adalah tetangga dari $n$ maka sampel
$p$ yang ada di dalam $nneighbours$ diganti dengan tetangga yang selanjutnya
($k+1$).
Fungsi ini mengembalikan semua tetangga $nneighbours$ yang kelasnya minoritas.

Fungsi \texttt{RandomGap} memiliki empat parameter yaitu dataset ($D$), sebuah
sampel minoritas ($p$) yang ada di dalam $D$, sebuah tetangga dari $p$ ($n$),
dan jumlah tetangga yang akan dicari ($k$).
Fungsi ini menghitung jarak acak antara sampel minoritas $p$ dengan tetangganya
$n$ berdasarkan nilai \textit{safe-level} dari $p$ dan $n$.
Jika \textit{safe-level} $n$ sama dengan 0 dan \textit{safe-level} dari $p$
besar dari 0, artinya tidak ada tetangga dari $n$ yang minoritas yang mana $n$
berarti adalah \textit{outlier}, maka fungsi ini mengembalikan nilai jarak 0.
Jika kedua \textit{safe-level} dari $p$ dan $n$ besar dari 0 maka dihitung
rasionya ($slratio$).
Jika $slratio$ sama dengan 1, artinya sampel $p$ dan $n$ memiliki jumlah
tetangga yang sama, maka fungsi ini mengembalikan jarak yang diambil secara
acak dengan rentang antara 0 dan 1.
Jika $slratio$ besar dari 1, artinya jumlah tetangga $p$ lebih besar dari
jumlah tetangga $n$, maka fungsi mengembalikan nilai jarak acak dengan rentang
0 dan $1/slratio$.
Selain itu, fungsi mengembalikan jarak acak antara 0 sampai
$1-Random(slratio)$.
