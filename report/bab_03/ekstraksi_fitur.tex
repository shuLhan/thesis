Beberapa makalah sebelumnya mengelompokan fitur ke dalam tiga kelompok yaitu
\textit{metadata}, teks, dan bahasa.
Dalam tesis ini digunakan 4 fitur metadata, 11 fitur teks, dan 10 fitur
bahasa yang diambil dari hasil analisis makalah Mola-Velasco
\cite{mola2012wikipedia}.

Dari analisis fitur vandalisme tersebut kemudian ditransformasikan ke dalam
sebuah program.
Program ekstraksi fitur kemudian dijalankan dengan input dataset PAN-WVC-10
dan PAN-WVC-11 yang telah dibersihkan pada subbab
\ref{bab:03:persiapan_data_pelatihan} dan
\ref{bab:03:persiapan_data_pengujian},
sehingga menghasilkan dataset fitur yang bernilai kontinu.
Implementasi program pembersihan data dan generator fitur ini dapat dilihat dan
digunakan secara terbuka pada repositori \texttt{wvcgen}
\footnote{\url{https://github.com/shuLhan/wvcgen}}.

\subsection{Fitur Kelompok Metadata}

Kelompok metadata mengacu pada properti dari sebuah revisi yang secara langsung
dapat diambil, seperti identitas penyunting, komentar, atau ukuran perubahan.

Berikut daftar fitur metadata yang digunakan,

\begin{itemize}

\item \textbf{Anonim}.
Penyunting anonim yaitu yang tidak menggunakan akun Wikipedia saat melakukan
penyuntingan, sehingga yang tercatat hanya alamat IP bukan nama pengguna.
Fitur ini melihat apakah suntingan anonim atau bukan pada atribut
\textit{editor}, jika benar maka ditandai dengan nilai 1, atau 0 sebaliknya.
Vandal lebih condong berlaku anonim karena jika menggunakan akun asli akan
membuat akun mereka mudah diblokir dan membuat akun baru membutuhkan waktu dan
identitas berupa alamat surel.

\item \textbf{Panjang komentar}.
Melihat dari jumlah karakter yang diinputkan di kolom rangkuman suntingan, yang
tersimpan dalam atribut \textit{editcomment} pada dataset, saat menyimpan hasil
suntingan, tanpa mengikutkan bagian \textit{header} yaitu penanda di awal
komentar yang menunjuk ke bagian yang di sunting, biasanya dalam format
"\texttt{/* Nama header */}".
Komentar yang panjang mungkin mengindikasikan suntingan normal dan yang pendek
atau kosong mungkin menyarankan suatu vandalisme.

\item \textbf{Peningkatan ukuran}.
Peningkatan absolut dari ukuran konten artikel, dihitung dengan
\[
|\text{ukuran\_suntingan\_baru} - \text{ukuran\_suntingan\_lama}|
\]
Misalnya, berkurangnya ukuran dalam jumlah besar bisa mengindikasikan
pengosongan artikel.

\item \textbf{Rasio ukuran}.
Ukuran revisi baru relatif terhadap revisi lama, dihitung dengan cara
\[
\frac{1 + |baru|}{1 + |lama|}
\]

\end{itemize}


\subsection{Fitur Kelompok Teks}

Berikut daftar fitur berbasiskan teks yang digunakan,

\begin{itemize}

\item \textbf{Rasio huruf besar dan kecil}.
Pelaku vandal biasanya tidak mengikuti aturan huruf kapital, menulis semuanya
dengan huruf kecil atau huruf besar.
Rasio ini dihitung pada teks yang ditambahkan di revisi baru dengan menggunakan
rumus
\[
\frac{1 + |huruf\ besar|}{1 + |huruf\ kecil|}
\]

\item \textbf{Rasio huruf besar terhadap semua karakter}
Rasio ini dihitung dengan menggunakan rumus
\[
\frac{1 + |huruf\ besar|}{1 + |huruf\ besar| + |huruf\ kecil|}
\]

\item \textbf{Rasio angka}.
Rasio semua karakter terhadap angka, yaitu
\[
\frac{1 + |angka|}{1 + |semua karakter|}
\]
Fitur ini membantu menemukan perubahan kecil yang hanya mengubah angka.
Contoh kasusnya perubahan sebuah tanggal atau perhitungan untuk menyebabkan
kesalahan informasi.

\item \textbf{Rasio non-alfanumerik}.
Rasio semua karakter terhadap karakter selain huruf dan angka, yaitu
\[
\frac{1 + |non\ alfanumerik|}{1 + |semua\ karakter|}
\]
Penggunaan karakter selain angka-huruf yang berlebihan bisa mengindikasikan
penggunaan \textit{emoticon}, tanda baca, atau kata tak bermakna.

\item \textbf{Diversitas karakter}.
Menghitung jumlah karakter berbeda yang digunakan pada penambahan, dibandingkan dengan panjang teks yang dimasukan,
dihitung dengan rumus,
\[
length \frac{1}{karakter\ berbeda}
\]
Fitur ini membantu menemukan penggunaan karakter secara acak dan kata tak
bermakna.

\item \textbf{Distribusi karakter}.
Menggunakan Divergensi Kullback-Leibler dari distribusi karakter yang dimasukan
terhadap ekspektasi.
Fitur ini berguna untuk mendeteksi kata tak bermakna.

\item \textbf{Laju kompresi}.
Melihat tingkat kompres dari penambahan teks menggunakan algoritma kompresi
LZW.
Fitur ini berguna untuk mendeteksi kata tak bermakna, pengulangan kata atau
karakter, dll.
Vandalisme biasanya memiliki ukuran kompresi yang rendah.

\item \textbf{Token umum}.
Menghitung token yang biasanya jarang digunakan oleh vandal yaitu sintaks
wiki, seperti \textit{\_\_TOC\_\_}. Daftar token dapat dilihat pada lampiran
\ref{lampiran:words_wiki_token}.

\item \textbf{Frekuensi rerata kata}.
Frekuensi relatif rerata dari kata yang dimasukan pada revisi baru.
Pada artikel yang panjang, semakin banyak kata yang dimasukan yang tidak ada
pada artikel mengindikasikan bahwa suntingan tersebut bisa tak bermakna atau
tidak berhubungan dengan isinya.

\item \textbf{Kata terpanjang}.
Panjang dari kata yang dimasukan.
Nilainya akan 0 jika tidak ada kata yang dimasukan.
Fitur ini berguna untuk mendeteksi suntingan tak-bermakna.

\item \textbf{Urutan karakter terpanjang}.
Urutan terpanjang dari karakter yang sama pada teks yang dimasukan sering
digunakan pada vandalisme, contohnya \textit{aaarrrrggghhh! sooo huge}.

\end{itemize}

\subsection{Fitur Kelompok Bahasa}

Fitur kelompok bahasa didasarkan pada jumlah kata tertentu yang ditambahkan
pada suntingan atau revisi yang baru.
Untuk setiap kategori kata dihitung dua jenis fiturnya: frekuensi dan impak.

Fitur frekuensi yaitu menghitung frekuensi dari kategori kata terhadap
total seluruh kata pada suntingan yang baru, yaitu
\[
	\frac{jumlah\ kategori\ kata}
		{jumlah\ seluruh\ kata}
\]

Fitur impak yaitu menghitung persentase peningkatan kategori kata di revisi
yang baru, dihitung dengan,
\[
	\frac{jumlah\ kategori\ kata\ di\ revisi\ lama}
		{jumlah\ kategori\ kata\ di\ revisi\ lama +
		jumlah\ kategori\ kata\ di\ revisi\ baru}
\]

Berikut daftar kategori kata yang digunakan,

\begin{itemize}
\item \textbf{Vulgarisme}.
Menghitung kata-kata vulgar, kasar dan menghina.
Daftar kategori kata ini dapat dilihat pada lampiran
\ref{lampiran:words_vulgar}.

\item \textbf{Subjek}.
Menghitung kata subjek pertama atau kedua, termasuk pengejaan tidak baku,
misalnya \textit{I}, \textit{you}, \textit{ya}.
Daftar kategori kata ini dapat dilihat pada lampiran
\ref{lampiran:words_pronoun}.

\item \textbf{Bias}.
Menghitung penggunaan kata sehari-hari yang mengandung bias, misalnya
\textit{coolest}, \textit{huge}.
Daftar kategori kata ini dapat dilihat pada lampiran
\ref{lampiran:words_bias}.

\item \textbf{Pornografi}.
Menghitung penggunaan kata berhubungan dengan pornografi.
Daftar kategori kata ini dapat dilihat pada lampiran
\ref{lampiran:words_sex}.

\item \textbf{Kata buruk}.
Menghitung penggunaan kata sehari-hari dan beberapa penulisan yang buruk
(misalnya, \textit{wanna}, \textit{gotcha}).
Daftar kategori kata ini dapat dilihat pada lampiran
\ref{lampiran:words_bad}.

\item \textbf{Seluruh kategori kata}.
Gabungan dari kategori kata vulgarisme, subjek, bias, pornografi, dan kata
buruk.

\end{itemize}

