Implementasi SMOTE dibuat ke dalam sebuah program
\footnote{\url{%
https://github.com/shuLhan/go-mining/tree/master/resampling/smote
}}
berdasarkan makalah aslinya
\parencite{chawla2002smote}.
Gambaran umum dari implementasi dapat dilihat pada algoritma \ref{alg:smote}.

Program SMOTE dijalankan dengan memberikan input sampel minoritas dari dataset
fitur PAN-WVC-10.
Untuk mendapatkan jumlah sampel positif yang mendekati jumlah sampel negatif,
maka parameter persentase jumlah sampel sintetis diberi dengan nilai 1.100\%,
yang berarti 12 kali jumlah sampel minoritas.
Parameter untuk jumlah tetangga terdekat yang dicari untuk dapat membuat sebuah
sampel diberi dengan nilai 5.
Nilai ini diberikan untuk menghindari sampel minoritas yang \textit{outliers}
tapi juga masih dapat mencakup kelompok-kelompok kelas minoritas yang kecil.
Dari ketiga parameter tersebut didapat hasil sampel ulang dengan jumlah sampel
sintetis positif yaitu 28.728 sampel, digabung dengan sampel positif asli
totalnya yaitu 31.122 sampel positif.

	\begin{center}
	\captionof{algorithm}{SMOTE}
	\label{alg:smote}
	\begin{algorithmic}[1]
\Require \\
$ P $: dataset minoritas \\
$ o $: persentase jumlah sampel sintetis yang akan dibuat \\
$ k $: jumlah tetangga terdekat dari sampel yang akan dibuat
\\
\Function{SMOTE}{$ P, o, k $}
	\State $ S \gets [] $
	\Comment Penampung untuk semua sampel sintetis yang akan dibuat.

	\State $ nsyn \gets 0 $
	\Comment Jumlah sampel sintetis yang akan dibuat.

	\State $ lenP \gets \Call{Length}{P} $

	\If{$o < 100$}
		\label{o_less_100}
		\State $ nsyn \gets (o / 100) \times lenP $
		\State $ P \gets \Call{RandomPickWithoutReplacement}{P, nsyn} $
	\Else
		\State $ nsyn \gets o / 100.0 $
	\EndIf

	\For{\textbf{each} $sample$ \textbf{in} $P$}
		\State $ neighbors \gets \Call{FindNeighbours}{P, sample, k} $
		\Comment Cari k buah tetangga terdekat dari $sample$ di P.

		\State $ s \gets \Call{CreateSynthetic}{sample, neighbors, nsyn} $

		\State $ \Call{S.Push}{s} $
	\EndFor

	\State \Return{S}
\EndFunction
\\
\Function{CreateSynthetic}{$ sample, neighbors, nsyn $}
	\State $ s \gets [] $
	\Comment Penampung untuk semua sampel sintetis yang akan dibuat.

	\State $ lenAttr \gets \Call{LengthAttribute}{sample} $
	\Comment Simpan jumlah atribut dari sampel.

	\For{$ x \gets 1,nsyn $}
		\State $ neighbor \gets \Call{RandomPick}{neighbors} $
		\State $ newsample \gets [] $
		\Comment Penampung untuk sampel sintetis yang baru.

		\For{$ y \gets 1,lenAttr $}
			\If{y \textbf{is} classAttribute}
				\State continue
			\EndIf

			\State $ nattr \gets neighbor.\Call{AttributeAt}{y} $
			\State $ sattr \gets sample.\Call{AttributeAt}{y} $
			\State $ diff \gets sattr - nattr $
			\State $ gap \gets \Call{Random}{0,1} $
			\State $ newAttr \gets nattr + (gap \times diff) $

			\State $ newsample.\Call{PushAttribute}{newAttr} $
		\EndFor

		\State $ \Call{s.Push}{newsample} $
	\EndFor

	\State \Return{s}
\EndFunction

	\end{algorithmic}
\end{center}


Sampel ulang SMOTE memiliki tiga parameter yaitu dataset minoritas yang akan
disampel ulang ($P$), persentase sampel sintetis yang akan
dibuat ($o$), dan jumlah tetangga terdekat yang dicari pada sampel
yang akan dibuat ($k$).
Dataset minoritas yaitu dataset yang berisi hanya sampel dengan kelas
minoritas.
Jika parameter persentase lebih kecil dari 100 (persen) maka hanya $o$ persen
dari dataset $P$ yang disampel ulang.
Parameter jumlah tetangga terdekat ($k$) menentukan jumlah sampel minoritas
lain yang dijadikan sebagai patokan dalam pembuatan sampel sintentis.
Semakin besar nilai $k$ maka distribusi sampel sintetis di dalam dataset juga
semakin luas, semakin kecil nilai $k$ maka antara sampel sintetis
dengan sampel asli semakin mirip/dekat.
Berikut penjelasan fungsi-fungsi tambahan yang ada dalam algoritma.

Fungsi \texttt{RandomPickWithoutReplacement} memiliki dua parameter yaitu
dataset ($P$) dan jumlah sampel yang akan diambil ($n$).  Fungsi ini
mengembalikan $n$ buah sampel dari $P$ yang diambil secara acak tanpa diganti,
artinya setelah sampel diambil sampel tersebut dihapus dari $P$.

Fungsi \texttt{FindNeighbours} memiliki tiga parameter yaitu dataset ($P$),
sebuah sampel yang akan dicari tetangganya ($s$), dan jumlah tetangga yang akan
dicari di sekitar $s$ ($k$).
Fungsi ini mengembalikan $k$ sampel tetangga dari $s$ di dalam dataset $P$.
Jumlah sampel tetangga yang dikembalikan bisa lebih kecil dari nilai $k$.

Fungsi \texttt{CreateSynthetic} memiliki tiga parameter yaitu sebuah sampel
(\textit{sample}), sampel-sampel tetangga dari $sample$ ($neighbors$), dan
jumlah sampel sintetis yang akan dibuat ($nsyn$).
Fungsi ini membuat dan mengembalikan $nsyn$ buah sampel sintetis dari $sample$
yang berdekatan dengan tetangganya ($neighbors$).

Fungsi \texttt{RandomPick} memiliki sebuah parameter yaitu sebuah dataset dan
fungsi ini mengembalikan sebuah sampel yang diambil secara acak dari dataset.

Fungsi \texttt{Random} memiliki dua parameter bilangan yaitu nilai minimum dan
maksimum, fungsi ini mengembalikan bilangan acak real dengan rentang antara
minimum dan maksimum.
