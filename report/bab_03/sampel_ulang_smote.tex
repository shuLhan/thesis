Implementasi SMOTE dibuat ke dalam sebuah program
\footnote{\url{%
https://github.com/shuLhan/go-mining/tree/master/resampling/smote
}}
berdasarkan makalah aslinya \cite{chawla2002smote}.
Gambaran umum dari implementasi dapat dilihat pada algoritma \ref{alg:smote}.

Sampel ulang SMOTE memiliki tiga input yaitu dataset minoritas yang akan
disampel ulang, persentase jumlah sampel sintetis yang akan
dibuat, dan jumlah tetangga terdekat dari sampel yang akan
dibuat.
Untuk mendapatkan jumlah sampel positif yang mendekati jumlah sampel negatif,
maka diberikan nilai 1.200\% untuk persentase jumlah sampel sintetis yang
berarti 12 kali jumlah sampel minoritas.
Jumlah tetangga terdekat yang dicari untuk dapat membuat sebuah sampel diberi
dengan nilai 5.
Nilai ini diberikan untuk menghindari sampel minoritas yang \textit{outliers}
tapi juga masih dapat mencakup kelompok-kelompok kelas minoritas yang kecil.

Program SMOTE dijalankan dengan memberikan input sampel minoritas dari dataset
fitur PAN-WVC-10, sehingga didapat hasil sampel ulang dengan jumlah sampel
sintetis positif yaitu 28.728 sampel, ditambah dengan sampel positif asli
totalnya yaitu 31.122 sampel positif.
