Klasifikasi pohon keputusan yang digunakan dalam penelitian ini yaitu CART
(\textit{Classification and Regression Tree})
\parencite{breiman1984classification}.
Referensi utama yang digunakan pada implementasi berdasarkan pada buku
\textcite{han2011data}.
Hasil implementasi dapat digunakan secara terbuka pada repositori
\texttt{go-mining}%
\footnote{%
	\url{%
	https://github.com/shuLhan/go-mining/tree/master/classifier/rf
}}.
Gambaran umum dari algoritma pohon keputusan CART dapat dilihat pada Algoritma
\ref{alg:cart}.

Klasifikasi CART yaitu pohon keputusan dua kelas (\textit{binary classifier}).
Setiap $node$ pada pohon yang bukan ujung (\textit{leaf}) menentukan fitur
pemisah dari dataset yang didapat dari menghitung probabilitas kelas
berdasarkan nilai \textit{Gini Index} dari semua fitur.
Pohon CART dibangun tanpa adanya pembersihan (\textit{pruning}) yaitu
pemotongan pada $node$ ujung yang memiliki kesamaan kelas dengan $node$ yang
lainnya.

	\begin{center}
	\captionof{algorithm}{CART}
	\label{alg:cart}
	\begin{algorithmic}[1]

\Require \\
$D$: dataset \\
$m$: jumlah fitur acak untuk membangun pohon

\Function{BuildTree}{$ D, m $}
	\State $ majorClass \gets $ ambil kelas mayoritas di $ D $
	\State $ root \gets $ \Call{SplitByGain}{$ D, m, majorClass $}
	\State \Return{$ root $}
\EndFunction
\\
\Function{SplitByGain}{$ D, m, majorClass $}
	\State $ node \gets nil $
	\If{jumlah sampel $ D <= 0 $}
		\State $ node.isLeaf \gets true $
		\State $ node.class \gets majorClass $
		\State \Return{$node$}
	\EndIf

	\If{semua kelas di $D$ sama}
		\State $ node.isLeaf \gets true $
		\State $ node.class \gets $ kelas mayoritas di $ D $
		\State \Return{$node$}
	\EndIf

	\Comment Hitung fitur dengan nilai gain paling tinggi.

	\State $ fiturGains \gets $ \Call{ComputeGain}{$ D, m $}
	\State $ maxGainFitur \gets $ ambil fitur dengan nilai gain paling tinggi
	berdasarkan $fiturGains$
	\State $ maxGainValue \gets $ ambil nilai maksimum pada $fiturGains$
	\State $ maxGainIdx \gets $ ambil indeks pada $maxGainFitur$ dengan
	nilai gain tertinggi

	\If{$ maxGainValue <= 0 $}
		\State $ node.class \gets $ kelas mayoritas di $ D $
		\State \Return{$node$}
	\EndIf

	\Comment Fitur dengan gain tertinggi menjadi pemisah pada $node$

	\State Urutkan $ D $ berdasarkan indeks terurut dari $maxGainFitur.sortedIdx$

	\State $ node.isLeaf \gets false $
	\State $ node.fitur \gets maxGainFitur $
	\State $ node.splitIdx \gets maxGainIdx $
	\State $ node.splitValue \gets maxGainValue $

	\Comment Bagi dua dataset $D$ berdasarkan nilai $maxGainValue$
	\State $ Dleft \gets $ sampel dengan nilai $gain$ kecil sama dengan
	$maxGainValue$
	\State $ Dright \gets $ sampel dengan $gain$ nilai besar dari
	$maxGainValue$

	\State $ node.left \gets $ \Call{SplitByGain}{$ Dleft, m, majorClass $}
	\State $ node.right \gets $ \Call{SplitByGain}{$ Dright, m, majorClass $}

	\State \Return{$node$}
\EndFunction
\\
\Function{ComputeGain}{$ D, m $}
	\State $ gains \gets [] $
	\State $ d \gets $ pilih $m$ fitur acak pada $D$
	\State $ T \gets $ fitur kelas pada $D$
	\For{\textbf{each} fitur \textbf{in} $d$}
		\If{fitur adalah atribut kelas}
			\State continue
		\EndIf

		\State $ fitur.sortedIdx \gets $ urutkan $fitur$ dengan
		\textit{indirect-sort}

		\State Urutkan fitur kelas $T$ dengan $fitur.sortedIdx$

		\State $ parts \gets \Call{CreatePartitition}{fitur} $
		\State $ giniValue \gets \Call{Gini}{fitur} $
		\State $ n \gets fitur_{length} $

		\For{$ x \gets 1,parts_{length} $}
			\State $ probLeft \gets \frac{x}{n} $
			\State $ probRight \gets \frac{n-x}{n} $

			\State $ Tleft \gets T[0:x] $
			\State $ Tright \gets T[x:n] $

			\State $ gainLeft \gets \Call{Gini}{Tleft} $
			\State $ gainRight \gets \Call{Gini}{Tright} $

			\State $ gain \gets giniValue - ((probLeft \times gainLeft) +
			(probRight \times gainRight)) $

			\State Tambahkan $gain$ ke $gains$
		\EndFor
	\EndFor

	\State \Return{$ gains $}
\EndFunction
\\
\Function{CreatePartition}{$fitur$}
	\State $ parts \gets [] $
	\State $ exist \gets false $

	\For{$ x \gets 1,fitur_{length - 1} $}
		\State $ med \gets (fitur[x] + fitur[x+1]) / 2 $

		\If{$ med = 0 $}
			\Comment Lanjutkan jika median antara dua nilai fitur
			adalah 0, karena akan menyebabkan nilai gain maksimum.
			\State $continue$
		\EndIf

		\For{$ y \gets 1,x $}
			\If{$ med = fitur[y] $}
				\Comment Tolak nilai $med$ yang sama dengan
				nilai di fitur.
				\State $ exist \gets true $
				\State break
			\EndIf
		\EndFor

		\If{\textbf{not} exist}
			\State Simpan nilai $med$ ke $parts$
		\EndIf
	\EndFor

	\State \Return{$parts$}
\EndFunction
\\
\Function{Gini}{$ T $}
	\If{$ T_{length} <= 0 $}
		\State \Return{0}
	\EndIf

	\State $ classCounts \gets $ jumlah setiap kelas di $T$
	\State $ sum \gets 0 $

	\For{\textbf{each} count \textbf{in} classCounts}
		\State $ sum \gets sum + (\frac{count}{T_{length}})^2 $
	\EndFor

	\State \Return{$ 1 - sum $}
\EndFunction
\\
\Function{Classify}{$tree, sample$}
	\State $ node \gets tree.root $

	\While{ \textbf{not} node.isLeaf }
		\State $ sampleValue \gets sample[node.splitIdx] $
		\If{$ sampleValue < node.splitValue $}
			\State $ node \gets node.left $
		\Else
			\State $ node \gets node.right $
		\EndIf
	\EndWhile

	\State \Return{$node.class$}
\EndFunction
	\end{algorithmic}
\end{center}


Fungsi \texttt{BuildTree} yaitu fungsi utama untuk membuat pohon keputusan CART.
Fungsi ini memiliki tiga parameter yaitu dataset yang akan dibuat
menjadi pohon keputusan ($D$), daftar kelas pada dataset $D$ ($C$), dan jumlah
fitur yang diambil secara acak untuk membangun pohon ($m$).
Fungsi ini mengembalikan akar ($root$) dari pohon keputusan.

Fungsi \texttt{SplitByGain} mencari fitur yang akan menjadi pemisah dari dataset
$D$.
Jika dataset $D$ kosong maka fungsi ini mengembalikan $node$ ujung dengan kelas
berisi kelas mayoritas pada $D$.
Jika semua sampel dari dataset $D$ hanya terdiri dari satu kelas, fungsi ini
akan mengembalikan $node$ dengan kelas tunggal tersebut.
Fitur pemisah ditentukan dengan menghitung \textit{gain} pada setiap fitur.
Fitur dengan nilai $gain$ paling tinggi kemudian memisahkan dataset $D$ menjadi
dua berdasarkan nilai $gain$.
Hasil pemisahan dataset tersebut kemudian dihitung kembali $gain$-nya
masing-masing sampai dataset kosong atau terdiri dari satu kelas saja.

Fungsi \texttt{ComputeGain} menghitung nilai $gain$ dari setiap fitur.
Fungsi ini memiliki dua parameter yaitu dataset yang akan dihitung setiap
$gain$ pada fiturnya ($D$), dan jumlah fitur yang diambil secara acak untuk
dihitung $m$.
Fungsi ini mengembalikan nilai $gain$ dari setiap fitur pada $D$.

Fungsi \texttt{CreatePartition} membuat partisi dari setiap nilai pada sebuah
fitur.
Fungsi ini memiliki satu parameter yaitu fitur yang akan dipartisi.
Partisi didapat dengan menghitung \textit{median} antara dua nilai dalam fitur
secara berkelanjutan dan mengembalikan semua nilai tersebut dalam bentuk
laring.

Fungsi \texttt{Gini} menghitung \textit{Gini index} dari sebuah dataset $T$.
\textit{Gini index} mengukur \textit{impurity}, nilai yang menentukan bobot
kelas pada sebuah fitur terhadap semua dataset, dengan menghitung jumlah
probabilitas dari setiap kelas pada $T$ dengan rentang dari 0 sampai 1.

Fungsi \texttt{Classify} mengembalikan label kelas dari sebuah $sample$ dengan
menelusuri setiap $node$ pada pohon dari atas sampai node paling ujung sesuai
dengan nilai pemisah fitur pada setiap $node$.
