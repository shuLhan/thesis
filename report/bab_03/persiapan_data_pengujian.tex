Dataset yang digunakan untuk pengujian yaitu \textit{PAN Wikipedia Vandalism
Corpus 2011} (PAN-WVC-11)
\parencite{potthast:2010b}
yang dapat diambil dari situs Universitas Universitas Bauhaus Weimar
\footnote{%
	\RaggedRight\url{%
http://www.uni-weimar.de/en/media/chairs/webis/corpora/corpus-pan-wvc-11/
	}
}
Korpus ini terdiri dari 9.985 suntingan bahasa Inggris, 9.990 suntingan bahasa
Jerman, dan 9.974 suntingan bahasa Spanyol.
Korpus yang digunakan untuk pengujian hanya yang bahasa Inggris.

Dataset PAN-WVC-11 hanya satu berkas, tidak terpisah seperti pada PAN-WVC-10,
dengan atribut yang sama dengan PAN-WVC-10 yaitu
\textit{editid},
\textit{editor},
\textit{oldrevisionid},
\textit{newrevisionid},
\textit{diffurl},
\textit{class},
\textit{annotators},
\textit{totalannotators},
\textit{edittime},
\textit{editcomment},
\textit{articleid}, dan
\textit{articletitle}.
Atribut \textit{annotators} dan \textit{totalannotators} dihilangkan dan
ditambah dengan dua atribut baru yaitu \textit{deletions} yang berisi teks yang
dihapus pada revisi lama, dan \textit{additions} yang berisi teks yang
ditambahkan pada revisi yang baru.
Nilai dari atribut \textit{class} diganti dari string menjadi integer, yaitu
dari "vandalism" menjadi bilangan \texttt{1}
dan "regular" menjadi bilangan \texttt{0}.
Contoh data mentah untuk PAN-WVC-11 dapat dilihat pada Lampiran
\ref{lampiran:dataset_wvc11_mentah}.

Untuk proses pembersihan data dan penambahan atribut dilakukan prosedur yang
sama seperti pada PAN-WVC-10, dengan contoh hasil dapat dilihat pada Lampiran
\ref{lampiran:dataset_wvc11_gabungan}.
