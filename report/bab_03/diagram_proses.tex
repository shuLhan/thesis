\begin{figure}[htbp]
\centering
\resizebox{.5\textwidth} {!} {%
\mytikzinput{diagramproses}
\begin{tikzpicture}[
	framed,
	nodes = {
		align = center
	},
	lines/.style={
		line width=2pt,
		>=latex
	},
	data/.style={
		circle,
		draw=black,
		text centered
	},
	proc/.style={
		rectangle,
		draw=black,
		text centered
	}
]
	\node[data] (training_raw) {
		Dataset pelatihan\\
		mentah\\
		(\textit{PAN-WVC-10})
	};

	\node[proc] (wvcgen) [below=of training_raw] {
		Ekstraksi\\fitur
	};

	\node[proc] (trainingset) [below=of wvcgen] {
		Sampel ulang
	};

	\node[proc] (c)  [below=of trainingset] {
		Pembangkitan model\\
		deteksi
	};

	\node[proc] (m)  [below=of c] {
		Model deteksi\\
		vandalisme
	};

	\node[data] (o)  [below=of m] {
		Hasil deteksi
	};
	%%
	\node[data] (testset_raw) [right=of training_raw] {
		Dataset pengujian\\mentah\\
		(\textit{PAN-WVC-11})
	};

	\node[proc] (testset_wvcgen) [below=of testset_raw] {
		Ekstraksi\\fitur
	};

	\node[data] (testset) [below=of testset_wvcgen] {
		Dataset\\
		pengujian
	};

	\draw[lines,->] (training_raw) -- (wvcgen);
	\draw[lines,->] (wvcgen) -- (trainingset);
	\draw[lines,->] (trainingset) -- (c);
	\draw[lines,->] (c) -- (m);
	\draw[lines,->] (m) -- (o);

	\draw[lines,->] (testset_raw) -- (testset_wvcgen);
	\draw[lines,->] (testset_wvcgen) -- (testset);

	\draw[lines,->] (testset) |- (m);
\end{tikzpicture}
}
\caption{
	Proses implementasi deteksi vandalisme
}
\label{fig:diagramproses}
\end{figure}
