\begin{figure}[htbp]
\centering
\resizebox {\columnwidth} {!} {
\begin{tikzpicture}[
	framed,
	nodes = {
		align = center
	},
	lines/.style={
		line width=2pt,
		>=latex
	},
	data/.style={
		trapezium,
		trapezium left angle=70,
		trapezium right angle=110,
		draw=black,
		text centered
	},
	proc/.style={
		rectangle,
		draw=black,
		text centered
	}
]
	\node[data] (wvc10_raw) {
		Dataset pelatihan mentah\\
		(\textit{PAN-WVC-10})
	};
	\node[proc] (wvcgen_10raw) [below=of wvc10_raw] {
		Pembuatan\\fitur
	};

	\node[data] (wvc10_fs) [below=of wvcgen_10raw] {
		Fitur set\\
		tanpa sampel ulang
	};
	\node[proc] (smote)         [below right=of wvc10_fs, xshift=1cm] {
		Sampel ulang\\SMOTE
	};
	\node[proc] (lnsmote)       [right=of smote] {
		Sampel ulang\\LNSMOTE
	};
	\node[data] (wvc10_smote)   [below=of smote] {Fitur set\\SMOTE};
	\node[data] (wvc10_lnsmote) [below=of lnsmote] {Fitur set\\LNSMOTE};

	\node[proc] (c2)  [below=of wvc10_smote] {Pengklasifikasi};
	\node[proc] (c3)  [below=of wvc10_lnsmote] {Pengklasifikasi};
	\node[proc] (c1)  [left=of  c2] {Pengklasifikasi};

	\node[proc] (m1)  [below=of c1] {Model};
	\node[proc] (m2)  [below=of c2] {Model};
	\node[proc] (m3)  [below=of c3] {Model};

	\node[data] (o1)  [below=of m1] {Hasil tanpa\\sampel ulang};
	\node[data] (o2)  [below=of m2] {Hasil\\SMOTE};
	\node[data] (o3)  [below=of m3] {Hasil\\LNSMOTE};

	\node[data] (testset) [right=of c3,xshift=1cm] {Dataset pengujian};
	\node[proc] (wvcgen_11raw) [above=of testset] {
		Pembuatan\\fitur
	};
	\node[data] (wvc11_raw) [above=of wvcgen_11raw] {
		Dataset pengujian mentah\\
		(\textit{PAN-WVC-11})
	};

	\draw[lines,->] (wvc10_raw) -- (wvcgen_10raw);
	\draw[lines,->] (wvcgen_10raw) -- (wvc10_fs);
	\draw[lines,->] (wvc10_fs) -| (smote);
	\draw[lines,->] (wvc10_fs) -| (lnsmote);
	\draw[lines,->] (smote) -- (wvc10_smote);
	\draw[lines,->] (lnsmote) -- (wvc10_lnsmote);
	\draw[lines,->] (wvc10_fs) -- (c1);
	\draw[lines,->] (wvc10_smote) -- (c2);
	\draw[lines,->] (wvc10_lnsmote) -- (c3);
	\draw[lines,->] (c1) -- (m1);
	\draw[lines,->] (c2) -- (m2);
	\draw[lines,->] (c3) -- (m3);

	\draw[lines,->] (wvc11_raw) -- (wvcgen_11raw);
	\draw[lines,->] (wvcgen_11raw) -- (testset);

	\draw[lines,dashed,->] (testset) -- (m1);
	\draw[lines,dashed,->] (testset) -- (m2);
	\draw[lines,dashed,->] (testset) -- (m3);

	\draw[lines,->] (m1) -- (o1);
	\draw[lines,->] (m2) -- (o2);
	\draw[lines,->] (m3) -- (o3);
\end{tikzpicture}
}
\caption{
	Proses implementasi deteksi vandalisme
}
\label{fig:proses}
\end{figure}
