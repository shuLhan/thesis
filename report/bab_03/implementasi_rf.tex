Implementasi RF berdasarkan pada makalah Breiman \cite{breiman2001random}
dan beberapa referensi tambahan dari sejumlah sumber lainnya
(situs, makalah, dan video).
Hasil dari implementasi dapat dilihat dan digunakan secara terbuka pada
repositori \texttt{go-mining}
\footnote{
\url{%
https://github.com/shuLhan/go-mining/tree/master/classifiers/rf
}}.
Gambaran umum dari pengklasifikasi RF dapat dilihat pada algoritma
\ref{alg:rf}.

RF memiliki empat parameter yaitu dataset pelatihan, jumlah pohon dalam
\textit{forest}, jumlah sampel acak yang digunakan untuk membangun setiap
pohon, dan jumlah fitur acak yang digunakan untuk membangun setiap pohon.
Untuk dataset pelatihan yang digunakan terdiri dari tiga jenis yaitu
dataset fitur tanpa sampel ulang, dataset yang telah disampel ulang dengan
SMOTE, dan dataset yang telah disampel ulang dengan LNSMOTE.
Untuk setiap dataset, jumlah pohon yang digunakan yaitu 200 pohon, dengan
parameter sampel acak yaitu 66\% dari total sampel pada dataset, dan 5 untuk
jumlah fitur acak.
