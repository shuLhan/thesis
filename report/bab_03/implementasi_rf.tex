Implementasi RF berdasarkan pada makalah Breiman \cite{breiman2001random}
dan beberapa referensi dari sejumlah sumber lainnya.
Hasil dari implementasi dapat dilihat dan digunakan secara terbuka pada
repositori \texttt{go-mining}
\footnote{
\url{%
https://github.com/shuLhan/go-mining/tree/master/classifiers/rf
}}.
Gambaran umum dari pengklasifikasi RF dapat dilihat pada algoritma
\ref{alg:rf}.

	\newpage
	\begin{center}
\captionof{algorithm}{Random Forest}
\label{alg:rf}
	\begin{algorithmic}[1]
\Require \\
$ TSET $: dataset pelatihan \\
$ T $: jumlah pohon untuk \textit{forest} \\
$ B $: persentase sampel yang dipilih secara acak tanpa diganti dari $TSET$ \\
$ m $: jumlah fitur acak \\

\Function{RandomForest}{$ TSET, T, B, m $}
	\State $ n \gets $ jumlah sampel pada dataset $ TSET $
	\State $ b \gets n * (B / 100) $
	\Comment Jumlah sampel untuk setiap pohon.

	\State $ forest \gets nil $
	\For{$ t \gets 1,T $}
		\State $ tp, tn, tree \gets \Call{GrowTree}{forest, TSET, b, m} $
		\State $ \Call{forest.Add}{tree} $
	\EndFor
	
	\State \Return $forest$
\EndFunction
\\
\Function{GrowTree}{$ forest, TSET, b, m $}
	\label{bagging}
	\State $ bag, oob, bagIdx, oobIdx \gets \Call{RandomPickSample}{$ TSET,
	b $} $

	\State $ tree \gets \Call{BuildTree}{bag, m} $

	\State $ \Call{forest.SaveBagIndex}{bagIdx} $
	\Comment Simpan pohon dan indeks dari \textit{bagging}

	\State $ cm \gets \Call{Classify}{forest, oob, oobIdx} $
	\State $ tp, tn \gets \Call{ComputeStat}{cm} $
	\Comment Hitung laju galat OOB dari \textit{confusion matrix}
	$cm$.

	\State \Return{$ tp, tn, tree $}
\EndFunction
\\
\Function{Classify}{$ forest, set, setIdx $}
	\State $ predictions \gets nil $
	\For{\textbf{each} $ sample, idx $ \textbf{in} $ set $}
		\State $ votes \gets \Call{ForestVotes}{forest, sample, idx} $
		\State $ class \gets \Call{Majority}{votes} $
		\Comment ambil mayoritas suara kelas pada $votes$
		\State $ predictions.push(class) $
	\EndFor

	\State \Return $predictions$
\EndFunction
\\
\Function{ForestVotes}{$ forest, sample, idx $}
	\State $ votes \gets nil $
	\For{\textbf{each} $ tree $ \textbf{in} $forest$ }
		\State $ exist \gets \Call{IsSampleInTheBag}{idx} $
		\Comment Periksa apakah sampel indeks digunakan pada pohon ini.

		\If{exist}
			\State continue
			\Comment Jika digunakan, lanjutkan ke pohon berikutnya.
		\EndIf

		\State $ class \gets \Call{tree.Classify}{sample} $

		\State $ \Call{votes.push}{class} $
	\EndFor

	\State \Return $ votes $
\EndFunction
	\end{algorithmic}
\end{center}

	\newpage

Fungsi \texttt{RandomForest} memiliki empat parameter yaitu dataset pelatihan
($TSET$),
jumlah pohon dalam \textit{forest} ($T$), jumlah sampel acak yang digunakan untuk
membangun setiap pohon ($B$), dan jumlah fitur acak yang digunakan untuk membangun
setiap pohon ($m$).
Fungsi ini mengembalikan sebuah \textit{forest} yang berisi kumpulan
pohon-pohon keputusan dengan jumlah $T$.

Fungsi \texttt{GrowTree} memiliki empat parameter yaitu sebuah $forest$, sebuah
dataset $TSET$, jumlah sampel acak untuk pembangunan pohon $b$, dan jumlah
fitur acak $m$.
Fungsi ini mengambil $b$ buah sampel acak dari $TSET$ ($bag$) dan menyimpan
indeks dari sampel yang terambil ($bagIdx$) untuk proses penghitungan galat
OOB, nantinya.
Fungsi ini mengembalikan sebuah pohon keputusan $tree$.

Fungsi \texttt{Classify} memiliki tiga parameter yaitu sebuah pengklasifikasi
RandomForest ($forest$), dataset yang akan diklasifikasi ($set$), dan indeks
dari setiap sampel dalam $set$ ($setIdx$, opsional).
Fungsi ini memberi label kelas pada setiap sampel dalam $set$ dengan
mengumpulkan hasil keputusan dari setiap pohon yang ada di dalam $forest$.

Fungsi \texttt{ForestVotes} memiliki tiga parameter yaitu sebuah
pengklasifikasi RandomForest ($forest$), sebuah sampel yang akan beri label
kelasnya (\textit{sample}), dan nilai index dari sampel tersebut ($idx$).
Fungsi ini memanggil fungsi klasifikasi pada setiap pohon di dalam $forest$ dan
mengumpulkan hasil klasifikasi ($class$) dari semua pohon tersebut untuk
kemudian dihitung mayoritas hasil klasifikasinya.
Sebelum pemanggilkan fungsi klasifikasi pada pohon $t$, diperiksa terlebih
dahulu apakah $sample$ tersebut digunakan untuk membangun pohon $t$, jika benar
maka fungsi klasifikasi tidak dilewati dan dilanjutkan dengan pohon
selanjutnya.
