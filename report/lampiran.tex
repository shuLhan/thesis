\vspace*{\fill}
\begin{center}
	\begin{minipage}{\textwidth}
		\centering
		\textbf{\Large\tUpLampiran}
		\addcontentsline{toc}{chapter}{\tUpLampiran}
	\end{minipage}
\end{center}
\vfill

\myappendix{Daftar Kategori Kata dan Token}
	\label{lampiran:daftar_token_dan_kata}

Lampiran ini berisi daftar kategori kata dan token yang digunakan untuk
penghitungan fitur vandalisme.

%%{{{ TOKEN WIKI
\subappendix{Token Wiki}
	\label{lampiran:words_wiki_token}

Kategori sintaks wiki digunakan dalam penghitungan fitur \textit{token umum}.
Berikut daftar token yang digunakan,

	\lstinputlisting[style=data,basicstyle=\scriptsize\ttfamily]
		{lampiran/token_wiki}
%%}}}

%%{{{ KATA VULGAR
\subappendix{Kategori Kata Vulgar}
\label{lampiran:words_vulgar}

Kategori kata vulgar yaitu kata yang kasar dan menghina.
Berikut daftar kata vulgar yang digunakan,

	\lstinputlisting[style=data,basicstyle=\scriptsize\ttfamily]
		{lampiran/kata_vulgar}
%%}}}

%%{{{ KATA SUBJEK
\subappendix{Kategori Kata Subjek}
\label{lampiran:words_pronoun}

Kategori kata subjek yaitu kata yang merujuk pada pihak pertama dan kedua yang
digunakan dalam percakapan sehari-hari.
Berikut daftar kata subjek yang digunakan,

	\lstinputlisting[style=data,basicstyle=\scriptsize\ttfamily]
		{lampiran/kata_subjek}
%%}}}

%%{{{ KATA BIAS
\subappendix{Kategori Kata Bias}
\label{lampiran:words_bias}

Kategori kata bias berisi kata yang mengesankan penekanan yang berlebihan
sehingga cenderung membuat bias.
Berikut daftar kata bias yang digunakan,

	\lstinputlisting[style=data,basicstyle=\scriptsize\ttfamily]
		{lampiran/kata_bias}
%%}}}

%%{{{ KATA PORNOGRAFI
\subappendix{Kategori Kata Pornografi}
\label{lampiran:words_sex}

Berikut daftar kata pornografi yang digunakan,

	\lstinputlisting[style=data,basicstyle=\scriptsize\ttfamily]
		{lampiran/kata_pornografi}
%%}}}

%%{{{ KATA BURUK
\subappendix{Kategori Kata Buruk}
\label{lampiran:words_bad}

Kategori kata buruk mengikutkan kata-kata yang bersifat negatif, bukan vulgar
dan bukan pornografi, atau kata slang yang digunakan dalam sehari-hari.
Berikut daftar kata buruk yang digunakan,

	\lstinputlisting[style=data,basicstyle=\scriptsize\ttfamily]
		{lampiran/kata_buruk}

%%}}}


\myappendix{Contoh Dataset}
	\label{lampiran:dataset}

Lampiran ini berisi contoh dataset yang digunakan pada pelatihan dan pengujian
model klasifikasi.

%%{{{ WVC-10 SUNTINGAN
\subappendix{Dataset Suntingan PAN-WVC-10}
	\label{lampiran:dataset_wvc10_suntingan}

Berikut contoh dataset suntingan pada PAN-WVC-10,

\lstinputlisting[style=data,basicstyle=\scriptsize\ttfamily]
	{lampiran/dataset_wvc10_suntingan}
%%}}}

%%{{{ WVC-10 ANOTASI
\subappendix{Dataset Anotasi PAN-WVC-10}
	\label{lampiran:dataset_wvc10_anotasi}

Berikut contoh dataset anotasi pada PAN-WVC-10,

\lstinputlisting[style=data,basicstyle=\scriptsize\ttfamily]
	{lampiran/dataset_wvc10_anotasi}
%%}}}

%%{{{ WVC-10 GABUNGAN
\subappendix{Dataset Gabungan PAN-WVC-10}
	\label{lampiran:dataset_wvc10_gabungan}

Berikut contoh dataset PAN-WVC-10 yang telah dibersihkan dan ditambahkan dengan
atribut \textit{additions} dan \textit{deletions},

\lstinputlisting[
	style=data
,	basicstyle=\scriptsize\ttfamily
,	texcl=true
]{lampiran/dataset_wvc10_gabungan}
%%}}}

%%{{{ WVC-11 MENTAH
\subappendix{Dataset Mentah PAN-WVC-11}
	\label{lampiran:dataset_wvc11_mentah}

Berikut contoh dataset mentah dari PAN-WVC-11 Bahasa Inggris,

\lstinputlisting[
	style=data
,	basicstyle=\scriptsize\ttfamily
,	texcl=true
]{lampiran/dataset_wvc11_mentah}
%%}}}

%%{{{ WVC-11 GABUNGAN
\subappendix{Dataset Gabungan PAN-WVC-11}
	\label{lampiran:dataset_wvc11_gabungan}

Berikut contoh dataset PAN-WVC-11 yang telah dibersihkan dan ditambahkan
atribut \textit{additions} dan \textit{deletions},

\lstinputlisting[
	style=data
,	basicstyle=\scriptsize\ttfamily
,	texcl=true
,	alsoletter={\\}
,	morekeywords={\"}
]{lampiran/dataset_wvc11_gabungan}
%%}}}
