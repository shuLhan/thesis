\label{fig:rocpoints}
\begin{figure}[htbp]
\centering
\begin{tikzpicture}[framed]
	\pgfplotsset{
		width=12cm,
		xtick distance=0.1,
		ytick distance=0.1,
	}
	\begin{axis}[
		title=Performansi pengklasifikasi pada ruang ROC,
		xlabel=$FPR$,
		ylabel=$TPR$,
		legend columns=3,
		legend entries={
			RF,
			RF-SMOTE,
			RF-LNSMOTE,
			CRF-200-1,
			CRF-200-1-SMOTE,
			CRF-200-1-LNSMOTE,
			CRF-100-2,
			CRF-100-2-SMOTE,
			CRF-100-2-LNSMOTE,
			CRF-50-4,
			CRF-50-4-SMOTE,
			CRF-50-4-LNSMOTE,
		},
		legend to name=class_legend
	]
		\addplot[
			scatter,
			only marks,
			point meta=explicit symbolic,
			every mark/.append style={
				/tikz/mark size=3pt
			},
			scatter/classes={
				rf-noresampling={
					mark=text,
					text mark=$\Box$
				},
				rf-smote={
					mark=text,
					text mark=$\boxplus$
				},
				rf-lnsmote={
					mark=text,
					text mark=$\boxtimes$
				},
				crf-200-1={
					mark=text,
					text mark=$\triangle$
				},
				crf-200-1-smote={
					mark=text,
					text mark=$\triangleleft$
				},
				crf-200-1-lnsmote={
					mark=text,
					text mark=$\triangleright$
				},
				crf-100-2={
					mark=text,
					text mark=$\medcircle$
				},
				crf-100-2-smote={
					mark=text,
					text mark=$\oplus$
				},
				crf-100-2-lnsmote={
					mark=text,
					text mark=$\otimes$
				},
				crf-50-4={
					mark=text,
					text mark=$\Diamond$
				},
				crf-50-4-smote={
					mark=text,
					text mark=$\diamondplus$
				},
				crf-50-4-lnsmote={
					mark=text,
					text mark=$\diamondtimes$
				}
			},
		]
		table[
			x={FPR},
			y={TPR},
			meta=klasifikasi,
		]
		{../result/cm.csv};

		\addplot[
			gray
		]
		coordinates{
			(0,0)
			(1,1)
		};
		\addplot[
			gray
		]
		coordinates{
			(0,1)
			(1,0)
		};
	\end{axis}
\end{tikzpicture}
\newline
\ref{class_legend}
\newline
\caption{
Titik ROC untuk semua pengklasifikasi dan dataset.
Untuk informasi tambahan terhadap legenda, CRF-200-1 yaitu klasifikasi CRF
dengan 200 tingkat dan 1 pohon, CRF-100-2 yaitu klasifikasi CRF dengan 100
tingkat dan 2 pohon, dan CRF-50-4 yaitu klasifikasi CRF dengan 50 tingkat dan 4
pohon.
}
\end{figure}
