Hasil pengujian diberikan dalam bentuk performansi pengklasifikasi pada tabel
\ref{tab:stats} dan kecepatan pelatihan model pada tabel
\ref{tab:runtimes}.

\DTLsetseparator{;}
\DTLloaddb{stats}{../result/stats.csv}
\DTLmaxforcolumn{stats}{TPR}{\maxtpr}
\DTLminforcolumn{stats}{FPR}{\minfpr}
\DTLmaxforcolumn{stats}{TNR}{\maxtnr}
\DTLmaxforcolumn{stats}{Presisi}{\maxprec}
\DTLmaxforcolumn{stats}{F-Measure}{\maxfm}
\DTLmaxforcolumn{stats}{Akurasi}{\maxacc}
\DTLmaxforcolumn{stats}{AUC}{\maxauc}

\begin{table}[htbp]
\caption{Performansi Klasifikasi RF dan CRF}
\centering
\footnotesize
\begin{tabular}{p{2cm} p{2cm} rrrrrrr}
\hline
\textbf{Klasifikasi} &
\textbf{Dataset} &
\textbf{TPR} &
\textbf{FPR} &
\textbf{TNR} &
\textbf{Pres.} &
\textbf{F-Mea.} &
\textbf{Aku.} &
\textbf{AUC}
\DTLforeach*{stats}{%
	\cl=Klasifikasi,%
	\ds=Dataset,%
	\tpr=TPR,%
	\fpr=FPR,%
	\tnr=TNR,%
	\prec=Presisi,%
	\fm=F-Measure,%
	\acc=Akurasi,%
	\auc=AUC%
}{%
	\DTLifnullorempty{\cl}
		{\\ \cline{2-9}}
		{\\ \hline \hline}
	\DTLifnullorempty{\cl}
		{}
		{
			\multirow{4}{2cm}{\cl}
		}
	& \ds
	& \DTLifnumeq{\tpr}{\maxtpr}{\textbf{\tpr}}{\tpr}
	& \DTLifnumeq{\fpr}{\minfpr}{\textbf{\fpr}}{\fpr}
	& \DTLifnumeq{\tnr}{\maxtnr}{\textbf{\tnr}}{\tnr}
	& \DTLifnumeq{\prec}{\maxprec}{\textbf{\prec}}{\prec}
	& \DTLifnumeq{\fm}{\maxfm}{\textbf{\fm}}{\fm}
	& \DTLifnumeq{\acc}{\maxacc}{\textbf{\acc}}{\acc}
	& \DTLifnumeq{\auc}{\maxauc}{\textbf{\auc}}{\auc}
}
\\
\hline
\end{tabular}
\label{tab:stats}
\end{table}

\DTLloaddb{runtimes}{../result/runtimes.csv}

\begin{table}[htbp]
\caption{Kecepatan pelatihan model}
\centering
\footnotesize
\begin{tabular}{p{4cm} p{4cm} r}
\hline
\textbf{Klasifikasi} &
\textbf{Dataset} &
\textbf{Waktu (menit)}
\DTLforeach*{runtimes}{%
		\cl=Klasifikasi,
		\ds=Dataset,
		\time=Waktu (menit)%
}{%
	\DTLifnullorempty{\cl}
		{\\ \cline{2-3}}
		{\\ \hline \hline}
	\DTLifnullorempty{\cl}
		{}
		{
			\multirow{3}{4cm}{\cl}
		}
	& \ds
	& \time
}
\\
\hline
\end{tabular}
\label{tab:runtimes}
\end{table}

Klasifikasi CRF LNSMOTE dengan 200 tingkat 1 pohon memberikan nilai TPR paling
tinggi yaitu $0,9904$ tapi dengan nilai FPR yang paling tinggi yaitu $0,8558$
dan TNR yang paling rendah yaitu $0,1442$ di antara model yang lainnya.
Kebalikannya, pengklasifikasi RF tanpa sampel ulang memberikan nilai TNR paling
tinggi yaitu $0,9988$ dan nilai FPR paling rendah yaitu $0,0012$.
Untuk presisi, RF tanpa sampel ulang memberikan nilai tertinggi yaitu $0,9450$,
dan nilai terendah diberikan oleh klasifikasi CRF LNSMOTE dengan 200 tingkat 1
pohon.
Untuk nilai \textit{F-Measure}, nilai tertinggi diberikan oleh klasifikasi CRF
tanpa sampel ulang dengan 50 tingkat dan 4 pohon yaitu $0,5353$, dengan nilai
terendah diberikan oleh klasifikasi CRF LNSMOTE 200 tingkat 1 pohon.
Untuk nilai akurasi tertinggi didapat dengan klasifikasi RF LNSMOTE dengan
yang terendah diberikan oleh klasifikasi CRF LNSMOTE 200 tingkat 1 pohon.
Klasifikasi dengan nilai AUC tertinggi yaitu CRF SMOTE 100 tingkat 2 pohon
dengan yang terendah diberikan oleh klasifikasi CRF tanpa sampel ulang dengan
100 tingkat 2 pohon.

Dari segi kecepatan pelatihan model, pengklasifikasi CRF lebih cepat dari RF
baik pada semua dataset pelatihan.
Sebagai pembanding, dapat dilihat pada klasifikasi RF dan CRF 50 tingkat 4
pohon.
CRF tanpa sampel ulang lebih cepat 11 kali daripada RF, dan pada sampel ulang
SMOTE dan LNSMOTE, klasifikasi CRF 1,6 kali lebih cepat daripada RF.
