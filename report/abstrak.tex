\begin{center}
\textbf{\large
	\MakeUppercase{\judul{}} \\
	\bigskip
	\textnormal{Oleh} \\
	\myname{} \\
	NIM: \mysid{} \\
	(\mydept{}) \\
}
\end{center}

\bigskip
\bigskip
\bigskip

Wikipedia.org adalah ensiklopedia daring yang dapat disunting oleh siapa saja.
Sifat wiki ini dapat mempercepat perbaikan dan pertumbuhan artikel,
kekurangannya yaitu dapat menimbulkan vandalisme dalam bentuk suntingan dengan
informasi yang salah, penghapusan, iklan, atau teks yang tidak bermakna.
Tesis ini membahas deteksi vandalisme menggunakan pembelajaran mesin
dengan menerapkan metode klasifikasi
\textit{Cascaded Random Forest} (CRF)
yang dilatih pada dataset
\textit{Wikipedia Vandalism Corpus 2010}
yang telah disampel ulang dengan
\textit{Local Neighbourhood Synthetic Minority Over-sampling Technique}
(LNSMOTE).
Kedua metode tersebut dibandingkan dengan klasifikasi
\textit{Random Forest} (RF)
dan teknik sampel ulang
\textit{Synthetic Minority Oversampling Technique} (SMOTE).
Hasil sampel ulang dan pemodelan diuji dengan dataset
\textit{Wikipedia Vandalism Corpus 2011}
menunjukan hasil sampel ulang dengan LNSMOTE meningkatkan laju
\textit{true-positive} lebih tinggi daripada dataset yang tidak disampel ulang
dan yang disampel ulang dengan SMOTE pada kedua pengklasifikasi.
Dari segi performansi, CRF dengan sampel ulang LNSMOTE dengan 200 tingkat dan 1
pohon memberikan hasil yang paling baik dengan nilai laju
\textit{true-positive} yaitu $0,9904$.
Dari segi kecepatan pemodelan pengklasifikasi CRF lebih cepat 1,6 kali daripada
RF pada data yang telah disampel ulang.

Kata kunci: wikipedia, vandalisme, dataset timpang, dataset sampel ulang,
\textit{cascaded random forest}
