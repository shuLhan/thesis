\begin{center}
\textbf{\large
	ABSTRAK \\
	\bigskip
	\MakeUppercase{\mytitle{}} \\
	\textnormal{\center Oleh} \\
	\myname{} \\
	NIM: \mysid{} \\
	(\mydept{}) \\
}
\end{center}

\bigskip
\bigskip
\bigskip

Wikipedia.org adalah ensiklopedia daring yang dapat disunting oleh siapa saja.
Sifat wiki ini dapat mempercepat perbaikan dan pertumbuhan artikel,
kekurangannya yaitu dapat menimbulkan vandalisme dalam bentuk suntingan dengan
informasi yang salah, penghapusan, iklan, atau teks yang tidak bermakna.
Tesis ini membahas deteksi vandalisme menggunakan pembelajaran mesin
dengan menerapkan metode klasifikasi
\textit{Cascaded Random Forest} (CRF)
yang dilatih pada dataset
\textit{Wikipedia Vandalism Corpus 2010} (WVC2010)
yang
disampel ulang dengan
\textit{Local Neighbourhood Synthetic Minority Over-sampling Technique}
(LN-SMOTE).
Kedua metode tersebut dibandingkan dengan klasifikasi
\textit{Random Forest} (RF)
dan teknik sampel ulang SMOTE.
Hasil sampel ulang dan pemodelan diuji dengan dataset
\textit{Wikipedia Vandalism Corpus 2011} (WVC2011)
menunjukan hasil sampel ulang dengan LN-SMOTE meningkatkan laju
\textit{true-positive} daripada SMOTE pada pengklasifikasi RF dan CRF, namun
pada CRF, LN-SMOTE meningkatkan laju \textit{false-positive} lebih tinggi
daripada SMOTE.
Sementara pada pengklasifikasi RF laju \textit{false-positive} dari hasil
sampel ulang LN-SMOTE tidak begitu signifikan dengan SMOTE.
Dari segi performansi, CRF dengan sampel ulang SMOTE dengan 100 tingkat dan 2
pohon memberikan hasil yang paling baik dengan nilai AUC yaitu $0,8694$.
Dari segi kecepatan pemodelan pengklasifikasi CRF lebih cepat 1,6 kali daripada
RF pada data yang telah disampel ulang.


Kata kunci: wikipedia, vandalisme, dataset timpang, dataset sampel ulang,
\textit{cascaded random forest}
