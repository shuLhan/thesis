\label{fig:roc}
\begin{figure}[htbp]
\centering
\begin{tikzpicture}[framed]
	\pgfplotsset{
		small,
	}
	\matrix{
		\begin{axis}[
			title=(a),
			ylabel=$TPR$,
			legend columns=-1,
			legend entries={Tanpa sampel ulang\ ,SMOTE\ ,LNSMOTE},
			legend to name=roc_legend
		]
			\addplot table[
				x={FPR},
				y={TPR},
			]
			{../result/rf.csv};
			\addplot table[
				x={FPR},
				y={TPR},
			]
			{../result/rf_smote.csv};
			\addplot table[
				x={FPR},
				y={TPR},
			]
			{../result/rf_lnsmote.csv};
		\end{axis}
		&
		\begin{axis}[
			title=(b),
		]
			\addplot table[
				x={FPR},
				y={TPR},
			]
			{../result/crf_200_1.csv};
			\addplot table[
				x={FPR},
				y={TPR},
			]
			{../result/crf_200_1_smote.csv};
			\addplot table[
				x={FPR},
				y={TPR},
			]
			{../result/crf_200_1_lnsmote.csv};
			title=(b),
		\end{axis}
		\\
		\begin{axis}[
			title=(c),
			xlabel=$FPR$,
			ylabel=$TPR$,
		]
			\addplot table[
				x={FPR},
				y={TPR},
			]
			{../result/crf_100_2.csv};
			\addplot table[
				x={FPR},
				y={TPR},
			]
			{../result/crf_100_2_smote.csv};
			\addplot table[
				x={FPR},
				y={TPR},
			]
			{../result/crf_100_2_lnsmote.csv};
		\end{axis}
		&
		\begin{axis}[
			title=(d),
			xlabel=$FPR$,
		]
			\addplot table[
				x={FPR},
				y={TPR},
			]
			{../result/crf_50_4.csv};
			\addplot table[
				x={FPR},
				y={TPR},
			]
			{../result/crf_50_4_smote.csv};
			\addplot table[
				x={FPR},
				y={TPR},
			]
			{../result/crf_50_4_lnsmote.csv};
		\end{axis}
		\\
	};
\end{tikzpicture}
\ref{roc_legend}
\caption{
Kurva ROC untuk klasifikasi RF dan CRF pada tiga dataset yaitu yang
tanpa sampel ulang, yang disampel ulang dengan SMOTE dan LNSMOTE.
(a) RF dengan 200 pohon
(b) CRF dengan 200 tingkat 1 pohon
(c) CRF dengan 100 tingkat 2 pohon
(d) CRF dengan 50 tingkat 4 pohon
}
\end{figure}
