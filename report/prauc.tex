%% pgfplots setting.
\pgfplotsset{
	/pgf/number format/read comma as period,
	/pgf/text mark as node=false,
	table/col sep=semicolon,
	xmax=1,
	xmin=0,
	ymax=1,
	ymin=0,
	xtick distance=0.2,
	ytick distance=0.2,
	grid=major,
	cycle list name=linestyles,
	compat=1.13
}

\begin{figure}[htbp]
\centering
\begin{tikzpicture}[framed]
	\pgfplotsset{
		small,
	}
	\matrix{
		\begin{axis}[
			title=(a),
			ylabel=$Precision$,
			legend columns=-1,
			legend entries={Tanpa sampel ulang\ ,SMOTE\ ,LNSMOTE},
			legend to name=rf_crf_prauc_legend
		]
			\addplot table[
				x={TPR},
				y={PREC},
			]
			{../result/rf.csv};
			\addplot table[
				x={TPR},
				y={PREC},
			]
			{../result/rf_smote.csv};
			\addplot table[
				x={TPR},
				y={PREC},
			]
			{../result/rf_lnsmote.csv};
		\end{axis}
		&
		\begin{axis}[
			title=(b),
		]
			\addplot table[
				x={TPR},
				y={PREC},
			]
			{../result/crf_200_1.csv};
			\addplot table[
				x={TPR},
				y={PREC},
			]
			{../result/crf_200_1_smote.csv};
			\addplot table[
				x={TPR},
				y={PREC},
			]
			{../result/crf_200_1_lnsmote.csv};
			title=(b),
		\end{axis}
		\\
		\begin{axis}[
			title=(c),
			xlabel=$Recall$,
			ylabel=$Precision$,
		]
			\addplot table[
				x={TPR},
				y={PREC},
			]
			{../result/crf_100_2.csv};
			\addplot table[
				x={TPR},
				y={PREC},
			]
			{../result/crf_100_2_smote.csv};
			\addplot table[
				x={TPR},
				y={PREC},
			]
			{../result/crf_100_2_lnsmote.csv};
		\end{axis}
		&
		\begin{axis}[
			title=(d),
			xlabel=$Recall$,
		]
			\addplot table[
				x={TPR},
				y={PREC},
			]
			{../result/crf_50_4.csv};
			\addplot table[
				x={TPR},
				y={PREC},
			]
			{../result/crf_50_4_smote.csv};
			\addplot table[
				x={TPR},
				y={PREC},
			]
			{../result/crf_50_4_lnsmote.csv};
		\end{axis}
		\\
	};
\end{tikzpicture}
\ref{rf_crf_prauc_legend}
\caption{
Kurva PR-AUC untuk klasifikasi RF dan CRF pada tiga dataset yaitu yang
tanpa sampel ulang, yang disampel ulang dengan SMOTE dan LNSMOTE.
(a) RF dengan 200 pohon
(b) CRF dengan 200 tingkat 1 pohon
(c) CRF dengan 100 tingkat 2 pohon
(d) CRF dengan 50 tingkat 4 pohon
}
\label{fig:prauc}
\end{figure}
