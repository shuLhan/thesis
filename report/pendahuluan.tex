\chapter{Pendahuluan}

\section{Latar Belakang}
\label{sec:latar-belakang}

Wikipedia.org adalah ensiklopedia daring dan terbuka, yang mana artikel di
Wikipedia merupakan hasil kolaborasi para penyunting dari seluruh dunia.
Terbuka artinya siapa pun dapat menyunting artikel tanpa perlu melakukan
registrasi terlebih dahulu.
Ensiklopedia daring ini memiliki artikel dari berbagai bahasa, dari bahasa umum
dunia seperti Bahasa Inggris, sampai bahasa daerah seperti Bahasa Jawa.

Vandalisme menurut Kamus Besar Bahasa Indonesia daring adalah,
1) perbuatan merusak dan menghancurkan hasil karya seni dan barang berharga
lainnya;
2) perusakan dan penghancuran secara kasar dan ganas.
Dalam konteks Wikipedia.org, vandalisme dapat berbentuk suntingan yang mengubah
konten dari artikel sehingga memberikan isi yang salah, penghapusan secara
menyeluruh, penghapusan sebagian, isi yang menghina, iklan, dan/atau teks yang
tidak ada maknanya.

Jumlah artikel Bahasa Inggris pada situs en.wikipedia.org pada bulan Juli 2015
yaitu sebanyak 4,932,627 artikel, dengan pengguna aktif, atau disebut juga
penyunting, sebanyak 31,369 orang.
Berarti, jika diasumsikan semua penyunting benar aktif, maka setiap pengguna
aktif harus mengawasi kurang lebih 157 artikel.
Menemukan dan memperbaiki vandalisme tersebut dapat mengganggu penyunting dari
menulis artikel dan pekerjaan penting lainnya, dan membuat pembaca bisa
mendapatkan informasi yang salah atau tidak mendapatkan informasi sama sekali.


\section{Rumusan Masalah}
\label{sec:rumusan-masalah}

Korpus yang umum digunakan untuk pembelajaran vandalisme pada Wikipedia yaitu
\textit{PAN Wikipedia Vandalism Corpus} 2010 (PAN-WVC-10)
\cite{potthast:2010b}
atau
\textit{PAN Wikipedia Vandalism Corpus} 2011 (PAN-WVC-11)
\cite{potthast:2010b}
dengan tingkat bias yang tinggi pada data.
Kedua korpus tersebut memiliki jumlah yang tidak seimbang antara suntingan
biasa dengan jumlah suntingan vandal.
PAN-WVC-10 untuk artikel Wikipedia bahasa Inggris memiliki 32.439 sampel dengan
2394, atau 7\%, diantaranya adalah vandal, sedangkan korpus PAN-WVC-11 untuk
artikel Wikipedia bahasa Inggris memiliki jumlah 9985 suntingan dengan 1144,
atau 8\%, diantaranya adalah vandal.

\newpage
Menerapkan klasifikasi dengan dataset yang bias bisa menyebabkan performansi
deteksi yang rendah.
Hal ini bisa disebabkan oleh,
\begin{enumerate}
	\item Jika sebuah klasifikasi belajar dengan meminimalkan keseluruhan
	galat, maka instan dari kelas yang minoritas memiliki kontribusi yang
	rendah terhadap galat.
	Hal ini menyebabkan bias yang condong pada kelas klasifikasi yang
	mayoritas.
	\item Pada umumnya klasifikasi mengasumsikan distribusi kelas yang
	seimbang antara kelas minoritas dan mayoritas, yang terkadang pada
	dunia nyata kasusnya tidak selalu seperti itu.
	\item Sering kali klasifikasi secara implisit mengasumsikan biaya yang
	sama untuk mis-klasifikasi pada kedua kelas tersebut, yang mana
	terkadang tidak masuk akal.
	Sebagai contohnya, biaya untuk mengklasifikasikan kanker sebagai bukan
	kanker lebih tinggi dari pada sebaliknya.
	Secara tidak adanya data kanker bisa menyebabkan tidak dilakukannya
	terapi, misklasifikasi bisa membahayakan nyawa.
\end{enumerate}

Untuk mengatasi masalah ketimpangan pada korpus, Gotze
\cite{gotze2014advanced}
mengaplikasikan teknik
\textit{random undersampling}
bernama
\textit{Synthetic Minority Over-sampling TEchnique} (SMOTE)
yang diajukan oleh Chawla
\cite{chawla2002smote},
dan kombinasi dari SMOTE dan
\textit{random undersampling}.
Dataset latihan yang orisinal dan hasil sampel ulang diuji dengan
pengklasifikasi satu-kelas dan dua-kelas.
Pengklasifikasi satu-kelas yang diterapkan diantaranya Hempstalk dkk.
\cite{hempstalk2008one}
dan SVM oleh Schölkopf dkk.
\cite{scholkopf1999support}
yang diimplementasikan oleh Chang dan Lin
\cite{chang2011libsvm}
pada korpus PAN-WVC.
Pengklasifikasi dua-kelas yang diterapkan diantaranya
\textit{Logistic Regression},
\textit{RealAdaBoost},
\textit{Random Forest} (RF), dan
\textit{Bayesian Network}.
Hasil percobaan yang didapat memperlihatkan performansi pengklasifikasi
satu-kelas tidak kompetitif dengan satu pun pengklasifikasi dua-kelas.
Hal ini bisa disebabkan karena tidak sesuainya kelompok fitur yang digunakan
untuk menjelaskan suntingan vandalisme, sebagaimana juga parameter pengaturan
yang tidak sesuai pada pendekatan yang digunakan.
Hasil dari pelatihan pada dataset orisinal memperlihatkan RF
lebih unggul dari pengklasifikasi lainnya.
Hasil dari pelatihan pada dataset hasil sampel ulang memperlihatkan adanya
peningkatkan pada semua pengklasifikasi kecuali pada RF.

Kelemahan RF yaitu walaupun sejumlah besar pohon-pohon individu
bisa menghasilkan performansi yang tinggi, hal ini juga bisa menambah waktu
komputasi yang dibutuhkan untuk klasifikasi terutama untuk pelatihan model
klasifikasi.
Untuk dataset yang besar yang terdiri dari 10.000 sampel (seperti pada kasus
korpus PAN-WVC-10) hal ini bisa menyebabkan waktu pelatihan sampai pada dua
digit menit.
Salah satu solusinya mungkin bisa dengan menggunakan kerangka kerja
\textit{Cascaded Random Forest} (CRF)
yang dikembangkan oleh Baumann dkk.
\cite{baumann2013cascaded}.
Pendekatan CRF menghasilkan pelatihan model yang lebih cepat dan performansi
klasifikasi yang meningkat dibandingkan dengan pengklasifikasi \textit{Random
Forest}.

Penggunaan sampel ulang pada dataset latihan menggunakan pendekatan dasar
terkadang menyebabkan performansi klasifikasi yang rendah.
Kelemahan teknik SMOTE yaitu bisa terlalu menggeneralisasi wilayah kelas
minoritas karena tidak mempertimbangkan distribusi tetangga lainnya dari
kelas-kelas mayoritas
\cite{maciejewski2011local}.
Melakukan teknik pembersihan dataset lanjut, seperti yang diajukan oleh
Laurikkala
\cite{laurikkala2001improving}
dan Batista dkk.
\cite{batista2004study},
sebagaimana juga teknik sampel ulang lanjut, seperti
\textit{borderline-SMOTE}
yang diajukan oleh Han dkk.
\cite{han2005borderline}
atau ekstensi SMOTE, LN-SMOTE, yang diajukan oleh Maciejewski dan Stefanowski 
\cite{maciejewski2011local},
mungkin bisa meningkatkan performansi klasifikasi.

Tesis ini mencoba menjawab permasalahan dataset yang tidak seimbang pada
PAN-WVC dengan mengkaji teknik sampel ulang dan klasifikasi yang belum pernah
digunakan sebelumnya pada korpus tersebut.
Teknik sampel ulang yang digunakan yaitu
\textit{Local Neighborhood SMOTE} (LNSMOTE),
yang diajukan oleh Maciejewski dan
Stefanowski 
\cite{maciejewski2011local}.
Hasil sampel ulang dataset digunakan untuk pembelajaran mesin dengan menerapkan
pengklasifikasi CRF dan dibandingkan dengan pengklasifikasi RF untuk melihat
performansi klasifikasi yang lebih baik.


\section{Tujuan}
\label{sec:tujuan}

Tesis ini bertujuan mengidentifikasi performansi dari teknik sampel ulang
LNSMOTE dan teknik klasifikasi
\textit{Cascaded Random Forest}.
Untuk teknik LNSMOTE dibandingkan dengan SMOTE pada sampel ulang dataset
PAN-WVC-11.
Sedangkan untuk klasifikasi
\textit{Cascaded Random Forest}
dibandingkan dengan klasifikasi
\textit{Random Forest}
untuk dilihat performansi pada akurasi dan kecepatan pemrosesan.

%%{{{ SECTION: Batasan Masalah
%%
\section{Batasan Masalah}
\label{sec:batasan-masalah}

Tesis ini hanya melakukan analisis untuk artikel Wikipedia Bahasa Inggris yang
terdapat pada korpus PAN-WVC-10 dan PAN-WVC-11.
Dataset yang digunakan untuk sampel ulang dan pelatihan model yaitu PAN-WVC-10.
Dataset yang digunakan dalam melakukan pengujian yaitu PAN-WVC-11.
Teknik sampel ulang yang dilakukan yaitu LNSMOTE yang dibandingkan dengan
SMOTE.
Teknik pengklasifikasi yang dijadikan dalam pelatihan yaitu CRF yang
dibandingkan dengan pengklasifikasi RF.

%%}}}

%%{{{ SECTION: Metodologi
%%
\section{Metodologi}
\label{sec:metodologi}

\textbf{Persiapan Data dan Lingkungan Penelitian}.
Dataset PAN-WVC-10 dan PAN-WVC-11 dapat diambil di situs Universitas Bauhaus
Weimar
\footnote{%
	\RaggedRight\url{%
	http://www.uni-weimar.de/en/media/chairs/webis/corpora/%
}}.
Sebelum dapat digunakan dalam pelatihan dan pengujian, kedua dataset diproses
ke dalam fitur terlebih dahulu.
Hasil fitur pada PAN-WVC-10 kemudian di sampel ulang dengan SMOTE dan LNSMOTE.
Hasil fitur pada PAN-WVC-11 tidak disampel ulang, hanya digunakan untuk
pengujian model.
Sistem operasi yang digunakan dalam penelitian ini yaitu GNU/Linux dengan
bahasa pemrograman yang digunakan untuk implementasi adalah Go
\footnote{\RaggedRight\url{https://golang.org}}.


\textbf{Implementasi dan Pengujian}.
Dataset tanpa sampel ulang dan yang telah disampel ulang dijadikan
pelatihan untuk pengklasifikasi CRF dan RF satu per satu, kemudian hasil
pemodelan diuji langsung dengan diberikan input dari fitur dataset PAN-WVC-11.
Hasil dari pengujian ini digunakan pada tahap analisis.
Implementasi fitur, sampel ulang SMOTE dan LNSMOTE, pengklasifikasi CRF dan RF
dibuat dari awal.

\textbf{Analisis}.
Hasil dari setiap pengklasifikasi pada dataset yang tidak disampel ulang dan
yang disampel ulang dibandingkan untuk dilihat performansinya dengan memetakan
laju \textit{false-positive} dan laju \textit{true-positive} pada ruang
\textit{Receiver Operating Characteristic} (ROC).
Untuk performansi pengklasifikasi selain dilihat dari akurasinya juga dilihat
dari kecepatan dalam pemodelannya.

%%}}}



\section{Sistematika Penulisan}
\label{sec:sistematika-penulisan}

Laporan tesis ini dibagi menjadi beberapa bab berikut,
\begin{enumerate}
	\item Bab I Pendahuluan, berisi Latar Belakang, Rumusan Masalah,
	Tujuan, Batasan Masalah, Metodologi, dan Sistematika Penulisan.
	\item Bab II Landasan Teori, berisi ilmu dan konsep yang mendukung
	pembahasan tesis ini.
	\item Bab III Metodologi Penelitian, berisi deskripsi tentang analisis, tahap implementasi, dan tahap pengujian yang dilakukan selama penelitian.
	\item Bab IV Hasil dan Analisis, berisi penjelasan dari hasil penelitian.
	\item Bab V Penutup, berisi kesimpulan yang dapat diambil dari hasil penelitian ini beserta saran untuk pengembangan selanjutnya.
\end{enumerate}
%%}}}

