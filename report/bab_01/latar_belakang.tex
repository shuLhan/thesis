Wikipedia.org adalah ensiklopedia daring dan terbuka, yang mana artikel di
Wikipedia merupakan hasil kolaborasi para penyunting dari seluruh dunia
Terbuka artinya siapa pun dapat menyunting artikel tanpa perlu melakukan
registrasi terlebih dahulu.
Sifat Wikipedia yang daring dan terbuka tersebut memiliki keuntungan dengan
cepatnya perkembangan dan perbaikan artikel, kelemahannya yaitu rentan terhadap
vandalisme.

Vandalisme menurut Kamus Besar Bahasa Indonesia daring adalah
1) perbuatan merusak dan menghancurkan hasil karya seni dan barang berharga
lainnya;
2) perusakan dan penghancuran secara kasar dan ganas.
Dalam konteks Wikipedia, vandalisme berbentuk suntingan yang mengubah
konten artikel sehingga memberikan isi yang salah, menghina, iklan,
dan/atau teks yang tidak ada maknanya, atau penghapusan secara
menyeluruh atau sebagian yang menyebabkan hilangnya informasi bagi pembaca.

Jumlah artikel bahasa Inggris pada situs en.wikipedia.org pada Juli 2015
yaitu 4,932,627 artikel, dengan pengguna aktif atau disebut juga
penyunting sebanyak 31,369 orang.
Berarti, jika diasumsikan semua penyunting benar aktif, maka setiap pengguna
aktif harus mengawasi kurang lebih 157 artikel setiap hari.
Menemukan dan memperbaiki vandalisme pada setiap artikel dapat mengganggu
penyunting dari menulis artikel dan pekerjaan penting lainnya, dan membuat
pembaca bisa mendapatkan informasi yang salah atau tidak mendapatkan informasi
sama sekali.

Untuk mencegah vandalisme, komunitas Wikipedia telah
melakukan beberapa cara.
Cara awal yaitu dengan melakukan pemblokiran terhadap alamat IP
(\textit{internet protocol}) atau pengguna teregistrasi yang melakukan
vandalisme selama waktu tertentu dan menggunakan \textit{bot} yang memeriksa
setiap suntingan terhadap sebuah daftar kata yang cenderung digunakan oleh para
pelaku vandal.
Teknik ini tidak efisien, karena administrator Wikipedia harus melakukan
pembaruan daftar IP dan pengguna yang diblokir secara manual setiap saat
vandalisme terjadi dan pembaruan daftar kata untuk bentuk baru suntingan vandal
yang tidak terdeteksi.
Mengatasi ketidakefisienan tersebut, \textcite{potthast2008automatic}
mengumpulkan suntingan yang vandal dan yang bukan vandal kemudian melakukan
pendekatan pembelajaran mesin dengan membangun sebuah model deteksi vandalisme
menggunakan
fitur tekstual dan metadata dengan \textit{Logistic Regression}.
Beberapa peneliti lain mengembangkan penelitian Potthast dengan menambah fitur
dan menggunakan pengklasifikasi yang berbeda. Sebagai contohnya,
\textcite{mola2012wikipedia} menambahkan fitur daftar kata dan menggunakan
pengklasifikasi \textit{Random Forest} (RF), \textcite{gotze2014advanced}
menggabungkan fitur dari penelitian Potthast, Mola-Velasco, dan beberapa
penelitian lain kemudian melakukan sampel ulang dengan SMOTE (\textit{Synthetic
Minority Oversampling Technique}) yang kemudian dilatih dengan pengklasifikasi
dua-kelas \textit{Logistic Regression}, \textit{RealAdaBoost}, RF, dan
\textit{Bayesian Network}.
Hasil penelitian Gotze memperlihatkan sampel ulang SMOTE meningkatkan
performansi pengklasifikasi dengan RF lebih unggul daripada yang lainnya.

Korpus yang umum digunakan untuk pembelajaran vandalisme pada Wikipedia yaitu
\textit{PAN Wikipedia Vandalism Corpus} 2010 (PAN-WVC-10)
atau
\textit{PAN Wikipedia Vandalism Corpus} 2011 (PAN-WVC-11)
\parencite{potthast:2010b}.
Kedua korpus tersebut memiliki jumlah yang tidak seimbang antara suntingan
biasa dengan suntingan vandal.
PAN-WVC-10 untuk artikel Wikipedia bahasa Inggris memiliki 32.439 sampel dengan
2.394 ($\sim 7\%$) adalah vandal, dan PAN-WVC-11 untuk
artikel Wikipedia bahasa Inggris memiliki sejumlah 9.985 suntingan dengan 1.144
($\sim 8\%$) adalah vandal.

Menerapkan pengklasifikasi pada dataset yang timpang bisa menyebabkan
performansi deteksi yang rendah.
Hal ini bisa disebabkan oleh,
\begin{itemize}
	\item instan dari kelas minoritas berkontribusi rendah
	terhadap model klasifikasi, sehingga menyebabkan bias yang condong pada
	kelas mayoritas.
	\item Pada umumnya pengklasifikasi mengasumsikan distribusi kelas yang
	seimbang antara kelas minoritas dan mayoritas, yang terkadang pada
	dunia nyata kasusnya tidak selalu seperti itu.
	\item Sering kali pengklasifikasi secara implisit mengasumsikan biaya
	(\textit{cost}) yang sama untuk klasifikasi pada setiap kelas.
	Sebagai contohnya, biaya untuk mengklasifikasikan kanker sebagai bukan
	kanker lebih tinggi dari pada sebaliknya.
	Secara tidak adanya data kanker bisa menyebabkan tidak dilakukannya
	terapi, salah klasifikasi bisa membahayakan nyawa.
\end{itemize}

\vspace{1.5em}
Dataset timpang dapat diatasi dengan melakukan penghapusan untuk sampel yang
mayoritas (\textit{undersampling}) atau penambahan sampel untuk kelas
yang minoritas (\textit{oversampling}).
Salah satu metode \textit{oversampling} yaitu \textit{Synthetic Minority
Oversampling TEchnique} (SMOTE) \parencite{chawla2002smote}.
Kelemahan SMOTE yaitu adanya tumpang tindih antara sampel sintetis dengan
sampel mayoritas dan tidak dipertimbangkannya \textit{outlier} dari sampel
minoritas.
Melakukan teknik sampel ulang lanjut, seperti
\textit{Local-Neighbourhoods SMOTE} (LNSMOTE)
\parencite{maciejewski2011local} yang mengatasi kelemahan-kelemahan tersebut
mungkin bisa meningkatkan performansi klasifikasi.

Pengklasifikasi yang digunakan oleh \textcite{mola2012wikipedia} dan
\textcite{gotze2014advanced} adalah \textit{Random Forest} (RF).
RF memiliki beberapa kelemahan yaitu waktu komputasi yang lama
dibutuhkan untuk klasifikasi terutama pelatihan model klasifikasi.
Pada dataset yang besar (misalkan lebih dari 10.000 sampel, seperti
PAN-WVC-10) hal ini membuat waktu pelatihan selesai dalam beberapa jam.
Kelemahan RF lainnya yaitu apabila dataset pelatihan timpang, RF condong
mendukung kelas mayoritas \parencite{strobl2007bias}.
Salah satu solusinya bisa dengan menggunakan pengklasifikasi \textit{Cascaded
Random Forest} (CRF) yang dikembangkan oleh \textcite{baumann2013cascaded}.
Pendekatan CRF menghasilkan pelatihan model yang lebih cepat dan performansi
klasifikasi yang meningkat dibandingkan dengan pengklasifikasi RF.
