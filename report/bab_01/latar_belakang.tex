Wikipedia.org adalah ensiklopedia daring dan terbuka, yang mana artikel di
Wikipedia merupakan hasil kolaborasi para penyunting dari seluruh dunia
Terbuka artinya siapa pun dapat menyunting artikel tanpa perlu melakukan
registrasi terlebih dahulu.
Sifat Wikipedia yang daring dan terbuka tersebut memiliki keuntungan dengan
cepatnya perkembangan dan perbaikan artikel, kelemahannya yaitu rentan terhadap
vandalisme.

Vandalisme menurut Kamus Besar Bahasa Indonesia daring adalah
1) perbuatan merusak dan menghancurkan hasil karya seni dan barang berharga
lainnya;
2) perusakan dan penghancuran secara kasar dan ganas.
Dalam konteks Wikipedia, vandalisme berbentuk suntingan yang mengubah
konten artikel sehingga memberikan isi yang salah, menghina, iklan,
dan/atau teks yang tidak ada maknanya, atau penghapusan secara
menyeluruh atau sebagian yang menyebabkan hilangnya informasi.

Jumlah artikel Bahasa Inggris pada situs en.wikipedia.org pada bulan Juli 2015
yaitu sebanyak 4,932,627 artikel, dengan pengguna aktif, atau disebut juga
penyunting, sebanyak 31,369 orang.
Berarti, jika diasumsikan semua penyunting benar aktif, maka setiap pengguna
aktif harus mengawasi kurang lebih 157 artikel.
Menemukan dan memperbaiki vandalisme tersebut dapat mengganggu penyunting dari
menulis artikel dan pekerjaan penting lainnya, dan membuat pembaca bisa
mendapatkan informasi yang salah atau tidak mendapatkan informasi sama sekali.
