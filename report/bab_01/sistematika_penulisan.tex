Laporan tesis ini dibagi menjadi beberapa bab berikut,
\begin{enumerate}
	\item Bab I Pendahuluan, berisi Latar Belakang, Rumusan Masalah,
	Tujuan, Batasan Masalah, Metodologi, dan Sistematika Penulisan.
	\item Bab II Tinjauan Pustaka, berisi ilmu dan konsep yang mendukung
	pembahasan tesis ini beserta makalah mengenai pekerjaan sebelumnya
	dalam deteksi vandalisme di Wikipedia.
	\item Bab III Proses Deteksi Vandalisme, berisi tahap dalam persiapan
	data, fitur, sampel ulang, implemetentasi, pelatihan model dan
	pengujian.
	\item Bab IV Evaluasi, berisi penjelasan dari hasil penelitian.
	\item Bab V Kesimpulan, berisi rangkuman yang dapat diambil dari hasil
	penelitian ini beserta saran untuk pengembangan selanjutnya.
\end{enumerate}
