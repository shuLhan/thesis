Tesis ini menggunakan dataset bahasa Inggris yang terdapat pada korpus
PAN-WVC-10 dan PAN-WVC-11.
Dataset yang digunakan untuk sampel ulang dan pelatihan model yaitu PAN-WVC-10.
Dataset yang digunakan dalam melakukan pengujian yaitu PAN-WVC-11.
Tesis ini tidak membangun sebuah sistem yang siap pakai untuk deteksi
vandalisme, namun memberikan sebuah alternatif kerangka kerja dari pembangunan
sistem deteksi vandalisme pada Wikipedia dengan memperlihatkan performansi yang
didapat setelah melakukan sampel ulang dengan LNSMOTE, pembangunan model
dan hasil pengklasifikasi \textit{Cascaded Random Forest}.
