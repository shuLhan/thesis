Tesis ini mencoba menjawab permasalahan dataset yang tidak seimbang pada
PAN-WVC-10 yang menyebabkan performansi deteksi yang rendah dan condong pada
kelas mayoritas dengan mengkaji teknik sampel ulang dan pengklasifikasi yang
belum pernah digunakan sebelumnya pada korpus tersebut.
Teknik sampel ulang yang digunakan yaitu LNSMOTE yang diajukan oleh
\textcite{maciejewski2011local}
dan teknik pengklasifikasi yang digunakan yaitu \textit{Cascaded Random Forest}
(CRF) yang diajukan oleh \textcite{baumann2013cascaded}.
