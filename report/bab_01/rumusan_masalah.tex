Korpus yang umum digunakan untuk pembelajaran vandalisme pada Wikipedia yaitu
\textit{PAN Wikipedia Vandalism Corpus} 2010 (PAN-WVC-10)
\cite{potthast:2010b}
atau
\textit{PAN Wikipedia Vandalism Corpus} 2011 (PAN-WVC-11)
\cite{potthast:2010b}
Kedua korpus tersebut memiliki jumlah yang tidak seimbang antara suntingan
biasa dengan jumlah suntingan vandal.
PAN-WVC-10 untuk artikel Wikipedia bahasa Inggris memiliki 32.439 sampel dengan
2394, atau 7\%, diantaranya adalah vandal, sedangkan korpus PAN-WVC-11 untuk
artikel Wikipedia bahasa Inggris memiliki jumlah 9985 suntingan dengan 1144,
atau 8\%, diantaranya adalah vandal.

Menerapkan klasifikasi dengan dataset yang timpang bisa menyebabkan performansi
deteksi yang rendah.
Hal ini bisa disebabkan oleh,
\begin{enumerate}
	\item instan dari kelas yang minoritas memiliki kontribusi yang rendah
	terhadap galat, sehingga menyebabkan bias yang condong pada kelas
	klasifikasi yang mayoritas.
	\item Pada umumnya klasifikasi mengasumsikan distribusi kelas yang
	seimbang antara kelas minoritas dan mayoritas, yang terkadang pada
	dunia nyata kasusnya tidak selalu seperti itu.
	\item Sering kali klasifikasi secara implisit mengasumsikan biaya yang
	sama untuk mis-klasifikasi pada kedua kelas tersebut.
	Sebagai contohnya, biaya untuk mengklasifikasikan kanker sebagai bukan
	kanker lebih tinggi dari pada sebaliknya.
	Secara tidak adanya data kanker bisa menyebabkan tidak dilakukannya
	terapi, misklasifikasi bisa membahayakan nyawa.
\end{enumerate}

Dataset timpang dapat diatasi dengan melakukan penghapusan untuk sampel yang
mayoritas (\textit{undersampling}) atau penambahan sampel untuk kelas
yang minoritas (\textit{oversampling}).
Salah satu metode \textit{oversampling} yaitu SMOTE.
Kelemahan teknik SMOTE yaitu menggeneralisasi wilayah kelas
minoritas karena tidak mempertimbangkan distribusi tetangga lainnya dari
kelas mayoritas
\cite{maciejewski2011local}.
Melakukan teknik sampel ulang lanjut, seperti
\textit{Borderline-SMOTE}
yang diajukan oleh Han dkk.
\cite{han2005borderline}
atau ekstensi SMOTE yaitu \textit{Local-Neighbourhoods SMOTE}, yang diajukan
oleh Maciejewski dan Stefanowski
\cite{maciejewski2011local},
mungkin bisa meningkatkan performansi klasifikasi.

Kelemahan \textit{Random Forest} (RF) yaitu walaupun sejumlah besar pohon-pohon
individu bisa menghasilkan performansi yang tinggi, hal ini juga bisa menambah
waktu komputasi yang dibutuhkan untuk klasifikasi terutama untuk pelatihan
model klasifikasi.
Untuk dataset yang besar (misalkan lebih dari 10.000 sampel, seperti
PAN-WVC-10) hal ini membuat waktu pelatihan selesai dalam beberapa jam.
Salah satu solusinya bisa dengan menggunakan kerangka kerja
\textit{Cascaded Random Forest} (CRF)
yang dikembangkan oleh Baumann dkk.
\cite{baumann2013cascaded}.
Pendekatan CRF menghasilkan pelatihan model yang lebih cepat dan performansi
klasifikasi yang meningkat dibandingkan dengan pengklasifikasi RF.

Tesis ini mencoba menjawab permasalahan dataset yang tidak seimbang pada
PAN-WVC dengan mengkaji teknik sampel ulang dan klasifikasi yang belum pernah
digunakan sebelumnya pada korpus tersebut.
Teknik sampel ulang yang digunakan yaitu
\textit{Local Neighborhood SMOTE} (LNSMOTE),
yang diajukan oleh Maciejewski dan
Stefanowski
\cite{maciejewski2011local}.
Hasil sampel ulang dataset digunakan untuk pembelajaran mesin dengan menerapkan
pengklasifikasi CRF dan dibandingkan dengan pengklasifikasi RF untuk melihat
performansi klasifikasi yang lebih baik.
