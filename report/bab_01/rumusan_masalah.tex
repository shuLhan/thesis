Korpus yang umum digunakan untuk pembelajaran vandalisme pada Wikipedia yaitu
\textit{PAN Wikipedia Vandalism Corpus} 2010 (PAN-WVC-10)
\cite{potthast:2010b}
atau
\textit{PAN Wikipedia Vandalism Corpus} 2011 (PAN-WVC-11)
\cite{potthast:2010b}
dengan tingkat bias yang tinggi pada data.
Kedua korpus tersebut memiliki jumlah yang tidak seimbang antara suntingan
biasa dengan jumlah suntingan vandal.
PAN-WVC-10 untuk artikel Wikipedia bahasa Inggris memiliki 32.439 sampel dengan
2394, atau 7\%, diantaranya adalah vandal, sedangkan korpus PAN-WVC-11 untuk
artikel Wikipedia bahasa Inggris memiliki jumlah 9985 suntingan dengan 1144,
atau 8\%, diantaranya adalah vandal.

\newpage
Menerapkan klasifikasi dengan dataset yang bias bisa menyebabkan performansi
deteksi yang rendah.
Hal ini bisa disebabkan oleh,
\begin{enumerate}
	\item Jika sebuah klasifikasi belajar dengan meminimalkan keseluruhan
	galat, maka instan dari kelas yang minoritas memiliki kontribusi yang
	rendah terhadap galat.
	Hal ini menyebabkan bias yang condong pada kelas klasifikasi yang
	mayoritas.
	\item Pada umumnya klasifikasi mengasumsikan distribusi kelas yang
	seimbang antara kelas minoritas dan mayoritas, yang terkadang pada
	dunia nyata kasusnya tidak selalu seperti itu.
	\item Sering kali klasifikasi secara implisit mengasumsikan biaya yang
	sama untuk mis-klasifikasi pada kedua kelas tersebut, yang mana
	terkadang tidak masuk akal.
	Sebagai contohnya, biaya untuk mengklasifikasikan kanker sebagai bukan
	kanker lebih tinggi dari pada sebaliknya.
	Secara tidak adanya data kanker bisa menyebabkan tidak dilakukannya
	terapi, misklasifikasi bisa membahayakan nyawa.
\end{enumerate}

Untuk mengatasi masalah ketimpangan pada korpus, Gotze
\cite{gotze2014advanced}
mengaplikasikan teknik
\textit{random oversampling}
bernama
\textit{Synthetic Minority Over-sampling TEchnique} (SMOTE)
yang diajukan oleh Chawla
\cite{chawla2002smote},
dan kombinasi dari SMOTE dan
\textit{random undersampling}.
Dataset latihan yang orisinal dan hasil sampel ulang diuji dengan
pengklasifikasi satu-kelas dan dua-kelas.
Pengklasifikasi satu-kelas yang diterapkan diantaranya Hempstalk dkk.
\cite{hempstalk2008one}
dan SVM oleh Schölkopf dkk.
\cite{scholkopf1999support}
yang diimplementasikan oleh Chang dan Lin
\cite{chang2011libsvm}
pada korpus PAN-WVC.
Pengklasifikasi dua-kelas yang diterapkan diantaranya
\textit{Logistic Regression},
\textit{RealAdaBoost},
\textit{Random Forest} (RF), dan
\textit{Bayesian Network}.
Hasil percobaan yang didapat memperlihatkan performansi pengklasifikasi
satu-kelas tidak kompetitif dengan satu pun pengklasifikasi dua-kelas.
Hal ini bisa disebabkan karena tidak sesuainya kelompok fitur yang digunakan
untuk menjelaskan suntingan vandalisme, sebagaimana juga parameter pengaturan
yang tidak sesuai pada pendekatan yang digunakan.
Hasil dari pelatihan pada dataset orisinal memperlihatkan RF
lebih unggul dari pengklasifikasi lainnya.
Hasil dari pelatihan pada dataset hasil sampel ulang memperlihatkan adanya
peningkatkan pada semua pengklasifikasi kecuali pada RF.

Kelemahan RF yaitu walaupun sejumlah besar pohon-pohon individu
bisa menghasilkan performansi yang tinggi, hal ini juga bisa menambah waktu
komputasi yang dibutuhkan untuk klasifikasi terutama untuk pelatihan model
klasifikasi.
Untuk dataset yang besar yang terdiri dari 10.000 sampel (seperti pada kasus
korpus PAN-WVC-10) hal ini bisa menyebabkan waktu pelatihan sampai pada dua
digit menit.
Salah satu solusinya mungkin bisa dengan menggunakan kerangka kerja
\textit{Cascaded Random Forest} (CRF)
yang dikembangkan oleh Baumann dkk.
\cite{baumann2013cascaded}.
Pendekatan CRF menghasilkan pelatihan model yang lebih cepat dan performansi
klasifikasi yang meningkat dibandingkan dengan pengklasifikasi \textit{Random
Forest}.

Penggunaan sampel ulang pada dataset latihan menggunakan pendekatan dasar
terkadang menyebabkan performansi klasifikasi yang rendah.
Kelemahan teknik SMOTE yaitu bisa terlalu menggeneralisasi wilayah kelas
minoritas karena tidak mempertimbangkan distribusi tetangga lainnya dari
kelas mayoritas
\cite{maciejewski2011local}.
Melakukan teknik pembersihan dataset lanjut, seperti yang diajukan oleh
Laurikkala
\cite{laurikkala2001improving}
dan Batista dkk.
\cite{batista2004study},
sebagaimana juga teknik sampel ulang lanjut, seperti
\textit{Borderline-SMOTE}
yang diajukan oleh Han dkk.
\cite{han2005borderline}
atau ekstensi SMOTE yaitu \textit{Local-Neighbourhoods SMOTE}, yang diajukan
oleh Maciejewski dan Stefanowski
\cite{maciejewski2011local},
mungkin bisa meningkatkan performansi klasifikasi.

Tesis ini mencoba menjawab permasalahan dataset yang tidak seimbang pada
PAN-WVC dengan mengkaji teknik sampel ulang dan klasifikasi yang belum pernah
digunakan sebelumnya pada korpus tersebut.
Teknik sampel ulang yang digunakan yaitu
\textit{Local Neighborhood SMOTE} (LNSMOTE),
yang diajukan oleh Maciejewski dan
Stefanowski
\cite{maciejewski2011local}.
Hasil sampel ulang dataset digunakan untuk pembelajaran mesin dengan menerapkan
pengklasifikasi CRF dan dibandingkan dengan pengklasifikasi RF untuk melihat
performansi klasifikasi yang lebih baik.
