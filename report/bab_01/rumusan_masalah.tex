Korpus yang umum digunakan untuk pembelajaran vandalisme pada Wikipedia yaitu
\textit{PAN Wikipedia Vandalism Corpus} 2010 (PAN-WVC-10)
atau
\textit{PAN Wikipedia Vandalism Corpus} 2011 (PAN-WVC-11)
\parencite{potthast:2010b}.
Kedua korpus tersebut memiliki jumlah yang tidak seimbang antara suntingan
biasa dengan suntingan vandal.
PAN-WVC-10 untuk artikel Wikipedia bahasa Inggris memiliki 32.439 sampel dengan
2.394 (7\%) diantaranya adalah vandal, sedangkan korpus PAN-WVC-11 untuk
artikel Wikipedia bahasa Inggris memiliki jumlah 9985 suntingan dengan 1.144
(8\%) diantaranya adalah vandal.

Menerapkan klasifikasi pada dataset yang timpang bisa menyebabkan performansi
deteksi yang rendah.
Hal ini bisa disebabkan oleh,
\begin{itemize}
	\item instan dari kelas yang minoritas memiliki kontribusi yang rendah
	terhadap model klasifikasi, sehingga menyebabkan bias yang condong pada
	kelas klasifikasi yang mayoritas.
	\item Pada umumnya pengklasifikasi mengasumsikan distribusi kelas yang
	seimbang antara kelas minoritas dan mayoritas, yang terkadang pada
	dunia nyata kasusnya tidak selalu seperti itu.
	\item Sering kali klasifikasi secara implisit mengasumsikan biaya
	(\textit{cost}) yang sama untuk klasifikasi pada setiap kelas.
	Sebagai contohnya, biaya untuk mengklasifikasikan kanker sebagai bukan
	kanker lebih tinggi dari pada sebaliknya.
	Secara tidak adanya data kanker bisa menyebabkan tidak dilakukannya
	terapi, salah klasifikasi bisa membahayakan nyawa.
\end{itemize}

\vspace{1.5em}
Dataset timpang dapat diatasi dengan melakukan penghapusan untuk sampel yang
mayoritas (\textit{undersampling}) atau penambahan sampel untuk kelas
yang minoritas (\textit{oversampling}).
Salah satu metode \textit{oversampling} yaitu \textit{Synthetic Minority
Oversampling TEchnique} (SMOTE).
Kelemahan teknik SMOTE yaitu menggeneralisasi wilayah kelas
minoritas karena tidak mempertimbangkan distribusi tetangga lainnya dari
kelas mayoritas
\parencite{maciejewski2011local}.
Melakukan teknik sampel ulang lanjut, seperti
\textit{Borderline-SMOTE}
yang diajukan oleh
\textcite{han2005borderline}
atau ekstensi dari SMOTE yaitu \textit{Local-Neighbourhoods SMOTE} (LNSMOTE), yang
diajukan oleh
\textcite{maciejewski2011local},
mungkin bisa meningkatkan performansi klasifikasi.

Kelemahan \textit{Random Forest} (RF) yaitu walaupun sejumlah besar pohon-pohon
individu bisa menghasilkan performansi yang tinggi, hal ini juga meningkatkan
waktu komputasi yang dibutuhkan untuk klasifikasi terutama pelatihan
model klasifikasi.
Pada dataset yang besar (misalkan lebih dari 10.000 sampel, seperti
PAN-WVC-10) hal ini membuat waktu pelatihan selesai dalam beberapa jam.
Salah satu solusinya bisa dengan menggunakan kerangka kerja
\textit{Cascaded Random Forest} (CRF)
yang dikembangkan oleh
\textcite{baumann2013cascaded}.
Pendekatan CRF menghasilkan pelatihan model yang lebih cepat dan performansi
klasifikasi yang meningkat dibandingkan dengan pengklasifikasi RF.

Tesis ini mencoba menjawab permasalahan dataset yang tidak seimbang pada
PAN-WVC dengan mengkaji teknik sampel ulang dan klasifikasi yang belum pernah
digunakan sebelumnya pada korpus tersebut.
Teknik sampel ulang yang digunakan yaitu LNSMOTE yang diajukan oleh
\textcite{maciejewski2011local}.
Hasil sampel ulang dataset digunakan untuk pembelajaran mesin dengan menerapkan
pengklasifikasi CRF dan dibandingkan dengan pengklasifikasi RF untuk melihat
performansi klasifikasi yang lebih baik.
