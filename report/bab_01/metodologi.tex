\textbf{Persiapan Data dan Lingkungan Penelitian}.
Dataset PAN-WVC-10 dan PAN-WVC-11 dapat diambil di situs Universitas
Bauhaus Weimar
\footnote{%
	\RaggedRight\url{%
	http://www.uni-weimar.de/en/media/chairs/webis/corpora/%
}}.
Sebelum dapat digunakan dalam pelatihan dan pengujian, kedua dataset diproses
ke dalam fitur terlebih dahulu.
Hasil fitur pada PAN-WVC-10 kemudian disampel ulang dengan SMOTE dan LNSMOTE.
Hasil fitur pada PAN-WVC-11 tidak disampel ulang, hanya digunakan untuk
pengujian model.
Sistem operasi yang digunakan dalam penelitian ini yaitu GNU/Linux dengan
bahasa pemrograman yang digunakan untuk implementasi adalah Go
\footnote{\RaggedRight\url{https://golang.org}}.

\textbf{Implementasi dan Pengujian}.
Implementasi pengklasifikasi dilakukan bertahap, dari impementasi pohon
keputusan CART, yang digunakan untuk implementasi RF, dan juga
digunakan untuk implementasi CRF.
Dataset tanpa sampel ulang dan yang telah disampel ulang dijadikan
pelatihan untuk pengklasifikasi RF dan CRF satu per satu, kemudian hasil
pemodelan diuji langsung dengan diberikan input dari fitur dataset PAN-WVC-11.
Hasil dari pengujian ini digunakan pada tahap evaluasi.
Implementasi fitur, sampel ulang SMOTE dan LNSMOTE, pengklasifikasi CRF dan RF
dibuat dari awal.

\textbf{Evaluasi}.
Hasil dari setiap pengklasifikasi pada dataset yang tidak disampel ulang dan
yang disampel ulang dibandingkan untuk dilihat performansinya berdasarkan laju
\textit{true-positive}.
Untuk performansi pengklasifikasi selain dilihat dari akurasinya juga dilihat
dari kecepatan dalam pemodelannya.
