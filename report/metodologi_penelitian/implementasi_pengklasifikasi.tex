Implementasi pengklasifikasi dilakukan bertahap karena keterkaitan antara
modul.
Dimulai dari implementasi untuk pohon keputusan CART yang digunakan untuk
implementasi \textit{Random Forest} (RF) yang kemudian digunakan dalam
implementasi \textit{Cascaded Random Forest} (CRF).
Implementasi dari algoritma CART dalam tesis ini mengacu pada buku Jiawei Han,
dkk. bab 8 \cite{han2011data}.
Untuk implementasi RF berdasarkan pada gabungan dari beberapa sumber (situs,
makalah, dan video) dengan rujukan utama tetap pada makalah Breiman
\cite{breiman2001random}.
Implementasi CRF berdasarkan pada makalah Bauman dkk.
\cite{baumann2013cascaded}.
Hasil dari implementasi dapat dilihat dan digunakan secara terbuka pada
repositori \texttt{go-mining}
\footnote{\url{https://github.com/shuLhan/go-mining/tree/master/classifiers}}.
