Dataset yang digunakan untuk pelatihan yaitu PAN-WVC-10
\cite{potthast:2010b}.
Dataset PAN-WVC-10 terbagi menjadi dua yaitu dataset suntingan dari artikel
Wikipedia dan dataset anotasi yang berisi hasil klasifikasi vandalisme pada
dataset suntingan tersebut.
Kedua dataset memiliki format yang sama yaitu menggunakan \textit{Comma
Separated Value} (CSV).
Dataset suntingan memiliki atribut sebagai berikut,

\begin{itemize}
	\item \textbf{editid}, format angka, berisi identifikasi (ID) unik dari setiap suntingan.
	\item \textbf{editor}, format string, berisi nama penyunting.
	\item \textbf{oldrevisionid}, format angka, berisi ID untuk suntingan lama.
	\item \textbf{newrevisionid}, format angka, berisi ID untuk suntingan baru.
	\item \textbf{diffurl}, format string, berisi URL yang mengacu pada perbedaan suntingan baru dengan lama.
	\item \textbf{edittime}, format string, berisi tanggal dan pukul sesuai dengan ISO 8601.
	\item \textbf{editcomment}, format string, berisi komentar yang ditambahkan oleh penyunting saat menyimpan hasil suntingan.
	\item \textbf{articleid}, format angka, berisi ID unik dari artikel.
	\item \textbf{articletitle}, format string, berisi judul dari artikel yang disunting.
\end{itemize}

Dataset anotasi memiliki atribut sebagai berikut,
\begin{itemize}
	\item \textbf{editid}, format angka, mengacu pada ID yang sama pada
	dataset suntingan.
	\item \textbf{class}, format string, berisi tipe suntingan yang
	bernilai "regular" yang menyatakan bahwa suntingan tersebut bukan
	vandalisme, dan "vandalism" yang menyatakan bahwa suntingan tersebut
	adalah vandalisme.
	\item \textbf{annotators}, format angka, berisi jumlah orang yang
	menandai (penanda) bahwa suntingan dengan ID tersebut termasuk ke dalam
	kelas "regular" atau "vandalism".
	\item \textbf{totalannotators}, format angka, berisi jumlah total penanda yang memeriksa suntingan.
\end{itemize}

Kedua dataset kemudian digabung untuk menghasilkan atribut \textit{editid},
\textit{class}, \textit{oldrevisionid}, \textit{newrevisionid},
\textit{edittime}, \textit{editor}, \textit{articletitle}, \textit{deletions},
dan \textit{additions}.
Atribut \textit{additions} berisi teks yang dihapus dalam revisi yang lama
Atribut \textit{deletions} berisi teks yang ditambahkan dalam revisi yang baru.
Kedua atribut tersebut didapat dengan membandingkan isi dari revisi yang lama
dengan yang baru.

Pemrosesan selanjutnya yaitu membuat berkas revisi yang bersih dari sintaks
wiki.
Tujuan dari revisi ini yaitu untuk mempercepat proses pembuatan fitur dan
supaya tidak ada \textit{noise}.
Setiap berkas revisi dibaca kemudian dilakukan pembersihan berikut,

\begin{itemize}
\item penghapusan URI atau \textit{links} yang berawalan dengan
\textit{http://}, \textit{https://}, \textit{ftp://}, dan \textit{ftps}.
\item menghapus \textit{mark-up} wiki yaitu baris yang berisi awalan dan
akhiran berikut,
	\begin{itemize}
	\item \texttt{[[Category:} dan \texttt{]]}
	\item \texttt{[[:Category:} dan \texttt{]]}
	\item \texttt{[[File:} dan \texttt{]]}
	\item \texttt{[[Help:} dan \texttt{]]}
	\item \texttt{[[Image:} dan \texttt{]]}
	\item \texttt{[[Special:} dan \texttt{]]}
	\item \texttt{[[Wikipedia:} dan \texttt{]]}
	\item \texttt{\{\{DEFAULTSORT:} dan \texttt{\}\}}
	\item \texttt{\{\{Template:} dan \texttt{\}\}}
	\item \texttt{<ref} dan \texttt{/>}
	\end{itemize}
\item mengganti karakter dan token berikut dengan karakter kosong (spasi):
\texttt{[}, \texttt{]}, \texttt{\{}, \texttt{\}}, \texttt{|}, \texttt{=},
\texttt{\#}, \texttt{'s}, \texttt{'}, \texttt{<ref>}, \texttt{</ref>},
\texttt{<br />}, \texttt{<br/>}, \texttt{<br>}, \texttt{<nowiki>},
\texttt{</nowiki>}, \texttt{\&nbsp;}.
\item menghapus karakter kosong yang berlebihan.
\end{itemize}

Hasil pembersihan disimpan dalam direktori yang berbeda tapi dengan nama yang
sama.
Hasil ini nanti berguna pada saat melakukan komputasi pembuatan fitur.
