Pembuatan model klasifikasi dilakukan dengan cara menjalankan program
klasifikasi RF dan CRF pada masing-masing dataset fitur PAN-WVC-10 yang belum
disampel ulang, yang telah disampel ulang dengan SMOTE, dan yang telah disampel
ulang dengan LNSMOTE.
Setelah modelnya terbangun, model diuji dengan diberikan input dataset fitur
PAN-WVC-11. Hasil dari pengujian digunakan untuk analisis.

Lingkungan pelatihan dan pengujian dilakukan pada mesin Intel\textregistered\
 Core\texttrademark \ i7-4750HQ CPU 2,00 GHz, dengan jumlah \textit{RAM} 8
GB. Setiap pelatihan model dilakukan satu per satu untuk menghindari adanya
\textit{cache miss} yang berpengaruh pada kecepatan dan waktu pemrosesan.

\section{Dataset}

Dataset yang digunakan untuk pelatihan model yaitu PAN-WVC-10 yang terdiri dari
tiga jenis yaitu dataset tanpa sampel ulang, dataset yang telah disampel ulang
dengan SMOTE, dan dataset yang telah disampel ulang dengan LNSMOTE.
Jumlah sampel pada dataset yang tidak disampel yaitu 2394 positif dan 30045
negatif dengan total 32439 sampel.
Jumlah sampel positif pada dataset hasil sampel ulang dengan SMOTE yaitu 28728
sampel dengan total 58773 sampel.
Jumlah sampel positif pada dataset hasil sampel ulang dengan LNSMOTE yaitu
28588 sampel dengan total 58633 sampel.

Dataset yang digunakan untuk pengujian model yaitu PAN-WVC-11 yang terdiri dari
1143 sampel positif dan 8842 sampel negatif dengan total 9985 sampel.
Jumlah fitur pada PAN-WVC-11 sama dengan PAN-WVC-10 yaitu 26 fitur.

\section{Parameter Pelatihan Model}

Supaya konsisten antara pengklasifikasi, digunakan parameter umum yang sama,
seperti jumlah pohon, jumlah fitur acak, dan persentase \textit{bootstrapping};
yaitu 200 pohon, 5 fitur acak, dan $ 64\% $ untuk \textit{bootstrapping}.
Untuk klasifikasi CRF dilakukan tiga pemodelan dan pengujian dengan parameter
yang berbeda yaitu 200 tingkat dengan 1 pohon, 100 tingkat dengan 2 pohon, dan
50 tingkat dengan 4 pohon; dengan jumlah pohon yang tetap sama yaitu 200.
Hal ini dilakukan untuk melihat pengaruh dari jumlah pohon terhadap tingkat dan
hasil klasifikasi.
Parameter lain pada pemodelan CRF yaitu nilai ambang batas TPR dan TNR diset
pada nilai $0,95$ dan $0,95$ untuk mendapatkan hasil klasifikasi yang bagus dan
jumlah pohon yang konsisten.

\section{Hasil Pengujian}

Hasil pengujian diberikan dalam bentuk performansi pengklasifikasi pada tabel
\ref{tab:stats} dan kecepatan pelatihan model pada tabel
\ref{tab:runtimes}.

\DTLsetseparator{;}
\DTLloaddb{stats}{../result/stats.csv}
\DTLmaxforcolumn{stats}{TPR}{\maxtpr}
\DTLminforcolumn{stats}{FPR}{\minfpr}
\DTLmaxforcolumn{stats}{TNR}{\maxtnr}
\DTLmaxforcolumn{stats}{Presisi}{\maxprec}
\DTLmaxforcolumn{stats}{F-Measure}{\maxfm}
\DTLmaxforcolumn{stats}{Akurasi}{\maxacc}
\DTLmaxforcolumn{stats}{AUC}{\maxauc}

\begin{table}[htbp]
\caption{Performansi Klasifikasi RF dan CRF}
\centering
\footnotesize
\begin{tabular}{p{2cm} p{2cm} rrrrrrr}
\hline
\textbf{Klasifikasi} &
\textbf{Dataset} &
\textbf{TPR} &
\textbf{FPR} &
\textbf{TNR} &
\textbf{Pres.} &
\textbf{F-Mea.} &
\textbf{Aku.} &
\textbf{AUC}
\DTLforeach*{stats}{%
	\cl=Klasifikasi,%
	\ds=Dataset,%
	\tpr=TPR,%
	\fpr=FPR,%
	\tnr=TNR,%
	\prec=Presisi,%
	\fm=F-Measure,%
	\acc=Akurasi,%
	\auc=AUC%
}{%
	\DTLifnullorempty{\cl}
		{\\ \cline{2-9}}
		{\\ \hline \hline}
	\DTLifnullorempty{\cl}
		{}
		{
			\multirow{4}{2cm}{\cl}
		}
	& \ds
	& \DTLifnumeq{\tpr}{\maxtpr}{\textbf{\tpr}}{\tpr}
	& \DTLifnumeq{\fpr}{\minfpr}{\textbf{\fpr}}{\fpr}
	& \DTLifnumeq{\tnr}{\maxtnr}{\textbf{\tnr}}{\tnr}
	& \DTLifnumeq{\prec}{\maxprec}{\textbf{\prec}}{\prec}
	& \DTLifnumeq{\fm}{\maxfm}{\textbf{\fm}}{\fm}
	& \DTLifnumeq{\acc}{\maxacc}{\textbf{\acc}}{\acc}
	& \DTLifnumeq{\auc}{\maxauc}{\textbf{\auc}}{\auc}
}
\\
\hline
\end{tabular}
\label{tab:stats}
\end{table}

\DTLloaddb{runtimes}{../result/runtimes.csv}

\begin{table}[htbp]
\caption{Kecepatan pelatihan model}
\centering
\footnotesize
\begin{tabular}{p{4cm} p{4cm} r}
\hline
\textbf{Klasifikasi} &
\textbf{Dataset} &
\textbf{Waktu (menit)}
\DTLforeach*{runtimes}{%
		\cl=Klasifikasi,
		\ds=Dataset,
		\time=Waktu (menit)%
}{%
	\DTLifnullorempty{\cl}
		{\\ \cline{2-3}}
		{\\ \hline \hline}
	\DTLifnullorempty{\cl}
		{}
		{
			\multirow{3}{4cm}{\cl}
		}
	& \ds
	& \time
}
\\
\hline
\end{tabular}
\label{tab:runtimes}
\end{table}

Klasifikasi CRF LNSMOTE dengan 200 tingkat 1 pohon memberikan nilai TPR paling
tinggi yaitu $0,9904$ tapi dengan nilai FPR yang paling tinggi yaitu $0,8558$
dan TNR yang paling rendah yaitu $0,1442$ di antara model yang lainnya.
Kebalikannya, pengklasifikasi RF tanpa sampel ulang memberikan nilai TNR paling
tinggi yaitu $0,9988$ dan nilai FPR paling rendah yaitu $0,0012$.
Untuk presisi, RF tanpa sampel ulang memberikan nilai tertinggi yaitu $0,9450$,
dan nilai terendah diberikan oleh klasifikasi CRF LNSMOTE dengan 200 tingkat 1
pohon.
Untuk nilai \textit{F-Measure}, nilai tertinggi diberikan oleh klasifikasi CRF
tanpa sampel ulang dengan 50 tingkat dan 4 pohon yaitu $0,5353$, dengan nilai
terendah diberikan oleh klasifikasi CRF LNSMOTE 200 tingkat 1 pohon.
Untuk nilai akurasi tertinggi didapat dengan klasifikasi RF LNSMOTE dengan
yang terendah diberikan oleh klasifikasi CRF LNSMOTE 200 tingkat 1 pohon.
Klasifikasi dengan nilai AUC tertinggi yaitu CRF SMOTE 100 tingkat 2 pohon
dengan yang terendah diberikan oleh klasifikasi CRF tanpa sampel ulang dengan
100 tingkat 2 pohon.

Dari segi kecepatan pelatihan model, pengklasifikasi CRF lebih cepat dari RF
baik pada semua dataset pelatihan.
Sebagai pembanding, dapat dilihat pada klasifikasi RF dan CRF 50 tingkat 4
pohon.
CRF tanpa sampel ulang lebih cepat 11 kali daripada RF, dan pada sampel ulang
SMOTE dan LNSMOTE, klasifikasi CRF 1,6 kali lebih cepat daripada RF.

\chapter{Analisis}

Hasil pelatihan dan pengujian model dianalisis dalam tiga bagian.
Pertama, analisis dari metode sampel ulang LNSMOTE, kedua, analisis terhadap
pengklasifikasi CRF, dan yang terakhir yaitu analisis terhadap hasil
keseluruhan.

\section{Analisis Sampel-ulang}

Pengaruh sampel ulang SMOTE dan LNSMOTE berbeda pada klasifikasi RF dan CRF.
Pada klasifikasi RF, sampel ulang dengan LNSMOTE lebih baik dari SMOTE dengan
meningkatnya nilai akurasi $0,4\%$.
Pada klasifikasi CRF, metode SMOTE secara keseluruhan memberikan performansi
lebih baik daripada LNSMOTE dengan nilai AUC tertinggi didapat dengan CRF 100
tingkat 2 pohon.
Perbedaan  sampel ulang dengan SMOTE dan LNSMOTE yaitu meningkatkan nilai
\textit{F-Measure} dan akurasi pada RF, sedangkan pada CRF sebaliknya, seperti
yang terlihat pada kolom \textit{F-Measure} dan akurasi di tabel
\ref{tab:stats}.
Kelemahan dari LNSMOTE yaitu meningkatkan nilai FPR dari
klasifikasi lebih tinggi dari SMOTE, seperti yang terlihat pada kurva ROC pada
gambar \ref{fig:roc} (c) dan (d).

\label{fig:roc}
\begin{figure}[htbp]
\centering
\begin{tikzpicture}
	\pgfplotsset{
		small,
	}
	\matrix{
		\begin{axis}[
			title=(a),
			ylabel=$TPR$,
			legend columns=-1,
			legend entries={Tanpa sampel ulang\ ,SMOTE\ ,LN-SMOTE},
			legend to name=roc_legend
		]
			\addplot table[
				x={FPR},
				y={TPR},
			]
			{../result/rf.csv};
			\addplot table[
				x={FPR},
				y={TPR},
			]
			{../result/rf_smote.csv};
			\addplot table[
				x={FPR},
				y={TPR},
			]
			{../result/rf_lnsmote.csv};
		\end{axis}
		&
		\begin{axis}[
			title=(b),
		]
			\addplot table[
				x={FPR},
				y={TPR},
			]
			{../result/crf_200_1.csv};
			\addplot table[
				x={FPR},
				y={TPR},
			]
			{../result/crf_200_1_smote.csv};
			\addplot table[
				x={FPR},
				y={TPR},
			]
			{../result/crf_200_1_lnsmote.csv};
			title=(b),
		\end{axis}
		\\
		\begin{axis}[
			title=(c),
			xlabel=$FPR$,
			ylabel=$TPR$,
		]
			\addplot table[
				x={FPR},
				y={TPR},
			]
			{../result/crf_100_2.csv};
			\addplot table[
				x={FPR},
				y={TPR},
			]
			{../result/crf_100_2_smote.csv};
			\addplot table[
				x={FPR},
				y={TPR},
			]
			{../result/crf_100_2_lnsmote.csv};
		\end{axis}
		&
		\begin{axis}[
			title=(d),
			xlabel=$FPR$,
		]
			\addplot table[
				x={FPR},
				y={TPR},
			]
			{../result/crf_50_4.csv};
			\addplot table[
				x={FPR},
				y={TPR},
			]
			{../result/crf_50_4_smote.csv};
			\addplot table[
				x={FPR},
				y={TPR},
			]
			{../result/crf_50_4_lnsmote.csv};
		\end{axis}
		\\
	};
\end{tikzpicture}
\ref{roc_legend}
\caption{
Kurva ROC untuk klasifikasi RF dan CRF pada tiga dataset yaitu yang
tanpa sampel ulang, yang disampel ulang dengan SMOTE dan LN-SMOTE.
(a) RF dengan 200 pohon
(b) CRF dengan 200 tingkat 1 pohon
(c) CRF dengan 100 tingkat 2 pohon
(d) CRF dengan 50 tingkat 4 pohon
}
\end{figure}


\section{Analisis CRF}

Fokus dari pengklasifikasi CRF yaitu pada pembelajaran sampel negatif.
Pada setiap tingkatan CRF, model diuji ulang dengan semua sampel negatif, yang
didapat dari proses tingkatan sebelumnya, dan hasil yang negatif dihapus dari
sampel pelatihan dan dijadikan sebagai dataset uji pada tingkatan selanjutnya.
Berbeda dengan RF, yang mana tidak ada sampel yang dihapus saat
pelatihan, CRF membuat dataset yang tadinya condong pada kelas mayoritas
sedikit demi sedikit pada setiap tingkatan menjadi sama.
Sehingga, melakukan sampel ulang pada pengklasifikasi dengan CRF tidak membantu
dalam memperbaiki performansi modelnya, namun membuat akurasi dari model hasil
pelatihan menurun.
Hal ini bisa dilihat pada kurva PR-AUC pada gambar \ref{fig:prauc}.

Pada gambar \ref{fig:prauc} (b), yaitu CRF dengan 200 tingkatan dan 1 pohon,
performansi tanpa sampel ulang lebih baik dari yang lainnya, dengan nilai AUC
$0,8673$, dan performansi yang buruk dari dataset LNSMOTE.
Saat jumlah pohon pada setiap tingkatan dinaikan menjadi 2 (dan jumlah
tingkatan diturunkan menjadi 100 supaya jumlah keseluruhan pohon tetap 200),
sampel ulang SMOTE menghasilkan nilai AUC yang paling tinggi diantara
model lainnya yaitu $0,8694$, sementara LNSMOTE masih dengan performansi yang
paling rendah, seperti yang terlihat pada gambar \ref{fig:prauc} (b).
Pada pengujian terakhir dengan 50 tingkatan dan 4 pohon, nilai AUC tertinggi
memang didapat dari SMOTE, tapi rerata performansi keseluruhan dihasilkan dari
CRF tanpa sampel ulang, dengan nilai \textit{F-Measure} yang paling tinggi
diantara pengujian lainnya yaitu $0,5353$.

\label{fig:prauc}
\begin{figure}[htbp]
\centering
\begin{tikzpicture}
	\pgfplotsset{
		small,
	}
	\matrix{
		\begin{axis}[
			title=(a),
			ylabel=$Precision$,
			legend columns=-1,
			legend entries={Tanpa sampel ulang\ ,SMOTE\ ,LN-SMOTE},
			legend to name=rf_crf_prauc_legend
		]
			\addplot table[
				x={TPR},
				y={PREC},
			]
			{../result/rf.csv};
			\addplot table[
				x={TPR},
				y={PREC},
			]
			{../result/rf_smote.csv};
			\addplot table[
				x={TPR},
				y={PREC},
			]
			{../result/rf_lnsmote.csv};
		\end{axis}
		&
		\begin{axis}[
			title=(b),
		]
			\addplot table[
				x={TPR},
				y={PREC},
			]
			{../result/crf_200_1.csv};
			\addplot table[
				x={TPR},
				y={PREC},
			]
			{../result/crf_200_1_smote.csv};
			\addplot table[
				x={TPR},
				y={PREC},
			]
			{../result/crf_200_1_lnsmote.csv};
			title=(b),
		\end{axis}
		\\
		\begin{axis}[
			title=(c),
			xlabel=$Recall$,
			ylabel=$Precision$,
		]
			\addplot table[
				x={TPR},
				y={PREC},
			]
			{../result/crf_100_2.csv};
			\addplot table[
				x={TPR},
				y={PREC},
			]
			{../result/crf_100_2_smote.csv};
			\addplot table[
				x={TPR},
				y={PREC},
			]
			{../result/crf_100_2_lnsmote.csv};
		\end{axis}
		&
		\begin{axis}[
			title=(d),
			xlabel=$Recall$,
		]
			\addplot table[
				x={TPR},
				y={PREC},
			]
			{../result/crf_50_4.csv};
			\addplot table[
				x={TPR},
				y={PREC},
			]
			{../result/crf_50_4_smote.csv};
			\addplot table[
				x={TPR},
				y={PREC},
			]
			{../result/crf_50_4_lnsmote.csv};
		\end{axis}
		\\
	};
\end{tikzpicture}
\ref{rf_crf_prauc_legend}
\caption{
Kurva PR-AUC untuk klasifikasi RF dan CRF pada tiga dataset yaitu yang
tanpa sampel ulang, yang disampel ulang dengan SMOTE dan LN-SMOTE.
(a) RF dengan 200 pohon
(b) CRF dengan 200 tingkat 1 pohon
(c) CRF dengan 100 tingkat 2 pohon
(d) CRF dengan 50 tingkat 4 pohon
}
\end{figure}


\section{Analisis Pengklasifikasi Vandalisme}

\label{fig:rocpoints}
\begin{figure}[htbp]
\centering
\begin{tikzpicture}[framed]
	\pgfplotsset{
		width=12cm,
		xtick distance=0.1,
		ytick distance=0.1,
	}
	\begin{axis}[
		title=Performansi pengklasifikasi pada ruang ROC,
		xlabel=$FPR$,
		ylabel=$TPR$,
		legend columns=3,
		legend entries={
			RF,
			RF-SMOTE,
			RF-LNSMOTE,
			CRF-200-1,
			CRF-200-1-SMOTE,
			CRF-200-1-LNSMOTE,
			CRF-100-2,
			CRF-100-2-SMOTE,
			CRF-100-2-LNSMOTE,
			CRF-50-4,
			CRF-50-4-SMOTE,
			CRF-50-4-LNSMOTE,
		},
		legend to name=class_legend
	]
		\addplot[
			scatter,
			only marks,
			point meta=explicit symbolic,
			every mark/.append style={
				/tikz/mark size=3pt
			},
			scatter/classes={
				rf-noresampling={
					mark=text,
					text mark=$\Box$
				},
				rf-smote={
					mark=text,
					text mark=$\boxplus$
				},
				rf-lnsmote={
					mark=text,
					text mark=$\boxtimes$
				},
				crf-200-1={
					mark=text,
					text mark=$\triangle$
				},
				crf-200-1-smote={
					mark=text,
					text mark=$\triangleleft$
				},
				crf-200-1-lnsmote={
					mark=text,
					text mark=$\triangleright$
				},
				crf-100-2={
					mark=text,
					text mark=$\medcircle$
				},
				crf-100-2-smote={
					mark=text,
					text mark=$\oplus$
				},
				crf-100-2-lnsmote={
					mark=text,
					text mark=$\otimes$
				},
				crf-50-4={
					mark=text,
					text mark=$\Diamond$
				},
				crf-50-4-smote={
					mark=text,
					text mark=$\diamondplus$
				},
				crf-50-4-lnsmote={
					mark=text,
					text mark=$\diamondtimes$
				}
			},
		]
		table[
			x={FPR},
			y={TPR},
			meta=klasifikasi,
		]
		{../result/cm.csv};

		\addplot[
			gray
		]
		coordinates{
			(0,0)
			(1,1)
		};
		\addplot[
			gray
		]
		coordinates{
			(0,1)
			(1,0)
		};
	\end{axis}
\end{tikzpicture}
\newline
\ref{class_legend}
\newline
\caption{
Titik ROC untuk semua pengklasifikasi dan dataset.
CRF-200-1 yaitu klasifikasi CRF
dengan 200 tingkat dan 1 pohon, CRF-100-2 yaitu klasifikasi CRF dengan 100
tingkat dan 2 pohon, dan CRF-50-4 yaitu klasifikasi CRF dengan 50 tingkat dan 4
pohon.
}
\end{figure}


Untuk lebih mudah membandingkan semua pengklasifikasi, digunakan titik
ROC yang digambarkan pada grafik \ref{fig:rocpoints}.
Pada bagian kiri bawah adalah klasifikasi RF, berurutan dari bawah ke atas
yaitu RF tanpa sampel ulang ($\Box$) dengan nilai TPR $0,165$ dan FPR $0,001$,
RF pada sampel ulang SMOTE ($\boxplus$) dengan nilai TPR $0,207$ dan FPR
$0,004$, dan yang
paling atas adalah RF pada sampel ulang LNSMOTE ($\boxtimes$) dengan nilai TPR
$0,235$ dan FPR $0,005$.
RF memberikan performansi dengan TPR dan FPR yang paling rendah di antara semua
pengklasifikasi.
Sampel ulang SMOTE meningkatkan nilai TPR $0,25$ kali dan FPR 4 kali.
Sampel ulang LNSMOTE meningkatkan nilai TPR $0,4$ kali dan nilai FPR 5 kali.

Klasifikasi CRF 200 tingkat 1 pohon (CRF-200-1) memiliki nilai TPR paling
tinggi, dengan titik ROC berada pada bagian paling atas dari semua klasifikasi.
Berurutan dari kiri ke kanan yaitu CRF-200-1 tanpa sampel ulang ($\triangle$)
dengan TPR $0,966$ dan FPR $0,467$, CRF-200-1 SMOTE ($\triangleleft$) di tengah
dengan TPR $0,979$ dan FPR $0,63$, dan CRF-200-1 LNSMOTE ($\triangleright$)
pada ujung kanan dengan TPR $0,99$ dan FPR $0,855$.
Pengklasifikasi CRF-200-1 akan mengembalikan kelas sampel positif dengan benar
tetapi dengan galat yang juga tinggi.
Bisa dilihat juga pengaruh SMOTE dan LNSMOTE terhadap nilai FPR.
Sampel ulang SMOTE memberi pengaruh kenaikan TPR $0,1$ dan $0,4$ untuk FPR dari
dataset awal.
Sampel ulang LNSMOTE meningkatkan TPR $0,3$ dan FPR $0,8$.

Klasifikasi CRF 100 tingkat dengan 2 pohon (CRF-100-2) berada pada bagian atas
tengah dari ROC, yang ditandai dengan marka bulat.
CRF-100-2 tanpa sampel ulang ($\medcircle$) berada paling bawah dan kiri dengan
nilai TPR $0,812$ dan FPR $0,24$.
CRF-100-2 SMOTE ($\oplus$) berada di tengah dengan nilai AUC paling tinggi
dengan TPR $0,903$ dan FRP $0,3603$.
CRF-100-2 LNSMOTE ($\otimes$) dengan TPR paling tinggi dari keduanya yaitu TPR
$0,953$ dan FPR $0,585$.
Sampel ulang SMOTE meningkatkan TPR $0,11$ kali dan FPR $0,5$ kali.
Sampel ulang LNSMOTE meningkatkan TPR $0,17$ dan FPR $1,4$ kali.

Klasifikasi CRF 50 tingkat dengan 4 pohon (CRF-50-4) berada pada bagian tengah
kiri atas pada ROC, yang ditandai dengan marka wajik.
CRF-50-4 tanpa sampel ulang ($\Diamond$) menghasilkan nilai TPR $0,607$ dan FPR
$0,08$.
CRF-50-4 dengan sampel ulang ($\diamondplus$) SMOTE meningkatkan nilai TPR
menjadi $0,783$ dan meningkatkan nilai FPR menjadi $0,223$.
CRF-50-4 dengan sampel ulang LNSMOTE ($\diamondtimes$) meningkatan nilai TPR
menjadi $0,895$ dan juga meningkatkan nilai FPR menjadi $0,388$.
Sampel ulang dengan SMOTE meningkatkan nilai TPR $0,3$ kali dan FPR $1,75$
kali.
Sampel ulang dengan LNSMOTE meningkatkan nilai TPR $0,48$ kali dan FPR $3,75$.

