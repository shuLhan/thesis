Ulasan pekerjaan yang telah dilakukan pada minggu-minggu sebelumnya,

\begin{itemize}

\item \textbf{Minggu 3}. Eksplorasi fitur yang digunakan untuk klasifikasi
yang telah digunakan oleh makalah sebelumnya dan memperkirakan fitur yang akan
digunakan pada tesis ini.
Mengembangkan sebuah fitur membutuhkan waktu, sebagai tahap awal hanya akan
menggunakan 20 fitur yaitu 4 fitur metadata, 11 fitur teks, dan 5 fitur bahasa
yang diambil dari hasil analisis makalah Mola-Velasco \cite{mola2012wikipedia}.
Pada saat implementasi sudah berjalan dengan benar, baru nanti secara iteratif
ditambahkan fitur-fitur dari makalah lainnya satu persatu.

\item \textbf{Minggu 4}.
Telah membaca makalah tentang SMOTE \cite{chawla2002smote} dan LN-SMOTE
\cite{maciejewski2011local} dan membuat rangkuman tentang perbedaan dari kedua
metode tersebut berdasarkan kekurangan dan solusi yang diajukan dalam makalah.

\item \textbf{Minggu 5, 6}. Implementasi algoritma SMOTE.

\item \textbf{Minggu 7-9}.
	\begin{itemize}
	\item membaca makalah dari \textit{Random Forest} \cite{breiman2001random},
	\item mempelajari dan mengimplementasikan algoritma \textit{decisition
	tree} \textit{Classification and Regression Tree} (CART), yang kemudian
	mengarah ke
	\item mempelajari tentang Gini Index, yaitu ukuran untuk menghitung
	ketidakseimbangan diantara nilai dari frekuensi distribusi, yang
	digunakan untuk menghitung \textit{Gini's gain} pada atribut dari
	dataset yang digunakan untuk menentukan nilai pemisah.
	\item mengetahui tentang \textit{Stirling Numbers of Second Kind},
	yaitu jumlah sebuah set dapat dipartisi menjadi $ k $ objek tanpa
	adanya subset yang kosong.
	\item mengetahui dan membuat algoritma partisi untuk sebuah set dengan
	metode rekursif.
	\end{itemize}

\item \textbf{Minggu 10}.  Perbaikan algoritma CART.

\item \textbf{Minggu 11-14}. Menyelesaikan pekerjaan di luar tesis.
Kebetulan di saat yang bersamaan ada dua dari pekerjaan saya yang terdahulu
yang meminta untuk ditambahkan fitur, laporan, dan migrasi data.

\end{itemize}
