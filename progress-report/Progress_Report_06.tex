\documentclass[12pt,a4paper,titlepage,oneside]{report}

%%{{{ packages
%% font encoding.
\usepackage[utf8]{inputenc}
\usepackage[T1]{fontenc}
\usepackage{lmodern}
\usepackage{couriers}
%% Customize chapters.
\usepackage{titlesec}
%% bibliography.
\usepackage[
	backend=bibtex
,	style=ieee
,	sorting=nyt
]{biblatex}
%% \includegraphics{name}
\usepackage{graphicx}
%% \url{something}
\usepackage{url}
%% \multirow{package}{width}{text}
\usepackage{multirow}
%% \cellcolor
\usepackage[table]{xcolor}
%% \caption{title}
\usepackage{caption}
\usepackage{subcaption}
%% \forloop
\usepackage{forloop}
%% fix underfull on footnote with URL.
\usepackage{ragged2e}
%% source code listing
\usepackage{listings}
%% \printindex
\usepackage{makeidx}
%% table of content
\usepackage{tocloft}
%% text emphasis, including strikeout.
\usepackage[normalem]{ulem}
%% mathematics
\usepackage{mathtools}
%% arithmetic
\usepackage{calc}
%% algorithm
\usepackage{algorithm}
\usepackage{algpseudocode}
%% multiple columns
\usepackage{multicol}
%% Change the margin
\usepackage[a4paper]{geometry}
\geometry{
	a4paper,
	top=3cm,
	right=3cm,
	bottom=3cm,
	left=4cm
}
%%
\usepackage{parskip}
%% Package for reading CSV to database.
\usepackage{datatool}
%% Package for scatter and line plot.
\usepackage{dataplot}
%% Long table
\usepackage{longtable}
%% Tikz
\usepackage{tikz}
\usetikzlibrary{backgrounds}

\usepackage{pgfplots}

%% MnSymbol
\usepackage{MnSymbol}

%% Compact list.
\usepackage{paralist}
%%}}}

%%{{{ hyphenation: sorted in ascending.
%%
\hyphenation{
	Ja-nu-a-ri
	SIGKDD
	Wiki-pedia
	a-kan
	a-ku-ra-si
	ang-ka
	ba-gai-ma-na
	bayes-ian
	ber-kas
	ber-ma-sa-lah
	ber-mak-na
	bi-a-ya
	da-lam
	data-set
	de-ngan
	di-ha-sil-kan
	di-pi-lih
	di-sing-kat
	di-tam-bah-kan
	dis-krit
	fung-si
	ga-bung-an
	ke-las
	ke-le-mah-an
	ke-mung-ki-nan
	ke-tak-se-imbang-an
	lan-guage
	ma-yo-ri-tas
	me-laku-kan
	me-me-rik-sa
	me-mi-lih
	me-ne-rap-kan
	meng-a-pli-ka-si-kan
	me-ning-kat-kan
	me-nye-dia-kan
	me-nye-im-bang-kan
	me-sin
	me-thod
	me-to-de
	mem-vi-sua-li-sa-si
	meng-gu-na-kan
	meng-hi-lang-kan
	meng-hu-bung-kan
	meng-i-kut-kan
	meng-i-kuti
	meng-im-ple-men-ta-si-kan
	meng-in-di-ka-si-kan
	mi-sal-nya
	mung-kin
	o-ver-sam-pling
	pa-ra-lel
	pe-la-ti-han
	pe-mi-sah
	pe-nan-da
	pe-ne-li-ti-an
	pe-nu-li-san
	pe-nyun-ting
	pem-ban-ding
	pen-de-kat-an
	peng-kla-si-fi-ka-si
	peng-a-pli-ka-si-an
	per-for-man-si-nya
	po-ten-si-al
	pro-ba-bi-li-tas
	pro-ses
	sam-pel
	se-im-bang
	se-jum-lah
	sun-ting-an
	ting-kat
	un-der-sam-pling
	wa-lau-pun
}
%%}}}

%%{{{ override default latex setting.
%%

%% Space between paragraphs.
\setlength{\parskip}{1.5em}

%% Add dot to TOC.
\renewcommand{\cftsecleader}{\cftdotfill{\cftdotsep}}
\renewcommand{\contentsname}{}

\lstset{
	basicstyle=\scriptsize\ttfamily
,	breaklines=true
,	stringstyle=\scriptsize\ttfamily
,	numbers=left
,	numberstyle=\tiny\ttfamily
,	numbersep=5pt
,	tabsize=4
,	frame=single
}

\makeatletter
\def\lst@lettertrue{\let\lst@ifletter\iffalse}
\makeatother

%% Format chapter and section.
%%
%%% Set chapter name to Bab.
\titleformat{\chapter}[hang]
{\bfseries\large\centering}
{Bab \thechapter}
{1em}
{}

\titleformat{\section}[hang]
{\bfseries\large}
{\thesection}
{1em}{}

%% Set spacing for sections.
\titlespacing{\chapter}{0ex}{0ex}{1.5em}
\titlespacing{\section}{0ex}{0ex}{0em}

%% Set roman on chapter number.
\def\thechapter{\Roman{chapter}}

%% Alter latex default title on table.
\captionsetup[table]{name=Tabel}
\captionsetup[figure]{name=Gambar}

\renewcommand{\arraystretch}{1.5}
\setlength{\tabcolsep}{3pt}

%% Change bibliography title.
\defbibheading{bibliography}{\centerline{
	\textbf{DAFTAR PUSTAKA}}
}

%%% uncomment this to show overrule in black box
\overfullrule=2cm

%% Algorithmicx
\makeatletter
\renewcommand{\ALG@beginalgorithmic}{\footnotesize}
\makeatother

%% multicolumn setting
\setlength{\columnsep}{1cm}

%% pgfplots setting.
\pgfplotsset{
	/pgf/number format/read comma as period,
	/pgf/text mark as node=false,
	table/col sep=semicolon,
	xmax=1,
	xmin=0,
	ymax=1,
	ymin=0,
	xtick distance=0.2,
	ytick distance=0.2,
	grid=major,
	cycle list name=linestyles,
}
%%}}}

%%{{{ variables
%%
\newcommand{\mytitle}{Deteksi Vandalisme pada Wikipedia Bahasa Inggris menggunakan klasifikasi Cascaded Random Forest}
\newcommand{\myname}{Muhamad Sulhan}
\newcommand{\mysid}{23513014}
\newcommand{\myadvisorname}{Dwi Hendratmo Widyantoro}
\newcommand{\myadvisorid}{196812071994021001}
\newcommand{\mydept}{Program Studi Magister Informatika}
\newcommand{\itb}{Institut Teknologi Bandung}

%%% My images directory
\graphicspath{{../images/}}
\newcommand{\myitbcover}{ITB-logo-hitam}

%%% My bibligraphy file
\addbibresource{bibliography.bib}
%%}}}

%%{{{ document's meta-data
%%
\author{\myname}
\title{\mytitle}
%%}}}

%%{{{ document's macros
%%
%%% two column signature.
\def\myadvisorsig#1{%
	\vbox{\hsize=6cm
		\textbf{#1}\\
		\addvspace{2cm}%
		\hbox to \hsize{%
			\strut\hfil%
			\myadvisorname%
			\hfil%
		}
		\hrule\kern1ex
		\hbox to \hsize{%
			\strut\hfil%
			NIP\hspace{1ex}\myadvisorid%
			\hfil%
		}
	}
}

%%% one column signature.
\def\mysignature#1#2#3{%
	\vbox{
		\textbf{#1}\\
		\addvspace{2cm}%
		\hbox to \hsize{%
			\strut\hfil%
			{#2}%
			\hfil%
		}
		\makebox[6cm][c]{
			\hrulefill
		}
		\hbox to \hsize{%
			\strut\hfil%
			NIP\hspace{1ex}{#3}%
			\hfil%
		}
	}
}

%%% source code listing
\lstdefinelanguage{go}
{
	morekeywords={package,import,const,func,for,type,var,struct}
,	sensitive=true
,	morecomment=[l]{//}
,	morecomment=[s]{/*}{*/}
}
\lstdefinestyle{go}{%
	language=go
,	keywordstyle=\color{black}\bfseries
,	commentstyle=\color{gray}
,	breakatwhitespace=true
,	lineskip={-2.5pt}
}
\newcommand{\includecodego}[2][c]{
	\lstinputlisting[caption=#2,escapechar=,style=go]
		{/home/ms/go/work/src/github.com/shuLhan/#2}
}

%%% data listing
\lstdefinestyle{data}{%
	breakatwhitespace=false
,	breakautoindent=false
,	literate={\,}{}{0\discretionary{,}{}{,}},
}
\newcommand{\includedata}[2][c]{
	\lstinputlisting[caption=#2,style=data,linerange={1-10}]
		{/home/ms/go/work/src/github.com/shuLhan/#2}
}
%%}}}

%%
%% Create cover with parameters
%% 1: report number
%% 2: report weeks, in string
%% 3: report date, string
%%
\newcommand{\reportcover}[3]{
	\thispagestyle{empty}
	\begin{center}\textbf{%
		\mytitle
		\vfill
		Laporan Progres Tesis, Catatan, dan Pekerjaan Selanjutnya
		\linebreak
		Laporan ke-#1
		\linebreak
		Minggu #2
		\linebreak
		\linebreak
		#3
		\vfill
		Oleh \\
		\myname \\
		\mysid \\
		\vfill
		\uppercase{%
			Program Studi Magister Informatika \\
			Sekolah Teknik Elektro dan Informatika \\
			Institut Teknologi Bandung \\
			2015
		}
	}
	\end{center}
	\newpage
}

%%
%% Macro for advisor's signature
%%
\newcommand{\advisorsignature}{
	\vfill
	\begin{center}
		Diketahui oleh,
		\linebreak
		\linebreak
		\hbox to \hsize{%
			\myadvisorsig{Pembimbing,\quad}\hfil
		}
	\end{center}
}

%%
%% Macro for create schedule using parameter
%% 1: number of weeks has passed on task 1
%% 2: number of weeks has passed on task 2
%% 3: number of weeks has passed on task 3
%% 4: number of weeks has passed on task 4
%% 5: should we begin in newpage?
%%

% function to fill cell with color
\newcommand{\tand}{&}
\newcounter{cnt}
\newcommand{\fillcell}[1]{%
	\forloop{cnt}{0}{\value{cnt}<#1}{%
		{\cellcolor[gray]{0.7}} \tand
	}%
}
% function to create empty cell
\newcommand{\emptycell}[2]{%
	\forloop{cnt}{0}{\value{cnt}<#1}{%
		\tand
	}%
	\ifthenelse{#2 = 1}{\\}{\tand}%
}
% function to fill week in progress.
\newcommand{\progresscell}[1]{%
	\forloop{cnt}{0}{\value{cnt}<#1}{%
		{\cellcolor{red!80}} \tand
	}%
}

\newcommand{\schedule}[4]{
	\section{Penjadwalan}\label{sec:penjadwalan}

Tabel di bawah menampilkan jadwal yang direncanakan dalam pengembangan tesis dari bulan ke I, September 2015, sampai dengan bulan ke VI, Januari 2016.

Warna merah menandakan minggu yang telah lewat sampai minggu dari laporan progres ini, untuk warna abu-abu menandakan waktu pelaksanaan yang akan datang.

\newcounter{planone}
\newcounter{plantwo}
\newcounter{planthree}
\newcounter{planfour}

\newcounter{progressone}
\newcounter{progresstwo}
\newcounter{progressthree}
\newcounter{progressfour}

\setcounter{progressone}{#1}
\setcounter{progresstwo}{#2}
\setcounter{progressthree}{#3}
\setcounter{progressfour}{#4}

\setcounter{planone}{4 - \theprogressone}
\setcounter{plantwo}{17 - \theprogresstwo}
\setcounter{planthree}{14 - \theprogressthree}
\setcounter{planfour}{2 - \theprogressfour}

\begin{table}[h!]
	\centering
	{\footnotesize
	\begin{tabular}{|c|p{0.2\textwidth}
	|c|c|c|c
	|c|c|c|c
	|c|c|c|c
	|c|c|c|c
	|c|c|c|c
	|c|c|c|c|}
		\hline
		\multirow{2}{*}{No.}
			& \multirow{2}{*}{Kegiatan}
			& \multicolumn{4}{c|}{Bulan I}
			& \multicolumn{4}{c|}{Bulan II}
			& \multicolumn{4}{c|}{Bulan III}
			& \multicolumn{4}{c|}{Bulan IV}
			& \multicolumn{4}{c|}{Bulan V}
			& \multicolumn{4}{c|}{Bulan VI}\\
		\cline{3-26}
		& &
			1 & 2 & 3 & 4 &
			1 & 2 & 3 & 4 &
			1 & 2 & 3 & 4 &
			1 & 2 & 3 & 4 &
			1 & 2 & 3 & 4 &
			1 & 2 & 3 & 4\\
		\hline
		1 & Persiapan\ \  Data dan\ \ Lingkungan Penelitian &
			\progresscell{\theprogressone}
			\fillcell{\theplanone}
			\emptycell{19}{1}
		\hline
		2 & Implementasi dan Pengujian &
			\emptycell{2}{0}
			\progresscell{\theprogresstwo}
			\fillcell{\theplantwo}
			\emptycell{3}{1}
		\hline
		4 & Analisis &
			\emptycell{7}{0}
			\progresscell{\theprogressthree}
			\fillcell{\theplanthree}
			\emptycell{1}{1}
		\hline
		5 & Evaluasi &
			\emptycell{20}{0}
			\progresscell{\theprogressfour}
			\fillcell{\theplanfour}
			\emptycell{0}{1}
		\hline
	\end{tabular}
	}
	\caption{Jadwal penelitian tesis}
\end{table}
}




\begin{document}
\reportcover{V}{17}{12 Januari 2016}

\section{Pendahuluan}

\newpage
\section{Dataset Forensic Glass}
\label{appendix:dataset_glass}

Berikut 10 baris atas dan bawah dari dataset \textit{Forensic Glass}.

\begin{lstlisting}
1,1.52101,13.64,4.49,1.10,71.78,0.06,8.75,0.00,0.00,1
2,1.51761,13.89,3.60,1.36,72.73,0.48,7.83,0.00,0.00,1
3,1.51618,13.53,3.55,1.54,72.99,0.39,7.78,0.00,0.00,1
4,1.51766,13.21,3.69,1.29,72.61,0.57,8.22,0.00,0.00,1
5,1.51742,13.27,3.62,1.24,73.08,0.55,8.07,0.00,0.00,1
6,1.51596,12.79,3.61,1.62,72.97,0.64,8.07,0.00,0.26,1
7,1.51743,13.30,3.60,1.14,73.09,0.58,8.17,0.00,0.00,1
8,1.51756,13.15,3.61,1.05,73.24,0.57,8.24,0.00,0.00,1
9,1.51918,14.04,3.58,1.37,72.08,0.56,8.30,0.00,0.00,1
10,1.51755,13.00,3.60,1.36,72.99,0.57,8.40,0.00,0.11,1
...
204,1.51658,14.80,0.00,1.99,73.11,0.00,8.28,1.71,0.00,7
205,1.51617,14.95,0.00,2.27,73.30,0.00,8.71,0.67,0.00,7
206,1.51732,14.95,0.00,1.80,72.99,0.00,8.61,1.55,0.00,7
207,1.51645,14.94,0.00,1.87,73.11,0.00,8.67,1.38,0.00,7
208,1.51831,14.39,0.00,1.82,72.86,1.41,6.47,2.88,0.00,7
209,1.51640,14.37,0.00,2.74,72.85,0.00,9.45,0.54,0.00,7
210,1.51623,14.14,0.00,2.88,72.61,0.08,9.18,1.06,0.00,7
211,1.51685,14.92,0.00,1.99,73.06,0.00,8.40,1.59,0.00,7
212,1.52065,14.36,0.00,2.02,73.42,0.00,8.44,1.64,0.00,7
213,1.51651,14.38,0.00,1.94,73.61,0.00,8.48,1.57,0.00,7
214,1.51711,14.23,0.00,2.08,73.36,0.00,8.62,1.67,0.00,7
\end{lstlisting}

\newpage
\section{Sumber Kode Random Forest}
\label{appendix:sumber_kode_randomforest}

\includecodego{go-mining/classifiers/ensemble/randomforest.go}

\newpage
\section{Sumber Kode \textit{Cascaded Random Forest}}
\label{appendix:sumber_kode_cascadedrf}

\includecodego{go-mining/classifiers/cascadedrf/cascadedrf.go}

\includecodego{go-mining/classifiers/cascadedrf/stage.go}


\section{Laporan Singkat Progres}

Selama seminggu ini saya telah,
\begin{itemize}
\item mempelajari tentang kurva ROC dan UAC,
\item mempelajari makalah \textit{Cascaded Random Forest}
\cite{baumann2013cascaded},
\item mengimplementasikan sebagian algoritma \textit{Cascaded Random Forest}.
\end{itemize}

\section{Catatan}

\input{notes/report_06_roc}
\subsection{Cascaded Random Forests}

Berikut rangkuman dari makalah Baumann dkk. \cite{baumann2013cascaded},

\begin{itemize}
\item Sebuah cascade terdiri dari beberapa \textit{stage} dengan kompleksitas
yang meningkat.
\item Setiap \textit{stage} paling kurang memiliki satu pohon.
\item Pohon ditambahkan ke stage sampai \textit{true positive rate} dan
\textit{true negative rate} dicapai.
\item Label dari sampel ditentukan oleh probabilitas kelas dari \textit{tree}
yang berbeda di stage $S$.
\item Untuk mengurangi pengaruh \textit{stage} yang performansinya rendah maka
ditambahkan faktor bobot $\alpha$ untuk setiap \textit{stage}, yaitu,
\[
	\alpha_{s} = exp(fmeasure)
\]
yang mana $fmeasure$ didapat dari,
\[
	fmeasure = 2 \cdot \frac{precision \cdot recall}{precision + recall}
\]
\item $\alpha_{s}$ secara linear dinormalkan menjadi rentang antara 0 dan 1.
\item Bobot dari \textit{stage} yang memiliki performansi rendah dikurangi
supaya pengaruh mereka pada mayoritas voting berkurang.
\end{itemize}

\begin{algorithm}[t!]
\caption{Algoritma Cascaded Random Forest}
\label{alg:cascadedrf}
\begin{algorithmic}
	\Require \\
	S: jumlah stage \\
	T: maksimum jumlah tree di setiap stage \\
	mintp: ambang batas untuk nilai minimum true positive \\
	mintn: ambang batas untuk nilai minimum true negative \\
	\Function{CascadedRandomForest}{$S, T, mintp, mintn$}
		\For{$ s = 1 \to S $}
			\For{$ t = 1 \to T $}
				\State buat \textit{tree}
				\If{$TP_{s,t} > mintp $ AND $ TN_{s,t} > mintn $}
					\State stage selesai.
				\EndIf
			\EndFor
			\State Hitung bobot $ \alpha = exp(fmeasure) $
			\State Hapus \textit{true-negative}
			\State Isi ulang dengan \textit{false-positive}
		\EndFor
	\EndFunction
\end{algorithmic}
\end{algorithm}

Dari algoritma Cascaded Random Forest \ref{alg:cascadedrf}, ada beberapa
pertanyaan untuk makalah tersebut,
\begin{itemize}
\item Untuk jenis dataset multi-kelas bukankah nilai TN sama dengan TP? Yang
mana TN = 1 - FP Rate.
\item Apa yang harus dilakukan jika maksimum jumlah tree tercapai tapi nilai
$TP_{s,t}$ dan $TN_{s,t}$ tidak mencapai $mintp$ dan $mintn$?
\item Apa yang dimaksud dengan "Hapus \textit{true negative}"? Apakah dihapus
stagenya? Treenya? Samplenya? Makalah tidak secara eksplisit menjelaskan
bagian ini
\item Apa yang dimaksud dengan "Isi ulang dengan \textit{false-positive}"?
\end{itemize}

Karena masih ada yang belum dipahami pada algoritmanya, implementasi
\textit{Cascaded Random Forest} untuk pengklasifikasi dataset multi-kelas belum
selesai.


\clearpage
\section{Pekerjaan Selanjutnya}

Berikut garis besar tahap yang akan dikerjakan untuk pekerjaan selanjutnya,

\begin{itemize}
\item perbaikan implementasi \textit{Cascaded Random Forest}.
\item Implementasi algoritma LN-SMOTE.
\item Implementasi fitur vandalisme di Wikipedia.
\item Menerapkan algoritma SMOTE dan LN-SMOTE pada hasil penghitungan fitur.
\item Menerapkan \textit{Random Forest} pada hasil \textit{resampling} SMOTE
dan LN-SMOTE.
\item Menerapkan algoritma \textit{Cascade Random Forest} dari hasil SMOTE dan
LN-SMOTE.
\item Analisis hasil dari klasifikasi \textit{Random Forest} dan
\textit{Cascaded Random Forest}.
\end{itemize}

\clearpage
\schedule{4}{14}{0}{0}

\advisorsignature

\clearpage
\addcontentsline{toc}{section}{Daftar Referensi}
\printbibliography

\newpage
\appendix
\newpage
\section{Sumber Kode \textit{Cascaded Random Forest}}
\label{appendix:sumber_kode_cascadedrf}

\includecodego{go-mining/classifiers/cascadedrf/cascadedrf.go}

\includecodego{go-mining/classifiers/cascadedrf/stage.go}


\end{document}
