\subsection{Cascaded Random Forests}

Berikut rangkuman dari makalah Baumann dkk. \cite{baumann2013cascaded},

\begin{itemize}
\item Sebuah cascade terdiri dari beberapa \textit{stage} dengan kompleksitas
yang meningkat.
\item Setiap \textit{stage} paling kurang memiliki satu pohon.
\item Pohon ditambahkan ke stage sampai \textit{true positive rate} dan
\textit{true negative rate} dicapai.
\item Label dari sampel ditentukan oleh probabilitas kelas dari \textit{tree}
yang berbeda di stage $S$.
\item Untuk mengurangi pengaruh \textit{stage} yang performansinya rendah maka
ditambahkan faktor bobot $\alpha$ untuk setiap \textit{stage}, yaitu,
\[
	\alpha_{s} = exp(fmeasure)
\]
yang mana $fmeasure$ didapat dari,
\[
	fmeasure = 2 \cdot \frac{precision \cdot recall}{precision + recall}
\]
\item $\alpha_{s}$ secara linear dinormalkan menjadi rentang antara 0 dan 1.
\item Bobot dari \textit{stage} yang memiliki performansi rendah dikurangi
supaya pengaruh mereka pada mayoritas voting berkurang.
\end{itemize}

\begin{algorithm}[t!]
\caption{Algoritma Cascaded Random Forest}
\label{alg:cascadedrf}
\begin{algorithmic}
	\Require \\
	S: jumlah stage \\
	T: maksimum jumlah tree di setiap stage \\
	mintp: ambang batas untuk nilai minimum true positive \\
	mintn: ambang batas untuk nilai minimum true negative \\
	\Function{CascadedRandomForest}{$S, T, mintp, mintn$}
		\For{$ s = 1 \to S $}
			\For{$ t = 1 \to T $}
				\State buat \textit{tree}
				\If{$TP_{s,t} > mintp $ AND $ TN_{s,t} > mintn $}
					\State stage selesai.
				\EndIf
			\EndFor
			\State Hitung bobot $ \alpha = exp(fmeasure) $
			\State Hapus \textit{true-negative}
			\State Isi ulang dengan \textit{false-positive}
		\EndFor
	\EndFunction
\end{algorithmic}
\end{algorithm}

Dari algoritma Cascaded Random Forest \ref{alg:cascadedrf}, ada beberapa
pertanyaan untuk makalah tersebut,
\begin{itemize}
\item Untuk jenis dataset multi-kelas bukankah nilai TN sama dengan TP? Yang
mana TN = 1 - FP Rate.
\item Apa yang harus dilakukan jika maksimum jumlah tree tercapai tapi nilai
$TP_{s,t}$ dan $TN_{s,t}$ tidak mencapai $mintp$ dan $mintn$?
\item Apa yang dimaksud dengan "Hapus \textit{true negative}"? Apakah dihapus
stagenya? Treenya? Samplenya? Makalah tidak secara eksplisit menjelaskan
bagian ini
\item Apa yang dimaksud dengan "Isi ulang dengan \textit{false-positive}"?
\end{itemize}

Karena masih ada yang belum dipahami pada algoritmanya, implementasi
\textit{Cascaded Random Forest} untuk pengklasifikasi dataset multi-kelas belum
selesai.
