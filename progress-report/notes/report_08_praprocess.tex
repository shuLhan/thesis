\subsection{Persiapan dataset pelatihan}

Dataset yang digunakan untuk pelatihan yaitu PAN-WVC-2010
\cite{potthast:2010b}.
Dataset PAN-WVC-2010 terbagi menjadi dua yaitu dataset suntingan dan dataset
anotasi yang berisi hasil klasifikasi.
Kedua dataset memiliki format yang sama yaitu menggunakan \textit{Coma
Separated Value} (CSV).
Dataset suntingan memiliki atribut sebagai berikut,

\begin{itemize}
	\item \textbf{editid}, format angka, berisi identifikasi (ID) unik dari setiap suntingan.
	\item \textbf{editor}, format string, berisi nama penyunting.
	\item \textbf{oldrevisionid}, format angka, berisi ID untuk suntingan lama.
	\item \textbf{newrevisionid}, format angka, berisi ID untuk suntingan baru.
	\item \textbf{diffurl}, format string, berisi URL yang mengacu pada perbedaan suntingan baru dengan lama.
	\item \textbf{edittime}, format string, berisi tanggal dan pukul sesuai dengan ISO 8601.
	\item \textbf{editcomment}, format string, berisi komentar yang ditambahkan oleh penyunting saat menyimpan hasil suntingan.
	\item \textbf{articleid}, format angka, berisi ID unik dari artikel.
	\item \textbf{articletitle}, format string, berisi judul dari artikel yang disunting.
\end{itemize}

Dataset anotasi memiliki atribut sebagai berikut,
\begin{itemize}
	\item \textbf{editid}, format angka, mengacu pada ID yang sama pada
	dataset suntingan.
	\item \textbf{class}, format string, berisi tipe suntingan yang
	bernilai "regular" yang menyatakan bahwa suntingan tersebut bukan
	vandalisme, dan "vandalism" yang menyatakan bahwa suntingan tersebut
	adalah vandalisme.
	\item \textbf{annotators}, format angka, berisi jumlah orang yang
	menandai (penanda) bahwa suntingan dengan ID tersebut termasuk ke dalam
	kelas "regular" atau "vandalism".
	\item \textbf{totalannotators}, format angka, berisi jumlah total penanda yang memeriksa suntingan.
\end{itemize}

Kedua dataset kemudian digabung untuk menghasilkan atribut \textit{editid},
\textit{class}, \textit{oldrevisionid}, \textit{newrevisionid},
\textit{edittime}, \textit{editor}, \textit{articletitle}, \textit{deletions},
dan \textit{additions}.
Atribut \textit{additions} berisi teks yang dihapus dalam revisi yang lama
Atribut \textit{deletions} berisi teks yang ditambahkan dalam revisi yang baru.

Pemrosesan selanjutnya yaitu membuat berkas revisi yang bersih dari sintaks
wiki.
Setiap berkas revisi dibaca kemudian dilakukan pembersihan berikut,

\begin{itemize}
\item penghapusan URI atau \textit{links} yang berawalan dengan
\textit{http://}, \textit{https://}, \textit{ftp://}, dan \textit{ftps}.
\item menghapus \textit{mark-up} wiki yaitu baris yang berisi awalan dan
akhiran berikut,
	\begin{itemize}
	\item \texttt{[[Category:} dan \texttt{]]}
	\item \texttt{[[:Category:} dan \texttt{]]}
	\item \texttt{[[File:} dan \texttt{]]}
	\item \texttt{[[Help:} dan \texttt{]]}
	\item \texttt{[[Image:} dan \texttt{]]}
	\item \texttt{[[Special:} dan \texttt{]]}
	\item \texttt{[[Wikipedia:} dan \texttt{]]}
	\item \texttt{\{\{DEFAULTSORT:} dan \texttt{\}\}}
	\item \texttt{\{\{Template:} dan \texttt{\}\}}
	\item \texttt{<ref} dan \texttt{/>}
	\end{itemize}
\item mengganti karakter dan token berikut dengan karakter kosong (spasi):
\texttt{[}, \texttt{]}, \texttt{\{}, \texttt{\}}, \texttt{|}, \texttt{=},
\texttt{\#}, \texttt{'s}, \texttt{'}, \texttt{<ref>}, \texttt{</ref>},
\texttt{<br />}, \texttt{<br/>}, \texttt{<br>}, \texttt{<nowiki>},
\texttt{</nowiki>}, \texttt{\&nbsp;}.
\item menghapus karakter kosong yang berlebihan.
\end{itemize}

Hasil pembersihan disimpan dalam direktori yang berbeda tapi dengan nama yang
sama.
Hasil ini nanti berguna pada saat melakukan komputasi penghitungan fitur
bahasa.


\subsection{Persiapan dataset pengujian}

Dataset yang digunakan untuk pengujian yaitu PAN-WVC-2011
\cite{potthast:2010b}.
Korpus PAN-WVC-2011 terdiri dari tiga bahasa yaitu Inggris, Jerman, dan
Spanyol, yang digunakan untuk pelatihan hanya yang bahasa Iggris.
Dataset asli memiliki atribut yang sama dengan PAN-WVC-2010, yaitu
\textit{editid},
\textit{editor},
\textit{oldrevisionid},
\textit{newrevisionid},
\textit{diffurl},
\textit{class},
\textit{annotators},
\textit{totalannotators},
\textit{edittime},
\textit{editcomment},
\textit{articleid}, dan
\textit{articletitle}.

Atribut \textit{annotators} dan \textit{totalannotators} dihilangkan dan
ditambah dengan dua atribut baru yaitu \textit{deletions} yang berisi teks yang
dihapus pada revisi lama, dan \textit{additions} yang berisi teks yang
ditambahkan pada revisi yang baru.
Nilai dari atribut \textit{class} diganti dari string menjadi integer, yaitu
dari "vandalism" \texttt{1}
dan "regular" menjadi \texttt{0}.

Untuk proses pembersihan data dilakukan prosedur yang sama seperti pada
PAN-WVC-2010.
