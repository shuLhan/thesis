\documentclass[12pt,a4paper,titlepage]{article}

%%{{{ document's packages
%%
\usepackage{graphicx}		% \includegraphics{name}
\usepackage{multirow}		% \multirow{package}{width}{text}
\usepackage{forloop}		% \forloop
\usepackage{caption}		% \caption{title}
\usepackage[table]{xcolor}
\usepackage{url}
%% fix underfull on footnote with URL.
\usepackage{ragged2e}
%%% uncomment this to show overrule in black box
\overfullrule=2cm
%%
\hyphenation{
	ber-kas
	data-set
	pe-nan-da
	sun-ting-an
}

%% reconfigure table.
%%% Alter latex default title
\captionsetup[table]{name=Tabel}
\renewcommand{\arraystretch}{1.5}
\setlength{\tabcolsep}{3pt}

%%}}}

%%{{{ document's variables
%%
\newcommand{\mytitle}{Deteksi Vandalisme pada Wikipedia Bahasa Inggris menggunakan klasifikasi Cascaded Random Forest}
\newcommand{\myname}{Muhamad Sulhan}
\newcommand{\mysid}{23513014}
%%}}}

%%{{{ document's meta-data
%%
\author{\myname}
\title{\mytitle}
%%}}}

%%{{{ document's functions
%%
%%% signature
\def\myadvisorsig#1{%
	\vbox{\hsize=6cm
		\textbf{#1}\\
		\addvspace{2cm}%
		\hbox to \hsize{%
			\strut\hfil%
			Dwi Hendratmo Widyantoro%
			\hfil%
		}
		\hrule\kern1ex
		\hbox to \hsize{%
			\strut\hfil%
			NIP\hspace{1ex}196812071994021001%
			\hfil%
		}
	}
}
%%}}}

%%
%% DOCUMENT
%%
\begin{document}

%% {{{ COVER
\thispagestyle{empty}
\begin{center}
	\textbf{%
		\mytitle
		\vfill
		Laporan Progres Tesis, Catatan dan Pekerjaan Selanjutnya
		\linebreak
		\linebreak
		11 September 2015
		\vfill
		Oleh \\
		\myname \\
		\mysid \\
		\vfill
		\uppercase{%
			Program Studi Magister Informatika \\
			Sekolah Teknik Elektro dan Informatika \\
			Institut Teknologi Bandung \\
			2015
		}
	}
\end{center}
\newpage
%% }}}

%%{{{ Laporan Progress
\section{Laporan Progres}

Bagian ini mencatat progres pekerjaan yang telah dilakukan setiap minggu secara ringkas.
Rincian dari progres berada pada bagian Catatan.

\subsection{Minggu I}

\begin{itemize}
	\item \textbf{Data korpus untuk PAN-WVC-11 telah diunduh.}
Ukuran berkas terkompres yaitu 371 MB, dan tidak terkompres sebesar 1,1 GB.
Jumlah suntingan yaitu 9985, dengan 8734 diantaranya adalah suntingan regular dan 1251 (sekitar 12\%) adalah vandalisme.

	\item \textbf{Perbaikan proposal tesis.}
\end{itemize}

\subsection{Minggu II}

\begin{itemize}
	\item \textbf{Data korpus untuk PAN-WVC-10 telah diunduh.}
Ukuran berkas terkompres yaitu 439 MB, dan tidak terkompres sebesar 1,3 GB.
Jumlah dataset yaitu 32439 revisi, dengan 30045 diantaranya adalah suntingan regular dan 2394 (sekitar 7\%) diantaranya adalah vandalisme.

	\item \textbf{Pemasangan \textit{Database Management System} (DBMS).}
DBMS yang digunakan adalah MariaDB, yang merupakan \textit{fork} dari MySQL.
DBMS nantinya digunakan untuk menyimpan data ekspor dari hasil \textit{dump} riwayat penyuntingan Wikipedia.
Wikimedia (perangkat lunak wiki yang digunakan oleh Wikipedia), menggunakan MySQL sebagai DBMS, itulah kenapa MariaDB dipilih dalam pengembangan tesis ini.

	\item \textbf{Data \textit{dump} dari riwayat wikipedia telah diunduh.} \footnote{\RaggedRight\url{http://dumps.wikimedia.org/enwiki/20150805/enwiki-20150805-stub-meta-history1.xml.gz}}
Berkas hanya satu dari 27 berkas \textit{dump} yang memiliki total 43,5 GB.
Ukuran data terkompres yaitu 503 MB dan tidak terkompres 3,3 GB.
Format data berupa xml.
\end{itemize}
%%}}}

%%{{{ Catatan
\newpage
\section{Catatan}

Bagian ini berisi catatan yang didapat pada setiap minggu selama mengerjakan tesis.

\subsection{Minggu I dan II}

Korpus PAN-WVC-10 terbagi menjadi dua yaitu dataset suntingan dan dataset anotasi yang berisi hasil klasifikasi.
Kedua dataset memiliki format yang sama yaitu menggunakan \textit{Coma Separated Value} (CSV).
Dataset suntingan memiliki atribut sebagai berikut,
\begin{itemize}
	\item \textbf{editid}, format angka, berisi identifikasi (ID) unik dari setiap suntingan.
	\item \textbf{editor}, format string, berisi nama penyunting.
	\item \textbf{oldrevisionid}, format angka, berisi ID untuk suntingan lama.
	\item \textbf{newrevisionid}, format angka, berisi ID untuk suntingan baru.
	\item \textbf{diffurl}, format string, berisi URL yang mengacu pada perbedaan suntingan baru dengan lama.
	\item \textbf{edittime}, format string, berisi tanggal dan pukul sesuai dengan ISO 8601.
	\item \textbf{editcomment}, format string, berisi komentar yang ditambahkan oleh penyunting saat menyimpan hasil suntingan.
	\item \textbf{articleid}, format angka, berisi ID unik dari artikel.
	\item \textbf{articletitle}, format string, berisi judul dari artikel yang disunting.
\end{itemize}

Dataset anotasi memiliki atribut sebagai berikut,
\begin{itemize}
	\item \textbf{editid}, format angka, mengacu pada ID yang sama pada dataset suntingan.
	\item \textbf{class}, format string, berisi tipe suntingan yang bernilai "regular" yang menyatakan bahwa suntingan tersebut bukan vandalisme, dan "vandalism" yang menyatakan bahwa suntingan tersebut adalah vandalisme.
	\item \textbf{annotators}, format angka, berisi jumlah orang yang menandai (penanda) bahwa suntingan dengan ID tersebut termasuk ke dalam kelas "regular" atau "vandalism".
	\item \textbf{totalannotators}, format angka, berisi jumlah total penanda yang memeriksa suntingan.
\end{itemize}

Korpus PAN-WVC-11 hanya memiliki satu dataset untuk Bahasa Inggris yaitu dataset suntingan, dengan atribut yang menggabungkan dataset suntingan dan dataset anotasi pada PAN-WVC-10.

Rencananya, korpus PAN-WVC-10 dijadikan sebagai dataset untuk \textit{resampling} dan pelatihan klasifikasi karena kuantitasnya lebih banyak daripada PAN-WVC-11.
Korpus PAN-WVC-11 dijadikan sebagai dataset untuk pengujian model.
%%}}}

%%{{{ Pekerjaan Selanjutnya
\section{Pekerjaan Selanjutnya}

Bagian ini berisi pekerjaan yang akan dilakukan oleh penulis pada minggu selanjutnya dari tanggal laporan progres ini.
Sub-bagian terurut menaik, dengan minggu terbesar atau minggu depan berada pada urutan paling atas dan minggu-minggu yang telah terlewati di bawahnya, untuk mempermudah pembaca.

\subsection{Minggu III}

Mempelajari dan mengimplementasikan algoritma LN-SMOTE pada bahasa pemrograman Go.

%%}}}

%%{{{ SECTION: Penjadwalan
%%
\newpage
\section{Penjadwalan}\label{sec:penjadwalan}

Tabel \ref{tab:jadwal} menampilkan jadwal yang direncanakan dalam pengembangan tesis dari bulan ke I, September 2015, sampai dengan bulan ke VI, Januari 2016.
Warna merah menandakan minggu dari laporan progres ini, untuk warna abu-abu menandakan waktu pelaksanaan yang akan datang.

% function to fill cell with color
\newcommand{\tand}{&}
\newcounter{cnt}
\newcommand{\fillcell}[1]{%
	\forloop{cnt}{0}{\value{cnt}<#1}{%
		{\cellcolor[gray]{0.7}} \tand
	}%
}
% function to create empty cell
\newcommand{\emptycell}[2]{%
	\forloop{cnt}{0}{\value{cnt}<#1}{%
		\tand
	}%
	\ifthenelse{#2 = 1}{\\}{\tand}%
}
% function to fill week in progress.
\newcommand{\progresscell}[1]{%
	\forloop{cnt}{0}{\value{cnt}<#1}{%
		{\cellcolor{red!80}} \tand
	}%
}

\begin{table}[h!]
	\centering
	{\footnotesize
	\begin{tabular}{|c|p{0.2\textwidth}
	|c|c|c|c
	|c|c|c|c
	|c|c|c|c
	|c|c|c|c
	|c|c|c|c
	|c|c|c|c|}
		\hline
		\multirow{2}{*}{No.}
			& \multirow{2}{*}{Kegiatan}
			& \multicolumn{4}{c|}{Bulan I}
			& \multicolumn{4}{c|}{Bulan II}
			& \multicolumn{4}{c|}{Bulan III}
			& \multicolumn{4}{c|}{Bulan IV}
			& \multicolumn{4}{c|}{Bulan V}
			& \multicolumn{4}{c|}{Bulan VI}\\
		\cline{3-26}
		& &
			1 & 2 & 3 & 4 &
			1 & 2 & 3 & 4 &
			1 & 2 & 3 & 4 &
			1 & 2 & 3 & 4 &
			1 & 2 & 3 & 4 &
			1 & 2 & 3 & 4\\
		\hline
		1 & Persiapan\ \  Data dan\ \ Lingkungan Penelitian &
			\progresscell{2}
			\fillcell{2}
			\emptycell{19}{1}
		\hline
		2 & Implementasi dan Pengujian &
			\emptycell{2}{0}
			\fillcell{17}
			\emptycell{3}{1}
		\hline
		4 & Analisis &
			\emptycell{7}{0}
			\fillcell{14}
			\emptycell{1}{1}
		\hline
		5 & Evaluasi &
			\emptycell{20}{0}
			\fillcell{2}
			\emptycell{0}{1}
		\hline
	\end{tabular}
	}
	\caption{Jadwal penelitian tesis}
	\label{tab:jadwal}
\end{table}
%%}}}

%%{{{ Advisor's signature
%%
\begin{center}
	Diketahui oleh,
	\linebreak
	\linebreak
	\hbox to \hsize{%
		\myadvisorsig{Pembimbing I,\quad}\hfil
	}
\end{center}
%%
%%}}}

\end{document}
