\documentclass[12pt,a4paper,titlepage]{article}

%%{{{ document's packages
%%
\usepackage{graphicx}		% \includegraphics{name}
\usepackage{multirow}		% \multirow{package}{width}{text}
\usepackage{forloop}		% \forloop
\usepackage{caption}		% \caption{title}
\usepackage[table]{xcolor}
\usepackage{url}
%% fix underfull on footnote with URL.
\usepackage{ragged2e}
%% source code listing
\usepackage{listings}
\lstset{
	basicstyle=\small\ttfamily,
	columns=flexible,
	breaklines=true
}
%% text emphasis, including strikeout.
\usepackage[normalem]{ulem}
%% mathematics
\usepackage{mathtools}
%%% comment/uncomment this to show overrule in black box
\overfullrule=2cm
%% reconfigure table.
%%% Alter latex default title
\captionsetup[table]{name=Tabel}
\renewcommand{\arraystretch}{1.5}
\setlength{\tabcolsep}{3pt}
%%}}}

%%{{{ hyphenation
\hyphenation{
	ba-gai-ma-na
	ber-kas
	ber-mak-na
	data-set
	meng-i-kuti
	meng-in-di-ka-si-kan
	mi-sal-nya
	pe-nan-da
	sun-ting-an
}
%%}}}

%%{{{ document's variables
%%
\newcommand{\mytitle}{Deteksi Vandalisme pada Wikipedia Bahasa Inggris menggunakan klasifikasi Cascaded Random Forest}
\newcommand{\myname}{Muhamad Sulhan}
\newcommand{\mysid}{23513014}
%%}}}

%%{{{ document's meta-data
%%
\author{\myname}
\title{\mytitle}
%%}}}

%%{{{ document's functions
%%
%%% signature
\def\myadvisorsig#1{%
	\vbox{\hsize=6cm
		\textbf{#1}\\
		\addvspace{2cm}%
		\hbox to \hsize{%
			\strut\hfil%
			Dwi Hendratmo Widyantoro%
			\hfil%
		}
		\hrule\kern1ex
		\hbox to \hsize{%
			\strut\hfil%
			NIP\hspace{1ex}196812071994021001%
			\hfil%
		}
	}
}
%%}}}

%%
%% DOCUMENT
%%
\begin{document}

%% {{{ COVER
\thispagestyle{empty}
\begin{center}
	\textbf{%
		\mytitle
		\vfill
		Laporan Progres Tesis, Catatan dan Pekerjaan Selanjutnya
		\linebreak
		\linebreak
		11 September 2015
		\vfill
		Oleh \\
		\myname \\
		\mysid \\
		\vfill
		\uppercase{%
			Program Studi Magister Informatika \\
			Sekolah Teknik Elektro dan Informatika \\
			Institut Teknologi Bandung \\
			2015
		}
	}
\end{center}
\newpage
%% }}}

\section{Laporan Progres}

Bagian ini mencatat progres pekerjaan yang telah dilakukan setiap minggu secara ringkas.
Rincian dari progres berada pada bagian \ref{sec:catatan}.

%%{{{ Laporan Progress: Minggu I
\subsection{Minggu I}

\begin{itemize}
	\item \textbf{Data korpus untuk PAN-WVC-11 telah diunduh.}
Ukuran berkas terkompres yaitu 371 MB, dan tidak terkompres sebesar 1,1 GB.
Jumlah suntingan yaitu 9985, dengan 8734 diantaranya adalah suntingan regular dan 1251 (sekitar 12\%) adalah vandalisme.

	\item \textbf{Perbaikan proposal tesis.}
\end{itemize}
%%}}}

%%{{{ Laporan Progress: Minggu II
\subsection{Minggu II}

\begin{itemize}
	\item \textbf{Data korpus untuk PAN-WVC-10 telah diunduh.}
Ukuran berkas terkompres yaitu 439 MB, dan tidak terkompres sebesar 1,3 GB.
Jumlah dataset yaitu 32439 revisi, dengan 30045 diantaranya adalah suntingan regular dan 2394 (sekitar 7\%) diantaranya adalah vandalisme.

	\item \textbf{Pemasangan \textit{Database Management System} (DBMS).}
DBMS yang digunakan adalah MariaDB, yang merupakan \textit{fork} dari MySQL.
DBMS nantinya digunakan untuk menyimpan data ekspor dari hasil \textit{dump} riwayat penyuntingan Wikipedia.
Wikimedia (perangkat lunak wiki yang digunakan oleh Wikipedia), menggunakan MySQL sebagai DBMS, itulah kenapa MariaDB dipilih dalam pengembangan tesis ini.

	\item \textbf{Data \textit{dump} dari riwayat wikipedia telah diunduh.} \footnote{\RaggedRight\url{http://dumps.wikimedia.org/enwiki/20150805/enwiki-20150805-stub-meta-history1.xml.gz}}
Berkas hanya satu dari 27 berkas \textit{dump} yang memiliki total 43,5 GB.
Ukuran data terkompres yaitu 503 MB dan tidak terkompres 3,3 GB.
Format data berupa xml.
\end{itemize}
%%}}}

%%{{{ Laporan Progress: Minggu III
\subsection{Minggu III}

Eksplorasi fitur-fitur yang digunakan untuk klasifikasi yang telah digunakan oleh makalah-makalah sebelumnya dan memperkirakan fitur yang akan digunakan pada tesis ini.
%%}}}

\newpage
\section{Catatan} \label{sec:catatan}

Bagian ini berisi catatan yang didapat pada setiap minggu selama mengerjakan tesis.

%%{{{ Catatan: Minggu I dan II
\subsection{Minggu I dan II}

Korpus PAN-WVC-10 terbagi menjadi dua yaitu dataset suntingan dan dataset anotasi yang berisi hasil klasifikasi.
Kedua dataset memiliki format yang sama yaitu menggunakan \textit{Coma Separated Value} (CSV).
Dataset suntingan memiliki atribut sebagai berikut,
\begin{itemize}
	\item \textbf{editid}, format angka, berisi identifikasi (ID) unik dari setiap suntingan.
	\item \textbf{editor}, format string, berisi nama penyunting.
	\item \textbf{oldrevisionid}, format angka, berisi ID untuk suntingan lama.
	\item \textbf{newrevisionid}, format angka, berisi ID untuk suntingan baru.
	\item \textbf{diffurl}, format string, berisi URL yang mengacu pada perbedaan suntingan baru dengan lama.
	\item \textbf{edittime}, format string, berisi tanggal dan pukul sesuai dengan ISO 8601.
	\item \textbf{editcomment}, format string, berisi komentar yang ditambahkan oleh penyunting saat menyimpan hasil suntingan.
	\item \textbf{articleid}, format angka, berisi ID unik dari artikel.
	\item \textbf{articletitle}, format string, berisi judul dari artikel yang disunting.
\end{itemize}

Dataset anotasi memiliki atribut sebagai berikut,
\begin{itemize}
	\item \textbf{editid}, format angka, mengacu pada ID yang sama pada dataset suntingan.
	\item \textbf{class}, format string, berisi tipe suntingan yang bernilai "regular" yang menyatakan bahwa suntingan tersebut bukan vandalisme, dan "vandalism" yang menyatakan bahwa suntingan tersebut adalah vandalisme.
	\item \textbf{annotators}, format angka, berisi jumlah orang yang menandai (penanda) bahwa suntingan dengan ID tersebut termasuk ke dalam kelas "regular" atau "vandalism".
	\item \textbf{totalannotators}, format angka, berisi jumlah total penanda yang memeriksa suntingan.
\end{itemize}

Korpus PAN-WVC-11 hanya memiliki satu dataset untuk Bahasa Inggris yaitu dataset suntingan, dengan atribut yang menggabungkan dataset suntingan dan dataset anotasi pada PAN-WVC-10.

Rencananya, korpus PAN-WVC-10 dijadikan sebagai dataset untuk \textit{resampling} dan pelatihan klasifikasi karena kuantitasnya lebih banyak daripada PAN-WVC-11.
Korpus PAN-WVC-11 dijadikan sebagai dataset untuk pengujian model.
%%}}}

%%{{{ Catatan: Minggu III
\subsection{Minggu III}

\subsubsection{Eksplorasi Fitur Vandalisme pada Wikipedia}

Beberapa makalah sebelumnya mengelompokan fitur-fitur ke dalam kelompok \textit{metadata}, teks, dan bahasa.

\paragraph{Fitur Kelompok Metadata}

Kelompok metadata mengacu pada properti dari sebuah revisi yang secara langsung dapat diambil, seperti identitas penyunting, atau waktu suntingan.

Berikut daftar fitur-fitur yang dapat digunakan pada klasifikasi vandalisme berbasiskan metadata,

\begin{itemize}
\item \textbf{Anonim}.
Melihat apakah penyunting memiliki akun atau anonim (hanya tercatat alamat \textit{IP}-nya).
Vandal lebih condong berlaku anonim karena jika menggunakan akun asli akan membuat akun mereka mudah diblokir.

\item \textbf{Panjang komentar}.
Melihat dari jumlah karakter yang digunakan di kolom "rangkuman suntingan" saat menyimpan hasil suntingan.
Komentar yang panjang mungkin mengindikasikan suntingan biasa dan yang pendek atau kosong mungkin menyarankan suatu vandalisme.
Namun, fitur ini sedikit lemah karena banyak suntingan biasa meninggalkan komentar yang kosong.

\item \textbf{Peningkatan ukuran}.
Peningkatan absolut dari ukuran konten artikel, dihitung dengan
\[
|ukuran\ suntingan\ baru| - |ukuran\ suntingan\ lama|
\]
Misalnya, berkurangnya ukuran dalam jumlah besar bisa mengindikasikan pengosongan artikel.

\item \textbf{Rasio ukuran}.
Ukuran revisi baru relatif terhadap revisi lama, yaitu
\[
\frac{1 + |baru|}{1 + |lama|}
\]

\end{itemize}

\paragraph{Fitur Kelompok Teks}

Berikut daftar fitur-fitur yang dapat digunakan pada klasifikasi vandalisme berbasiskan teks,

\begin{itemize}

\item \textbf{Rasio huruf besar dan kecil}.
Pelaku vandal biasanya tidak mengikuti aturan huruf kapital, menulis semuanya dengan huruf kecil atau huruf besar.
Rasio ini dihitung dengan menggunakan rumus
\[
\frac{1 + |perubahan\ baru|}{1 + |perubahan\ lama|}
\]

\item \textbf{Rasio huruf besar terhadap semua huruf}
Rasio ini dihitung dengan menggunakan rumus
\[
\frac{1 + |huruf\ besar|}{1 + |huruf\ besar| + |huruf\ kecil|}
\]

\item \textbf{Rasio angka}.
Rasio semua karakter terhadap angka, yaitu
\[
\frac{1 + |angka|}{1 + |semua|}
\]
Fitur ini membantu menemukan suntingan kecil yang hanya mengubah angka.
Contoh kasusnya perubahan sebuah tanggal atau perhitungan untuk menyebabkan kesalahan informasi.

\item \textbf{Rasio non-alfanumerik}.
Rasio semua karakter terhadap karakter selain huruf dan angka, yaitu
\[
\frac{1 + |non\ alfanumerik|}{1 + |semua\ karakter|}
\]
Kelebihan penggunaan karakter selain angka-huruf bisa mengindikasikan penggunaan \textit{emoticon}, tanda baca, atau kata tak bermakna.

\item \textbf{Diversitas karakter}.
Menghitung karakter berbeda dibandingkan dengan panjang teks yang dimasukan, yaitu
\[
length \frac{1}{karakter\ berbeda}
\]
Fitur ini membantu menemukan penggunaan karakter secara acak dan kata tak bermakna.

\item \textbf{Distribusi karakter}.
Menggunakan Divergensi Kullback-Leibler dari distribusi karakter yang dimasukan terhadap ekspektasi.
Fitur ini berguna untuk mendeteksi kata tak bermakna.

\item \textbf{Kompresabilitas}.
Melihat tingkat kompres dari teks menggunakan algoritma LZW.
Fitur ini berguna untuk mendeteksi kata tak bermakna, pengulangan kata atau karakter, dll.
Vandalisme biasanya memiliki ukuran kompresi yang rendah.

\item \textbf{Token umum}.
Token yang biasanya jarang digunakan oleh vandal adalah sintaks wiki, seperti \textit{\_\_TOC\_\_}.

\item \textbf{Frekuensi rerata kata}.
Frekuensi relatif rerata dari kata yang dimasukan pada revisi baru.
Pada artikel yang panjang, semakin banyak kata yang dimasukan yang tidak ada pada artikel mengindikasikan bahwa suntingan tersebut bisa tak bermakna atau tidak berhubungan dengan isinya.

\item \textbf{Kata terpanjang}.
Panjang dari kata yang dimasukan.
Nilainya akan 0 jika tidak ada kata yang dimasukan.
Fitur ini berguna untuk mendeteksi suntingan tak-bermakna.

\item \textbf{Urutan karakter terpanjang}.
Urutan terpanjang dari karakter yang sama pada teks yang dimasukan sering digunakan pada vandalisme, contohnya \textit{aaarrrrggghhh! saaanggaatt besar}.

\end{itemize}

\paragraph{Fitur Kelompok Bahasa}

Fitur kelompok bahasa berdasarkan jumlah kata yang ditambahkan pada suntingan baru.
Dua fitur berikut dihitung relatif terhadap konten yang baru: frekuensi dan dampak.
Frekuensi adalah menghitung frekuensi dari kata tersebut dari total seluruh kata pada suntingan yang baru.
Dampak adalah menghitung persentase penambahan jumlah kata tersebut pada suntingan.
Berikut daftar fitur-fitur yang dapat digunakan pada klasifikasi vandalisme berbasiskan bahasa,

\begin{itemize}
\item \textbf{Vulgarisme}.
Menghitung kata-kata vulgar dan menghina, misalnya \textit{fuck}, \textit{suck}, \textit{stupid}.

\item \textbf{Subjek}.
Menghitung kata-kata subjek pertama atau kedua, termasuk pengejaan tidak baku, misalnya \textit{I}, \textit{you}, \textit{ya}.

\item \textbf{Bias}.
Menghitung penggunaan kata sehari-hari yang mengandung bias, misalnya \textit{coolest}, \textit{huge}.

\item \textbf{Seks}.
Menghitung penggunaan kata berhubungan dengan seks, misalnya \textit{sex}, \textit{penis}, \textit{nipple}.

\item \textbf{Kata buruk}.
Gabungan kategori untuk kata sehari-hari dan beberapa penulisan yang buruk (misalnya, \textit{wanna}, \textit{gotcha}).

\end{itemize}

\subsubsection{Eksplorasi Sumber Kode Referensi}

Penulis menemukan sumber kode yang digunakan oleh referensi utama
\footnote{
	\url{https://github.com/webis-de/wikipedia-vandalism-detection}
}.
Setelah mengunduh sumber kode dan mempersiapkan ketergantungan yang dibutuhkan untuk menjalankan aplikasi tersebut (seperti bahasa pemrograman dan semua ketergantungan pustaka yang dibutuhkan), penulis tidak dapat menjalankan sumber kode karena ada ketergantungan atau konfigurasi yang mungkin terlewat atau tidak tertulis sehingga tidak dapat menjalankan program.

Penulis masih mencoba memperbaiki kesalahan tersebut untuk tiga hari mendatang.
Seandainya masih tidak bisa, setidaknya ada gambaran bagaimana penerapan teknik \textit{resampling} dan penggunaan klasifikasi dilakukan.
%%}}}

%%{{{ Pekerjaan Selanjutnya
\section{Pekerjaan Selanjutnya}

Bagian ini berisi pekerjaan yang akan dilakukan oleh penulis pada minggu selanjutnya dari tanggal laporan progres ini.

\subsection{Minggu IV}

Mempelajari bagaimana mengaplikasikan fitur-fitur pada klasifikasi.

\subsection{Minggu III}

\sout{
Mempelajari dan mengimplementasikan algoritma LN-SMOTE pada bahasa pemrograman Go.
}

Mempelajari fitur-fitur untuk klasifikasi.

\subsection{Minggu I dan II}

Perbaikan proposal tesis.

%%}}}

%%{{{ SECTION: Penjadwalan
%%
\newpage
\section{Penjadwalan}\label{sec:penjadwalan}

Tabel \ref{tab:jadwal} menampilkan jadwal yang direncanakan dalam pengembangan tesis dari bulan ke I, September 2015, sampai dengan bulan ke VI, Januari 2016.

Warna merah menandakan minggu yang telah lewat sampai minggu dari laporan progres ini, untuk warna abu-abu menandakan waktu pelaksanaan yang akan datang.

% function to fill cell with color
\newcommand{\tand}{&}
\newcounter{cnt}
\newcommand{\fillcell}[1]{%
	\forloop{cnt}{0}{\value{cnt}<#1}{%
		{\cellcolor[gray]{0.7}} \tand
	}%
}
% function to create empty cell
\newcommand{\emptycell}[2]{%
	\forloop{cnt}{0}{\value{cnt}<#1}{%
		\tand
	}%
	\ifthenelse{#2 = 1}{\\}{\tand}%
}
% function to fill week in progress.
\newcommand{\progresscell}[1]{%
	\forloop{cnt}{0}{\value{cnt}<#1}{%
		{\cellcolor{red!80}} \tand
	}%
}

\begin{table}[h!]
	\centering
	{\footnotesize
	\begin{tabular}{|c|p{0.2\textwidth}
	|c|c|c|c
	|c|c|c|c
	|c|c|c|c
	|c|c|c|c
	|c|c|c|c
	|c|c|c|c|}
		\hline
		\multirow{2}{*}{No.}
			& \multirow{2}{*}{Kegiatan}
			& \multicolumn{4}{c|}{Bulan I}
			& \multicolumn{4}{c|}{Bulan II}
			& \multicolumn{4}{c|}{Bulan III}
			& \multicolumn{4}{c|}{Bulan IV}
			& \multicolumn{4}{c|}{Bulan V}
			& \multicolumn{4}{c|}{Bulan VI}\\
		\cline{3-26}
		& &
			1 & 2 & 3 & 4 &
			1 & 2 & 3 & 4 &
			1 & 2 & 3 & 4 &
			1 & 2 & 3 & 4 &
			1 & 2 & 3 & 4 &
			1 & 2 & 3 & 4\\
		\hline
		1 & Persiapan\ \  Data dan\ \ Lingkungan Penelitian &
			\progresscell{4}
			\emptycell{19}{1}
		\hline
		2 & Implementasi dan Pengujian &
			\emptycell{2}{0}
			\fillcell{17}
			\emptycell{3}{1}
		\hline
		4 & Analisis &
			\emptycell{7}{0}
			\fillcell{14}
			\emptycell{1}{1}
		\hline
		5 & Evaluasi &
			\emptycell{20}{0}
			\fillcell{2}
			\emptycell{0}{1}
		\hline
	\end{tabular}
	}
	\caption{Jadwal penelitian tesis}
	\label{tab:jadwal}
\end{table}
%%}}}

%%{{{ Advisor's signature
%%
\vfill
\begin{center}
	Diketahui oleh,
	\linebreak
	\linebreak
	\hbox to \hsize{%
		\myadvisorsig{Pembimbing I,\quad}\hfil
	}
\end{center}
%%
%%}}}

\newpage
\appendix
%%{{{ Appendix : Corpora
\section{Contoh data korpus}

\subsection{Contoh Isi Korpus PAN-WVC-10}

Berikut dataset untuk riwayat suntingan pada PAN-WVC-10.

\begin{lstlisting}
"editid","editor","oldrevisionid","newrevisionid","diffurl","edittime","editcomment","articleid","articletitle"
1,"TheHeartbreakKid15",328391343,328391582,"http://en.wikipedia.org/w/index.php?diff=328391582&oldid=328391343","2009-11-28T15:21:18Z","/* Episodes */",24477266,"Top Gear (series 14)"
2,"Stepopen",327585467,327607921,"http://en.wikipedia.org/w/index.php?diff=327607921&oldid=327585467","2009-11-24T04:43:37Z","removed factually wrong information",476288,"List of United Nations resolutions concerning Israel"
3,"93.6.135.185",328227083,328242890,"http://en.wikipedia.org/w/index.php?diff=328242890&oldid=328227083","2009-11-27T18:22:12Z","/* History */",174853,"W.A.S.P."
4,"Plasticspork",314955274,327191082,"http://en.wikipedia.org/w/index.php?diff=327191082&oldid=314955274","2009-11-21T23:12:24Z","Clean infobox + general fixes using [[Project:AutoWikiBrowser|AWB]]",1418363,"Psusennes II"
...
\end{lstlisting}

Berikut dataset anotasi untuk klasifikasi dari korpus PAN-WVC-10 pada berkas yang berbeda,

\begin{lstlisting}
"editid","class","annotators","totalannotators"
...
35,"regular",4,4
36,"regular",4,4
37,"vandalism",6,7
38,"regular",4,4
...
\end{lstlisting}

\newpage
\subsection{Contoh Isi Korpus PAN-WVC-11}

Tidak seperti pada PAN-WVC-10, PAN-WVC-11 menggabungkan riwayat suntingan dan hasil anotasi pada berkas yang sama.
Berikut dataset untuk riwayat suntingan pada PAN-WVC-11.

\begin{lstlisting}
"editid","editor","oldrevisionid","newrevisionid","diffurl","class","annotators","totalannotators","edittime","editcomment","articleid","articletitle"
416341,"YUL89YYZ",308672260,326825757,"http://en.wikipedia.org/w/index.php?diff=326825757&oldid=308672260","regular",8,9,"2009-11-19T23:25:19Z","Disambiguate [[CMG]] to [[Companion of the Order of St Michael and St George]] using [[:en:Wikipedia:Tools/Navigation_popups|popups]]",22605565,"Elwin Palmer"
416342,"Abyssal",326560892,326560971,"http://en.wikipedia.org/w/index.php?diff=326560971&oldid=326560892","regular",3,3,"2009-11-18T16:48:35Z","/* Thyreophorans */",25122995,"List of stratigraphic units with dinosaur tracks"
416343,"24.199.196.3",326540780,326785787,"http://en.wikipedia.org/w/index.php?diff=326785787&oldid=326540780","vandalism",20,21,"2009-11-19T19:46:03Z","/* Air Jordan XI */",1394509,"Air Jordan"
416345,"Brillbananaman",326886207,326886384,"http://en.wikipedia.org/w/index.php?diff=326886384&oldid=326886207","regular",3,3,"2009-11-20T06:53:06Z","/* Photographic career */",20307120,"Peter Dazeley"
...
\end{lstlisting}
%%}}}

\end{document}
