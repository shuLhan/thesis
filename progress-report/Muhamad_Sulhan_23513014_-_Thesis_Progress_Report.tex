\documentclass[12pt,a4paper,titlepage,oneside]{report}

%%{{{ packages
%% font encoding.
\usepackage[utf8]{inputenc}
\usepackage[T1]{fontenc}
\usepackage{lmodern}
\usepackage{couriers}
%% Customize chapters.
\usepackage{titlesec}
%% bibliography.
\usepackage[
	backend=bibtex
,	style=ieee
,	sorting=nyt
]{biblatex}
%% \includegraphics{name}
\usepackage{graphicx}
%% \url{something}
\usepackage{url}
%% \multirow{package}{width}{text}
\usepackage{multirow}
%% \cellcolor
\usepackage[table]{xcolor}
%% \caption{title}
\usepackage{caption}
\usepackage{subcaption}
%% \forloop
\usepackage{forloop}
%% fix underfull on footnote with URL.
\usepackage{ragged2e}
%% source code listing
\usepackage{listings}
%% \printindex
\usepackage{makeidx}
%% table of content
\usepackage{tocloft}
%% text emphasis, including strikeout.
\usepackage[normalem]{ulem}
%% mathematics
\usepackage{mathtools}
%% arithmetic
\usepackage{calc}
%% algorithm
\usepackage{algorithm}
\usepackage{algpseudocode}
%% multiple columns
\usepackage{multicol}
%% Change the margin
\usepackage[a4paper]{geometry}
\geometry{
	a4paper,
	top=3cm,
	right=3cm,
	bottom=3cm,
	left=4cm
}
%%
\usepackage{parskip}
%% Package for reading CSV to database.
\usepackage{datatool}
%% Package for scatter and line plot.
\usepackage{dataplot}
%% Long table
\usepackage{longtable}
%% Tikz
\usepackage{tikz}
\usetikzlibrary{backgrounds}

\usepackage{pgfplots}

%% MnSymbol
\usepackage{MnSymbol}

%% Compact list.
\usepackage{paralist}
%%}}}

%%{{{ hyphenation: sorted in ascending.
%%
\hyphenation{
	Ja-nu-a-ri
	SIGKDD
	Wiki-pedia
	a-kan
	a-ku-ra-si
	ang-ka
	ba-gai-ma-na
	bayes-ian
	ber-kas
	ber-ma-sa-lah
	ber-mak-na
	bi-a-ya
	da-lam
	data-set
	de-ngan
	di-ha-sil-kan
	di-pi-lih
	di-sing-kat
	di-tam-bah-kan
	dis-krit
	fung-si
	ga-bung-an
	ke-las
	ke-le-mah-an
	ke-mung-ki-nan
	ke-tak-se-imbang-an
	lan-guage
	ma-yo-ri-tas
	me-laku-kan
	me-me-rik-sa
	me-mi-lih
	me-ne-rap-kan
	meng-a-pli-ka-si-kan
	me-ning-kat-kan
	me-nye-dia-kan
	me-nye-im-bang-kan
	me-sin
	me-thod
	me-to-de
	mem-vi-sua-li-sa-si
	meng-gu-na-kan
	meng-hi-lang-kan
	meng-hu-bung-kan
	meng-i-kut-kan
	meng-i-kuti
	meng-im-ple-men-ta-si-kan
	meng-in-di-ka-si-kan
	mi-sal-nya
	mung-kin
	o-ver-sam-pling
	pa-ra-lel
	pe-la-ti-han
	pe-mi-sah
	pe-nan-da
	pe-ne-li-ti-an
	pe-nu-li-san
	pe-nyun-ting
	pem-ban-ding
	pen-de-kat-an
	peng-kla-si-fi-ka-si
	peng-a-pli-ka-si-an
	per-for-man-si-nya
	po-ten-si-al
	pro-ba-bi-li-tas
	pro-ses
	sam-pel
	se-im-bang
	se-jum-lah
	sun-ting-an
	ting-kat
	un-der-sam-pling
	wa-lau-pun
}
%%}}}

%%{{{ override default latex setting.
%%

%% Space between paragraphs.
\setlength{\parskip}{1.5em}

%% Add dot to TOC.
\renewcommand{\cftsecleader}{\cftdotfill{\cftdotsep}}
\renewcommand{\contentsname}{}

\lstset{
	basicstyle=\scriptsize\ttfamily
,	breaklines=true
,	stringstyle=\scriptsize\ttfamily
,	numbers=left
,	numberstyle=\tiny\ttfamily
,	numbersep=5pt
,	tabsize=4
,	frame=single
}

\makeatletter
\def\lst@lettertrue{\let\lst@ifletter\iffalse}
\makeatother

%% Format chapter and section.
%%
%%% Set chapter name to Bab.
\titleformat{\chapter}[hang]
{\bfseries\large\centering}
{Bab \thechapter}
{1em}
{}

\titleformat{\section}[hang]
{\bfseries\large}
{\thesection}
{1em}{}

%% Set spacing for sections.
\titlespacing{\chapter}{0ex}{0ex}{1.5em}
\titlespacing{\section}{0ex}{0ex}{0em}

%% Set roman on chapter number.
\def\thechapter{\Roman{chapter}}

%% Alter latex default title on table.
\captionsetup[table]{name=Tabel}
\captionsetup[figure]{name=Gambar}

\renewcommand{\arraystretch}{1.5}
\setlength{\tabcolsep}{3pt}

%% Change bibliography title.
\defbibheading{bibliography}{\centerline{
	\textbf{DAFTAR PUSTAKA}}
}

%%% uncomment this to show overrule in black box
\overfullrule=2cm

%% Algorithmicx
\makeatletter
\renewcommand{\ALG@beginalgorithmic}{\footnotesize}
\makeatother

%% multicolumn setting
\setlength{\columnsep}{1cm}

%% pgfplots setting.
\pgfplotsset{
	/pgf/number format/read comma as period,
	/pgf/text mark as node=false,
	table/col sep=semicolon,
	xmax=1,
	xmin=0,
	ymax=1,
	ymin=0,
	xtick distance=0.2,
	ytick distance=0.2,
	grid=major,
	cycle list name=linestyles,
}
%%}}}

%%{{{ variables
%%
\newcommand{\mytitle}{Deteksi Vandalisme pada Wikipedia Bahasa Inggris menggunakan klasifikasi Cascaded Random Forest}
\newcommand{\myname}{Muhamad Sulhan}
\newcommand{\mysid}{23513014}
\newcommand{\myadvisorname}{Dwi Hendratmo Widyantoro}
\newcommand{\myadvisorid}{196812071994021001}
\newcommand{\mydept}{Program Studi Magister Informatika}
\newcommand{\itb}{Institut Teknologi Bandung}

%%% My images directory
\graphicspath{{../images/}}
\newcommand{\myitbcover}{ITB-logo-hitam}

%%% My bibligraphy file
\addbibresource{bibliography.bib}
%%}}}

%%{{{ document's meta-data
%%
\author{\myname}
\title{\mytitle}
%%}}}

%%{{{ document's macros
%%
%%% two column signature.
\def\myadvisorsig#1{%
	\vbox{\hsize=6cm
		\textbf{#1}\\
		\addvspace{2cm}%
		\hbox to \hsize{%
			\strut\hfil%
			\myadvisorname%
			\hfil%
		}
		\hrule\kern1ex
		\hbox to \hsize{%
			\strut\hfil%
			NIP\hspace{1ex}\myadvisorid%
			\hfil%
		}
	}
}

%%% one column signature.
\def\mysignature#1#2#3{%
	\vbox{
		\textbf{#1}\\
		\addvspace{2cm}%
		\hbox to \hsize{%
			\strut\hfil%
			{#2}%
			\hfil%
		}
		\makebox[6cm][c]{
			\hrulefill
		}
		\hbox to \hsize{%
			\strut\hfil%
			NIP\hspace{1ex}{#3}%
			\hfil%
		}
	}
}

%%% source code listing
\lstdefinelanguage{go}
{
	morekeywords={package,import,const,func,for,type,var,struct}
,	sensitive=true
,	morecomment=[l]{//}
,	morecomment=[s]{/*}{*/}
}
\lstdefinestyle{go}{%
	language=go
,	keywordstyle=\color{black}\bfseries
,	commentstyle=\color{gray}
,	breakatwhitespace=true
,	lineskip={-2.5pt}
}
\newcommand{\includecodego}[2][c]{
	\lstinputlisting[caption=#2,escapechar=,style=go]
		{/home/ms/go/work/src/github.com/shuLhan/#2}
}

%%% data listing
\lstdefinestyle{data}{%
	breakatwhitespace=false
,	breakautoindent=false
,	literate={\,}{}{0\discretionary{,}{}{,}},
}
\newcommand{\includedata}[2][c]{
	\lstinputlisting[caption=#2,style=data,linerange={1-10}]
		{/home/ms/go/work/src/github.com/shuLhan/#2}
}
%%}}}

%%
%% Create cover with parameters
%% 1: report number
%% 2: report weeks, in string
%% 3: report date, string
%%
\newcommand{\reportcover}[3]{
	\thispagestyle{empty}
	\begin{center}\textbf{%
		\mytitle
		\vfill
		Laporan Progres Tesis, Catatan, dan Pekerjaan Selanjutnya
		\linebreak
		Laporan ke-#1
		\linebreak
		Minggu #2
		\linebreak
		\linebreak
		#3
		\vfill
		Oleh \\
		\myname \\
		\mysid \\
		\vfill
		\uppercase{%
			Program Studi Magister Informatika \\
			Sekolah Teknik Elektro dan Informatika \\
			Institut Teknologi Bandung \\
			2015
		}
	}
	\end{center}
	\newpage
}

%%
%% Macro for advisor's signature
%%
\newcommand{\advisorsignature}{
	\vfill
	\begin{center}
		Diketahui oleh,
		\linebreak
		\linebreak
		\hbox to \hsize{%
			\myadvisorsig{Pembimbing,\quad}\hfil
		}
	\end{center}
}

%%
%% Macro for create schedule using parameter
%% 1: number of weeks has passed on task 1
%% 2: number of weeks has passed on task 2
%% 3: number of weeks has passed on task 3
%% 4: number of weeks has passed on task 4
%% 5: should we begin in newpage?
%%

% function to fill cell with color
\newcommand{\tand}{&}
\newcounter{cnt}
\newcommand{\fillcell}[1]{%
	\forloop{cnt}{0}{\value{cnt}<#1}{%
		{\cellcolor[gray]{0.7}} \tand
	}%
}
% function to create empty cell
\newcommand{\emptycell}[2]{%
	\forloop{cnt}{0}{\value{cnt}<#1}{%
		\tand
	}%
	\ifthenelse{#2 = 1}{\\}{\tand}%
}
% function to fill week in progress.
\newcommand{\progresscell}[1]{%
	\forloop{cnt}{0}{\value{cnt}<#1}{%
		{\cellcolor{red!80}} \tand
	}%
}

\newcommand{\schedule}[4]{
	\section{Penjadwalan}\label{sec:penjadwalan}

Tabel di bawah menampilkan jadwal yang direncanakan dalam pengembangan tesis dari bulan ke I, September 2015, sampai dengan bulan ke VI, Januari 2016.

Warna merah menandakan minggu yang telah lewat sampai minggu dari laporan progres ini, untuk warna abu-abu menandakan waktu pelaksanaan yang akan datang.

\newcounter{planone}
\newcounter{plantwo}
\newcounter{planthree}
\newcounter{planfour}

\newcounter{progressone}
\newcounter{progresstwo}
\newcounter{progressthree}
\newcounter{progressfour}

\setcounter{progressone}{#1}
\setcounter{progresstwo}{#2}
\setcounter{progressthree}{#3}
\setcounter{progressfour}{#4}

\setcounter{planone}{4 - \theprogressone}
\setcounter{plantwo}{17 - \theprogresstwo}
\setcounter{planthree}{14 - \theprogressthree}
\setcounter{planfour}{2 - \theprogressfour}

\begin{table}[h!]
	\centering
	{\footnotesize
	\begin{tabular}{|c|p{0.2\textwidth}
	|c|c|c|c
	|c|c|c|c
	|c|c|c|c
	|c|c|c|c
	|c|c|c|c
	|c|c|c|c|}
		\hline
		\multirow{2}{*}{No.}
			& \multirow{2}{*}{Kegiatan}
			& \multicolumn{4}{c|}{Bulan I}
			& \multicolumn{4}{c|}{Bulan II}
			& \multicolumn{4}{c|}{Bulan III}
			& \multicolumn{4}{c|}{Bulan IV}
			& \multicolumn{4}{c|}{Bulan V}
			& \multicolumn{4}{c|}{Bulan VI}\\
		\cline{3-26}
		& &
			1 & 2 & 3 & 4 &
			1 & 2 & 3 & 4 &
			1 & 2 & 3 & 4 &
			1 & 2 & 3 & 4 &
			1 & 2 & 3 & 4 &
			1 & 2 & 3 & 4\\
		\hline
		1 & Persiapan\ \  Data dan\ \ Lingkungan Penelitian &
			\progresscell{\theprogressone}
			\fillcell{\theplanone}
			\emptycell{19}{1}
		\hline
		2 & Implementasi dan Pengujian &
			\emptycell{2}{0}
			\progresscell{\theprogresstwo}
			\fillcell{\theplantwo}
			\emptycell{3}{1}
		\hline
		4 & Analisis &
			\emptycell{7}{0}
			\progresscell{\theprogressthree}
			\fillcell{\theplanthree}
			\emptycell{1}{1}
		\hline
		5 & Evaluasi &
			\emptycell{20}{0}
			\progresscell{\theprogressfour}
			\fillcell{\theplanfour}
			\emptycell{0}{1}
		\hline
	\end{tabular}
	}
	\caption{Jadwal penelitian tesis}
\end{table}
}




%%
%% DOCUMENT
%%
\begin{document}
\reportcover{3}{5 dan 6}{16 Oktober 2015}
\tableofcontents

%%{{{ Progress summary.
%%
\section{Laporan Progres}

Bagian ini mencatat rangkuman dari progres pekerjaan yang telah dilakukan setiap minggu.
Rincian dari progres berada pada bagian \ref{sec:catatan}.

\subsection{Minggu I}

\begin{itemize}
	\item \textbf{Data korpus untuk PAN-WVC-11 telah diunduh.}
Ukuran berkas terkompres yaitu 371 MB, dan tidak terkompres sebesar 1,1 GB.
Jumlah suntingan yaitu 9985, dengan 8734 diantaranya adalah suntingan regular dan 1251 (sekitar 12\%) adalah vandalisme.

	\item \textbf{Perbaikan proposal tesis.}
\end{itemize}

\subsection{Minggu II}

\begin{itemize}
	\item \textbf{Data korpus untuk PAN-WVC-10 telah diunduh.}
Ukuran berkas terkompres yaitu 439 MB, dan tidak terkompres sebesar 1,3 GB.
Jumlah dataset yaitu 32439 revisi, dengan 30045 diantaranya adalah suntingan regular dan 2394 (sekitar 7\%) diantaranya adalah vandalisme.

	\item \textbf{Pemasangan \textit{Database Management System} (DBMS).}
DBMS yang digunakan adalah MariaDB, yang merupakan \textit{fork} dari MySQL.
DBMS nantinya digunakan untuk menyimpan data ekspor dari hasil \textit{dump} riwayat penyuntingan Wikipedia.
Wikimedia (perangkat lunak wiki yang digunakan oleh Wikipedia), menggunakan MySQL sebagai DBMS, itulah kenapa MariaDB dipilih dalam pengembangan tesis ini.

	\item \textbf{Data \textit{dump} dari riwayat wikipedia telah diunduh.} \footnote{\RaggedRight\url{http://dumps.wikimedia.org/enwiki/20150805/enwiki-20150805-stub-meta-history1.xml.gz}}
Berkas hanya satu dari 27 berkas \textit{dump} yang memiliki total 43,5 GB.
Ukuran data terkompres yaitu 503 MB dan tidak terkompres 3,3 GB.
Format data berupa xml.
\end{itemize}

\subsection{Minggu III}

\begin{itemize}
	\item Menambah lampiran yang menampilkan contoh data dari PAN-WVC-10 dan PAN-WVC-11.
	\item Eksplorasi fitur-fitur yang digunakan untuk klasifikasi yang telah digunakan oleh makalah-makalah sebelumnya dan memperkirakan fitur yang akan digunakan pada tesis ini.
Mengembangkan sebuah fitur membutuhkan waktu, sebagai tahap awal hanya akan menggunakan 20 fitur yaitu 4 fitur metadata, 11 fitur teks, dan 5 fitur bahasa yang diambil dari hasil analisis makalah Mola-Velasco \cite{mola2012wikipedia}. Pada saat implementasi sudah berjalan dengan benar, baru nanti secara iteratif ditambahkan fitur-fitur dari makalah lainnya satu persatu.
\end{itemize}

\subsection{Minggu IV}

Telah membaca makalah tentang SMOTE \cite{chawla2002smote} dan LN-SMOTE \cite{maciejewski2011local} dan membuat rangkuman tentang perbedaan dari kedua metode tersebut berdasarkan kekurangan dan solusi yang diajukan dalam makalah.

\subsection{Minggu V, VI}

Implementasi algoritma SMOTE pada bahasa pemrograman Go. Sumber kode dapat dilihat pada lampiran.

%%
%%}}}

%%{{{ Notes
\newpage
\section{Catatan} \label{sec:catatan}

Bagian ini berisi pengetahuan yang didapat pada setiap minggu selama mengerjakan tesis.

\subsection{Minggu I dan II}
\input{notes/week_01-02}

\subsection{Minggu III}
\input{notes/week_03}

\subsection{Minggu IV}
\subsubsection{Metode SMOTE}

Metode \textit{Synthetic Minority Over-sampling Technique} (SMOTE)
\cite{chawla2002smote} menggunakan pendekatan \textit{over-sampling} yang mana
kelas minoritas ditambah dengan membuat sampel "sintetis" bukan dengan
mengganti sampel dari kelas mayoritas menjadi kelas minoritas.
Sampel sintetis dibuat lebih kurang lewat aplikasi, dengan beroperasi pada
"ruang fitur" bukan pada "ruang data".
Kelas minoritas ditambah dengan mengambil setiap sampel-sampel dari kelas
minoritas dan membuat sampel sintetis diantara segmen garis yang menggabungkan
setiap/semua \textit{k-nearest-neighbors} dari kelas minoritas.
Bergantung kepada jumlah \textit{over-sampling} yang dibutuhkan, instan dari
\textit{k-nearest-neighbors} dipilih secara acak.

\begin{figure}[b]
	\centering
	\includegraphics[keepaspectratio=true,scale=0.6]{SMOTE-example}
	\caption{Ilustrasi pembuatan sampel sintetis pada SMOTE.
\textit{p} adalah sampel minoritas, \textit{n} adalah salah satu
\textit{k-nearest-neighbors} dari \textit{p}.
Sampel sintetis yang baru akan berada digaris antara \textit{p} dan \textit{n}.
	}
	\label{fig:smote}
\end{figure}

Sampel sintetis dibuat dengan cara berikut,
\begin{itemize}
	\item Hitung selisih antara vektor fitur (sampel) dengan tetangga
	terdekatnya.
	\item Kalikan selisih tersebut dengan angka ril acak antara 0 sampai 1,
	dan
	\item tambahkan hasilnya ke vektor fitur.
\end{itemize}

Cara ini membuat sampel secara acak pada segmen garis antara dua fitur yang
terpilih, seperti yang terlihat pada gambar \ref{fig:smote}.
Pendekatan ini secara efektif mendorong wilayah pembelajaran dari kelas
minoritas menjadi lebih besar tanpa menyebabkan \textit{overfitting}.

Ambil contoh sebuah sampel (6,4) dan (4,3) sebagai tetangga terdekatnya.
(6,4) adalah sampel yang akan dicari \textit{k-nearest-neighbors}-nya.
(4,3) adalah salah satu dari \textit{k-nearest-neighbors}-nya.
Misalkan,
\[
\begin{matrix}
f1\_1 = 6 & f1\_2 = 4 \\
f2\_1 = 4 & f2\_2 = 3
\end{matrix}
\]
\[
\begin{matrix}
f2\_1 - f1\_1 = 4 - 6 = -2 \\
f2\_2 - f1\_2 = 3 - 4 = -1
\end{matrix}
\]

Sampel baru dihasilkan dengan,
\[
(f1', f2') = (6,4) + random(0-1) * (-2,-1)
\]

Fungsi \texttt{random(0-1)} menghasilkan bilangan ril acak dari 0 sampai 1.


\subsubsection{Metode LN-SMOTE}

SMOTE method has several weakness.
First, all sample from minority class is used, this could be a problem because
not all the samples have equal benefit for learning.
Minority sample in the boundary region between minority and majority class
usually result in misclassified rather than sample located in the center of
region, while sample that located in the center may give a little contribution
to classifier.
One of the method to overcome this problem is by using sample in boundary of
minority class instead in the center which proposed by Han et al.
\cite{han2005borderline}, called Borderline-SMOTE.

Another weakness of SMOTE method is overgeneralization, where their method does
not take into consideration the distribution of minority sample in majority
class, or the outliers.
Maciejewski and Stefanowski \cite{maciejewski2011local}
introduced an extension of SMOTE called Local Neighbourhood
SMOTE (LNSMOTE) \cite{maciejewski2011local} by combining Borderline-SMOTE
with modified version of Safe-Level SMOTE (SL-SMOTE)
\cite{bunkhumpornpat2009safe}.

In SL-SMOTE, majority samples is taken into consideration before creating
synthetic sample by calculating a coefficient called safe-level.
For each minority sample, count the number of their $k$ nearest neighbours (KNN).
If KNN value is close to 0, then the sample will be considered as noise.
If KNN value is close to $k$, then the sample can be said in safe region in
minority class.
The main idea was to create synthetic sample that close to safe region.

The SL-SMOTE strategy has a problem especially when class distribution is bias
in which the minority class spread into small sub-region with low number
cardinality.
In this situation, creating synthetic sample with SLMOTE will cause an overlap
between class.
This problem is due to SL-SMOTE find only KNN for minority class.
If the sample candidate does not located in region with densed minority class,
then some of their neighbours could be far from sample candidate or surrounded
by majority class samples.
LNSMOTE overcome this overlap problem by taking into consideration the local
neighbourhood of minority sample candidate that can provide the number of
majority class around each of them.



\subsection{Minggu V, VI}
Penulis tidak menggunakan program seperti Weka untuk pemrosesan karena, pertama, penulis ingin belajar mengenai semua algoritma pembelajaran mesin yang digunakan dalam makalah ini lebih dalam. Kedua, penulis sekaligus ingin belajar menggunakan bahasa pemrograman Go.

Standar pustaka dari bahasa Go hanya menyediakan paket untuk membaca data teks dalam format CSV (\textit{Comma Separated Value}), yang memiliki kelemahan. Pertama, hanya terbatas pada data teks yang terpisah dengan koma, tidak bisa membaca kolom yang terpisah dengan karakter lain, misalnya karakter kosong, tab, dll. Kedua, konversi data tidak tersedia, hasil pembacaan semuanya dalam tipe string, sementara data yang diproses dapat berisi string dan bilangan real. Masalah kedua bisa diatasi dengan mengkonversi data setelah semua teks dibaca, tapi hal tersebut membuat kerja program bekerja lebih lama.

Untuk mengatasi permasalahan tersebut, sebelum mengimplementasikan algoritma SMOTE, penulis membuat librari untuk membaca data dalam teks yang terpisah oleh karakter (\textit{Delimited Separated Value}, DSV) yang dapat mengkonversi tipe data langsung ke format \textit{native}-nya sehingga tidak perlu pemrosesan ulang.
Librari ini nantinya berguna pada saat pembacaan dataset dan penulisan hasil yang didapat dari pemrosesan. Sumber kode dari librari dapat dilihat di lampiran
\ref{appendix:sumber_kode_dsv}
dan di internet
\footnote{\url{https://github.com/shuLhan/dsv}}. 

Penulis juga mengimplementasikan algoritma penghitungan jarak dari \textit{K-nearest-neighbors} (KNN) menggunakan metode \textit{Euclidian}, karena fungsi ini digunakan oleh algoritma SMOTE untuk mendapatkan sampel terdekat. Sumber kode dari hasil implementasi dapat dilihat di lampiran
\ref{appendix:sumber_kode_knn}
dan di internet
\footnote{\url{https://github.com/shuLhan/go-mining/tree/master/knn}}.

Setelah semua selesai dan diuji, implementasi SMOTE dilakukan dengan mengacu pada makalah Chawla \cite{chawla2002smote}. Hasil implementasi dapat dilihat di bagian lampiran
\ref{appendix:sumber_kode_smote}
dan di internet
\footnote{\url{https://github.com/shuLhan/go-mining/tree/master/resampling/smote}}.

Hasil implementasi SMOTE diuji pada dataset phoneme, dengan total sampel 5404 dengan jumlah kelas minoritas 1586.
Setelah menjalankan SMOTE, dengan nilai \textit{K} yaitu 5 dan persentase \textit{oversampling} sebesar 200, didapat data sintetis sejumlah 3172.
Input data dan hasilnya bisa dilihat pada lampiran
\ref{appendix:smote_raw}.

Tadinya penulis ingin memperlihatkan hasil implementasi SMOTE dengan mencoba menerapkannya pada salah satu dataset yang digunakan oleh Chawla, dkk.\cite{chawla2002smote}, yaitu dataset \texttt{phoneme} seperti yang diperlihatkan pada halaman 335.
Namun karena kesulitan dalam mengimplementasikan algoritma C4.5, penulis tidak bisa membandingkan hasil dari makalah Chawla, dkk. tersebut dengan hasil implementasi sendiri.


%%}}}

%%{{{ Pekerjaan Selanjutnya
%%
\newpage
\section{Pekerjaan Selanjutnya}

Bagian ini berisi pekerjaan yang akan dilakukan oleh penulis pada minggu selanjutnya dari tanggal laporan progres ini.

\subsection{Minggu I}

Perbaikan proposal tesis.

\subsection{Minggu II}

Perbaikan proposal tesis.

\subsection{Minggu III}

\sout{
Mempelajari dan mengimplementasikan algoritma LN-SMOTE pada bahasa pemrograman Go.
}

Mempelajari fitur-fitur untuk klasifikasi.

\subsection{Minggu IV}

Mempelajari makalah SMOTE dan LN-SMOTE.

\subsection{Minggu V}
Mempelajari dan mengimplementasikan algoritma SMOTE.

\subsection{Minggu VI}
Mempelajari dan mengimplementasikan algoritma SMOTE.

\subsection{Minggu VII}
Mempelajari dan mengimplementasikan algoritma Random Forest.

%%}}}

\newpage
\schedule{4}{3}{0}{0}
\advisorsignature

%%{{{ Appendix : Corpora
\newpage
\appendix
\input{appendix/korpus}
\input{appendix/sumber_kode_dsv-knn-smote}
\newpage
\section{Hasil Uji Implementasi SMOTE}
\label{appendix:smote_raw}

Contoh input dataset adalah sebagai berikut (menampilkan hanya 10 baris teratas),

\includedata{go-mining/resampling/smote/testdata/phoneme.dat}

Contoh hasil sampel sintetis yang didapat yaitu (menampilkan hanya 10 baris teratas),

\includedata{go-mining/resampling/smote/testdata/phoneme.csv}

%%}}}

\clearpage
\addcontentsline{toc}{section}{Daftar Referensi}
\printbibliography

\end{document}
