\documentclass[12pt,a4paper,titlepage,oneside]{report}

%%{{{ packages
%% font encoding.
\usepackage[utf8]{inputenc}
\usepackage[T1]{fontenc}
\usepackage{lmodern}
\usepackage{couriers}
%% Customize chapters.
\usepackage{titlesec}
%% bibliography.
\usepackage[
	backend=bibtex
,	style=ieee
,	sorting=nyt
]{biblatex}
%% \includegraphics{name}
\usepackage{graphicx}
%% \url{something}
\usepackage{url}
%% \multirow{package}{width}{text}
\usepackage{multirow}
%% \cellcolor
\usepackage[table]{xcolor}
%% \caption{title}
\usepackage{caption}
\usepackage{subcaption}
%% \forloop
\usepackage{forloop}
%% fix underfull on footnote with URL.
\usepackage{ragged2e}
%% source code listing
\usepackage{listings}
%% \printindex
\usepackage{makeidx}
%% table of content
\usepackage{tocloft}
%% text emphasis, including strikeout.
\usepackage[normalem]{ulem}
%% mathematics
\usepackage{mathtools}
%% arithmetic
\usepackage{calc}
%% algorithm
\usepackage{algorithm}
\usepackage{algpseudocode}
%% multiple columns
\usepackage{multicol}
%% Change the margin
\usepackage[a4paper]{geometry}
\geometry{
	a4paper,
	top=3cm,
	right=3cm,
	bottom=3cm,
	left=4cm
}
%%
\usepackage{parskip}
%% Package for reading CSV to database.
\usepackage{datatool}
%% Package for scatter and line plot.
\usepackage{dataplot}
%% Long table
\usepackage{longtable}
%% Tikz
\usepackage{tikz}
\usetikzlibrary{backgrounds}

\usepackage{pgfplots}

%% MnSymbol
\usepackage{MnSymbol}

%% Compact list.
\usepackage{paralist}
%%}}}

%%{{{ hyphenation: sorted in ascending.
%%
\hyphenation{
	Ja-nu-a-ri
	SIGKDD
	Wiki-pedia
	a-kan
	a-ku-ra-si
	ang-ka
	ba-gai-ma-na
	bayes-ian
	ber-kas
	ber-ma-sa-lah
	ber-mak-na
	bi-a-ya
	da-lam
	data-set
	de-ngan
	di-ha-sil-kan
	di-pi-lih
	di-sing-kat
	di-tam-bah-kan
	dis-krit
	fung-si
	ga-bung-an
	ke-las
	ke-le-mah-an
	ke-mung-ki-nan
	ke-tak-se-imbang-an
	lan-guage
	ma-yo-ri-tas
	me-laku-kan
	me-me-rik-sa
	me-mi-lih
	me-ne-rap-kan
	meng-a-pli-ka-si-kan
	me-ning-kat-kan
	me-nye-dia-kan
	me-nye-im-bang-kan
	me-sin
	me-thod
	me-to-de
	mem-vi-sua-li-sa-si
	meng-gu-na-kan
	meng-hi-lang-kan
	meng-hu-bung-kan
	meng-i-kut-kan
	meng-i-kuti
	meng-im-ple-men-ta-si-kan
	meng-in-di-ka-si-kan
	mi-sal-nya
	mung-kin
	o-ver-sam-pling
	pa-ra-lel
	pe-la-ti-han
	pe-mi-sah
	pe-nan-da
	pe-ne-li-ti-an
	pe-nu-li-san
	pe-nyun-ting
	pem-ban-ding
	pen-de-kat-an
	peng-kla-si-fi-ka-si
	peng-a-pli-ka-si-an
	per-for-man-si-nya
	po-ten-si-al
	pro-ba-bi-li-tas
	pro-ses
	sam-pel
	se-im-bang
	se-jum-lah
	sun-ting-an
	ting-kat
	un-der-sam-pling
	wa-lau-pun
}
%%}}}

%%{{{ override default latex setting.
%%

%% Space between paragraphs.
\setlength{\parskip}{1.5em}

%% Add dot to TOC.
\renewcommand{\cftsecleader}{\cftdotfill{\cftdotsep}}
\renewcommand{\contentsname}{}

\lstset{
	basicstyle=\scriptsize\ttfamily
,	breaklines=true
,	stringstyle=\scriptsize\ttfamily
,	numbers=left
,	numberstyle=\tiny\ttfamily
,	numbersep=5pt
,	tabsize=4
,	frame=single
}

\makeatletter
\def\lst@lettertrue{\let\lst@ifletter\iffalse}
\makeatother

%% Format chapter and section.
%%
%%% Set chapter name to Bab.
\titleformat{\chapter}[hang]
{\bfseries\large\centering}
{Bab \thechapter}
{1em}
{}

\titleformat{\section}[hang]
{\bfseries\large}
{\thesection}
{1em}{}

%% Set spacing for sections.
\titlespacing{\chapter}{0ex}{0ex}{1.5em}
\titlespacing{\section}{0ex}{0ex}{0em}

%% Set roman on chapter number.
\def\thechapter{\Roman{chapter}}

%% Alter latex default title on table.
\captionsetup[table]{name=Tabel}
\captionsetup[figure]{name=Gambar}

\renewcommand{\arraystretch}{1.5}
\setlength{\tabcolsep}{3pt}

%% Change bibliography title.
\defbibheading{bibliography}{\centerline{
	\textbf{DAFTAR PUSTAKA}}
}

%%% uncomment this to show overrule in black box
\overfullrule=2cm

%% Algorithmicx
\makeatletter
\renewcommand{\ALG@beginalgorithmic}{\footnotesize}
\makeatother

%% multicolumn setting
\setlength{\columnsep}{1cm}

%% pgfplots setting.
\pgfplotsset{
	/pgf/number format/read comma as period,
	/pgf/text mark as node=false,
	table/col sep=semicolon,
	xmax=1,
	xmin=0,
	ymax=1,
	ymin=0,
	xtick distance=0.2,
	ytick distance=0.2,
	grid=major,
	cycle list name=linestyles,
}
%%}}}

%%{{{ variables
%%
\newcommand{\mytitle}{Deteksi Vandalisme pada Wikipedia Bahasa Inggris menggunakan klasifikasi Cascaded Random Forest}
\newcommand{\myname}{Muhamad Sulhan}
\newcommand{\mysid}{23513014}
\newcommand{\myadvisorname}{Dwi Hendratmo Widyantoro}
\newcommand{\myadvisorid}{196812071994021001}
\newcommand{\mydept}{Program Studi Magister Informatika}
\newcommand{\itb}{Institut Teknologi Bandung}

%%% My images directory
\graphicspath{{../images/}}
\newcommand{\myitbcover}{ITB-logo-hitam}

%%% My bibligraphy file
\addbibresource{bibliography.bib}
%%}}}

%%{{{ document's meta-data
%%
\author{\myname}
\title{\mytitle}
%%}}}

%%{{{ document's macros
%%
%%% two column signature.
\def\myadvisorsig#1{%
	\vbox{\hsize=6cm
		\textbf{#1}\\
		\addvspace{2cm}%
		\hbox to \hsize{%
			\strut\hfil%
			\myadvisorname%
			\hfil%
		}
		\hrule\kern1ex
		\hbox to \hsize{%
			\strut\hfil%
			NIP\hspace{1ex}\myadvisorid%
			\hfil%
		}
	}
}

%%% one column signature.
\def\mysignature#1#2#3{%
	\vbox{
		\textbf{#1}\\
		\addvspace{2cm}%
		\hbox to \hsize{%
			\strut\hfil%
			{#2}%
			\hfil%
		}
		\makebox[6cm][c]{
			\hrulefill
		}
		\hbox to \hsize{%
			\strut\hfil%
			NIP\hspace{1ex}{#3}%
			\hfil%
		}
	}
}

%%% source code listing
\lstdefinelanguage{go}
{
	morekeywords={package,import,const,func,for,type,var,struct}
,	sensitive=true
,	morecomment=[l]{//}
,	morecomment=[s]{/*}{*/}
}
\lstdefinestyle{go}{%
	language=go
,	keywordstyle=\color{black}\bfseries
,	commentstyle=\color{gray}
,	breakatwhitespace=true
,	lineskip={-2.5pt}
}
\newcommand{\includecodego}[2][c]{
	\lstinputlisting[caption=#2,escapechar=,style=go]
		{/home/ms/go/work/src/github.com/shuLhan/#2}
}

%%% data listing
\lstdefinestyle{data}{%
	breakatwhitespace=false
,	breakautoindent=false
,	literate={\,}{}{0\discretionary{,}{}{,}},
}
\newcommand{\includedata}[2][c]{
	\lstinputlisting[caption=#2,style=data,linerange={1-10}]
		{/home/ms/go/work/src/github.com/shuLhan/#2}
}
%%}}}

%%
%% Create cover with parameters
%% 1: report number
%% 2: report weeks, in string
%% 3: report date, string
%%
\newcommand{\reportcover}[3]{
	\thispagestyle{empty}
	\begin{center}\textbf{%
		\mytitle
		\vfill
		Laporan Progres Tesis, Catatan, dan Pekerjaan Selanjutnya
		\linebreak
		Laporan ke-#1
		\linebreak
		Minggu #2
		\linebreak
		\linebreak
		#3
		\vfill
		Oleh \\
		\myname \\
		\mysid \\
		\vfill
		\uppercase{%
			Program Studi Magister Informatika \\
			Sekolah Teknik Elektro dan Informatika \\
			Institut Teknologi Bandung \\
			2015
		}
	}
	\end{center}
	\newpage
}

%%
%% Macro for advisor's signature
%%
\newcommand{\advisorsignature}{
	\vfill
	\begin{center}
		Diketahui oleh,
		\linebreak
		\linebreak
		\hbox to \hsize{%
			\myadvisorsig{Pembimbing,\quad}\hfil
		}
	\end{center}
}

%%
%% Macro for create schedule using parameter
%% 1: number of weeks has passed on task 1
%% 2: number of weeks has passed on task 2
%% 3: number of weeks has passed on task 3
%% 4: number of weeks has passed on task 4
%% 5: should we begin in newpage?
%%

% function to fill cell with color
\newcommand{\tand}{&}
\newcounter{cnt}
\newcommand{\fillcell}[1]{%
	\forloop{cnt}{0}{\value{cnt}<#1}{%
		{\cellcolor[gray]{0.7}} \tand
	}%
}
% function to create empty cell
\newcommand{\emptycell}[2]{%
	\forloop{cnt}{0}{\value{cnt}<#1}{%
		\tand
	}%
	\ifthenelse{#2 = 1}{\\}{\tand}%
}
% function to fill week in progress.
\newcommand{\progresscell}[1]{%
	\forloop{cnt}{0}{\value{cnt}<#1}{%
		{\cellcolor{red!80}} \tand
	}%
}

\newcommand{\schedule}[4]{
	\section{Penjadwalan}\label{sec:penjadwalan}

Tabel di bawah menampilkan jadwal yang direncanakan dalam pengembangan tesis dari bulan ke I, September 2015, sampai dengan bulan ke VI, Januari 2016.

Warna merah menandakan minggu yang telah lewat sampai minggu dari laporan progres ini, untuk warna abu-abu menandakan waktu pelaksanaan yang akan datang.

\newcounter{planone}
\newcounter{plantwo}
\newcounter{planthree}
\newcounter{planfour}

\newcounter{progressone}
\newcounter{progresstwo}
\newcounter{progressthree}
\newcounter{progressfour}

\setcounter{progressone}{#1}
\setcounter{progresstwo}{#2}
\setcounter{progressthree}{#3}
\setcounter{progressfour}{#4}

\setcounter{planone}{4 - \theprogressone}
\setcounter{plantwo}{17 - \theprogresstwo}
\setcounter{planthree}{14 - \theprogressthree}
\setcounter{planfour}{2 - \theprogressfour}

\begin{table}[h!]
	\centering
	{\footnotesize
	\begin{tabular}{|c|p{0.2\textwidth}
	|c|c|c|c
	|c|c|c|c
	|c|c|c|c
	|c|c|c|c
	|c|c|c|c
	|c|c|c|c|}
		\hline
		\multirow{2}{*}{No.}
			& \multirow{2}{*}{Kegiatan}
			& \multicolumn{4}{c|}{Bulan I}
			& \multicolumn{4}{c|}{Bulan II}
			& \multicolumn{4}{c|}{Bulan III}
			& \multicolumn{4}{c|}{Bulan IV}
			& \multicolumn{4}{c|}{Bulan V}
			& \multicolumn{4}{c|}{Bulan VI}\\
		\cline{3-26}
		& &
			1 & 2 & 3 & 4 &
			1 & 2 & 3 & 4 &
			1 & 2 & 3 & 4 &
			1 & 2 & 3 & 4 &
			1 & 2 & 3 & 4 &
			1 & 2 & 3 & 4\\
		\hline
		1 & Persiapan\ \  Data dan\ \ Lingkungan Penelitian &
			\progresscell{\theprogressone}
			\fillcell{\theplanone}
			\emptycell{19}{1}
		\hline
		2 & Implementasi dan Pengujian &
			\emptycell{2}{0}
			\progresscell{\theprogresstwo}
			\fillcell{\theplantwo}
			\emptycell{3}{1}
		\hline
		4 & Analisis &
			\emptycell{7}{0}
			\progresscell{\theprogressthree}
			\fillcell{\theplanthree}
			\emptycell{1}{1}
		\hline
		5 & Evaluasi &
			\emptycell{20}{0}
			\progresscell{\theprogressfour}
			\fillcell{\theplanfour}
			\emptycell{0}{1}
		\hline
	\end{tabular}
	}
	\caption{Jadwal penelitian tesis}
\end{table}
}




%%
%% DOCUMENT
%%
\begin{document}
\reportcover{IV}{7 dan 8}{6 November 2015}

%%{{{ Pendahuluan
\section{Pendahuluan}

Berbeda dengan laporan sebelumnya yang menggabungkan progres tesis dari minggu pertama sampai minggu terakhir laporan, untuk laporan ini dan laporan selanjutnya akan dibuat dalam satu dokumen, karena jumlah halaman yang digunakan sudah terlalu banyak sehingga tidak mudah bagi orang lain untuk membacanya.

Pada laporan sebelumnya, rencana pada dua minggu lewat ini adalah implementasi algoritma \textit{Random Forest}. Pada akhir minggu ke-8, implementasi tersebut belum tercapai, hanya baru sampai pada implementasi \textit{decision tree} (pohon keputusan) CART.

\section{Laporan Singkat Progres}

Selama dua minggu terakhir saya telah,
\begin{itemize}
\item membaca makalah dari \textit{Random Forest} \cite{breiman2001random},
\item mempelajari dan mengimplementasikan algoritma \textit{decisition tree} \textit{Classification and Regression Tree} (CART), yang kemudian mengarah ke
	\begin{itemize}
	\item mempelajari tentang Gini Index, yaitu ukuran untuk menghitung ketidakseimbangan diantara nilai dari frekuensi distribusi, yang digunakan untuk menghitung \textit{Gini's gain} pada atribut dari dataset yang digunakan untuk menentukan nilai pemisah.
	\item mengetahui tentang \textit{Stirling Numbers of Second Kind}, yaitu jumlah sebuah set dapat dipartisi menjadi $ k $ objek tanpa adanya subset yang kosong.
	\item mengetahui dan membuat algoritma partisi untuk sebuah set dengan metode rekursif.
	\end{itemize}
\end{itemize}
%%}}}

\section{Catatan}

\textit{Random Forest} secara sederhananya adalah klasifikasi dengan hasil dari mayoritas \textit{voting} dari beberapa pohon keputusan.
Untuk dapat mengimplementasikannya, dibutuhkan algoritma yang dapat membuat pohon keputusan.
Ada tiga pohon keputusan yang umum dikenal yaitu ID3, CART, dan C4.5.
Algoritma yang dipilih untuk pohon keputusan yang digunakan nantinya pada \textit{Random Forest} yaitu CART karena,
\begin{itemize}
\item algoritma ID3 yang asli tidak mendukung atribut dengan nilai kontinu,
\item penjelasan dari algoritma C4.5 dalam bentuk pseudo kode yang mudah dipahami sulit ditemukan
\item algoritma CART mudah dipahami, mendukung nilai kontinu dan diskrit
\item penulis dari makalah \textit{Random Forest} adalah salah satu penulis dari makalah algoritma CART
\cite{breiman1984classification}
itu sendiri, walaupun secara eksplisit tidak disebutkan harus menggunakan algoritma CART dalam makalah \textit{Random Forest}.
\end{itemize}
%%}}}

Implementasi dari algoritma CART mengacu pada buku Jiawei Han, dkk.\cite{han2011data}, bab 8; karena makalah asli dari CART tidak bisa ditemukan.
Hasil implementasi telah dapat membuat pohon keputusan untuk nilai kontinu dan diskrit, tapi belum ada \textit{pruning} dan penanganan nilai yang hilang.
Hasilnya dapat dilihat di lampiran 
\ref{appendix:sumber_kode_cart}
atau di internet
\footnote{\url{https://github.com/shuLhan/go-mining/tree/master/classifiers/cart}}.

Implementasi CART kemudian dicoba pada dataset \textit{iris} (contoh data bisa dilihat pada lampiran \ref{appendix:dataset_iris}), dengan 150 \textit{record} dan lima atribut. Empat atribut bertipe kontinu, dan satu atribut target bertipe diskrit. Hasilnya adalah sebagai berikut,

\begin{lstlisting}
CART Tree:
 {false  true 2 2.45}
        {false  true 3 1.55}
                {false  true 0 5.95}
                        {true Iris-virginica false 0 <nil>}
                        {true Iris-versicolor false 0 <nil>}
                {false  true 0 5.95}
                        {false  true 1 3.25}
                                {true Iris-virginica false 0 <nil>}
                                {false  true 1 3.05}
                                        {false  true 0 6.550000000000001}
                                                {true Iris-versicolor false 0 <nil>}
                                                {true Iris-virginica false 0 <nil>}
                                        {true Iris-versicolor false 0 <nil>}
                        {true Iris-versicolor false 0 <nil>}
        {true Iris-setosa false 0 <nil>}
\end{lstlisting}

Baris kedua adalah \textit{root} dari pohon, dengan nilai atribut pada node $ { false true 2 2.45 } $ yaitu $false$ menyatakan \textit{node} bukan \textit{leaf}, $true$ menyatakan node dipisahkan oleh nilai kontinu, $2$ menyatakan indeks dari atribut pemisah, dan $2.45$ menyatakan nilai pemisah. Misalkan, jika diberikan sampel dengan nilai atribut dari indeks ke 2 adalah 1 (kecil dari 2.45) maka kelas dari sampel tersebut adalah \textit{Iris-setosa}.

Implementasi pendukung lainnya yaitu penghitungan nilai Gini index dan Gini gain, yang digunakan untuk menentukan partisi mana yang akan digunakan sebagai pemisah dari atribut.
Formula untuk menghitung \textit{Gini index} dari sebuah partisi yang terdiri dari dua bagian yaitu,

\begin{equation*}
Gini_{A}(D) = \frac{|D_{1}|}{|D|} Gini(D_{1}) + \frac{|D_{2}|}{|D|} Gini(D_{2})
\end{equation*}%
%
dengan \textit{Gini gain} dihitung menggunakan,

\begin{equation*}
	\Delta Gini(A) = Gini(D) - Gini_{A}(D)
\end{equation*}%
%
Hasil implementasi penghitungan \textit{Gini index} dan \textit{gain} dapat dilihat pada lampiran \ref{appendix:sumber_kode_gini} atau di internet \footnote{\url{https://github.com/shuLhan/go-mining/tree/master/gain/gini}}.

\subsection{Hubungan dengan Bilangan Stirling dan Partisi Set}

Pada saat membentuk pohon keputusan, atribut dari dataset latihan bisa bernilai diskrit, seperti $ \{A,B,B,C,A,A\} $ dengan nilai nominal (atau nilai yang mungkin muncul) yaitu $ \{A,B,C\} $. Untuk menentukan atribut yang paling bagus dalam memisahkan data, maka harus dibentuk semua kemungkinan partisi dari set $ \{A,B,C\} $ dan kemudian dihitung \textit{Gini gain} dari setiap partisi.
Misalkan partisi-dua dari set tersebut adalah $ \{\{A,B\},\{C\}\} $, $ \{\{A,C\},\{B\}\}\} $, dan $\{\{A\}\{B,C\}\}$, dengan total partisi yaitu tiga. Untuk dapat mengetahui jumlah partisi yang dapat dibuat tanpa harus memproses, dapat dihitung dengan menggunakan \textit{Stirling number of the second kind}, yang rumusnya adalah sebagai berikut,%
%
\begin{equation*}
	\begin{Bmatrix}
		n \\
		k
	\end{Bmatrix} = \frac{1}{k!} \sum_{j=0}^{k} (-1)^{k-j} \binom{k}{j} j^{n}
\end{equation*}%
%
yang mana $n$ adalah jumlah objek di dalam set dan $k$ adalah jumlah partisi.

Hasil implementasi penghitungan Stirling ini dapat dilihat pada lampiran \ref{appendix:sumber_kode_math} atau di internet \footnote{\url{https://github.com/shuLhan/go-mining/tree/master/math}}

Untuk membentuk partisi dari sebuah set itu sendiri menggunakan algoritma fungsi rekursif dengan mengambil objek yang pertama dari $n$ kemudian mengulangi pemanggilan fungsi partisi sampai $n$ sama dengan $k$ atau $k$ sama dengan 1. Hasil implementasinya dapat dilihat pada lampiran \ref{appendix:sumber_kode_setstring} atau di internet
\footnote{\url{https://github.com/shuLhan/go-mining/tree/master/set}}.


\section{Pekerjaan Selanjutnya}

Implementasi algoritma \textit{Random Forest}.

\schedule{4}{6}{0}{0}

\advisorsignature

\clearpage
\addcontentsline{toc}{section}{Daftar Referensi}
\printbibliography

\newpage
\appendix
\section{Catatan}

\textit{Random Forest} secara sederhananya adalah klasifikasi dengan hasil dari mayoritas \textit{voting} dari beberapa pohon keputusan.
Untuk dapat mengimplementasikannya, dibutuhkan algoritma yang dapat membuat pohon keputusan.
Ada tiga pohon keputusan yang umum dikenal yaitu ID3, CART, dan C4.5.
Algoritma yang dipilih untuk pohon keputusan yang digunakan nantinya pada \textit{Random Forest} yaitu CART karena,
\begin{itemize}
\item algoritma ID3 yang asli tidak mendukung atribut dengan nilai kontinu,
\item penjelasan dari algoritma C4.5 dalam bentuk pseudo kode yang mudah dipahami sulit ditemukan
\item algoritma CART mudah dipahami, mendukung nilai kontinu dan diskrit
\item penulis dari makalah \textit{Random Forest} adalah salah satu penulis dari makalah algoritma CART
\cite{breiman1984classification}
itu sendiri, walaupun secara eksplisit tidak disebutkan harus menggunakan algoritma CART dalam makalah \textit{Random Forest}.
\end{itemize}
%%}}}

Implementasi dari algoritma CART mengacu pada buku Jiawei Han, dkk.\cite{han2011data}, bab 8; karena makalah asli dari CART tidak bisa ditemukan.
Hasil implementasi telah dapat membuat pohon keputusan untuk nilai kontinu dan diskrit, tapi belum ada \textit{pruning} dan penanganan nilai yang hilang.
Hasilnya dapat dilihat di lampiran 
\ref{appendix:sumber_kode_cart}
atau di internet
\footnote{\url{https://github.com/shuLhan/go-mining/tree/master/classifiers/cart}}.

Implementasi CART kemudian dicoba pada dataset \textit{iris} (contoh data bisa dilihat pada lampiran \ref{appendix:dataset_iris}), dengan 150 \textit{record} dan lima atribut. Empat atribut bertipe kontinu, dan satu atribut target bertipe diskrit. Hasilnya adalah sebagai berikut,

\begin{lstlisting}
CART Tree:
 {false  true 2 2.45}
        {false  true 3 1.55}
                {false  true 0 5.95}
                        {true Iris-virginica false 0 <nil>}
                        {true Iris-versicolor false 0 <nil>}
                {false  true 0 5.95}
                        {false  true 1 3.25}
                                {true Iris-virginica false 0 <nil>}
                                {false  true 1 3.05}
                                        {false  true 0 6.550000000000001}
                                                {true Iris-versicolor false 0 <nil>}
                                                {true Iris-virginica false 0 <nil>}
                                        {true Iris-versicolor false 0 <nil>}
                        {true Iris-versicolor false 0 <nil>}
        {true Iris-setosa false 0 <nil>}
\end{lstlisting}

Baris kedua adalah \textit{root} dari pohon, dengan nilai atribut pada node $ { false true 2 2.45 } $ yaitu $false$ menyatakan \textit{node} bukan \textit{leaf}, $true$ menyatakan node dipisahkan oleh nilai kontinu, $2$ menyatakan indeks dari atribut pemisah, dan $2.45$ menyatakan nilai pemisah. Misalkan, jika diberikan sampel dengan nilai atribut dari indeks ke 2 adalah 1 (kecil dari 2.45) maka kelas dari sampel tersebut adalah \textit{Iris-setosa}.

Implementasi pendukung lainnya yaitu penghitungan nilai Gini index dan Gini gain, yang digunakan untuk menentukan partisi mana yang akan digunakan sebagai pemisah dari atribut.
Formula untuk menghitung \textit{Gini index} dari sebuah partisi yang terdiri dari dua bagian yaitu,

\begin{equation*}
Gini_{A}(D) = \frac{|D_{1}|}{|D|} Gini(D_{1}) + \frac{|D_{2}|}{|D|} Gini(D_{2})
\end{equation*}%
%
dengan \textit{Gini gain} dihitung menggunakan,

\begin{equation*}
	\Delta Gini(A) = Gini(D) - Gini_{A}(D)
\end{equation*}%
%
Hasil implementasi penghitungan \textit{Gini index} dan \textit{gain} dapat dilihat pada lampiran \ref{appendix:sumber_kode_gini} atau di internet \footnote{\url{https://github.com/shuLhan/go-mining/tree/master/gain/gini}}.

\subsection{Hubungan dengan Bilangan Stirling dan Partisi Set}

Pada saat membentuk pohon keputusan, atribut dari dataset latihan bisa bernilai diskrit, seperti $ \{A,B,B,C,A,A\} $ dengan nilai nominal (atau nilai yang mungkin muncul) yaitu $ \{A,B,C\} $. Untuk menentukan atribut yang paling bagus dalam memisahkan data, maka harus dibentuk semua kemungkinan partisi dari set $ \{A,B,C\} $ dan kemudian dihitung \textit{Gini gain} dari setiap partisi.
Misalkan partisi-dua dari set tersebut adalah $ \{\{A,B\},\{C\}\} $, $ \{\{A,C\},\{B\}\}\} $, dan $\{\{A\}\{B,C\}\}$, dengan total partisi yaitu tiga. Untuk dapat mengetahui jumlah partisi yang dapat dibuat tanpa harus memproses, dapat dihitung dengan menggunakan \textit{Stirling number of the second kind}, yang rumusnya adalah sebagai berikut,%
%
\begin{equation*}
	\begin{Bmatrix}
		n \\
		k
	\end{Bmatrix} = \frac{1}{k!} \sum_{j=0}^{k} (-1)^{k-j} \binom{k}{j} j^{n}
\end{equation*}%
%
yang mana $n$ adalah jumlah objek di dalam set dan $k$ adalah jumlah partisi.

Hasil implementasi penghitungan Stirling ini dapat dilihat pada lampiran \ref{appendix:sumber_kode_math} atau di internet \footnote{\url{https://github.com/shuLhan/go-mining/tree/master/math}}

Untuk membentuk partisi dari sebuah set itu sendiri menggunakan algoritma fungsi rekursif dengan mengambil objek yang pertama dari $n$ kemudian mengulangi pemanggilan fungsi partisi sampai $n$ sama dengan $k$ atau $k$ sama dengan 1. Hasil implementasinya dapat dilihat pada lampiran \ref{appendix:sumber_kode_setstring} atau di internet
\footnote{\url{https://github.com/shuLhan/go-mining/tree/master/set}}.


\end{document}
