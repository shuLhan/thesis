\newcommand{\judul}{%
	Deteksi Vandalisme pada Wikipedia Bahasa Inggris Menggunakan %
	Sampel Ulang LNSMOTE %
	dan Pengklasifikasi Cascaded Random Forest%
}
\newcommand{\mytitle}{%
	Detecting Vandalism on English Wikipedia using LNSMOTE Resampling and %
	Cascaded Random Forest Classifier%
}
\newcommand{\myname}{Muhamad Sulhan}
\newcommand{\mysid}{23513014}
\newcommand{\myemail}{ms@students.itb.ac.id}
\newcommand{\myschool}{School of Electrical and Informatics Engineering}
\newcommand{\tfakultas}{Sekolah Teknik Elektro dan Informatika}

\newcommand{\myadvisorname}{Dwi Hendratmo Widyantoro}
\newcommand{\myadvisorshortname}{Dwi H. Widyantoro}
\newcommand{\myadvisorid}{196812071994021001}
\newcommand{\myadvisoremail}{dwi@stei.itb.ac.id}
\newcommand{\mydept}{Program Studi Magister Informatika}

\newcommand{\itb}{Institut Teknologi Bandung}
\newcommand{\itbaddress}{Ganesha 10, Bandung, Indonesia 40132}

\newcommand{\tUpAbstrak}{ABSTRAK}
\newcommand{\tupabstract}{ABSTRACT}
\newcommand{\tuppengesahan}{HALAMAN PENGESAHAN}

\newcommand{\daftarisi}{DAFTAR ISI}
\newcommand{\tupdaftarlampiran}{DAFTAR LAMPIRAN}
\newcommand{\daftargambar}{DAFTAR GAMBAR DAN ILUSTRASI}
\newcommand{\daftartabel}{DAFTAR TABEL}

\newcommand{\tDaftarPustaka}{DAFTAR PUSTAKA}
\newcommand{\tLampiran}{Lampiran}
\newcommand{\tUpLampiran}{LAMPIRAN}

\newcommand{\tuppedoman}{PEDOMAN PENGGUNAAN TESIS}

%%% My images directory
\graphicspath{{../images/} {images/}}
\newcommand{\myitbcover}{ITB-logo-hitam}
\newcommand{\myitbcoverblue}{ITB-logo-ganesha}
